\documentclass[12pt, a4paper]{article}
\usepackage[utf8]{inputenc}

\usepackage[a4paper,top=3cm,bottom=2cm,left=2.5cm,right=2.5cm,marginparwidth=1.75cm]{geometry}
%\usepackage[utf8]{inputenc}
%\usepackage[T1]{fontenc}
\usepackage{adjustbox}
\usepackage{amsmath}
\usepackage{amssymb}
\usepackage[ngerman, english]{babel}
\usepackage{booktabs}
\usepackage{csquotes}
\usepackage{float}
\usepackage[maxfloats=256]{morefloats}
\usepackage{graphicx}
\usepackage{longtable}
\usepackage{lscape}
\usepackage{pgffor}
\usepackage{subfig,caption}
\usepackage{tabularx}
\usepackage{threeparttablex}
\usepackage{xcolor}
\usepackage{xtab}
\usepackage{hyperref}

\maxdeadcycles = 1000

% everything concerning biblatex

\usepackage[backend = biber,
            style=authoryear-icomp,
            citestyle=authoryear-comp,
            url=false,
            isbn=false,
            doi=false,
            maxcitenames=2,
            uniquelist=false,
            date=year,
            uniquename=false]{biblatex}

\addbibresource{References.bib} % TBD

% Setup
\graphicspath{{../1_Figures/}}

% new environment

\newenvironment{subcaption}
{\strut
\vspace{-5pt}
\begin{minipage}[b]{0.9\textwidth}
  \hspace*{-\parindent}
  \footnotesize}
 {\end{minipage}}

 % Title

\title{Making sense of heterogeneity in carbon intensity of consumption}
\author{Leonard Missbach\thanks{Mercator Research Institute on Global Commons and Climate Change, Berlin, Germany; Department Economics of Climate Change, Technische Universität Berlin, Berlin, Germany}}
\date{January 2023}

\begin{document}

\maketitle
\begin{abstract}
  In this study ...
\end{abstract}

\smallskip

\noindent \small \textit{Keywords:} Climate mitigation, inequality, poverty

\noindent \small \textit{JEL Codes:}

\thispagestyle{empty}
\clearpage
\setcounter{page}{1}

\section{Main} \label{sec:main}

\begin{itemize}
  \item Policy instruments that help to combat $CO_{2}$-emissions may have strengths and weaknesses with respect to efficiency, effectiveness, tractability, institutional feasibility, but also political feasibility, which may also depend on equity and fairness.
  \item Inequality in carbon intensity of consumption can matter for political feasibility evaluation. Different climate policy instruments affect consumers differently depending, among others, on the carbon intensity of their respective consumption. Other factors include household-specific abilities to substitute away from carbon-intensive goods and services, industries' ability to reduce sectoral carbon intensity of consumption, and household-level price elasticities of demand.
  \item Some policy instruments are considered superior with regard to aggregate cost-effectiveness, but might require supplementary measures, such as subsidies, allowances, or transfers to address unintended distributional consequences for households.
  \item Ex-ante economic assessments of distributional impacts of climate policy often focus on single-country and/or -policy contexts. Analyses of multiple countries often use more aggregate data, which allows for investigation of approximate trends.
  \item Here, we construct a unique dataset on household-level carbon intensity of consumption for 1.5 million single households representative for 5 billion people in 80 countries which comprise Y\% of global GDP and X\% of global $CO_{2}$-emissions.
  \item Granularity of data allows for detailed analysis of factors associated to country- and sector-level particularities, household-level consumption patterns and sociodemographic factors that are associated to larger or lower levels of carbon intensity of consumption.
  \item Our analysis informs further research on potential drivers of carbon emissions and inequality in carbon-intensive consumption. It might inform policy to develop international, but also national complementary measures to combat unintended effects of effective climate mitigation policies.
  \item We deploy multiple analysis tools to support robust conclusions.
\end{itemize}

% motivation and research gap

\begin{itemize}
  \item Many criteria influence the evaluation of different policy instruments. One important aspect are distributional implications. In the context of climate policy people may ask: Who loses from climate policy and why?
  \item One important metric that helps assessing the impacts of climate policy is the carbon intensity of consumption or production. For households, the carbon intensity expresses the amount of $CO_{2}$-emissions that link to the consumption of consumed goods and services per unit of expenditure. The carbon intensity of consumption gives an accurate approximation of the cost burden resulting from any policy instrument that directly or indirectly increases the price of $CO_{2}$-emissions.
  \item Households that consume more carbon-intensive than others spend relatively more money on carbon-intensive goods and services, such as transport fuels, heating fuels, cooking fuels, or electricity.
  \item Evidence suggests many socio-economic socio-demographic charateristics to drive or correlate with more carbon-intensive consumption.
  \item Studies often focus on single-country contexts or aggregated cross-country analyses. What is missing is a systematic cross-country analyses with detailed household-level information.
\end{itemize}

% research question

We analyse the heterogeneity in carbon intensity of consumption within countries and compare results across countries. Transgressing traditional analyses of vertical and horizontal heterogeneity, we study country-level household characteristics that correlate with and drive the carbon intensity of consumption.

% answer

We find and confirm that households whose consumption is more carbon-intensive than others are in general

\begin{itemize}
  \item poorer than others are in richer countries,
  \item richer than others are in poorer countries,
  \item more likely to own (and use) a motorized vehicle, most notably a car,
  \item more likely to use fossil fuels for cooking, such as coal, gas, or LPG,
  \item less likely to use firewood, charcoal or biomass for cooking, compared to households cooking with electricity,
  \item more likely to be connected to the electricity grid,
  \item more likely to having had secondary and higher education,
  \item more likely to live in rural areas,
  \item more/less likely to identify with ethnic majorities/minorities,
  \item more/less likely to have a female household head,
  \item more/less likely to receive income from governmental cash transfers.
\end{itemize}

On a country-level, we observe differences across space, districts, and provinces.

% relevant literature

Heterogeneity in carbon-intensity can be attributed to differences in observable characteristics, which have been identified extensively before.

% Main body of paper
\begin{itemize}
  \item Why distributional implications of climate policy are subject to academic research.
  \item Theory and empirical results on vertical impacts of taxes, subdidies, standards.
  \item Show differences between poor and rich households.
  \item Theory and empirical results on horizontal impacts of taxes, subsidies, standards.
  \item Show differences within poor and rich households.
  \item Discuss main arguments for single characteristics.
  \item Show results for different characteristics.
\end{itemize}
% Discussion

% conclusion


\clearpage

\section{Data and Methods} \label{sec:data_and_methods}

\subsection{Multi-regional input-output data}

% Critical discussion

\subsection{Household expenditure data}

% Critical discussion

\subsection{Microsimulation}

% Critical discussion

\subsection{Econometric analysis}

% Critical discussion

\subsection{Critical appraisal}

% Critical discussion

\subsection{Data}

\paragraph{Household-level consumption data}

\paragraph{Multi-regional input-output data}

\subsection{Method}

\paragraph{Household-level expenditure shares on consumption sectors}

\paragraph{Multi-regional input-output model}

\paragraph{Household-level carbon intensity of consumption}

\subsection{Econometric analysis}

\paragraph{Descriptive statistics}

\paragraph{OLS-regression}

\paragraph{Logit-regression}

\paragraph{Inequality decomposition}

\paragraph{Boosted regression trees}

\paragraph{Methodological discussion}

\clearpage

\printbibliography

\clearpage

\appendix

\section{Appendix} \label{sec:appendix}

\subsection{Data cleaning}

% tracking and documenting removals

% transfer schemes ?

% Electriciy generation

\clearpage

\renewcommand\thefigure{\thesection.\arabic{figure}}
\renewcommand\thetable{\thesection.\arabic{table}}
\setcounter{figure}{0}
\setcounter{table}{0}

\subsection{Supplementary figures} \label{sec:figures}

\begin{figure}[ht!]
  \centering
  \caption{Engel curves: expenditure shares over total household expenditures - Part A} \label{fig:A1}
  \includegraphics{Analysis_Parametric_Engel_Curves/Parametric_EC_0_A}
  \begin{subcaption}
    This figure displays fitted lines for parametric and quadratic Engel curves for each consumption category in 20 countries of our sample. Black vertical lines indicate average household expenditures per capita for each expenditure quintile and country.
  \end{subcaption}

\end{figure}

\clearpage

\begin{figure}[ht!]
  \centering
  \caption{Engel curves: expenditure shares over total household expenditures - Part B} \label{fig:A2}
  \includegraphics{Analysis_Parametric_Engel_Curves/Parametric_EC_0_B}
  \begin{subcaption}
    This figure displays fitted lines for parametric and quadratic Engel curves for each consumption category in 20 countries of our sample. Black vertical lines indicate average household expenditures per capita for each expenditure quintile and country.
  \end{subcaption}

\end{figure}

\clearpage

\begin{figure}[ht!]
  \centering
  \caption{Engel curves: expenditure shares over total household expenditures - Part C} \label{fig:A3}
  \includegraphics{Analysis_Parametric_Engel_Curves/Parametric_EC_0_C}
  \begin{subcaption}
    This figure displays fitted lines for parametric and quadratic Engel curves for each consumption category in 20 countries of our sample. Black vertical lines indicate average household expenditures per capita for each expenditure quintile and country.
  \end{subcaption}

\end{figure}

\clearpage

\begin{figure}[ht!]
  \centering
  \caption{Engel curves: expenditure shares over total household expenditures - Part D} \label{fig:A4}
  \includegraphics{Analysis_Parametric_Engel_Curves/Parametric_EC_0_D}
  \begin{subcaption}
    This figure displays fitted lines for parametric and quadratic Engel curves for each consumption category in 20 countries of our sample. Black vertical lines indicate average household expenditures per capita for each expenditure quintile and country.
  \end{subcaption}

\end{figure}

\clearpage

% Carbon intensities

\begin{figure}[ht!]
  \centering
  \caption{Sectoral carbon intensities from GTAP - Part A} \label{fig:B1}
  \includegraphics{Analysis_Carbon_Intensities_GTAP/Figure_2.1.1_A}
  \begin{subcaption}
    This figure displays...
  \end{subcaption}

\end{figure}

\clearpage

\begin{figure}[ht!]
  \centering
  \caption{Sectoral carbon intensities from GTAP - Part B} \label{fig:B2}
  \includegraphics{Analysis_Carbon_Intensities_GTAP/Figure_2.1.1_B}
  \begin{subcaption}
    This figure displays...
  \end{subcaption}

\end{figure}

\clearpage

\begin{figure}[ht!]
  \centering
  \caption{Sectoral carbon intensities from GTAP - Part C} \label{fig:B3}
  \includegraphics{Analysis_Carbon_Intensities_GTAP/Figure_2.1.1_C}
  \begin{subcaption}
    This figure displays...
  \end{subcaption}

\end{figure}

\clearpage

\begin{figure}[ht!]
  \centering
  \caption{Sectoral carbon intensities from GTAP - Part D} \label{fig:B4}
  \includegraphics{Analysis_Carbon_Intensities_GTAP/Figure_2.1.1_D}
  \begin{subcaption}
    This figure displays...
  \end{subcaption}

\end{figure}

\clearpage

% Carbon intensity of consumption over total household expenditures

\begin{figure}[ht!]
  \centering
  \caption{Carbon intensity of consumption over total household expenditures - Part A} \label{fig:C1}
  \includegraphics{Analysis_Carbon_Intensity_Curve/All_Panel_A}
  \begin{subcaption}
    This figure displays carbon intensity of aggregate consumption (in $kgCO_{2}/USD$) over total household expenditures in USD for 20 countries in our sample. Household expenditures are inflated (or deflated) to 2014. Points represent single households. The red line represents a fitted curve for a quadratic OLS-regression including a 95\%-confidence interval.
  \end{subcaption}

\end{figure}

\clearpage

\begin{figure}[ht!]
  \centering
  \caption{Carbon intensity of consumption over total household expenditures - Part B} \label{fig:C2}
  \includegraphics{Analysis_Carbon_Intensity_Curve/All_Panel_B}
  \begin{subcaption}
    This figure displays carbon intensity of aggregate consumption (in $kgCO_{2}/USD$) over total household expenditures in USD for 20 countries in our sample. Household expenditures are inflated (or deflated) to 2014. Points represent single households. The red line represents a fitted curve for a quadratic OLS-regression including a 95\%-confidence interval.
  \end{subcaption}

\end{figure}

\clearpage

\begin{figure}[ht!]
  \centering
  \caption{Carbon intensity of consumption over total household expenditures - Part C} \label{fig:C3}
  \includegraphics{Analysis_Carbon_Intensity_Curve/All_Panel_C}
  \begin{subcaption}
    This figure displays carbon intensity of aggregate consumption (in $kgCO_{2}/USD$) over total household expenditures in USD for 20 countries in our sample. Household expenditures are inflated (or deflated) to 2014. Points represent single households. The red line represents a fitted curve for a quadratic OLS-regression including a 95\%-confidence interval.
  \end{subcaption}

\end{figure}

\clearpage

\begin{figure}[ht!]
  \centering
  \caption{Carbon intensity of consumption over total household expenditures - Part D} \label{fig:C4}
  \includegraphics{Analysis_Carbon_Intensity_Curve/All_Panel_D}
  \begin{subcaption}
    This figure displays carbon intensity of aggregate consumption (in $kgCO_{2}/USD$) over total household expenditures in USD for 20 countries in our sample. Household expenditures are inflated (or deflated) to 2014. Points represent single households. The red line represents a fitted curve for a quadratic OLS-regression including a 95\%-confidence interval.
  \end{subcaption}

\end{figure}

\clearpage

\begin{figure}[ht!]
  \centering
 \caption{Average marginal effects of car ownership} \label{fig:D1_Car}
  \includegraphics{Analysis_OLS_ME_Carbon_Footprint/AME_OLS_FP_car.01}
  \begin{subcaption}
    This figure displays ...
  \end{subcaption}

\end{figure}

\clearpage

\begin{figure}[ht!]
  \centering
 \caption{Average marginal effects of electricity access} \label{fig:D2_Electricity}
  \includegraphics{Analysis_OLS_ME_Carbon_Footprint/AME_OLS_FP_electricity.access}
  \begin{subcaption}
    This figure displays ...
  \end{subcaption}

\end{figure}

\clearpage

\begin{figure}[ht!]
  \centering
 \caption{Average marginal effects of household size} \label{fig:D3_Size}
  \includegraphics{Analysis_OLS_ME_Carbon_Footprint/AME_OLS_FP_hh_size}
  \begin{subcaption}
    This figure displays ...
  \end{subcaption}

\end{figure}

\clearpage

\begin{figure}[ht!]
  \centering
 \caption{Average marginal effects of household expenditures} \label{fig:D4_Expenditures}
  \includegraphics{Analysis_OLS_ME_Carbon_Footprint/AME_OLS_FP_log_hh_expenditures_USD_2014}
  \begin{subcaption}
    This figure displays ...
  \end{subcaption}

\end{figure}

\clearpage

\begin{figure}[ht!]
  \centering
 \caption{Average marginal effects of urban citizenship} \label{fig:D5_Urban}
  \includegraphics{Analysis_OLS_ME_Carbon_Footprint/AME_OLS_FP_urban_01}
  \begin{subcaption}
    This figure displays ...
  \end{subcaption}

\end{figure}

\clearpage

\begin{figure}[ht!]
  \centering
 \caption{Average marginal effects of cooking fuel choice - Part A} \label{fig:D6_Electricity_A}
  \includegraphics{Analysis_OLS_ME_Carbon_Footprint/AME_OLS_FP_CF_Electricity A}
  \begin{subcaption}
    This figure displays ...
  \end{subcaption}

\end{figure}

\clearpage

\begin{figure}[ht!]
  \centering
 \caption{Average marginal effects of cooking fuel choice - Part B} \label{fig:D7_Electricity_B}
  \includegraphics{Analysis_OLS_ME_Carbon_Footprint/AME_OLS_FP_CF_Electricity B}
  \begin{subcaption}
    This figure displays ...
  \end{subcaption}

\end{figure}

\clearpage

\begin{figure}[ht!]
  \centering
 \caption{Average marginal effects of cooking fuel choice - Part C} \label{fig:D8_LPG}
  \includegraphics{Analysis_OLS_ME_Carbon_Footprint/AME_OLS_FP_CF_LPG}
  \begin{subcaption}
    This figure displays ...
  \end{subcaption}

\end{figure}

\clearpage

\begin{figure}[ht!]
  \centering
 \caption{Average marginal effects of cooking fuel choice - Part D} \label{fig:D9_Charcoal}
  \includegraphics{Analysis_OLS_ME_Carbon_Footprint/AME_OLS_FP_CF_Charcoal}
  \begin{subcaption}
    This figure displays ...
  \end{subcaption}

\end{figure}

\clearpage

\begin{figure}[ht!]
  \centering
 \caption{Average marginal effects of secondary education} \label{fig:D10_sec_edu}
  \includegraphics{Analysis_OLS_ME_Carbon_Footprint/AME_OLS_FP_secondary_education}
  \begin{subcaption}
    This figure displays ... Finland special case
  \end{subcaption}

\end{figure}

\clearpage

\begin{figure}[ht!]
  \centering
 \caption{Average marginal effects of higher education} \label{fig:D11_high_edu}
  \includegraphics{Analysis_OLS_ME_Carbon_Footprint/AME_OLS_FP_higher_education}
  \begin{subcaption}
    This figure displays ... Finland special case
  \end{subcaption}

\end{figure}

\clearpage

\begin{figure}[ht!]
  \centering
 \caption{Average marginal effects of car ownership} \label{fig:E1_Car}
  \includegraphics{Analysis_OLS_ME_Carbon_Intensity/AME_OLS_CI_car.01}
  \begin{subcaption}
    This figure displays ...
  \end{subcaption}

\end{figure}

\clearpage

\begin{figure}[ht!]
  \centering
 \caption{Average marginal effects of electricity access} \label{fig:E2_Electricity}
  \includegraphics{Analysis_OLS_ME_Carbon_Intensity/AME_OLS_CI_electricity.access}
  \begin{subcaption}
    This figure displays ...
  \end{subcaption}

\end{figure}

\clearpage

\begin{figure}[ht!]
  \centering
 \caption{Average marginal effects of household size} \label{fig:E3_Size}
  \includegraphics{Analysis_OLS_ME_Carbon_Intensity/AME_OLS_CI_hh_size}
  \begin{subcaption}
    This figure displays ...
  \end{subcaption}

\end{figure}

\clearpage

\begin{figure}[ht!]
  \centering
 \caption{Average marginal effects of household expenditures} \label{fig:E4_Expenditures}
  \includegraphics{Analysis_OLS_ME_Carbon_Intensity/AME_OLS_CI_log_hh_expenditures_USD_2014}
  \begin{subcaption}
    This figure displays ...
  \end{subcaption}

\end{figure}

\clearpage

\begin{figure}[ht!]
  \centering
 \caption{Average marginal effects of urban citizenship} \label{fig:E5_Urban}
  \includegraphics{Analysis_OLS_ME_Carbon_Intensity/AME_OLS_CI_urban_01}
  \begin{subcaption}
    This figure displays ...
  \end{subcaption}

\end{figure}

\clearpage

\begin{figure}[ht!]
  \centering
 \caption{Average marginal effects of cooking fuel choice - Part A} \label{fig:E6_Electricity_A}
  \includegraphics{Analysis_OLS_ME_Carbon_Intensity/AME_OLS_CI_CI_Electricity A}
  %\includegraphics{Analysis_OLS_ME_Carbon_Intensity/AME_OLS_CI_CF_Electricity A}
  \begin{subcaption}
    This figure displays ...
  \end{subcaption}

\end{figure}

\clearpage

\begin{figure}[ht!]
  \centering
 \caption{Average marginal effects of cooking fuel choice - Part B} \label{fig:E7_Electricity_B}
  \includegraphics{Analysis_OLS_ME_Carbon_Intensity/AME_OLS_CI_CI_Electricity B}
  %\includegraphics{Analysis_OLS_ME_Carbon_Intensity/AME_OLS_CI_CF_Electricity B}
  \begin{subcaption}
    This figure displays ...
  \end{subcaption}

\end{figure}

\clearpage

\begin{figure}[ht!]
  \centering
 \caption{Average marginal effects of cooking fuel choice - Part C} \label{fig:E8_LPG}
  \includegraphics{Analysis_OLS_ME_Carbon_Intensity/AME_OLS_CI_CI_LPG}
  %\includegraphics{Analysis_OLS_ME_Carbon_Intensity/AME_OLS_CI_CF_LPG}
  \begin{subcaption}
    This figure displays ...
  \end{subcaption}

\end{figure}

\clearpage

\begin{figure}[ht!]
  \centering
 \caption{Average marginal effects of cooking fuel choice - Part D} \label{fig:E9_Charcoal}
  \includegraphics{Analysis_OLS_ME_Carbon_Intensity/AME_OLS_CI_CI_Charcoal}
  %\includegraphics{Analysis_OLS_ME_Carbon_Intensity/AME_OLS_CI_CF_Charcoal}
  \begin{subcaption}
    This figure displays ...
  \end{subcaption}

\end{figure}

\clearpage

\begin{figure}[ht!]
  \centering
 \caption{Average marginal effects of secondary education} \label{fig:E10_sec_edu}
  \includegraphics{Analysis_OLS_ME_Carbon_Intensity/AME_OLS_CI_secondary_education}
  \begin{subcaption}
    This figure displays ... Special case finland
  \end{subcaption}

\end{figure}

\clearpage

\begin{figure}[ht!]
  \centering
 \caption{Average marginal effects of higher education} \label{fig:E11_high_edu}
  \includegraphics{Analysis_OLS_ME_Carbon_Intensity/AME_OLS_CI_higher_education}
  \begin{subcaption}
    This figure displays ... Special case finland
  \end{subcaption}

\end{figure}

\clearpage

 \begin{figure}[ht!]
   \centering
  \caption{Average marginal effects of car ownership} \label{fig:F1_Car}
   \includegraphics{Analysis_Logit_Models_Marginal_Effects/Average_Marginal_Effects_affected_upper_80_car.01}
   \begin{subcaption}
     This figure displays ...
   \end{subcaption}

 \end{figure}

 \clearpage
%
 \begin{figure}[ht!]
   \centering
   \caption{Average marginal effects of electricity.access} \label{fig:F2_Electricity}
   \includegraphics{Analysis_Logit_Models_Marginal_Effects/Average_Marginal_Effects_affected_upper_80_electricity.access}
   \begin{subcaption}
     This figure displays ...
   \end{subcaption}
 \end{figure}

 \clearpage

 \begin{figure}[ht!]
   \centering
   \caption{Average marginal effects of household size} \label{fig:F3_Size}
   \includegraphics{Analysis_Logit_Models_Marginal_Effects/Average_Marginal_Effects_affected_upper_80_hh_size}
   \begin{subcaption}
     This figure displays ...
   \end{subcaption}
 \end{figure}

 \clearpage

 \begin{figure}[ht!]
   \centering
   \caption{Average marginal effects of household expenditures } \label{fig:F4_Expenditures}
   \includegraphics{Analysis_Logit_Models_Marginal_Effects/Average_Marginal_Effects_affected_upper_80_log_hh_expenditures_USD_2014}
   \begin{subcaption}
     This figure displays ...
   \end{subcaption}
 \end{figure}

 \clearpage

 \begin{figure}[ht!]
   \centering
   \caption{Average marginal effects of urban citizenship} \label{fig:F5_Urban}
   \includegraphics{Analysis_Logit_Models_Marginal_Effects/Average_Marginal_Effects_affected_upper_80_urban_01}
   \begin{subcaption}
     This figure displays ...
   \end{subcaption}
 \end{figure}

 \clearpage

 \begin{figure}[ht!]
   \centering
   \caption{Average marginal effects of cooking fuel choice - Part A} \label{fig:F6_Electricity_A}
   \includegraphics{Analysis_Logit_Models_Marginal_Effects/Average_Marginal_Effects_affected_upper_80_CF_Electricity A}
   \begin{subcaption}
     This figure displays ...
   \end{subcaption}
 \end{figure}

 \clearpage

 \begin{figure}[ht!]
   \centering
   \caption{Average marginal effects of cooking fuel choice - Part B} \label{fig:F7_Electricity_B}
   \includegraphics{Analysis_Logit_Models_Marginal_Effects/Average_Marginal_Effects_affected_upper_80_CF_Electricity B}
   \begin{subcaption}
     This figure displays ...
   \end{subcaption}
 \end{figure}

 \clearpage

 \begin{figure}[ht!]
   \centering
   \caption{Average marginal effects of cooking fuel choice - Part C} \label{fig:F8_LPG}
   \includegraphics{Analysis_Logit_Models_Marginal_Effects/Average_Marginal_Effects_affected_upper_80_CF_LPG}
   \begin{subcaption}
     This figure displays ...
   \end{subcaption}
 \end{figure}

 \clearpage

 \begin{figure}[ht!]
   \centering
   \caption{Average marginal effects of cooking fuel choice - Part D} \label{fig:F9_Charcoal}
   \includegraphics{Analysis_Logit_Models_Marginal_Effects/Average_Marginal_Effects_affected_upper_80_CF_Charcoal}
   \begin{subcaption}
     This figure displays ...
   \end{subcaption}
 \end{figure}

 \clearpage

 \begin{figure}[ht!]
   \centering
   \caption{Average marginal effects of secondary education} \label{fig:F10_sec_edu}
   \includegraphics{Analysis_Logit_Models_Marginal_Effects/Average_Marginal_Effects_affected_upper_80_secondary_education}
   \begin{subcaption}
     This figure displays ...
   \end{subcaption}
 \end{figure}

 \clearpage

 \begin{figure}[ht!]
   \centering
   \caption{Average marginal effects of higher education} \label{fig:F11_high_edu}
   \includegraphics{Analysis_Logit_Models_Marginal_Effects/Average_Marginal_Effects_affected_upper_80_higher_education}
   \begin{subcaption}
     This figure displays ...
   \end{subcaption}
 \end{figure}

 \clearpage

\subsection{Supplementary tables} \label{sec:tables}

\begingroup\fontsize{9}{11}\selectfont

\begin{ThreePartTable}
\begin{TableNotes}
\item \textit{Note: } 
\item This table provides summary statistics for households in our sample. All values (except observations) are household-weighted averages.
\end{TableNotes}
\begin{longtable}[t]{>{\raggedright\arraybackslash}p{1.5 cm}|>{\raggedleft\arraybackslash}p{1.5 cm}>{\centering\arraybackslash}p{1.5 cm}>{\centering\arraybackslash}p{1.5 cm}>{\centering\arraybackslash}p{1.5 cm}>{\centering\arraybackslash}p{1.5 cm}>{\centering\arraybackslash}p{1.5 cm}>{\centering\arraybackslash}p{1.5 cm}}
\caption{\label{tab:A1}Summary statistics}\\
\toprule
Country & Observations & Average 
Household Size & Urban 
Population & Electricity 
Access & Average 
Household 
Expenditures [USD] & Car 
Ownership & Share of 
Firewood or 
 Charcoal Cons.\\
\midrule
\endfirsthead
\caption[]{Summary statistics \textit{(continued)}}\\
\toprule
Country & Observations & Average 
Household Size & Urban 
Population & Electricity 
Access & Average 
Household 
Expenditures [USD] & Car 
Ownership & Share of 
Firewood or 
 Charcoal Cons.\\
\midrule
\endhead

\endfoot
\bottomrule
\insertTableNotes
\endlastfoot
Argentina & 21,540 & 3.19 &  & 99.9\% & 15,810 & 49\% & 5\%\\
Armenia & 7,776 & 3.63 & 66\% & 99.8\% & 4,779 & 32\% & 1\%\\
Austria & 7,162 & 2.23 &  &  & 38,002 & 77\% & 28\%\\
Bangladesh & 12,240 & 4.50 & 27\% & 55.2\% & 2,438 & 1\% & 39\%\\
Barbados & 2,434 & 2.62 &  & 94.7\% & 17,652 & 52\% & 0\%\\
Belgium & 6,133 & 2.31 & 96\% &  & 32,310 &  & 9\%\\
Benin & 8,012 & 5.21 & 47\% & 33.1\% & 2,690 & 3\% & 97\%\\
Bolivia & 11,859 & 3.34 & 69\% & 94.7\% & 4,089 & 17\% & 12\%\\
Brazil & 57,889 & 3.01 & 86\% & 99.5\% & 11,075 & 46\% & 3\%\\
Bulgaria & 2,964 & 2.37 & 71\% &  & 5,357 &  & 37\%\\
Burkina Faso & 7,010 & 6.51 & 31\% & 24.4\% & 2,660 & 4\% & 92\%\\
Cambodia & 1,206 & 4.34 & 27\% &  & 5,630 & 11\% & 73\%\\
Canada & 4,012 & 2.32 &  &  & 48,762 & 86\% & 0\%\\
Chile & 15,237 & 3.29 &  &  & 19,014 &  & 11\%\\
Colombia & 86,866 & 3.35 & 79\% & 98.3\% & 6,856 & 14\% & 9\%\\
Costa Rica & 7,046 & 3.24 & 71\% & 99.7\% & 11,830 & 45\% & 5\%\\
Côte d’Ivoire & 12,992 & 4.48 & 52\% & 64.1\% & 3,247 & 3\% & 77\%\\
Croatia & 2,029 & 2.89 & 59\% &  & 11,890 &  & 51\%\\
Cyprus & 2,876 & 2.70 & 74\% &  & 26,575 &  & 21\%\\
Czechia & 2,905 & 2.22 & 67\% &  & 11,098 &  & 22\%\\
Denmark & 2,205 & 2.12 & 67\% &  & 37,759 &  & 21\%\\
Dominican Republic & 8,884 & 3.21 & 81\% & 97.5\% & 7,549 & 21\% & 7\%\\
Ecuador & 28,263 & 3.68 & 69\% & 90.5\% & 6,831 & 19\% & 5\%\\
Egypt & 12,485 & 4.17 & 46\% & 99.5\% & 2,449 & 7\% & 0\%\\
El Salvador & 23,622 & 3.67 & 64\% & 95.7\% & 5,758 & 15\% & 12\%\\
Estonia & 3,395 & 2.24 & 51\% &  & 11,994 &  & 33\%\\
Ethiopia & 6,767 & 4.48 & 32\% & 55.9\% & 1,167 & 1\% & 96\%\\
Finland & 3,673 & 2.02 & 71\% &  & 31,618 &  & 43\%\\
France & 16,978 & 2.23 & 69\% &  & 26,865 &  & 0\%\\
Georgia & 13,247 & 2.44 & 61\% & 100\% & 2,436 & 29\% & 5\%\\
Germany & 52,388 & 2.00 & 90\% &  & 28,683 &  & 0\%\\
Ghana & 13,521 & 3.91 & 56\% & 83.1\% & 2,380 & 4\% & 83\%\\
Greece & 6,140 & 2.58 & 72\% &  & 19,219 &  & 28\%\\
Guatemala & 11,535 & 4.77 & 54\% & 81\% & 5,677 & 17\% & 70\%\\
Guinea-Bissau & 5,351 & 8.18 & 47\% & 21.7\% & 3,691 & 3\% & 99\%\\
Hungary & 7,183 & 2.34 & 56\% &  & 8,385 &  & 42\%\\
India & 101,581 & 4.43 & 31\% & 79.9\% & 1,612 & 4\% & 63\%\\
Indonesia & 295,116 & 3.77 & 55\% & 98.5\% & 2,838 & 11\% & 29\%\\
Iraq & 24,994 & 6.73 & 72\% & 99.3\% & 14,006 & 35\% & 3\%\\
Ireland & 6,837 & 2.73 & 65\% &  & 33,816 &  & 31\%\\
Israel & 8,786 & 3.28 & 90\% &  & 39,035 & 72\% & 0\%\\
Italy & 14,636 & 2.37 & 82\% &  & 23,955 &  & 15\%\\
Jordan & 4,850 & 5.11 & 83\% &  & 11,973 & 51\% & 0\%\\
Kenya & 21,714 & 3.98 & 44\% & 56.4\% & 2,468 &  & 82\%\\
Latvia & 3,844 & 2.37 & 56\% &  & 10,195 &  & 0\%\\
Liberia & 8,332 & 4.27 & 52\% & 16.7\% & 2,568 & 2\% & 99\%\\
Lithuania & 3,441 & 2.15 & 47\% &  & 8,884 &  & 33\%\\
Luxembourg & 3,163 & 2.42 & 81\% &  & 50,165 &  & 0\%\\
Malawi & 11,374 & 4.40 & 16\% & 10.7\% & 707 & 2\% & 99\%\\
Maldives & 4,749 & 5.19 &  &  & 20,199 & 5\% & 0\%\\
Mali & 6,602 & 7.14 & 28\% & 27.5\% & 3,458 & 4\% & 99\%\\
Mexico & 74,158 & 3.61 & 77\% & 99.5\% & 5,945 & 38\% & 16\%\\
Mongolia & 11,197 & 3.58 & 66\% &  & 5,939 &  & 44\%\\
Morocco & 15,970 & 4.74 & 65\% &  & 7,374 &  & 21\%\\
Mozambique & 11,335 & 5.01 & 31\% & 25.3\% & 2,872 & 1\% & 96\%\\
Myanmar (Burma) & 3,648 & 4.53 & 29\% & 63\% & 2,347 & 4\% & 88\%\\
Netherlands & 14,407 & 2.19 & 90\% &  & 34,292 &  & 1\%\\
Nicaragua & 6,850 & 4.38 & 60\% & 86.8\% & 4,799 & 8\% & 51\%\\
Niger & 6,024 & 5.96 & 17\% & 15.7\% & 1,901 & 2\% & 97\%\\
Nigeria & 22,110 & 5.08 & 40\% & 63.4\% & 3,013 & 8\% & 70\%\\
Norway & 3,363 & 2.77 & 82\% &  & 53,131 & 88\% & 0\%\\
Pakistan & 23,886 & 6.32 & 37\% & 90.1\% & 3,491 &  & 25\%\\
Paraguay & 5,410 & 3.90 & 61\% & 97.8\% & 7,393 & 25\% & 29\%\\
Peru & 34,542 & 3.56 & 77\% & 95.6\% & 4,673 & 12\% & 15\%\\
Philippines & 41,540 & 4.60 & 44\% & 91.1\% & 4,468 & 7\% & 45\%\\
Poland & 37,115 & 2.80 & 64\% &  & 12,779 &  & 6\%\\
Portugal & 11,392 & 2.53 & 73\% &  & 17,731 &  & 9\%\\
Romania & 30,605 & 2.66 & 58\% &  & 5,094 &  & 9\%\\
Russia & 4,831 & 2.60 &  &  & 7,511 & 41\% & 3\%\\
Rwanda & 14,577 & 4.39 & 19\% &  & 1,262 & 1\% & 41\%\\
Senegal & 7,156 & 8.91 & 53\% & 63.7\% & 6,705 & 5\% & 86\%\\
Serbia & 6,350 & 2.68 & 62\% & 99.9\% & 7,608 & 91\% & 14\%\\
Slovakia & 4,785 & 2.93 & 71\% &  & 12,839 &  & 19\%\\
South Africa & 22,964 & 3.53 & 70\% & 92.7\% & 6,958 & 27\% & 10\%\\
Spain & 22,127 & 2.50 & 75\% &  & 22,569 &  & 0\%\\
Suriname & 2,025 & 3.39 & 72\% &  & 7,589 & 38\% & 0\%\\
Sweden & 2,871 & 2.13 & 45\% &  & 29,741 &  & 0\%\\
Switzerland & 9,955 & 2.14 &  &  & 76,279 & 77\% & 0\%\\
Taiwan & 16,528 & 3.02 &  &  & 20,687 & 61\% & 0\%\\
Thailand & 42,711 & 3.04 & 36\% & 99.8\% & 3,747 & 14\% & 26\%\\
Togo & 6,171 & 4.23 & 47\% & 51.8\% & 2,381 & 3\% & 92\%\\
Turkey & 10,060 & 3.64 & 70\% &  & 9,986 & 39\% & 4\%\\
Uganda & 15,627 & 4.82 & 28\% & 39.2\% & 1,262 & 3\% & 95\%\\
United Kingdom & 5,425 & 2.37 & 77\% &  & 35,305 & 75\% & 1\%\\
United States & 5,588 & 2.44 & 94\% &  & 43,740 &  & 0\%\\
Uruguay & 6,888 & 2.82 & 83\% & 99.7\% & 21,058 & 46\% & 13\%\\
Vietnam & 9,378 & 3.84 & 30\% & 97.8\% & 2,362 & 1\% & 15\%\\*
\end{longtable}
\end{ThreePartTable}
\endgroup{}
 \label{tab:A_1}

\clearpage

\begin{table}[H]

\caption{Average expenditures and average energy expenditure shares per expenditure quintile}
\centering
\resizebox{\linewidth}{!}{
\begin{threeparttable}
\begin{tabular}[t]{l|rrrrrr|rrrrrrl|rrrrrr|rrrrrrl|rrrrrr|rrrrrrl|rrrrrr|rrrrrrl|rrrrrr|rrrrrrl|rrrrrr|rrrrrrl|rrrrrr|rrrrrrl|rrrrrr|rrrrrrl|rrrrrr|rrrrrrl|rrrrrr|rrrrrrl|rrrrrr|rrrrrrl|rrrrrr|rrrrrrl|rrrrrr|rrrrrr}
\toprule
\multicolumn{1}{c}{ } & \multicolumn{6}{c}{Average household expenditures [USD]} & \multicolumn{6}{c}{Average energy expenditure shares} \\
\cmidrule(l{3pt}r{3pt}){2-7} \cmidrule(l{3pt}r{3pt}){8-13}
\multicolumn{2}{c}{ } & \multicolumn{5}{c}{Expenditure quintile} & \multicolumn{1}{c}{ } & \multicolumn{5}{c}{Expenditure quintile} \\
\cmidrule(l{3pt}r{3pt}){3-7} \cmidrule(l{3pt}r{3pt}){9-13}
Country & All & EQ1 & EQ2 & EQ3 & EQ4 & EQ5 & All & EQ1 & EQ2 & EQ3 & EQ4 & EQ5\\
\midrule
ARG & 14,437 & 5,485 & 9,224 & 12,236 & 17,668 & 27,586 & 13.6\% & 17.1\% & 15\% & 13.7\% & 12.5\% & 9.9\%\\
ARM & 3,265 & 933 & 1,601 & 2,149 & 2,864 & 8,780 & 0.2\% & 0.6\% & 0\% & 0.1\% & 0\% & 0\%\\
BEL & 36,432 & 25,561 & 32,483 & 33,973 & 38,374 & 51,782 & 11.9\% & 14.7\% & 12.8\% & 12.5\% & 11\% & 8.5\%\\
BEN & 3,127 & 1,153 & 2,034 & 2,861 & 3,939 & 5,652 & 8.2\% & 6.1\% & 7.3\% & 8.2\% & 8.8\% & 10.5\%\\
BFA & 3,095 & 997 & 1,722 & 2,424 & 3,728 & 6,615 & 6.9\% & 4.3\% & 5.1\% & 5.9\% & 8.1\% & 11\%\\
BGD & 2,125 & 943 & 1,394 & 1,790 & 2,428 & 4,072 & 4.1\% & 4\% & 3.9\% & 4.2\% & 4.3\% & 4.1\%\\
BGR & 6,376 & 3,802 & 4,741 & 5,552 & 7,559 & 10,228 & 18\% & 19.6\% & 18.7\% & 18.7\% & 17.9\% & 15\%\\
BOL & 3,688 & 1,743 & 2,860 & 3,630 & 4,383 & 5,822 & 6.2\% & 6.7\% & 6.3\% & 6.2\% & 6.4\% & 5.7\%\\
BRA & 12,212 & 2,886 & 5,748 & 8,714 & 13,351 & 30,544 & 14.3\% & 21.7\% & 15.3\% & 13.5\% & 11.8\% & 9.3\%\\
BRB & 16,842 & 6,877 & 12,169 & 16,180 & 18,957 & 29,988 & 12.8\% & 12.5\% & 12.8\% & 14\% & 13.4\% & 11.1\%\\
CHL & 19,547 & 7,224 & 12,118 & 16,290 & 22,404 & 39,721 & 8.9\% & 12.7\% & 9.6\% & 8.7\% & 7.7\% & 5.9\%\\
CIV & 3,718 & 1,636 & 2,736 & 3,695 & 4,567 & 5,959 & 5.8\% & 4.9\% & 6\% & 6\% & 5.6\% & 6.6\%\\
COL & 9,732 & 1,974 & 3,812 & 5,633 & 9,012 & 28,230 & 8.5\% & 12.2\% & 10.1\% & 8.7\% & 7\% & 4.5\%\\
CRI & 12,177 & 4,900 & 7,525 & 9,901 & 13,675 & 24,893 & 10.3\% & 12.9\% & 11.2\% & 10.2\% & 9.7\% & 7.7\%\\
CYP & 31,916 & 18,208 & 26,426 & 31,222 & 38,459 & 45,293 & 13.5\% & 16.1\% & 14.9\% & 13.2\% & 12.2\% & 10.9\%\\
CZE & 12,621 & 9,982 & 11,706 & 11,875 & 12,891 & 16,652 & 17.9\% & 19.7\% & 19.2\% & 18.9\% & 16.9\% & 14.9\%\\
DEU & 32,797 & 24,304 & 27,630 & 30,644 & 34,326 & 47,085 & 12.6\% & 15.2\% & 13.5\% & 12.8\% & 11.9\% & 9.8\%\\
DNK & 43,833 & 35,673 & 40,781 & 38,588 & 44,774 & 59,372 & 11.6\% & 13.1\% & 12.2\% & 12\% & 11.2\% & 9.5\%\\
DOM & 7,786 & 4,154 & 5,899 & 7,159 & 8,574 & 13,146 & 9.8\% & 9.4\% & 9.1\% & 9.5\% & 9.2\% & 11.8\%\\
ECU & 10,224 & 2,561 & 4,370 & 6,060 & 8,754 & 29,378 & 5.8\% & 7.5\% & 6\% & 5.6\% & 5.8\% & 4\%\\
ESP & 26,230 & 13,606 & 20,512 & 25,673 & 31,719 & 39,646 & 12\% & 14.3\% & 13.1\% & 12.2\% & 11.1\% & 9.2\%\\
EST & 13,508 & 6,189 & 9,161 & 12,098 & 15,224 & 24,900 & 15.4\% & 19.2\% & 17.1\% & 15.5\% & 13.8\% & 11.2\%\\
ETH & 1,100 & 297 & 600 & 842 & 1,420 & 2,341 & 2.8\% & 1.2\% & 1.1\% & 2.1\% & 4.6\% & 4.7\%\\
FIN & 36,784 & 26,609 & 30,756 & 34,229 & 37,958 & 54,384 & 8.1\% & 10.1\% & 8.9\% & 8.4\% & 7.3\% & 6\%\\
FRA & 31,144 & 19,309 & 26,515 & 30,680 & 34,281 & 44,950 & 10.9\% & 13.4\% & 11.8\% & 11.3\% & 10.1\% & 8.1\%\\
GHA & 2,380 & 1,120 & 1,887 & 2,349 & 2,868 & 3,679 & 8\% & 6\% & 7.9\% & 8.2\% & 9\% & 8.7\%\\
GNB & 4,172 & 1,706 & 2,838 & 3,781 & 4,990 & 7,551 & 4\% & 1.6\% & 1.9\% & 3.5\% & 5.5\% & 7.6\%\\
GRC & 22,591 & 13,041 & 16,819 & 20,443 & 24,340 & 38,320 & 13.8\% & 16.7\% & 15.6\% & 14.2\% & 12.3\% & 10\%\\
GTM & 4,830 & 2,190 & 3,401 & 4,321 & 5,513 & 8,732 & 16\% & 20\% & 16.3\% & 15\% & 14.6\% & 14.3\%\\
HRV & 14,049 & 8,835 & 11,506 & 13,362 & 16,029 & 20,535 & 18.2\% & 20.6\% & 19.7\% & 18.3\% & 17\% & 15.4\%\\
HUN & 9,596 & 6,305 & 8,156 & 9,190 & 10,778 & 13,553 & 20.3\% & 22.2\% & 21.2\% & 21.1\% & 19.6\% & 17.2\%\\
IDN & 2,799 & 1,084 & 1,789 & 2,450 & 3,359 & 5,317 & 12\% & 13.5\% & 12.3\% & 11.7\% & 11.4\% & 10.9\%\\
IND & 1,514 & 719 & 976 & 1,244 & 1,722 & 2,909 & 8.5\% & 6.9\% & 8.1\% & 8.8\% & 9.6\% & 9.1\%\\
IRL & 39,756 & 24,619 & 32,964 & 39,836 & 46,065 & 55,302 & 13.3\% & 16\% & 14.9\% & 13.1\% & 12.5\% & 10\%\\
IRQ & 14,489 & 5,795 & 9,063 & 11,788 & 15,771 & 30,022 & 9.1\% & 11.9\% & 10\% & 9.3\% & 8.1\% & 6.3\%\\
ISR & 39,641 & 20,252 & 30,396 & 38,556 & 46,804 & 62,217 & 7.6\% & 10\% & 8.4\% & 7.3\% & 7.1\% & 5.5\%\\
ITA & 27,761 & 15,011 & 22,139 & 26,943 & 32,672 & 42,044 & 14.4\% & 19\% & 15.9\% & 13.9\% & 12.7\% & 10.4\%\\
KEN & 2,372 & 653 & 1,338 & 2,009 & 2,801 & 5,060 & 6.3\% & 6\% & 6.5\% & 6.8\% & 6.4\% & 5.9\%\\
KHM & 5,426 & 2,165 & 3,449 & 4,515 & 6,341 & 10,674 & 10\% & 12.1\% & 11\% & 9.7\% & 8.7\% & 8.3\%\\
LBR & 16,611 & 899 & 1,731 & 2,570 & 3,460 & 58,528 & 3.4\% & 2.6\% & 2.4\% & 3.3\% & 3.9\% & 4.6\%\\
LTU & 10,073 & 6,009 & 7,382 & 8,790 & 11,923 & 16,266 & 18\% & 18.1\% & 18.3\% & 18.9\% & 19.1\% & 15.8\%\\
LUX & 57,716 & 37,957 & 47,098 & 57,618 & 66,726 & 79,202 & 8.6\% & 11.8\% & 9.4\% & 8.2\% & 7.5\% & 6.1\%\\
LVA & 11,726 & 5,939 & 7,891 & 9,919 & 12,974 & 21,912 & 17.4\% & 19.4\% & 19.5\% & 17.4\% & 17.1\% & 13.6\%\\
MAR & 8,194 & 4,348 & 5,959 & 7,177 & 9,066 & 14,425 & 7.8\% & 10.3\% & 8\% & 7.4\% & 6.9\% & 6.5\%\\
MDV & 19,238 & 10,074 & 15,158 & 18,870 & 23,676 & 28,443 & 6.8\% & 9.8\% & 8\% & 6.5\% & 5.4\% & 4.2\%\\
MEX & 6,846 & 3,038 & 4,878 & 6,181 & 7,814 & 12,319 & 11.2\% & 10.3\% & 11.1\% & 11.8\% & 12\% & 10.8\%\\
MLI & 4,011 & 1,388 & 2,361 & 3,470 & 5,136 & 7,703 & 6.3\% & 4.5\% & 5.5\% & 5.8\% & 7.5\% & 7.9\%\\
MMR & 2,541 & 1,166 & 1,723 & 2,249 & 2,951 & 4,619 & 5.3\% & 4.6\% & 5.1\% & 4.9\% & 5.5\% & 6.1\%\\
MNG & 7,174 & 3,576 & 5,052 & 6,198 & 7,767 & 13,280 & 9.8\% & 10.4\% & 11\% & 10.4\% & 9.8\% & 7.3\%\\
MWI & 790 & 171 & 372 & 553 & 849 & 2,004 & 2.8\% & 0.3\% & 0.7\% & 1.6\% & 3.6\% & 7.6\%\\
NER & 2,206 & 720 & 1,287 & 1,747 & 2,445 & 4,833 & 2.9\% & 0.6\% & 1.4\% & 1.9\% & 3.3\% & 7.1\%\\
NGA & 3,955 & 1,821 & 3,059 & 3,974 & 5,027 & 5,894 & 5\% & 3.6\% & 4.4\% & 5.1\% & 5.7\% & 6\%\\
NIC & 5,581 & 1,463 & 2,647 & 3,739 & 5,472 & 14,591 & 6\% & 4.3\% & 5.2\% & 6.2\% & 6.8\% & 7.5\%\\
NLD & 39,679 & 32,670 & 37,108 & 36,702 & 39,800 & 52,116 & 10\% & 12.5\% & 10.9\% & 9.9\% & 8.9\% & 7.8\%\\
NOR & 64,706 & 35,240 & 51,003 & 62,112 & 73,374 & 101,851 & 10.3\% & 13.6\% & 11.7\% & 10.2\% & 9\% & 7.1\%\\
PAK & 862 & 328 & 508 & 695 & 968 & 1,812 & 2.9\% & 2.6\% & 3\% & 3.2\% & 3.1\% & 2.7\%\\
PER & 4,866 & 1,668 & 3,251 & 4,532 & 5,848 & 9,033 & 8\% & 9\% & 8.7\% & 8\% & 7.6\% & 6.8\%\\
PHL & 4,838 & 1,946 & 2,951 & 4,143 & 5,790 & 9,360 & 5.7\% & 3.6\% & 5\% & 6.1\% & 6.9\% & 7.1\%\\
POL & 14,963 & 8,256 & 10,588 & 12,296 & 15,514 & 28,165 & 14.6\% & 16.1\% & 16.8\% & 16\% & 14.3\% & 9.9\%\\
PRT & 20,299 & 10,263 & 14,940 & 18,552 & 23,271 & 34,476 & 17.2\% & 22.4\% & 19.1\% & 17.2\% & 15.3\% & 12.1\%\\
PRY & 8,371 & 2,793 & 5,437 & 7,872 & 10,284 & 15,473 & 10.4\% & 9.7\% & 11\% & 10.3\% & 10.5\% & 10.5\%\\
ROU & 6,040 & 4,014 & 5,023 & 5,883 & 6,641 & 8,640 & 16.6\% & 13.5\% & 16.6\% & 17.9\% & 18.1\% & 17\%\\
RWA & 1,353 & 439 & 723 & 988 & 1,468 & 3,147 & 3.2\% & 1.2\% & 1.8\% & 2.6\% & 4.2\% & 6\%\\
SEN & 7,639 & 3,495 & 5,748 & 7,795 & 9,351 & 11,806 & 4.9\% & 2.5\% & 4\% & 5.5\% & 5.8\% & 6.5\%\\
SLV & 5,707 & 1,277 & 2,951 & 4,699 & 6,885 & 12,724 & 20\% & 25.9\% & 23\% & 20.4\% & 16.9\% & 13.9\%\\
SUR & 8,490 & 3,295 & 5,660 & 7,658 & 10,050 & 15,804 & 6\% & 8.3\% & 6.7\% & 5.8\% & 5.4\% & 3.9\%\\
SVK & 15,012 & 10,277 & 12,861 & 14,025 & 15,774 & 22,129 & 19.6\% & 23\% & 21.1\% & 20.8\% & 18.5\% & 14.5\%\\
SWE & 33,803 & 23,812 & 29,946 & 33,343 & 35,248 & 46,716 & 10.7\% & 13.1\% & 12.3\% & 10.9\% & 9.1\% & 8.1\%\\
TGO & 2,733 & 939 & 1,766 & 2,619 & 3,620 & 4,725 & 7.6\% & 3.6\% & 6.5\% & 8.2\% & 9.3\% & 10.3\%\\
THA & 3,917 & 1,084 & 1,957 & 3,133 & 4,961 & 8,451 & 19.8\% & 20.4\% & 23\% & 22.6\% & 18.8\% & 14.4\%\\
TUR & 12,906 & 6,400 & 9,001 & 11,595 & 14,389 & 23,145 & 11.4\% & 10.8\% & 12.2\% & 12.1\% & 11.8\% & 10.2\%\\
UGA & 1,494 & 341 & 776 & 1,225 & 1,900 & 3,223 & 5.2\% & 3.9\% & 3.4\% & 4.6\% & 6.4\% & 7.5\%\\
URY & 20,528 & 7,939 & 13,025 & 17,923 & 24,282 & 39,484 & 9.7\% & 13.5\% & 10.8\% & 9.5\% & 8.3\% & 6.6\%\\
ZAF & 7,223 & 1,826 & 2,979 & 4,125 & 6,966 & 20,224 & 11\% & 10.8\% & 10\% & 10.6\% & 11.9\% & 11.6\%\\
\bottomrule
\end{tabular}
\begin{tablenotes}
\item \textit{Note: } 
\item This table shows average household expenditures and average energy expenditure shares for households in our sample. We estimate household-weighted averages for the whole population and per expenditure quintile.
\end{tablenotes}
\end{threeparttable}}
\end{table}
 \label{tab:A_2}

\clearpage

\begin{table}[H]

\caption{Average carbon footprint and average USD/tCO$_{2}$ carbon price incidence per expenditure quintile}
\centering
\resizebox{\linewidth}{!}{
\begin{threeparttable}
\begin{tabular}[t]{l|rrrrrr|rrrrrrl|rrrrrr|rrrrrrl|rrrrrr|rrrrrrl|rrrrrr|rrrrrrl|rrrrrr|rrrrrrl|rrrrrr|rrrrrrl|rrrrrr|rrrrrrl|rrrrrr|rrrrrrl|rrrrrr|rrrrrrl|rrrrrr|rrrrrrl|rrrrrr|rrrrrrl|rrrrrr|rrrrrrl|rrrrrr|rrrrrr}
\toprule
\multicolumn{1}{c}{ } & \multicolumn{6}{c}{Average carbon footprint [tCO$_{2}$]} & \multicolumn{6}{c}{Average incidence from USD 40/tCO$_{2}$ carbon price} \\
\cmidrule(l{3pt}r{3pt}){2-7} \cmidrule(l{3pt}r{3pt}){8-13}
\multicolumn{2}{c}{ } & \multicolumn{5}{c}{Expenditure quintile} & \multicolumn{1}{c}{ } & \multicolumn{5}{c}{Expenditure quintile} \\
\cmidrule(l{3pt}r{3pt}){3-7} \cmidrule(l{3pt}r{3pt}){9-13}
Country & All & EQ1 & EQ2 & EQ3 & EQ4 & EQ5 & All & EQ1 & EQ2 & EQ3 & EQ4 & EQ5\\
\midrule
ARG & 10.4 & 5.0 & 7.7 & 9.6 & 12.8 & 16.6 & 3.19\% & 3.93\% & 3.44\% & 3.18\% & 2.93\% & 2.45\%\\
ARM & 1.2 & 0.2 & 0.3 & 0.5 & 0.7 & 4.1 & 0.92\% & 0.71\% & 0.74\% & 0.81\% & 0.89\% & 1.44\%\\
BEL & 13.4 & 11.4 & 13.3 & 13.4 & 13.9 & 15.1 & 1.63\% & 1.85\% & 1.75\% & 1.71\% & 1.55\% & 1.29\%\\
BEN & 1.3 & 0.4 & 0.7 & 1.0 & 1.4 & 3.1 & 1.47\% & 1.26\% & 1.34\% & 1.37\% & 1.43\% & 1.95\%\\
BFA & 1.9 & 0.5 & 0.9 & 1.3 & 2.1 & 4.7 & 2.16\% & 1.98\% & 2.02\% & 2.06\% & 2.17\% & 2.56\%\\
BGD & 0.9 & 0.3 & 0.4 & 0.6 & 1.0 & 1.9 & 1.48\% & 1.2\% & 1.24\% & 1.38\% & 1.63\% & 1.93\%\\
BGR & 4.7 & 2.8 & 3.4 & 4.5 & 5.7 & 7.1 & 2.94\% & 2.83\% & 2.84\% & 3.09\% & 3.05\% & 2.88\%\\
BOL & 2.3 & 1.2 & 1.9 & 2.4 & 2.8 & 3.3 & 2.64\% & 2.84\% & 2.72\% & 2.67\% & 2.62\% & 2.36\%\\
BRA & 5.7 & 1.8 & 3.1 & 4.6 & 6.7 & 12.4 & 2.17\% & 2.78\% & 2.23\% & 2.11\% & 1.98\% & 1.73\%\\
BRB & 9.9 & 4.4 & 7.6 & 10.6 & 12.0 & 14.8 & 2.49\% & 2.65\% & 2.58\% & 2.66\% & 2.5\% & 2.09\%\\
CHL & 7.9 & 4.1 & 5.8 & 7.2 & 9.2 & 13.3 & 1.85\% & 2.41\% & 2\% & 1.82\% & 1.65\% & 1.37\%\\
CIV & 1.8 & 0.8 & 1.3 & 1.7 & 2.0 & 3.0 & 1.8\% & 1.89\% & 1.84\% & 1.77\% & 1.69\% & 1.79\%\\
COL & 3.8 & 1.2 & 2.2 & 2.9 & 4.0 & 8.5 & 2.04\% & 2.52\% & 2.32\% & 2.11\% & 1.82\% & 1.44\%\\
CRI & 3.5 & 1.4 & 2.4 & 3.0 & 4.3 & 6.2 & 1.16\% & 1.14\% & 1.24\% & 1.19\% & 1.2\% & 1.04\%\\
CYP & 17.2 & 11.8 & 16.3 & 17.2 & 19.8 & 21.0 & 2.32\% & 2.61\% & 2.51\% & 2.29\% & 2.18\% & 2.01\%\\
CZE & 10.8 & 9.8 & 10.6 & 10.9 & 10.6 & 12.0 & 3.65\% & 4.11\% & 3.76\% & 3.87\% & 3.48\% & 3.04\%\\
DEU & 11.9 & 10.6 & 10.9 & 11.5 & 12.3 & 14.4 & 1.46\% & 1.68\% & 1.49\% & 1.45\% & 1.41\% & 1.26\%\\
DNK & 15.2 & 14.8 & 15.0 & 13.7 & 14.9 & 17.7 & 1.47\% & 1.73\% & 1.54\% & 1.46\% & 1.36\% & 1.25\%\\
DOM & 4.1 & 1.8 & 2.7 & 3.5 & 4.2 & 8.2 & 1.92\% & 1.78\% & 1.8\% & 1.88\% & 1.86\% & 2.29\%\\
ECU & 3.2 & 1.4 & 2.1 & 2.7 & 3.7 & 6.0 & 1.87\% & 2.5\% & 2\% & 1.84\% & 1.77\% & 1.22\%\\
ESP & 10.5 & 6.1 & 9.2 & 11.0 & 12.7 & 13.7 & 1.67\% & 1.8\% & 1.79\% & 1.73\% & 1.6\% & 1.41\%\\
EST & 8.5 & 4.6 & 6.5 & 8.2 & 9.5 & 13.9 & 2.72\% & 3\% & 2.94\% & 2.72\% & 2.56\% & 2.39\%\\
ETH & 0.1 & 0.0 & 0.1 & 0.1 & 0.1 & 0.2 & 0.4\% & 0.46\% & 0.4\% & 0.37\% & 0.38\% & 0.38\%\\
FIN & 12.0 & 9.3 & 10.7 & 12.2 & 12.4 & 15.3 & 1.32\% & 1.4\% & 1.36\% & 1.4\% & 1.28\% & 1.16\%\\
FRA & 10.6 & 7.9 & 10.3 & 11.5 & 11.3 & 12.2 & 1.46\% & 1.68\% & 1.57\% & 1.52\% & 1.37\% & 1.15\%\\
GHA & 0.7 & 0.3 & 0.5 & 0.7 & 1.0 & 1.3 & 1.11\% & 0.86\% & 0.99\% & 1.08\% & 1.25\% & 1.35\%\\
GNB & 1.2 & 0.3 & 0.6 & 0.9 & 1.4 & 2.9 & 0.98\% & 0.73\% & 0.76\% & 0.92\% & 1.09\% & 1.4\%\\
GRC & 14.5 & 9.9 & 12.1 & 14.0 & 15.5 & 20.8 & 2.75\% & 3.11\% & 2.94\% & 2.8\% & 2.6\% & 2.3\%\\
GTM & 2.3 & 0.5 & 1.1 & 1.8 & 2.7 & 5.2 & 1.59\% & 0.96\% & 1.22\% & 1.59\% & 1.92\% & 2.25\%\\
HRV & 8.4 & 5.0 & 7.3 & 8.2 & 9.7 & 11.8 & 2.31\% & 2.05\% & 2.4\% & 2.35\% & 2.37\% & 2.37\%\\
HUN & 6.2 & 3.9 & 5.5 & 6.3 & 7.1 & 8.1 & 2.56\% & 2.44\% & 2.64\% & 2.72\% & 2.6\% & 2.4\%\\
IDN & 2.6 & 0.9 & 1.6 & 2.3 & 3.2 & 5.2 & 3.79\% & 3.67\% & 3.62\% & 3.74\% & 3.89\% & 4.01\%\\
IND & 1.5 & 0.7 & 1.0 & 1.3 & 1.8 & 2.7 & 4.08\% & 4.2\% & 4.2\% & 4.16\% & 4.07\% & 3.77\%\\
IRL & 20.1 & 15.2 & 19.1 & 20.7 & 23.3 & 22.2 & 2.3\% & 2.79\% & 2.55\% & 2.24\% & 2.18\% & 1.71\%\\
IRQ & 8.2 & 3.9 & 5.8 & 7.4 & 9.4 & 14.3 & 2.53\% & 2.83\% & 2.63\% & 2.57\% & 2.45\% & 2.15\%\\
ISR & 17.2 & 11.8 & 15.6 & 17.6 & 19.7 & 21.4 & 1.92\% & 2.54\% & 2.08\% & 1.82\% & 1.73\% & 1.42\%\\
ITA & 13.6 & 9.3 & 12.4 & 13.7 & 15.6 & 17.4 & 2.12\% & 2.52\% & 2.28\% & 2.07\% & 1.97\% & 1.73\%\\
KEN & 1.4 & 0.3 & 0.7 & 1.1 & 1.7 & 3.5 & 2.08\% & 1.59\% & 1.92\% & 2.06\% & 2.23\% & 2.59\%\\
KHM & 1.9 & 0.8 & 1.3 & 1.6 & 2.2 & 3.6 & 1.42\% & 1.43\% & 1.5\% & 1.4\% & 1.39\% & 1.36\%\\
LBR & 2.6 & 0.1 & 0.3 & 0.6 & 0.9 & 8.7 & 0.83\% & 0.57\% & 0.68\% & 0.86\% & 0.92\% & 1.05\%\\
LTU & 3.6 & 2.1 & 2.7 & 3.2 & 4.4 & 5.4 & 1.4\% & 1.33\% & 1.37\% & 1.47\% & 1.47\% & 1.35\%\\
LUX & 17.0 & 14.7 & 15.5 & 16.9 & 18.5 & 19.2 & 1.32\% & 1.68\% & 1.42\% & 1.25\% & 1.21\% & 1.04\%\\
LVA & 8.8 & 6.6 & 7.1 & 7.6 & 10.0 & 12.8 & 3.51\% & 4.99\% & 3.83\% & 3.07\% & 3.2\% & 2.47\%\\
MAR & 3.5 & 1.9 & 2.5 & 3.0 & 3.8 & 6.3 & 1.68\% & 1.79\% & 1.67\% & 1.65\% & 1.63\% & 1.68\%\\
MDV & 7.2 & 4.8 & 6.7 & 7.4 & 8.2 & 8.7 & 1.61\% & 1.95\% & 1.8\% & 1.6\% & 1.44\% & 1.25\%\\
MEX & 4.6 & 2.0 & 3.4 & 4.4 & 5.5 & 7.6 & 2.75\% & 2.65\% & 2.79\% & 2.88\% & 2.85\% & 2.56\%\\
MLI & 1.5 & 0.5 & 0.8 & 1.2 & 1.9 & 3.0 & 1.37\% & 1.32\% & 1.34\% & 1.3\% & 1.4\% & 1.48\%\\
MMR & 1.1 & 0.4 & 0.6 & 0.9 & 1.3 & 2.4 & 1.54\% & 1.27\% & 1.41\% & 1.46\% & 1.59\% & 1.99\%\\
MNG & 11.8 & 7.1 & 9.5 & 10.9 & 12.7 & 18.9 & 7.25\% & 8.13\% & 7.82\% & 7.33\% & 6.92\% & 6.05\%\\
MWI & 0.2 & 0.0 & 0.0 & 0.1 & 0.1 & 0.7 & 0.65\% & 0.53\% & 0.52\% & 0.57\% & 0.63\% & 1.01\%\\
NER & 0.7 & 0.2 & 0.3 & 0.4 & 0.6 & 2.0 & 0.99\% & 0.9\% & 0.84\% & 0.88\% & 0.96\% & 1.38\%\\
NGA & 1.5 & 0.4 & 0.9 & 1.4 & 2.1 & 2.6 & 1.37\% & 0.96\% & 1.17\% & 1.41\% & 1.6\% & 1.71\%\\
NIC & 2.5 & 0.4 & 0.9 & 1.5 & 2.7 & 7.3 & 1.57\% & 1\% & 1.28\% & 1.52\% & 1.84\% & 2.24\%\\
NLD & 17.1 & 16.9 & 17.3 & 16.0 & 16.3 & 19.1 & 1.83\% & 2.16\% & 1.95\% & 1.82\% & 1.68\% & 1.53\%\\
NOR & 15.9 & 10.2 & 14.4 & 16.6 & 18.1 & 20.5 & 1.06\% & 1.11\% & 1.14\% & 1.13\% & 1.03\% & 0.88\%\\
PAK & 0.4 & 0.1 & 0.2 & 0.3 & 0.4 & 0.9 & 1.56\% & 1.26\% & 1.42\% & 1.59\% & 1.67\% & 1.85\%\\
PER & 2.2 & 1.0 & 1.8 & 2.2 & 2.6 & 3.5 & 2.16\% & 2.56\% & 2.43\% & 2.17\% & 1.95\% & 1.67\%\\
PHL & 2.2 & 0.6 & 1.1 & 1.8 & 2.8 & 4.8 & 1.64\% & 1.17\% & 1.44\% & 1.7\% & 1.9\% & 2.01\%\\
POL & 17.2 & 10.8 & 15.1 & 16.9 & 19.1 & 23.9 & 5.15\% & 5.05\% & 5.65\% & 5.66\% & 5.35\% & 4.05\%\\
PRT & 11.0 & 7.3 & 9.4 & 10.8 & 12.4 & 15.0 & 2.3\% & 2.81\% & 2.48\% & 2.3\% & 2.12\% & 1.81\%\\
PRY & 3.3 & 1.3 & 2.7 & 3.3 & 3.8 & 5.4 & 1.7\% & 1.77\% & 2.06\% & 1.75\% & 1.53\% & 1.39\%\\
ROU & 3.8 & 1.9 & 3.0 & 3.9 & 4.5 & 5.7 & 2.48\% & 1.93\% & 2.4\% & 2.63\% & 2.73\% & 2.7\%\\
RWA & 0.3 & 0.0 & 0.1 & 0.1 & 0.2 & 1.0 & 0.57\% & 0.43\% & 0.44\% & 0.5\% & 0.58\% & 0.92\%\\
SEN & 2.6 & 0.8 & 1.4 & 2.5 & 3.3 & 4.8 & 1.19\% & 0.82\% & 0.95\% & 1.23\% & 1.38\% & 1.56\%\\
SLV & 2.7 & 0.9 & 1.8 & 2.5 & 3.1 & 5.0 & 2.09\% & 2.75\% & 2.4\% & 2.04\% & 1.75\% & 1.52\%\\
SUR & 3.4 & 1.5 & 2.4 & 3.2 & 4.3 & 5.8 & 1.68\% & 1.8\% & 1.77\% & 1.7\% & 1.66\% & 1.46\%\\
SVK & 7.5 & 6.8 & 7.0 & 7.9 & 7.7 & 8.4 & 2.2\% & 2.66\% & 2.29\% & 2.36\% & 2.06\% & 1.65\%\\
SWE & 7.3 & 6.1 & 7.3 & 7.7 & 7.0 & 8.6 & 0.87\% & 0.97\% & 0.93\% & 0.9\% & 0.79\% & 0.78\%\\
TGO & 0.9 & 0.2 & 0.5 & 0.7 & 1.1 & 1.8 & 1.06\% & 0.76\% & 0.98\% & 1.01\% & 1.13\% & 1.41\%\\
THA & 3.8 & 1.2 & 2.2 & 3.5 & 5.0 & 7.2 & 4.06\% & 4.06\% & 4.47\% & 4.46\% & 3.96\% & 3.36\%\\
TUR & 11.5 & 7.2 & 10.2 & 11.8 & 12.7 & 15.5 & 4.04\% & 4.43\% & 4.74\% & 4.32\% & 3.76\% & 2.97\%\\
UGA & 0.4 & 0.1 & 0.1 & 0.2 & 0.4 & 1.1 & 0.91\% & 1.03\% & 0.75\% & 0.74\% & 0.83\% & 1.2\%\\
URY & 3.7 & 1.8 & 2.6 & 3.4 & 4.5 & 6.4 & 0.78\% & 0.92\% & 0.81\% & 0.77\% & 0.72\% & 0.66\%\\
ZAF & 13.0 & 4.0 & 6.2 & 8.5 & 14.0 & 32.3 & 8.51\% & 9.67\% & 8.79\% & 8.67\% & 8.36\% & 7.03\%\\
\bottomrule
\end{tabular}
\begin{tablenotes}
\item \textit{Note: } 
\item This table shows average carbon footprints in tCO$_{2}$ and average levels of carbon price incidence for households in all countries of our sample. We estimate household-weighted averages for the whole population and per expenditure quintile.
\end{tablenotes}
\end{threeparttable}}
\end{table}
 \label{tab:A3}

\clearpage

\begin{table}[H]

\caption{\label{tab:A4_CF}Share of households using cooking fuels}
\centering
\resizebox{\linewidth}{!}{
\begin{threeparttable}
\begin{tabular}[t]{l|rrrrr|rrrrr|rrrrrl|rrrrr|rrrrr|rrrrrl|rrrrr|rrrrr|rrrrrl|rrrrr|rrrrr|rrrrrl|rrrrr|rrrrr|rrrrrl|rrrrr|rrrrr|rrrrrl|rrrrr|rrrrr|rrrrrl|rrrrr|rrrrr|rrrrrl|rrrrr|rrrrr|rrrrrl|rrrrr|rrrrr|rrrrrl|rrrrr|rrrrr|rrrrrl|rrrrr|rrrrr|rrrrrl|rrrrr|rrrrr|rrrrrl|rrrrr|rrrrr|rrrrrl|rrrrr|rrrrr|rrrrrl|rrrrr|rrrrr|rrrrr}
\toprule
\multicolumn{1}{c}{ } & \multicolumn{5}{c}{Solid fuels} & \multicolumn{5}{c}{Liquid or gaseous fuels} & \multicolumn{5}{c}{Electricity} \\
\cmidrule(l{3pt}r{3pt}){2-6} \cmidrule(l{3pt}r{3pt}){7-11} \cmidrule(l{3pt}r{3pt}){12-16}
\multicolumn{1}{c}{ } & \multicolumn{5}{c}{Expenditure quintile} & \multicolumn{5}{c}{Expenditure quintile} & \multicolumn{5}{c}{Expenditure quintile} \\
\cmidrule(l{3pt}r{3pt}){2-6} \cmidrule(l{3pt}r{3pt}){7-11} \cmidrule(l{3pt}r{3pt}){12-16}
Country & EQ1 & EQ2 & EQ3 & EQ4 & EQ5 & EQ1 & EQ2 & EQ3 & EQ4 & EQ5 & EQ1 & EQ2 & EQ3 & EQ4 & EQ5\\
\midrule
Argentina & - & - & - & - & - & 99\% & 99\% & 99\% & 98\% & 96\% & 1\% & 0\% & 1\% & 2\% & 4\%\\
Barbados & 0\% & 0\% & - & - & - & 89\% & 95\% & 94\% & 94\% & 88\% & 4\% & 4\% & 5\% & 5\% & 11\%\\
Benin & 100\% & 100\% & 99\% & 96\% & 77\% & - & 0\% & 1\% & 3\% & 23\% & - & - & - & - & -\\
Bolivia & 36\% & 12\% & 6\% & 3\% & 2\% & 63\% & 87\% & 92\% & 93\% & 89\% & - & 0\% & 0\% & 0\% & 1\%\\
Brazil & 3\% & 1\% & 0\% & 0\% & 0\% & 95\% & 98\% & 98\% & 99\% & 98\% & 0\% & 1\% & 1\% & 1\% & 1\%\\
Burkina Faso & 99\% & 100\% & 98\% & 89\% & 43\% & 0\% & 0\% & 1\% & 11\% & 56\% & - & - & - & - & -\\
Cambodia & 82\% & 59\% & 59\% & 44\% & 24\% & 17\% & 41\% & 41\% & 54\% & 74\% & 1\% & 0\% & 1\% & 0\% & 2\%\\
Colombia & 28\% & 10\% & 4\% & 3\% & 1\% & 68\% & 86\% & 92\% & 92\% & 92\% & 3\% & 3\% & 3\% & 3\% & 5\%\\
Costa Rica & 11\% & 4\% & 3\% & 2\% & 1\% & 52\% & 54\% & 47\% & 44\% & 29\% & 36\% & 41\% & 50\% & 54\% & 69\%\\
Côte d’Ivoire & 97\% & 92\% & 73\% & 49\% & 27\% & 2\% & 8\% & 26\% & 49\% & 68\% & - & - & - & - & 0\%\\
Dominican Republic & 10\% & 4\% & 3\% & 2\% & 1\% & 89\% & 94\% & 93\% & 92\% & 91\% & 0\% & - & 0\% & 0\% & 0\%\\
Ecuador & 15\% & 4\% & 2\% & 1\% & 0\% & 80\% & 94\% & 95\% & 96\% & 95\% & 0\% & 0\% & 0\% & 0\% & 1\%\\
Egypt & 0\% & 0\% & 0\% & 0\% & - & 100\% & 100\% & 100\% & 100\% & 100\% & 0\% & 0\% & - & 0\% & 0\%\\
El Salvador & 32\% & 12\% & 7\% & 3\% & 2\% & 62\% & 87\% & 91\% & 95\% & 88\% & 0\% & 0\% & 1\% & 1\% & 4\%\\
Ethiopia & 99\% & 99\% & 98\% & 90\% & 64\% & 0\% & 1\% & 0\% & 1\% & 2\% & 0\% & 0\% & 1\% & 8\% & 29\%\\
Georgia & - & - & - & - & - & 95\% & 97\% & 98\% & 98\% & 99\% & - & - & - & - & -\\
Ghana & 97\% & 87\% & 70\% & 55\% & 31\% & 2\% & 11\% & 25\% & 35\% & 51\% & - & 0\% & 0\% & 0\% & 1\%\\
Guatemala & 98\% & 92\% & 75\% & 58\% & 28\% & 1\% & 7\% & 23\% & 41\% & 68\% & - & - & - & - & -\\
Guinea-Bissau & 100\% & 99\% & 98\% & 99\% & 93\% & - & 0\% & 0\% & 1\% & 6\% & - & - & - & - & -\\
India & 92\% & 84\% & 70\% & 41\% & 9\% & 2\% & 9\% & 25\% & 56\% & 79\% & 0\% & 0\% & 0\% & 0\% & 0\%\\
Indonesia & 42\% & 21\% & 12\% & 6\% & 2\% & 57\% & 78\% & 87\% & 92\% & 92\% & 0\% & 0\% & 0\% & 1\% & 1\%\\
Iraq & 2\% & 0\% & 0\% & 0\% & 0\% & 98\% & 99\% & 100\% & 99\% & 99\% & 1\% & 1\% & 0\% & 1\% & 0\%\\
Jordan & 0\% & 0\% & 0\% & - & - & 100\% & 100\% & 100\% & 100\% & 100\% & - & - & - & - & -\\
Kenya & 98\% & 94\% & 79\% & 52\% & 24\% & 1\% & 5\% & 18\% & 44\% & 70\% & 0\% & 0\% & 1\% & 2\% & 2\%\\
Liberia & 100\% & 99\% & 99\% & 99\% & 98\% & 0\% & 0\% & - & 0\% & 0\% & 0\% & - & - & 0\% & 0\%\\
Malawi & 100\% & 100\% & 100\% & 100\% & 95\% & - & - & - & - & - & - & - & 0\% & 0\% & 5\%\\
Maldives & 2\% & 0\% & 0\% & - & - & 96\% & 96\% & 98\% & 97\% & 95\% & 0\% & 1\% & 1\% & 1\% & 2\%\\
Mali & 100\% & 100\% & 100\% & 99\% & 94\% & - & - & - & 1\% & 5\% & - & - & - & - & -\\
Mexico & 43\% & 17\% & 9\% & 4\% & 2\% & 56\% & 81\% & 90\% & 94\% & 95\% & 1\% & 1\% & 1\% & 1\% & 2\%\\
Mozambique & 100\% & 100\% & 99\% & 99\% & 85\% & 0\% & 0\% & 0\% & 1\% & 11\% & - & 0\% & 0\% & 1\% & 4\%\\
Myanmar (Burma) & 95\% & 90\% & 85\% & 78\% & 66\% & 1\% & 0\% & 1\% & 1\% & 3\% & 3\% & 10\% & 14\% & 19\% & 30\%\\
Nicaragua & 94\% & 75\% & 49\% & 28\% & 10\% & 5\% & 24\% & 50\% & 70\% & 88\% & 0\% & 0\% & 1\% & 1\% & 0\%\\
Niger & 98\% & 99\% & 99\% & 98\% & 81\% & - & - & 0\% & 1\% & 18\% & - & - & - & - & -\\
Nigeria & 98\% & 91\% & 72\% & 47\% & 19\% & 1\% & 9\% & 27\% & 52\% & 77\% & - & - & - & - & -\\
Paraguay & 83\% & 56\% & 28\% & 17\% & 5\% & 12\% & 38\% & 65\% & 74\% & 81\% & 2\% & 4\% & 5\% & 8\% & 10\%\\
Peru & 31\% & 10\% & 4\% & 2\% & 0\% & 60\% & 85\% & 89\% & 87\% & 76\% & 1\% & 3\% & 5\% & 11\% & 21\%\\
Rwanda & - & - & - & - & 0\% & - & - & - & 0\% & 5\% & 99\% & 99\% & 99\% & 100\% & 94\%\\
Senegal & 98\% & 90\% & 71\% & 48\% & 18\% & 2\% & 10\% & 29\% & 51\% & 79\% & - & - & - & 0\% & 0\%\\
South Africa & 28\% & 13\% & 6\% & 2\% & 0\% & 8\% & 9\% & 9\% & 6\% & 8\% & 63\% & 77\% & 85\% & 91\% & 92\%\\
Suriname & - & - & - & - & - & 99\% & 98\% & 99\% & 97\% & 96\% & 0\% & 2\% & 0\% & 2\% & 2\%\\
Thailand & 56\% & 33\% & 16\% & 8\% & 4\% & 38\% & 63\% & 77\% & 76\% & 67\% & 1\% & 1\% & 2\% & 4\% & 7\%\\
Togo & 100\% & 99\% & 96\% & 90\% & 62\% & - & 0\% & 3\% & 9\% & 36\% & - & - & - & - & -\\
Turkey & 16\% & 3\% & 1\% & 1\% & 0\% & 80\% & 96\% & 98\% & 98\% & 98\% & 3\% & 1\% & 0\% & 1\% & 2\%\\
Uganda & 96\% & 98\% & 97\% & 95\% & 85\% & 0\% & 0\% & 0\% & 1\% & 6\% & 0\% & 0\% & 0\% & 1\% & 2\%\\
Uruguay & 3\% & 1\% & 1\% & 1\% & 0\% & 93\% & 96\% & 96\% & 94\% & 90\% & 3\% & 3\% & 3\% & 6\% & 10\%\\
\bottomrule
\end{tabular}
\begin{tablenotes}
\item \textit{Note: } 
\item This table shows the share of households using different cooking fuels, such as solid fuels (e.g., firewood, charcoal, coal, biomass), liquid fuels (e.g., LPG, natural gas, kerosene), or electricity over expenditure quintiles.
\end{tablenotes}
\end{threeparttable}}
\end{table}
 \label{tab:A4_CF}

\begin{table}[H]

\caption{\label{tab:A5_LF}Share of households using lighting fuels}
\centering
\resizebox{\linewidth}{!}{
\begin{threeparttable}
\begin{tabular}[t]{l|rrrrr|rrrrr|rrrrrl|rrrrr|rrrrr|rrrrrl|rrrrr|rrrrr|rrrrrl|rrrrr|rrrrr|rrrrrl|rrrrr|rrrrr|rrrrrl|rrrrr|rrrrr|rrrrrl|rrrrr|rrrrr|rrrrrl|rrrrr|rrrrr|rrrrrl|rrrrr|rrrrr|rrrrrl|rrrrr|rrrrr|rrrrrl|rrrrr|rrrrr|rrrrrl|rrrrr|rrrrr|rrrrrl|rrrrr|rrrrr|rrrrrl|rrrrr|rrrrr|rrrrrl|rrrrr|rrrrr|rrrrrl|rrrrr|rrrrr|rrrrr}
\toprule
\multicolumn{1}{c}{ } & \multicolumn{5}{c}{Kerosene} & \multicolumn{5}{c}{Electricity} & \multicolumn{5}{c}{Other lighting fuels} \\
\cmidrule(l{3pt}r{3pt}){2-6} \cmidrule(l{3pt}r{3pt}){7-11} \cmidrule(l{3pt}r{3pt}){12-16}
\multicolumn{1}{c}{ } & \multicolumn{5}{c}{Expenditure quintile} & \multicolumn{5}{c}{Expenditure quintile} & \multicolumn{5}{c}{Expenditure quintile} \\
\cmidrule(l{3pt}r{3pt}){2-6} \cmidrule(l{3pt}r{3pt}){7-11} \cmidrule(l{3pt}r{3pt}){12-16}
Country & EQ1 & EQ2 & EQ3 & EQ4 & EQ5 & EQ1 & EQ2 & EQ3 & EQ4 & EQ5 & EQ1 & EQ2 & EQ3 & EQ4 & EQ5\\
\midrule
Barbados & 1\% & 1\% & 1\% & 0\% & - & 88\% & 95\% & 97\% & 97\% & 97\% & 3\% & 3\% & 2\% & 2\% & 1\%\\
Benin & 1\% & 0\% & 1\% & 0\% & 1\% & 20\% & 30\% & 42\% & 60\% & 74\% & 80\% & 70\% & 58\% & 40\% & 25\%\\
Burkina Faso & 0\% & 0\% & 0\% & 0\% & 0\% & 29\% & 38\% & 44\% & 66\% & 91\% & 65\% & 59\% & 52\% & 30\% & 8\%\\
Cambodia & 2\% & 1\% & - & - & 1\% & 85\% & 94\% & 96\% & 96\% & 98\% & 12\% & 5\% & 4\% & 4\% & 1\%\\
Costa Rica & - & - & - & - & - & 99\% & 100\% & 100\% & 100\% & 100\% & - & - & - & - & -\\
Côte d’Ivoire & 0\% & 0\% & 0\% & 0\% & 0\% & 60\% & 74\% & 84\% & 90\% & 95\% & 37\% & 24\% & 15\% & 9\% & 4\%\\
Dominican Republic & 2\% & 2\% & 1\% & 1\% & 0\% & 96\% & 97\% & 98\% & 98\% & 99\% & 2\% & 1\% & 1\% & 1\% & 0\%\\
Ecuador & - & - & - & - & - & 95\% & 99\% & 99\% & 100\% & 100\% & - & - & - & - & -\\
Egypt & 0\% & 0\% & 0\% & 0\% & 1\% & 99\% & 100\% & 99\% & 99\% & 99\% & 0\% & 0\% & 0\% & 0\% & 0\%\\
El Salvador & 4\% & 1\% & 0\% & 0\% & 0\% & 87\% & 96\% & 98\% & 99\% & 99\% & 9\% & 3\% & 2\% & 1\% & 1\%\\
Ethiopia & 30\% & 27\% & 23\% & 14\% & 3\% & 30\% & 43\% & 48\% & 68\% & 90\% & 41\% & 29\% & 29\% & 18\% & 7\%\\
Ghana & 1\% & 1\% & 1\% & 1\% & - & 60\% & 80\% & 88\% & 92\% & 96\% & 36\% & 17\% & 11\% & 7\% & 4\%\\
Guatemala & - & - & - & - & - & 58\% & 82\% & 89\% & 96\% & 97\% & 37\% & 15\% & 9\% & 4\% & 2\%\\
Guinea-Bissau & 1\% & 0\% & 0\% & 0\% & 0\% & 43\% & 46\% & 49\% & 58\% & 72\% & 48\% & 48\% & 47\% & 37\% & 25\%\\
India & 48\% & 28\% & 15\% & 6\% & 2\% & 51\% & 72\% & 85\% & 94\% & 98\% & 0\% & 0\% & 0\% & 0\% & 0\%\\
Indonesia & - & - & - & - & - & 96\% & 98\% & 99\% & 100\% & 100\% & - & - & - & - & -\\
Iraq & 1\% & 0\% & 0\% & 0\% & 0\% & 99\% & 100\% & 100\% & 100\% & 100\% & 0\% & - & - & - & -\\
Kenya & 56\% & 53\% & 37\% & 20\% & 9\% & 23\% & 38\% & 57\% & 75\% & 88\% & 18\% & 8\% & 5\% & 4\% & 2\%\\
Liberia & - & 0\% & 0\% & - & - & 0\% & 3\% & 9\% & 20\% & 38\% & 98\% & 96\% & 90\% & 78\% & 59\%\\
Malawi & 1\% & 1\% & 0\% & 0\% & 0\% & 0\% & 1\% & 3\% & 10\% & 39\% & 97\% & 97\% & 95\% & 88\% & 58\%\\
Mali & 1\% & 1\% & 0\% & 0\% & 0\% & 61\% & 66\% & 68\% & 80\% & 94\% & 27\% & 26\% & 26\% & 18\% & 5\%\\
Mozambique & 11\% & 14\% & 17\% & 17\% & 8\% & 2\% & 5\% & 14\% & 39\% & 74\% & 87\% & 82\% & 68\% & 43\% & 18\%\\
Myanmar (Burma) & 13\% & 5\% & 4\% & 5\% & 2\% & 46\% & 55\% & 61\% & 69\% & 77\% & 41\% & 39\% & 35\% & 27\% & 21\%\\
Nicaragua & 14\% & 4\% & 3\% & 2\% & 0\% & 62\% & 85\% & 92\% & 96\% & 99\% & - & - & - & - & -\\
Niger & 1\% & 0\% & 0\% & 0\% & 0\% & 3\% & 6\% & 13\% & 25\% & 58\% & 95\% & 94\% & 87\% & 74\% & 41\%\\
Peru & 1\% & 0\% & 0\% & 0\% & 0\% & 86\% & 96\% & 98\% & 99\% & 99\% & - & - & - & - & -\\
Rwanda & - & - & - & - & - & 79\% & 83\% & 83\% & 85\% & 92\% & 20\% & 16\% & 16\% & 14\% & 8\%\\
Senegal & 1\% & 1\% & 0\% & 0\% & 0\% & 40\% & 61\% & 83\% & 91\% & 96\% & 55\% & 35\% & 14\% & 8\% & 3\%\\
South Africa & 3\% & 2\% & 2\% & 1\% & 0\% & 85\% & 89\% & 92\% & 96\% & 99\% & 12\% & 8\% & 6\% & 3\% & 0\%\\
Suriname & - & - & - & - & - & 89\% & 96\% & 99\% & 99\% & 99\% & 6\% & 2\% & 1\% & 0\% & 1\%\\
Togo & 0\% & 0\% & 1\% & 0\% & 0\% & 13\% & 36\% & 62\% & 79\% & 89\% & 85\% & 63\% & 37\% & 19\% & 10\%\\
Uganda & 44\% & 50\% & 40\% & 24\% & 10\% & 14\% & 21\% & 33\% & 52\% & 76\% & 8\% & 3\% & 3\% & 5\% & 4\%\\
Uruguay & 0\% & 0\% & - & - & - & 99\% & 100\% & 100\% & 100\% & 100\% & 1\% & 0\% & 0\% & 0\% & 0\%\\
Vietnam & 5\% & 1\% & 0\% & 0\% & 0\% & 94\% & 99\% & 100\% & 100\% & 100\% & - & - & - & - & -\\
\bottomrule
\end{tabular}
\begin{tablenotes}
\item \textit{Note: } 
\item This table shows the share of households using different lighting fuels over expenditure quintiles.
\end{tablenotes}
\end{threeparttable}}
\end{table}
 \label{tab:A5_LF}

\begin{table}[H]

\caption{Share of households possessing different assets}
\centering
\resizebox{\linewidth}{!}{
\begin{threeparttable}
\begin{tabular}[t]{l|rrr|rrr|rrr|rrr|rrrl|rrr|rrr|rrr|rrr|rrrl|rrr|rrr|rrr|rrr|rrrl|rrr|rrr|rrr|rrr|rrrl|rrr|rrr|rrr|rrr|rrrl|rrr|rrr|rrr|rrr|rrrl|rrr|rrr|rrr|rrr|rrrl|rrr|rrr|rrr|rrr|rrrl|rrr|rrr|rrr|rrr|rrrl|rrr|rrr|rrr|rrr|rrrl|rrr|rrr|rrr|rrr|rrrl|rrr|rrr|rrr|rrr|rrrl|rrr|rrr|rrr|rrr|rrrl|rrr|rrr|rrr|rrr|rrrl|rrr|rrr|rrr|rrr|rrrl|rrr|rrr|rrr|rrr|rrr}
\toprule
\multicolumn{1}{c}{ } & \multicolumn{3}{c}{Car} & \multicolumn{3}{c}{TV} & \multicolumn{3}{c}{Refrigerator} & \multicolumn{3}{c}{AC} & \multicolumn{3}{c}{Washing machine} \\
\cmidrule(l{3pt}r{3pt}){2-4} \cmidrule(l{3pt}r{3pt}){5-7} \cmidrule(l{3pt}r{3pt}){8-10} \cmidrule(l{3pt}r{3pt}){11-13} \cmidrule(l{3pt}r{3pt}){14-16}
Country & All & EQ1 & EQ5 & All & EQ1 & EQ5 & All & EQ1 & EQ5 & All & EQ1 & EQ5 & All & EQ1 & EQ5\\
\midrule
ARG & 26\% & 66\% & 49\% & 96\% & 97\% & 97\% & 95\% & 99\% & 98\% & 33\% & 72\% & 53\% & 81\% & 87\% & 87\%\\
ARM & 35\% & 35\% & 32\% & 99\% & 99\% & 99\% & 95\% & 97\% & 96\% & 5\% & 12\% & 8\% & 93\% & 94\% & 92\%\\
BEN & 0\% & 12\% & 3\% & 3\% & 52\% & 23\% & 0\% & 14\% & 4\% & 0\% & 1\% & 0\% & 0\% & 1\% & 0\%\\
BFA & 0\% & 17\% & 4\% & 3\% & 78\% & 30\% & 0\% & 38\% & 9\% & 0\% & 8\% & 2\% & 0\% & 0\% & 0\%\\
BGD & 0\% & 2\% & 1\% & 9\% & 71\% & 36\% & 0\% & 44\% & 12\% & - & - & - & 0\% & 1\% & 0\%\\
BOL & 5\% & 31\% & 17\% & 61\% & 92\% & 84\% & 28\% & 77\% & 61\% & 2\% & 22\% & 10\% & 2\% & 40\% & 18\%\\
BRA & 17\% & 76\% & 46\% & 94\% & 98\% & 96\% & 96\% & 99\% & 98\% & 6\% & 42\% & 20\% & 38\% & 87\% & 65\%\\
BRB & 21\% & 75\% & 52\% & 34\% & 61\% & 49\% & 84\% & 97\% & 94\% & 2\% & 18\% & 8\% & 60\% & 86\% & 75\%\\
CIV & 0\% & 10\% & 3\% & 15\% & 70\% & 45\% & 1\% & 35\% & 15\% & 0\% & 9\% & 2\% & 1\% & 5\% & 2\%\\
COL & 1\% & 39\% & 14\% & 81\% & 97\% & 92\% & 66\% & 92\% & 83\% & 1\% & 7\% & 4\% & 34\% & 82\% & 61\%\\
CRI & 19\% & 74\% & 45\% & 95\% & 98\% & 97\% & 92\% & 98\% & 96\% & - & - & - & - & - & -\\
DOM & 6\% & 45\% & 21\% & 83\% & 89\% & 87\% & 74\% & 87\% & 83\% & 2\% & 37\% & 14\% & 72\% & 84\% & 80\%\\
ECU & 2\% & 52\% & 19\% & 78\% & 98\% & 91\% & 56\% & 93\% & 80\% & 0\% & 17\% & 6\% & 15\% & 71\% & 45\%\\
ETH & 0\% & 4\% & 1\% & 1\% & 51\% & 18\% & 0\% & 25\% & 7\% & - & - & - & - & - & -\\
GHA & 1\% & 9\% & 4\% & 31\% & 85\% & 64\% & 7\% & 57\% & 36\% & 0\% & 3\% & 1\% & 0\% & 3\% & 1\%\\
GNB & 0\% & 12\% & 3\% & 5\% & 59\% & 26\% & 0\% & 40\% & 13\% & 0\% & 2\% & 1\% & 0\% & 1\% & 0\%\\
GTM & 2\% & 44\% & 17\% & 34\% & 92\% & 71\% & 0\% & 16\% & 5\% & - & - & - & 0\% & 36\% & 11\%\\
IDN & 1\% & 36\% & 11\% & 2\% & 38\% & 14\% & 25\% & 80\% & 57\% & 0\% & 29\% & 8\% & - & - & -\\
IND & 1\% & 15\% & 4\% & 23\% & 82\% & 59\% & 1\% & 58\% & 20\% & 2\% & 30\% & 12\% & 0\% & 32\% & 9\%\\
IRQ & 17\% & 62\% & 35\% & - & - & - & 83\% & 98\% & 92\% & 21\% & 59\% & 41\% & 41\% & 89\% & 69\%\\
ISR & 53\% & 82\% & 72\% & 76\% & 93\% & 88\% & 100\% & 100\% & 100\% & 89\% & 97\% & 93\% & 97\% & 94\% & 96\%\\
KHM & 2\% & 34\% & 11\% & - & - & - & - & - & - & - & - & - & - & - & -\\
LBR & 0\% & 6\% & 2\% & 1\% & 43\% & 18\% & 0\% & 15\% & 4\% & 0\% & 1\% & 0\% & - & - & -\\
MDV & 2\% & 8\% & 5\% & 86\% & 81\% & 87\% & 92\% & 82\% & 90\% & 58\% & 65\% & 68\% & 92\% & 82\% & 90\%\\
MEX & 21\% & 57\% & 40\% & 75\% & 54\% & 67\% & 74\% & 94\% & 88\% & 100\% & 100\% & 100\% & 50\% & 82\% & 71\%\\
MLI & 0\% & 17\% & 4\% & 13\% & 73\% & 37\% & 0\% & 34\% & 10\% & 0\% & 10\% & 2\% & 0\% & 0\% & 0\%\\
MMR & 0\% & 11\% & 4\% & 26\% & 72\% & 49\% & 1\% & 34\% & 14\% & 0\% & 11\% & 3\% & 0\% & 12\% & 4\%\\
MNG & - & - & - & 94\% & 99\% & 97\% & - & - & - & - & - & - & - & - & -\\
MWI & 0\% & 6\% & 2\% & 0\% & 38\% & 11\% & 0\% & 19\% & 4\% & 0\% & 0\% & 0\% & 0\% & 0\% & 0\%\\
NER & 0\% & 9\% & 2\% & 0\% & 41\% & 10\% & 0\% & 18\% & 4\% & 0\% & 4\% & 1\% & 0\% & 0\% & 0\%\\
NGA & 1\% & 19\% & 8\% & 11\% & 76\% & 48\% & 2\% & 49\% & 24\% & 0\% & 9\% & 3\% & 0\% & 8\% & 2\%\\
NIC & 0\% & 29\% & 8\% & 39\% & 95\% & 75\% & 7\% & 79\% & 40\% & 0\% & 6\% & 1\% & 0\% & 31\% & 10\%\\
NOR & 85\% & 93\% & 88\% & 96\% & 98\% & 97\% & 96\% & 97\% & 96\% & - & - & - & 93\% & 96\% & 94\%\\
PAK & 0\% & 16\% & 4\% & 26\% & 83\% & 56\% & 9\% & 79\% & 43\% & 0\% & 18\% & 5\% & 14\% & 79\% & 47\%\\
PER & 2\% & 29\% & 12\% & 52\% & 93\% & 81\% & 15\% & 80\% & 53\% & - & - & - & 3\% & 61\% & 30\%\\
PHL & 0\% & 27\% & 7\% & 45\% & 95\% & 77\% & 6\% & 81\% & 41\% & 0\% & 40\% & 12\% & 4\% & 72\% & 36\%\\
PRY & 2\% & 57\% & 25\% & 71\% & 93\% & 87\% & 59\% & 90\% & 80\% & 2\% & 60\% & 25\% & 40\% & 77\% & 66\%\\
RWA & 0\% & 5\% & 1\% & 0\% & 37\% & 10\% & 0\% & 8\% & 2\% & - & - & - & 0\% & 0\% & 0\%\\
SEN & 0\% & 20\% & 5\% & 17\% & 85\% & 58\% & 4\% & 65\% & 32\% & 0\% & 11\% & 2\% & 0\% & 2\% & 0\%\\
SLV & 1\% & 40\% & 15\% & 68\% & 95\% & 87\% & 36\% & 84\% & 67\% & 0\% & 5\% & 1\% & 2\% & 44\% & 17\%\\
SUR & 29\% & 44\% & 38\% & 66\% & 58\% & 66\% & 67\% & 84\% & 80\% & 10\% & 54\% & 31\% & 69\% & 88\% & 83\%\\
TGO & 0\% & 10\% & 3\% & 3\% & 70\% & 36\% & 0\% & 21\% & 6\% & 0\% & 3\% & 1\% & 0\% & 1\% & 0\%\\
THA & 1\% & 39\% & 14\% & 93\% & 97\% & 97\% & 82\% & 90\% & 90\% & 1\% & 45\% & 18\% & 39\% & 72\% & 63\%\\
TUR & 17\% & 65\% & 39\% & 23\% & 64\% & 41\% & 97\% & 100\% & 99\% & 13\% & 36\% & 21\% & 91\% & 98\% & 96\%\\
UGA & 0\% & 11\% & 3\% & 0\% & 52\% & 17\% & 0\% & 19\% & 5\% & - & - & - & - & - & -\\
URY & 26\% & 67\% & 46\% & 96\% & 97\% & 97\% & 97\% & 99\% & 99\% & 20\% & 60\% & 42\% & 74\% & 90\% & 85\%\\
ZAF & 3\% & 75\% & 27\% & 70\% & 91\% & 79\% & 54\% & 90\% & 69\% & - & - & - & 12\% & 69\% & 34\%\\
\bottomrule
\end{tabular}
\begin{tablenotes}
\item \textit{Note: } 
\item This table shows the share of households possessing differents assets for all households (first and fifth expenditure quintile, respectively) in different countries.
\end{tablenotes}
\end{threeparttable}}
\end{table}
 \label{tab:A6}

\clearpage

\foreach \country in {ARG, ARM, BEL, BEN, BFA,
BGD, BGR, BOL, BRA, BRB, CHL, %CIV,
 COL, CRI, CYP, CZE, DEU, DNK, DOM, ECU, ESP, EST, ETH, FIN, FRA, GHA, GRC, GTM,
 HRV, HUN, IDN, IND, IRL, IRQ, ISR, ITA, KEN, KHM, LBR, LTU, LUX, LVA, MAR, MDV, MEX, MLI, MMR, MNG, MWI, NER, NGA,  NIC, NLD, NOR, PAK, PER, PHL, POL, PRT, PRY, ROU, RWA, SEN, SLV, SUR, SVK, SWE, TGO, THA, TUR, UGA, URY, ZAF}{
  \input{../2_Tables/3c_OLS_Logit_combined/Table_CF_CI_UL20_\country}
}

\begingroup\fontsize{9}{11}\selectfont

\begin{ThreePartTable}
\begin{TableNotes}
\item \textit{Note: } 
\item This table shows the median carbon intensity in the first expenditure quintile ($\overline{e}_{r}^{1}$) and in the fifth quintile ($\overline{e}_{r}^{5}$). It displays the difference between the 5\textsuperscript{th} (20\textsuperscript{th}) and 95\textsuperscript{th} (80\textsuperscript{th}) within-quintile percentile for the first ($\overline{H}_{r}^{1}$ and $\overline{H}_{r}^{1*}$) and the fifth quintile ($\overline{H}_{r}^{5}$ and $\overline{H}_{r}^{5*}$). It also compares median carbon intensity in the first income quintile to that in the fifth quintile ($\widehat{V}$$_{r}^{1}$). Lastly it displays our comparison index facilitating the comparison of within-quintile variation between the first and fifth quintile ($\widehat{H}_{r}^{1}$ and $\widehat{H}_{r}^{1*}$, respectively).
\end{TableNotes}
\begin{longtable}[t]{l|cc|cccc|cccl|cc|cccc|cccl|cc|cccc|cccl|cc|cccc|cccl|cc|cccc|cccl|cc|cccc|cccl|cc|cccc|cccl|cc|cccc|cccl|cc|cccc|cccl|cc|cccc|ccc}
\caption{\label{tab:A7}Comparing median carbon intensity and horizontal heterogeneity between first and fifth expenditure quintile}\\
\toprule
\multicolumn{1}{c}{Country} & \multicolumn{1}{c}{$\overline{e}_{r}^{1}$} & \multicolumn{1}{c}{$\overline{e}_{r}^{5}$} & \multicolumn{1}{c}{$\overline{H}_{r}^{1}$} & \multicolumn{1}{c}{$\overline{H}_{r}^{5}$} & \multicolumn{1}{c}{$\overline{H}_{r}^{1*}$} & \multicolumn{1}{c}{$\overline{H}_{r}^{5*}$} & \multicolumn{1}{c}{$\widehat{V}_{r}^{1}$} & \multicolumn{1}{c}{$\widehat{H}_{r}^{1}$} & \multicolumn{1}{c}{$\widehat{H}_{r}^{1*}$} \\
\cmidrule(l{3pt}r{3pt}){1-1} \cmidrule(l{3pt}r{3pt}){2-2} \cmidrule(l{3pt}r{3pt}){3-3} \cmidrule(l{3pt}r{3pt}){4-4} \cmidrule(l{3pt}r{3pt}){5-5} \cmidrule(l{3pt}r{3pt}){6-6} \cmidrule(l{3pt}r{3pt}){7-7} \cmidrule(l{3pt}r{3pt}){8-8} \cmidrule(l{3pt}r{3pt}){9-9} \cmidrule(l{3pt}r{3pt}){10-10}
\endfirsthead
\caption[]{Comparing median carbon intensity and horizontal heterogeneity between first and fifth expenditure quintile \textit{(continued)}}\\
\toprule
\endhead

\endfoot
\bottomrule
\insertTableNotes
\endlastfoot
Argentina & 1.44 & 0.74 & 3.15 & 1.78 & 1.45 & 0.88 & 1.93 & 1.77 & 1.64\\
Armenia & 1.07 & 0.58 & 3.64 & 2.30 & 1.56 & 0.88 & 1.85 & 1.59 & 1.78\\
Australia & 0.91 & 0.50 & 1.41 & 0.80 & 0.64 & 0.35 & 1.83 & 1.75 & 1.80\\
Austria & 0.62 & 0.39 & 1.66 & 0.85 & 0.87 & 0.42 & 1.58 & 1.95 & 2.09\\
Bangladesh & 0.32 & 0.31 & 0.33 & 0.46 & 0.15 & 0.20 & 1.03 & 0.71 & 0.75\\
Barbados & 0.58 & 0.63 & 2.17 & 1.52 & 1.05 & 0.78 & 0.91 & 1.43 & 1.35\\
Belgium & 0.80 & 0.56 & 1.52 & 0.94 & 0.72 & 0.47 & 1.42 & 1.62 & 1.53\\
Benin & 0.18 & 0.38 & 1.26 & 1.42 & 0.71 & 0.61 & 0.49 & 0.88 & 1.17\\
Bolivia & 0.39 & 0.37 & 0.80 & 0.56 & 0.33 & 0.28 & 1.05 & 1.44 & 1.17\\
Brazil & 0.85 & 0.59 & 2.12 & 1.33 & 0.84 & 0.68 & 1.45 & 1.59 & 1.23\\
Bulgaria & 0.60 & 0.66 & 1.43 & 1.11 & 0.50 & 0.49 & 0.92 & 1.29 & 1.04\\
Burkina Faso & 0.02 & 0.46 & 1.58 & 1.41 & 0.65 & 0.52 & 0.04 & 1.12 & 1.25\\
Cambodia & 0.45 & 0.39 & 1.25 & 0.96 & 0.63 & 0.46 & 1.13 & 1.30 & 1.36\\
Canada & 0.66 & 0.56 & 1.73 & 0.77 & 0.79 & 0.36 & 1.19 & 2.24 & 2.22\\
Chile & 0.76 & 0.48 & 1.23 & 0.63 & 0.57 & 0.31 & 1.58 & 1.96 & 1.83\\
Colombia & 0.46 & 0.32 & 1.87 & 0.88 & 0.86 & 0.42 & 1.44 & 2.11 & 2.06\\
Costa Rica & 0.24 & 0.29 & 1.22 & 1.18 & 0.54 & 0.65 & 0.83 & 1.03 & 0.83\\
Côte d’Ivoire & 0.06 & 0.20 & 1.06 & 1.09 & 0.32 & 0.38 & 0.30 & 0.97 & 0.84\\
Croatia & 0.52 & 0.74 & 1.56 & 1.93 & 0.83 & 0.83 & 0.70 & 0.81 & 1.00\\
Cyprus & 0.79 & 0.61 & 1.68 & 1.04 & 0.89 & 0.54 & 1.29 & 1.61 & 1.66\\
Czechia & 1.56 & 1.41 & 2.83 & 2.14 & 1.12 & 0.90 & 1.11 & 1.32 & 1.26\\
Denmark & 0.39 & 0.34 & 1.58 & 1.09 & 0.68 & 0.40 & 1.13 & 1.45 & 1.69\\
Dominican Republic & 0.36 & 0.49 & 1.14 & 1.74 & 0.45 & 0.91 & 0.75 & 0.65 & 0.49\\
Ecuador & 0.34 & 0.27 & 1.02 & 0.78 & 0.37 & 0.37 & 1.26 & 1.31 & 1.00\\
Egypt & 0.59 & 0.61 & 0.46 & 0.73 & 0.21 & 0.30 & 0.98 & 0.63 & 0.71\\
El Salvador & 0.18 & 0.24 & 2.11 & 1.24 & 1.24 & 0.50 & 0.75 & 1.70 & 2.49\\
Estonia & 0.78 & 0.59 & 1.89 & 1.22 & 0.86 & 0.63 & 1.31 & 1.56 & 1.36\\
Ethiopia & 0.07 & 0.08 & 0.34 & 0.11 & 0.11 & 0.04 & 0.98 & 3.11 & 2.66\\
Finland & 0.45 & 0.40 & 1.25 & 0.96 & 0.63 & 0.52 & 1.12 & 1.30 & 1.23\\
France & 0.50 & 0.43 & 1.79 & 1.19 & 0.95 & 0.66 & 1.15 & 1.51 & 1.44\\
Georgia & 0.69 & 0.78 & 2.75 & 2.42 & 1.20 & 1.34 & 0.88 & 1.14 & 0.89\\
Germany & 1.29 & 0.93 & 2.18 & 1.49 & 1.02 & 0.69 & 1.38 & 1.46 & 1.48\\
Ghana & 0.07 & 0.16 & 0.49 & 1.10 & 0.11 & 0.23 & 0.42 & 0.45 & 0.49\\
Greece & 0.77 & 0.59 & 1.32 & 0.92 & 0.69 & 0.46 & 1.30 & 1.44 & 1.50\\
Guatemala & 0.06 & 0.50 & 0.54 & 1.72 & 0.13 & 0.83 & 0.13 & 0.32 & 0.16\\
Guinea-Bissau & 0.05 & 0.18 & 0.68 & 0.96 & 0.16 & 0.29 & 0.29 & 0.70 & 0.55\\
Hungary & 0.86 & 1.03 & 2.23 & 1.96 & 1.14 & 1.01 & 0.84 & 1.14 & 1.13\\
India & 0.98 & 0.99 & 0.81 & 1.11 & 0.40 & 0.50 & 0.99 & 0.73 & 0.80\\
Indonesia & 0.96 & 0.99 & 1.71 & 1.44 & 0.86 & 0.72 & 0.97 & 1.18 & 1.20\\
Iraq & 0.87 & 0.53 & 1.51 & 1.01 & 0.68 & 0.48 & 1.65 & 1.50 & 1.41\\
Ireland & 0.96 & 0.67 & 2.34 & 1.18 & 1.06 & 0.59 & 1.44 & 1.98 & 1.80\\
Israel & 0.65 & 0.38 & 1.52 & 0.93 & 0.69 & 0.46 & 1.72 & 1.62 & 1.49\\
Italy & 0.94 & 0.66 & 1.56 & 0.99 & 0.81 & 0.49 & 1.42 & 1.57 & 1.66\\
Jordan & 0.76 & 1.27 & 1.75 & 2.02 & 0.84 & 1.28 & 0.60 & 0.87 & 0.66\\
Kenya & 0.29 & 0.42 & 0.96 & 1.05 & 0.43 & 0.47 & 0.69 & 0.91 & 0.91\\
Latvia & 0.47 & 0.47 & 2.10 & 1.50 & 1.13 & 0.91 & 0.98 & 1.40 & 1.24\\
Liberia & 0.06 & 0.18 & 0.38 & 0.94 & 0.08 & 0.28 & 0.34 & 0.41 & 0.29\\
Lithuania & 0.24 & 0.45 & 1.42 & 1.39 & 0.69 & 0.70 & 0.54 & 1.02 & 0.98\\
Luxembourg & 0.61 & 0.37 & 1.33 & 0.76 & 0.66 & 0.40 & 1.65 & 1.75 & 1.67\\
Malawi & 0.01 & 0.02 & 0.02 & 0.32 & 0.01 & 0.04 & 0.37 & 0.07 & 0.15\\
Maldives & 0.29 & 0.18 & 0.50 & 0.36 & 0.22 & 0.17 & 1.57 & 1.42 & 1.24\\
Mali & 0.02 & 0.35 & 1.39 & 1.06 & 0.73 & 0.60 & 0.05 & 1.31 & 1.21\\
Mexico & 0.73 & 1.03 & 2.05 & 1.95 & 0.94 & 1.00 & 0.72 & 1.05 & 0.93\\
Mongolia & 1.20 & 0.52 & 3.84 & 1.98 & 2.29 & 0.93 & 2.30 & 1.94 & 2.46\\
Morocco & 0.63 & 0.53 & 0.75 & 0.81 & 0.32 & 0.39 & 1.19 & 0.93 & 0.82\\
Mozambique & 0.03 & 0.15 & 1.30 & 2.01 & 0.21 & 0.55 & 0.20 & 0.65 & 0.38\\
Myanmar (Burma) & 0.25 & 0.46 & 0.90 & 1.78 & 0.39 & 0.72 & 0.54 & 0.51 & 0.55\\
Netherlands & 0.93 & 0.65 & 1.18 & 0.89 & 0.59 & 0.45 & 1.42 & 1.32 & 1.32\\
Nicaragua & 0.03 & 0.26 & 0.46 & 1.33 & 0.14 & 0.47 & 0.11 & 0.34 & 0.29\\
Niger & 0.03 & 0.09 & 0.10 & 0.82 & 0.03 & 0.34 & 0.35 & 0.12 & 0.09\\
Nigeria & 0.18 & 0.54 & 0.82 & 0.94 & 0.39 & 0.43 & 0.34 & 0.87 & 0.90\\
Norway & 0.62 & 0.40 & 2.00 & 1.07 & 1.14 & 0.55 & 1.54 & 1.87 & 2.05\\
Pakistan & 0.29 & 0.70 & 0.83 & 1.05 & 0.40 & 0.54 & 0.41 & 0.79 & 0.75\\
Paraguay & 0.37 & 0.45 & 1.96 & 1.14 & 0.88 & 0.53 & 0.82 & 1.71 & 1.65\\
Peru & 0.70 & 0.47 & 2.51 & 0.84 & 1.31 & 0.39 & 1.50 & 2.97 & 3.34\\
Philippines & 0.26 & 0.52 & 0.50 & 0.71 & 0.20 & 0.33 & 0.50 & 0.71 & 0.61\\
Poland & 1.27 & 0.90 & 2.55 & 2.42 & 0.96 & 0.64 & 1.42 & 1.05 & 1.51\\
Portugal & 1.15 & 0.72 & 1.78 & 1.09 & 0.91 & 0.57 & 1.60 & 1.63 & 1.60\\
Romania & 0.49 & 0.75 & 1.37 & 1.75 & 0.61 & 0.86 & 0.66 & 0.78 & 0.71\\
Russia & 1.08 & 1.13 & 2.60 & 1.88 & 1.01 & 0.97 & 0.96 & 1.38 & 1.04\\
Rwanda & 0.00 & 0.02 & 0.04 & 0.68 & 0.00 & 0.04 & 0.17 & 0.06 & 0.10\\
Senegal & 0.07 & 0.28 & 0.51 & 0.98 & 0.15 & 0.28 & 0.27 & 0.53 & 0.53\\
Serbia & 0.73 & 0.78 & 1.26 & 2.29 & 0.55 & 0.54 & 0.94 & 0.55 & 1.03\\
Slovakia & 0.83 & 0.67 & 3.06 & 1.74 & 1.54 & 0.79 & 1.24 & 1.76 & 1.95\\
South Africa & 1.86 & 1.99 & 2.49 & 2.38 & 1.09 & 1.11 & 0.93 & 1.05 & 0.98\\
Spain & 0.62 & 0.59 & 1.57 & 1.04 & 0.76 & 0.55 & 1.05 & 1.50 & 1.38\\
Suriname & 0.18 & 0.12 & 0.83 & 0.54 & 0.31 & 0.16 & 1.53 & 1.54 & 1.94\\
Sweden & 0.45 & 0.38 & 1.65 & 0.98 & 0.91 & 0.58 & 1.19 & 1.69 & 1.56\\
Switzerland & 0.23 & 0.19 & 0.89 & 0.62 & 0.38 & 0.23 & 1.23 & 1.42 & 1.62\\
Taiwan & 1.13 & 1.16 & 0.93 & 1.12 & 0.47 & 0.63 & 0.98 & 0.84 & 0.74\\
Thailand & 1.45 & 1.51 & 2.23 & 2.10 & 1.20 & 1.28 & 0.96 & 1.06 & 0.93\\
Togo & 0.00 & 0.14 & 1.17 & 1.64 & 0.03 & 0.74 & 0.02 & 0.71 & 0.04\\
Turkey & 1.33 & 1.25 & 4.52 & 2.36 & 2.18 & 0.95 & 1.06 & 1.91 & 2.29\\
Uganda & 0.04 & 0.09 & 1.17 & 1.12 & 0.40 & 0.28 & 0.41 & 1.05 & 1.45\\
United Kingdom & 0.81 & 0.54 & 2.27 & 1.11 & 1.28 & 0.54 & 1.51 & 2.05 & 2.35\\
United States & 1.12 & 0.73 & 1.94 & 1.02 & 0.96 & 0.46 & 1.54 & 1.89 & 2.07\\
Uruguay & 0.24 & 0.20 & 0.88 & 0.56 & 0.35 & 0.29 & 1.21 & 1.57 & 1.24\\
Vietnam & 0.56 & 0.46 & 0.50 & 0.41 & 0.23 & 0.21 & 1.20 & 1.23 & 1.12\\*
\end{longtable}
\end{ThreePartTable}
\endgroup{}
 \label{tab:A7}

\end{document}
