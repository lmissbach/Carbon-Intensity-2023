\documentclass[12pt, a4paper]{article}
\usepackage[utf8]{inputenc}

\usepackage[a4paper,top=3cm,bottom=2cm,left=2.5cm,right=2.5cm,marginparwidth=2.55cm]{geometry}
%\usepackage[utf8]{inputenc}
%\usepackage[T1]{fontenc}
\usepackage{adjustbox}
\usepackage{amsmath}
\usepackage{amssymb}
\usepackage{authblk}
\usepackage[ngerman, english]{babel}
\usepackage{booktabs}
\usepackage{lineno}
\usepackage{csquotes}
\usepackage{float}
\usepackage[maxfloats=256]{morefloats}
\usepackage{graphicx}
\usepackage{longtable}
\usepackage{lscape}
\usepackage[skip = 1pt, indent = 20pt]{parskip}
\usepackage{pgffor}
\usepackage{caption}
\usepackage{subcaption}
\usepackage{setspace}
\usepackage{tabularx}
\usepackage{threeparttablex}
\usepackage{xcolor}
\usepackage{xtab}
\usepackage{hyperref}

\hypersetup{
    %pdftitle={Heterogeneity in carbon intensity of consumption},
    % pdfpagemode=FullScreen,
    %linkbordercolor = 0 1 1
}

\maxdeadcycles = 1000

\onehalfspace
\linenumbers
\modulolinenumbers[5]

% everything concerning biblatex

\usepackage[backend = biber,
            style=authoryear-icomp,
            citestyle=authoryear-comp,
            url=false,
            isbn=false,
            % doi=true,
            doi=false,
            maxcitenames=2,
            uniquelist=false,
            date=year,
            uniquename=false,
            sorting=ynt,
            sortcites]{biblatex}

\addbibresource{5_Literature/Literature.bib} % TBD

% Setup
\graphicspath{{../1_Figures/}}

% new environment

\newenvironment{subcaption2}
{\strut
\vspace{-5pt}
\begin{minipage}[b]{0.9\textwidth}
  \hspace*{-\parindent}
  \footnotesize}
 {\end{minipage}}

% For continous counting in subfigures
\renewcommand*{\thesubfigure}{\thefigure.\arabic{subfigure}}

 % Title
\title{A comprehensive analysis of distributional impacts of climate policy for 87 countries
%Distributional impacts of climate policy: A comprehensive analysis for 87 countries
}
% \title{Heterogeneity in carbon intensity of consumption}
\author[1,2,*]{Leonard Missbach}
\author[1,3]{Jan Christoph Steckel}
\affil[1]{\small Mercator Research Institute on Global Commons and Climate Change, Berlin, Germany}
\affil[2]{\small Technical University of Berlin, Berlin, Germany}
\affil[3]{\small Brandenburg University of Technology Cottbus Senftenberg, Cottbus, Germany}
\affil[*]{\normalsize Corresponding author: \href{mailto:missbach@mcc-berlin.net}{missbach@mcc-berlin.net}}

\date{December 2023}

\begin{document}

\maketitle
\begin{abstract}
  % Placement 1-2 sentence.
  % 1-2 sentences on what we do. 
  We analyze the distributional impacts of climate policy among households by examining heterogeneity in households' carbon intensity of consumption. We construct a novel dataset that covers information about the carbon intensity of 1.5 million individual households from 87 countries, representative for more than 5 billion people. Taking advantage of supervised machine learning, we analyze the non-linear contribution of household characteristics to predicting carbon intensity of consumption on a country-level.
  %Three sentences on results and implications
   Our results show that horizontal differences are generally large. Including household-level information beyond total household expenditures, such as information about vehicle ownership, space and energy use, increases accuracy of predicting households' carbon intensity. The importance of such features is country-specific while model accuracy varies across the sample. We identify six clusters of countries within which countries are more similar with respect to the distribution of climate policy costs and their determinants. Our results can facilitate more efficient, equitable and thus politically viable climate policy design.   
\end{abstract}

\smallskip

\noindent \small \textit{Keywords:} Climate mitigation, climate policy, inequality

\noindent \small \textit{JEL Codes:} C38, C55, D30, H23, Q56
%\normalsize Please do not distribute this manuscript without permission.

\thispagestyle{empty}
\clearpage
\setcounter{page}{1}

\normalsize

\section{Introduction} \label{sec:introduction}

% Setting the scene: Understanding distributional impacts are important because they matter to people.
Policy makers as well as the general public often judge a policy by its distributional implications. One reason is that the distribution of policy-induced costs between household groups can influence aggregate public perception and thus the political feasibility of policy reforms. In the context of climate change mitigation, unintended and heterogeneous policy impacts on households have been associated with public resistance \autocite{MaestreAndres.2019,Dechezlepretre.2022}, constraining the implementation of efficient and effective mitigation policies. For example, in the past, public protests contributed to reversal of envisaged climate policies, such as fossil fuel subsidy reforms \autocite{Clements.2013} or carbon pricing \autocite{Douenne.2020c}.

% Setting the scene 2: Why it is important to understand the distributional impacts of climate policy --> design of transfers.
It is possible to alleviate distributional impacts by complementing climate policy with (targeted) compensation measures. In practice, however, governments and other stakeholders face important information constraints. Ex-ante, it is often unclear how the costs of climate policy distribute among different households, if utilizing existing compensation measures are sufficient to remedy such costs, and which novel measures would be required to achieve any distribution of costs that is politically desirable. While compensating effectively, i.e. minimizing targeting errors is important to advance public support and to ease fiscal pressure, our understanding which compensation would be effective in achieving which distribution of costs \textit{post}-compensation is coarse.

% Research question
In this study, we analyze the heterogeneous impacts of climate policy on households within 87 countries. Transgressing traditional analyses of vertical and horizontal heterogeneity, we use supervised machine learning to disentangle the non-linear contribution of household characteristics that correlate with and drive the carbon intensity of consumption on a country-level. We argue that households' carbon intensity of consumption serves as an accurate representation of the short-term additional costs of climate policy, at least for any policy instrument that increases the marginal cost of emitting CO\textsubscript{2}.

% One paragraph sketching findings
We show that across the entire sample horizontal differences in carbon intensity exceed vertical differences. In addition, heterogeneity in total household expenditures by itself is often insufficient to predict heterogeneity in carbon intensity. Instead, other important household characteristics beyond household expenditures include vehicle ownership, information about space and energy use, such as cooking and heating fuels or appliance ownership. Our results point to country- and policy-specific distributional impacts, which call for compensation measures tailored to each country context, if households should be reimbursed effectively for additional costs. 

% One paragraph outlining contribution
Our contribution to a more comprehensive assessment of heterogeneous impacts of climate policy is threefold: First, we compile a novel and harmonized dataset on household-level carbon intensity of consumption. Our dataset contains granular information on 1.5 million single households representative for more than 5 billion people in 87 countries. In contrast, prior work often focuses on single country contexts or neglects within-country particularities on the household-level. Second, we use supervised machine learning to detect non-linear relationships between household characteristics and carbon intensity of consumption, while the (nascent) literature investigating horizontal heterogeneity primarily centers on linear models. Third, we identify different clusters of countries based on model outcomes. Countries in the same cluster are more similar to each other with respect to factors associated with the heterogeneity of carbon intensity. This approach helps to understand similarities and differences between the country-specific characteristics of distributional impacts more systematically. % Cross-country learning not really possible.

% Roadmap paragraph
We proceed as follows: In chapter \ref{sec:literature}, we revisit the literature on vertical and horizontal distributional impacts of climate policy and point to a knowledge gap around the analysis of horizontal heterogeneity and the design of complementary compensation policies. In chapter \ref{sec:data_methods}, we introduce our modelling approach combining household budget survey and multi-regional input-output data. We present empirical methods to describe both within-country heterogeneity and cross-country similarities. In chapter \ref{sec:results}, we analyze the relative importance of features (i.e. household characteristics) for predicting carbon intensity on the country-level and cluster countries, if features are similarly important. Lastly, we discuss our findings in light of on-going debates about how to circumvent or address unintended distributional impacts of climate policy in chapter \ref{sec:discussion} before we conclude in chapter \ref{sec:conclusion}.  

\section{Literature: Vertical and horizontal distributional impacts of climate policy} \label{sec:literature}

% Introduce distributional impacts of climate policy - importance for public support
One important criterion for climate policy analysis is the distribution of policy impacts across households. While mitigating climate change benefits all households, policy-induced costs may deviate for different households. Such distributional impacts can impede policy implementation, for example because governments oppose increasing inequality per se or because of lower public support \autocite{Bergquist.2022,Douenne.2022}. 

% Show why distributional impacts can happen (theoretical arguments).
Accordingly, there is a long-standing tradition in the economic literature that assesses distributional effects of environmental policy reforms \autocite{Cremer.2003,Poterba.1991,Sandmo.1975,Stiglitz.2019}. As a general result, households that consume more of a polluting good can expect higher absolute costs when pollution becomes more expensive, but relative additional costs depend on expenditure shares for polluting goods, which differ with income \autocite{Jacobs.2019,Dorband.2019}. Above all, the distribution of such additional costs depends on the distribution of less-polluting technology. If some households have access to less carbon-intensive technologies and others do not, then climate policy may lead to increasing inequality, posing a difficult trade-off between efficiency and equity \autocite{Hansel.2022,TerryDinan.2016}. 

% Show that vertical distributional impacts exist
Embarking on quantifying this trade-off, many researchers have studied the \textit{vertical} distributional effects of different climate policy instruments, i.e., heterogeneity in policy outcomes between relatively poorer and relatively richer households. For price-based climate policy (such as carbon pricing), such work includes analyses in single countries \autocite{Goulder.2019,Grainger.2010,Rausch.2011,Garaffa.2021,Sterner.2012,Wu.2022} or across countries \autocite{Budolfson.2021,Feindt.2021,Dorband.2019,Steckel.2021b,VogtSchilb.2019,Missbach.2024}. Particularly in many high-income countries, price-based policies that cover all sectors are often found to be regressive. In contrast, a meta-analysis \autocite{Ohlendorf.2021} documents more progressive results in lower income countries and for price-based policies directed at the transport sector.

% Other instruments
A different strand of research investigates vertical distributional impacts for other climate policy instruments, such as fossil fuel subsidy removal \autocite{Schaffitzel.2019,Giuliano.2020,DelArzeGranado.2012}, technology standards \autocite{Levinson.2019,Zhao.2022,Bruegge.2019}, subsidies on cleaner goods \autocite{Borenstein.2016,Vaishnav.2017,Winter.2019} or behavioural interventions \autocite{DellaValle.2020,Liebe.2021}. In substance, it emerges that all policy instruments entail some distributional consequences, which link closely to heterogeneous preferences for and endowment with less-polluting technologies. 

% Show that there is little sound understanding of horizontal impacts across many countries
More recently, researchers have started to become interested in the \textit{horizontal} distributional effects of climate policy, i.e., differences in policy impacts among similarly poor or rich households \autocite{Rausch.2011,Fischer.2019}. This follows partially from the empirical observation that variation \textit{within} expenditure quintiles can differ more substantially than \textit{between} quintiles \autocite{Cronin.2019,Steckel.2021b,Pizer.2019}. Such horizontal differences indicate that households use technologies with heterogeneous carbon intensity, which can however not be attributed to heterogeneous levels of household affluence \autocite{Hansel.2022}. In comparison to analyses of vertical distributional impacts, empirically investigating the determinants of horizontal impacts received little attention: Research highlights the role of energy use \autocite{Steckel.2021b,Missbach.2024}, space \autocite{Chan.2023,Burtraw.2009} or sociodemographic variables (such as household size, education, ethnicity and occupation) \autocite{Grainger.2010,Buchs.2013,Farrell.2017,Missbach.2023,Fremstad.2019} for horizontal heterogeneity.

% Explain why horizontal implications matter --> Redistribution
Horizontal heterogeneity matters for the design of complementing compensation policies, which is a frequently proposed option to address unintended distributional effects of climate policy \autocite{Klenert.2018,Baranzini.2017}. Often, researchers suggest lump-sum transfers to households as compensation for additional costs \autocite{Stiglitz.2017,Baranzini.2000,Metcalf.2009} that would indeed render the vertical distribution more progressive \autocite{Budolfson.2021,vanderPloeg.2022}, but would neglect or even increase horizontal differences \autocite{Cronin.2019,Hansel.2022}. Essentially, combining price-based climate policies with revenue-neutral lump-sum transfers would fall short of compensating those households that would bear the higher additional costs and such remaining costs would remain higher for larger levels of horizontal heterogeneity. Assuming climate policy should also serve the purpose of least distributive distortions \autocite{Fischer.2019} or of increasing public support by shielding especially vulnerable households, then appropriate design of compensation measures requires accounting for horizontal heterogeneity of policy impacts, rendering popular instruments, such as lump-sum transfers, less effective \autocite{Fullerton.2019,Missbach.2024}. 

% Short paragraph on targeting:  Literature on how to provide income to households in absence of information. Inclusion and exclusion errors. 
Horizontal heterogeneity may instead vindicate transfers targeted at households with large additional costs, albeit with detrimental consequences for aggregate efficiency \autocite{Hansel.2022}. In general, governments would face the challenge of ensuring that compensation reaches households that are eligible for compensation and does not reach households that are not \autocite[e.g.][]{Hanna.2018}. Research engaging with transfer design in the absence of information about recipients underpins that such \textit{targeting errors} can be sizeable, especially in non-industrialized countries \autocite{WorldBank.2018, Robles.2019}. One reason are limited institutional capacities \autocite[e.g.][]{Besley.2009} and larger administrative costs \autocite{Coady.2004} required to improve precision.

% Other conceivable transfer mechanisms
If (targeted) transfers are unavailable or likely ineffective, governments may draw on many other theoretically conceivable compensation policies: Lowering taxes on labor can be preferable on efficiency grounds \autocite{Pearce.1991,Goulder.1995,Bento.2018}, green spending can lead to increasing public support \autocite{Sommer.2022,Kotchen.2017,Dechezlepretre.2022}, funding public infrastructure can help promoting development goals \autocite{Franks.2018,Jakob.2016} and subsidizing or providing subsistence goods (including energy) may prevent detrimental impacts on the poorest households \autocite{Greve.2022,Schaffitzel.2019}. In any case, effectively accounting for the distributional effects of climate policy and designing complementing compensation policies requires precise information about the horizontal heterogeneity and household-level characteristics beyond income or total expenditures that help explain it. 

% Show that cross-country studies exist, but do not include the micro-level information
Our study also connects to research comparing within-country heterogeneity of carbon-intensive consumption across countries and time. For example, \textcite{Chancel.2022b} creates a time-series of country- and percentile-level carbon footprints, notably including households' investments decisions, and finds that \textit{carbon inequality} within countries has increased over the last thirty years. Others \autocite{Oswald.2020,Bruckner.2022} compare the distribution of carbon footprints across and within countries, but such macro-level studies usually remain calm about policy impacts and their associated vertical and horizontal distributional implications.

% Demonstrate gap and highlight contribution
Building on these findings, our paper provides a comprehensive analysis of vertical and horizontal distributional impacts of climate policy for a broad set of countries. This study contributes by highlighting the limitations of prevalent approaches to assess climate policies most prominently with respect to its \textit{vertical} distributional implications. Explicitly accounting for household characteristics beyond income or total expenditures, our paper analyzes characteristics associated with heterogeneous costs of policies. Our flexible modelling framework allows for the simulation of different policies - including, but not limited to price-based policies such as carbon pricing - and can help to inform about potential complementing compensation policies that are likely effective in addressing heterogeneous policy impacts on households.

\section{Data and methods: Analyzing heterogeneity in carbon intensity of consumption} \label{sec:data_methods}

% Introduce carbon intensity of consumption as major indicator
We infer about the heterogeneous costs of climate policy on households by analyzing heterogeneity in the carbon intensity of consumption: Assume household \textit{A}'s consumption is twice as carbon-intensive as the consumption of household \textit{B}, then climate policy will lead to costs twice as high for household \textit{A} compared to household \textit{B} and relative to total expenditures\footnote{This proposition holds under the assumption that climate policy increases supply-side input prices according to embedded CO\textsubscript{2}-emissions and that firms cannot react to changing input prices in the short-term. As a corollary, output prices for consumer goods and services would increase in equivalence to embedded (direct and indirect) CO\textsubscript{2}-emissions. More generally, the carbon intensity reflects additional costs of any policy increasing consumer prices proportional to embedded CO\textsubscript{2}-emissions, irrespective of existing policies. See also Appendix \ref{sec:policysimulation}.}.

In this chapter, we describe the construction of a novel dataset capturing household-level carbon intensities across countries. For each country, we explore and compare the vertical and horizontal heterogeneity of carbon intensities. Our preferred approach to analyze such heterogeneity is fitting boosted regression trees, which helps unravelling the contribution of single household-characteristics for predicting households' carbon intensity.

\subsection{Household-level carbon intensities: A novel dataset} \label{sec:data}

The carbon intensity of consumption of household \textit{i}, denoted by $e_{i}$, is the variable of interest in this study. It reflects the amount of CO\textsubscript{2}-emissions $E_{i}$ that one can reasonably attribute to production, transportation and retail of all goods and services purchased in household \textit{i}, over total consumption C$_{i}$. We express $e_{i}$ in $\frac{kgCO_{2}}{USD}$. More specifically, the carbon intensity of consumption represents carbon intensities of different sectors $e_{s}$, weighted by expenditure shares in household \textit{i} for goods and services from each sector \textit{s}, denoted as $w_{i,s}$:

% Argue backwards: Carbon intensity of consumption consists of household expenditure shares and sectoral carbon intensities

\begin{equation} \label{eq:ei}
e_{i} = \frac{E_{i}}{C_{i}} = \frac{\sum_{s} e_{s}*C_{i,s}}{\sum_{s} C_{i,s}} = \sum_{s} e_{s}*w_{i,s}
\end{equation}

% Show that carbon intensity is equivalent to first-order carbon pricing incidence (Detailed analysis in Appendix)
% Show that sector-specific carbon intensities can help to learn about other policies (to be continued in discussion)

Examining the household-level carbon intensity for different sectors \textit{s}, denoted $e_{i,s}$, allows for understanding heterogeneous impacts of different policies, for example of policies directed at the transport or electricity sector or of trade policies, such as carbon border adjustments.

%  Household expenditure shares: Data, cleaning, homogenizing and shortcomings
\paragraph{Sectoral expenditure shares} We collect information on sectoral expenditure shares at the household-level ($w_{i,s}=\frac{C_{i,s}}{\sum_{s}C_{i,s}}$) from household budget surveys (see table \ref{tab:datasets} for an overview). In such surveys, households report expenditures on goods and services on the item-level, from which we compute sectoral expenditure shares\footnote{We match consumption items to sectors with the help of matching tables. We share all matching tables through a stable online data repository. See Appendix \ref{code}.} \footnote{Figure \ref{fig:Engel} shows country-level Engel curves for energy, goods, services and food.}. Survey datasets are eligible for inclusion in our study, if they cover a nationally representative sample, include item-level expenditure information and if surveys were conducted between 2010 and 2019\footnote{We exclude more recent survey data to account for potential biases induced by large economic shocks, such as measures in the context of Covid-19.}. After several cleaning steps\footnote{Appendix \ref{sec:cleaning} lists details on cleaning and our efforts to harmonize household characteristics across countries.}, our resulting dataset contains information on more than 1.5 million individual households representative for 5 billion people from 87 countries that comprise 68\% of global GDP and 52\% of global CO\textsubscript{2}-emissions\footnote{We calculate these numbers with data for GDP and CO\textsubscript{2}-emissions from the World Development Indicators Database \autocite{WorldBankGroup.2023}.}.

Beyond expenditure shares, we include total household expenditures as a surrogate for household income because total consumption expenditures are a better-suited proxy for lifetime income \autocite{Poterba.1989,Poterba.1991,Cronin.2019} and because wage data from such surveys are often unreliable \autocite{Blundell.1998}. Acknowledging differences between using expenditures or income data in the context of calculating carbon footprints \autocite{Levay.2023}, we proceed considering total household expenditures and income as synonyms in the remainder of the study.

We also include socio-demographic information about household members (such as education, gender, nationality, self-identified ethnicity, religion or age of household representatives), detailed spatial information (such as province, district or village of households) and on energy use (such as main fuels used for cooking, lighting and heating) or appliance and vehicle ownership. Such household-level information (including total household expenditures) forms a set of variables $X_{i}^{'}$ that allows for analyzing differences between households with different characteristics\footnote{Table \ref{tab:A1} shows summary statistics for all countries in our sample; Table \ref{tab:A2} shows average household expenditures and average energy expenditure shares for each expenditure quintile and each country. We also show the share of households using different cooking fuels (table \ref{tab:A4_CF}), lighting fuels (table \ref{tab:A5_LF}) and the possession of different major appliances (table \ref{tab:A6}) for all countries where such data are available.}.

% Sectoral carbon intensities: Data, method, intuition and shortcomings
\paragraph{Sectoral carbon intensities} We complement data on expenditure shares with country- and sector-level carbon intensities $e_{s,r}$, which represent CO\textsubscript{2}-emissions that can directly or indirectly be attributed to one unit of (household) consumption from sector \textit{s} in region \textit{r}:

\begin{equation}
    e_{s,r} = \frac{E_{s,r}^{direct}+E_{s,r}^{indirect}}{\sum_{i} C_{i,s,r}}
\end{equation}

We derive total sectoral consumption ($\sum_{i} C_{i,s,r}$), direct ($E_{s}^{direct}$) and indirect ($E_{s}^{indirect}$) CO\textsubscript{2}-emissions from and with the help of multi-regional input-output (MRIO) data. This approach is popular among researchers as it accommodates trade flows between different countries and regions, but features sufficient detail for high sectoral resolution. 

We capitalize on trade data from the GTAP database \autocite[Version 11, ][]{Aguiar.2022} that we transform to MRIO data \autocite{Peters.2011}, capturing input-output relationships between 65 sectors \textit{s} in 160 countries \textit{r}. Subsequently, we compute the \textit{Leontief}-inverse $L_{r',s'}^{r,s}$ that captures information about required inputs from each sector \textit{s'} and region \textit{r'} for production of one unit of output in each sector \textit{s} and region \textit{r}. We derive indirect CO\textsubscript{2}-emissions $E_{s}^{indirect}$ as follows\footnote{See \textcite{Missbach.2024, Steckel.2021b,Feindt.2021,VogtSchilb.2019} for a detailed description of this approach.} \footnote{Simulation of different sectoral and regional policies is possible through exclusion of different sectors \textit{s} or countries \textit{r}. Our flexible framework also allows for analyzing the impacts of policies targeted at non-CO\textsubscript{2}-emissions, such as CH$_{4}$, N$_{2}$O or F-gases. In our main analysis we focus on national carbon intensities, i.e., how many CO\textsubscript{2}-emissions resulting from production within each country can be attributed to one unit of output. This would be equivalent to zero carbon intensities of imported products, but re-imported emissions would be included. See also Appendix \ref{sec:policysimulation}.}:

\begin{equation}
    E_{s}^{indirect} = \sum_{r'} \sum_{s'} e_{r',s'} L_{r',s'}^{r,s} C_{s}
\end{equation}

In addition, the GTAP database also includes information on direct CO\textsubscript{2}-emissions $E_{s}^{direct}$. It covers CO\textsubscript{2}-emissions resulting from household-level use of fossil fuels, such as gasoline, natural gas, LPG or hard coal.

Our result is a matrix containing information on the carbon intensity of (household) consumption $e_{s,r}$ for 65 sectors \textit{s} and 160 countries \textit{r}. This data reflect technologies, prices and trade relationships between sectors and countries for the year 2017. We show all country- and sector-level carbon intensities used in this study in figure \ref{fig:B}.

\paragraph{A novel cross-country dataset} 

Our resulting dataset integrates information on household characteristics and households' expenditure shares with country- and sector-level carbon intensities, as described in equation \ref{eq:ei}. Specifically, it consists of nationally representative accounts of households' carbon intensity of consumption $e_{i}$. Seizing detailed information about several household characteristics allows us to analyze heterogeneity in carbon intensity. To the best of our knowledge, such a dataset linking household-level information to sectoral expenditure shares, weighted by country- and sector-level carbon intensities, is unprecedented and may help to inform more detailed policy analysis in the future\footnote{See Appendices \ref{data_availability} for more information about data availability and \ref{code} for information about code written for cleaning, modelling and analysis.}.

\subsection{Vertical and horizontal heterogeneity in carbon intensity} \label{sec:descriptive}

% Descriptive analysis 
We proceed by analyzing the heterogeneity in carbon intensity of consumption descriptively to provide some intuition for the vertical and horizontal distributional impacts of climate policy at the country-level.

Across countries, the average CO\textsubscript{2}-intensity of household consumption is 0.68 kgCO\textsubscript{2}/USD. The average CO\textsubscript{2}-intensity is highest for South Africa (2.15 kgCO\textsubscript{2}/USD) followed by Turkey (1.75 kgCO\textsubscript{2}/USD) and Czech Republic (1.71 kgCO\textsubscript{2}/USD). The average CO\textsubscript{2}-intensity is lowest for Malawi (0.03 kgCO\textsubscript{2}/USD), Rwanda (0.04 kgCO\textsubscript{2}/USD), Ethiopia and Niger (both 0.1 kgCO\textsubscript{2}/USD)\footnote{See table \ref{tab:A3} for average CO\textsubscript{2}-intensities for all countries.}. Country-level CO\textsubscript{2}-intensities help to infer about the relative average costs of climate policy in countries: For example, a carbon price of USD 40 per tCO\textsubscript{2} \autocite{Stiglitz.2017} would be equivalent to relative average costs of 2.7\% of total annual expenditures in a country with an average CO\textsubscript{2}-intensity of 0.68 kgCO\textsubscript{2}/USD.

Analyses of distributional impacts of climate policy often focus on comparing average (or median) costs of policies for different income groups of households. One frequent approach is to assign households to income (or expenditure) quintiles to infer about vertical heterogeneity. Recently, researchers have also started to compute measures for within-group heterogeneity, such as the 25\textsuperscript{th} or 75\textsuperscript{th} quantile within each expenditure quintile \autocite{Cronin.2019, Missbach.2024}. Comparing such quantile costs across expenditure quintiles can help to infer about horizontal heterogeneity.

Figure \ref{fig:fig_1} displays the distribution of carbon intensity of consumption among the poorest quintile in all countries of our sample with the help of boxplots. Boxes and whiskers contain 90\% of all households in each quintile and epitomize the horizontal heterogeneity, i.e. differences among poorer households. In contrast, coloured bars show the difference between the lowest and the highest median carbon intensity across all quintiles for each country, describing the vertical heterogeneity, i.e. differences between poorer and richer households\footnote{See figure \ref{fig:Quint} for country-level comparisons across all expenditure quintiles and table \ref{tab:A3} for summary statistics on carbon footprints and carbon intensity of consumption.}.

\begin{figure}[ht!]
    \centering
    \includegraphics{Figure 1/Figure_1_2017}
    \caption{Vertical differences and horizontal distribution of carbon intensity within poorest quintiles}
    \label{fig:fig_1}
    \begin{subcaption2}
    Boxplots display the horizontal distribution of household-level carbon intensity within the poorest expenditure quintile in each of 87 countries in our sample: The boxes display the 25\textsuperscript{th} and 75\textsuperscript{th} percentile; whiskers display the 5\textsuperscript{th} and 95\textsuperscript{th} percentile, respectively. Rhombuses display the mean. Blue and red bars represent the vertical difference in household-level carbon intensity, i.e., the difference between the highest and the lowest median carbon intensity across quintiles. Red (blue) bars indicate that richer households consume less (more) carbon-intensively at the median compared to poorer households. Figure \ref{fig:Quint} shows the distribution of carbon intensities for all expenditure quintiles in all countries of our sample.
    \end{subcaption2}
\end{figure}

Figure \ref{fig:fig_1} illustrates that within-quintile heterogeneity exceeds between-quintile heterogeneity in \textit{all} countries. This underlines that analyses building on differences in income to explain differences in carbon intensity of consumption (or the impact of climate policy) might be insufficient, since they fall short of accounting for differences in carbon intensity at similar levels of income. Instead, we propose including household-level characteristics beyond income in such analyses to give a more nuanced description of which households' consumption is especially carbon-intensive.

This is also warranted, because - as we show in figure \ref{fig:fig_2} - within-quintile differences vary across quintiles. To facilitate comparison of such differences across countries, we compute two coefficients \autocite{Missbach.2024}\footnote{Arguably, many approaches are plausible to assess and compare the heterogeneity within and across expenditure quintiles. For example, \textcite{Cronin.2019} inspect the standard deviation of additional costs.}: The vertical distribution coefficient $\widehat{V_{r}}$ compares median carbon intensity of the poorest and the richest expenditure quintile:

\begin{equation}
    \widehat{V_{r}} = \frac{\overline{e_{EQ1}}}{\overline{e_{EQ5}}}
\end{equation}

If the median carbon intensity among poorer households exceeds (is smaller than) the median carbon intensity of richer households, then $\widehat{V_{r}}>1$ ($\widehat{V_{r}}<1$) and climate policy would likely lead to regressive (progressive) outcomes.

The horizontal distribution coefficient $\widehat{H_{r}}$ compares within-quintile differences of the poorest and the richest expenditure quintile:

\begin{equation}
    \widehat{H_{r}} = \frac{e_{EQ1}^{95} - e_{EQ1}^{5}}{e_{EQ5}^{95} - e_{EQ5}^{5}}
\end{equation}

$\widehat{H_{r}}>1$ ($\widehat{H_{r}}<1$) would indicate that within-quintile differences are larger (smaller) among poorer compared to richer households.

Figure \ref{fig:fig_2} illustrates that the average carbon intensity of consumption is larger among the poorest quintile in 43 out of 87 countries compared to the richest quintile. These countries are relatively more affluent than others, as expressed through a higher GDP per capita: We observe $\widehat{V_{r}}>1$ for 20 out of 20 countries in our sample with the highest GDP per capita. In these countries, climate policy is likely to have regressive effects. Similarly, the average carbon intensity is higher for richer quintiles compared to poorer quintiles ($\widehat{V_{r}}<1$ ) in 18 out of 20 countries in our sample with the lowest GDP per capita. In these countries, climate policy is likely to have progressive effects. Both findings are in line with inverse-U-shaped Engel curves for carbon-intensive goods and services across countries and income quintiles \autocite{Dorband.2019}. 

\begin{figure}[ht!]
    \centering
    \includegraphics{Figure 2/Figure_2_2017}
    %\includegraphics{Figure 2/Legend}
    \caption{Vertical and horizontal distribution coefficient}
    \label{fig:fig_2}
    \begin{subcaption2}
    The vertical distribution coefficient (y-axis) compares the median carbon intensity for the richest and the poorest quintile. The horizontal distribution coefficient (x-axis) compares the within-quintile differences (5\textsuperscript{th} to 95\textsuperscript{th} percentile within quintiles) of the richest and the poorest quintile. Rectangles (A) and (B) indicate higher carbon intensity (at the median) among the poorest quintile compared to the richest quintile; rectangles (C) and (D) indicate lower carbon intensity (at the median) among the poorest quintile compared to the richest quintile. Rectangles (A) and (C) indicate smaller within-quintile differences of carbon intensity among the richest quintile compared to the poorest quintile; rectangles (B) and (D) indicate larger within-quintile differences of carbon intensity among the richest quintile compared to the poorest quintile. Point colors indicate GDP per capita for 2018 (in log-transformed constant 2010 USD). Table \ref{tab:A7} lists all distribution coefficients for all countries.
    \end{subcaption2}
\end{figure}

% Check all numbers
In 58 out of 87 countries, within-quintile variation (expressed as the difference between the 5\textsuperscript{th} and the 95\textsuperscript{th} percentile within quintiles) is larger in the poorest quintile compared to the richest quintile. Differences in horizontal heterogeneity, i.e. within-quintile heterogeneity, exceed vertical differences, i.e. between-quintile heterogeneity in 66 countries, emphasizing the need for detailed investigation of household characteristics associated with higher levels of carbon intensity of consumption.

\subsection{Analyzing heterogeneity in carbon intensity} \label{sec:methods}

Such descriptive evidence suggests that horizontal heterogeneity of carbon intensity is consistently larger than vertical heterogeneity. In response, one core assumption of this study is that $e_{i}$, the carbon intensity of household \textit{i}, correlates with observable household characteristics $X_{i}^{'}$, including, but not limited to total household expenditures:

\begin{equation} \label{eq:relationship}
    e_{i} \sim X_{i}^{'}
\end{equation}

% Which criteria should be included in modelling and why?
% Discuss main arguments for single characteristics (possibly already in methods section).

To shed light on which household characteristics correlate with and possibly lead to higher levels of carbon intensity of consumption, we build on two analytical approaches, namely boosted regression trees (BRT) and a logit-model.

\paragraph{Boosted regression trees} Fitting boosted regression trees \autocite{Friedman.2003, Elith.2008} is a supervised machine learning method allowing for detection of non-linear relationships and interaction effects between an outcome and many predictor variables (\textit{features}). As an extension to regression trees, the BRT-algorithm (\texttt{XGBoost} by \textcite{Chen.2016}) fits many single regression trees, iteratively giving higher weights to observations with larger predicting errors. This leads to large predictive power, also if compared to the popular random forest algorithm. Nevertheless, fitting BRT is more computationally intensive and outcomes are more difficult to interpret compared to outcomes from other machine learning models. 

% Why BRT is helpful here.
Drawing on BRT serves the purpose of our analysis, because it is a priori ambiguous which variables justify inclusion in a model. In addition, research indicates that the impacts of climate policy (and accordingly the carbon intensity of consumption) distribute non-linearly across households with different characteristics, such as income, demographic groups \autocite{Missbach.2023}, energy use \autocite{Farrell.2017} and space \autocite{Chan.2023}. In contrast to other approaches, such as variance-based inequality decomposition \autocite{Farrell.2017,Sager.2019,Missbach.2024}, deploying BRT is well-suited to help identifying important predictors while also allowing for detection of non-linear relationships.

We fit BRT-models on the country-level to investigate characteristics associated with heterogeneous levels of carbon intensities within single countries. The carbon intensity $e_{i}$ is the outcome variable. For each country-level model we use the entire (\textit{rich}) set of household-level characteristics $X_{i}^{'}$ as possible features and perform several feature engineering steps (see also Appendix \ref{sec:featureengineering}). In addition, we include only total household expenditures as single feature for prediction in a \textit{sparse} model. Comparing sparse and rich models helps to distil the contribution of additional features that are often not included in such analyses.

The predictive performance of BRT-models critically hinges on several hyperparameters. For hyperparameter tuning, we randomly split country-level data into a training and a test sample, comprising 80\% and 20\% of observations, respectively. We use five-fold cross-validation on the training sample to fit 1,000 trees - following the recommendations by \textcite{Elith.2008} - along with 30 different combinations of learning rate ($\eta \in [0.001,0.3]$), the maximum depth of trees (\texttt{max\_depth} $\in \{x \in \mathbb{N} \mid 1  \leq x \leq 15 \}$) and a fraction of features included in each tree (\texttt{mtry} $\in \{0.5,0.7,1\}$). We select combinations of hyperparameters such that combinations distribute evenly across the possible combination space. For each country, we select the combination of hyperparameters that minimizes mean absolute error (MAE).

Building on selected hyperparameters, we use five-fold cross-validation on all observations, i.e., from the training and test sample, for model evaluation. We evaluate model performance with the help of MAE, root mean squared error (RMSE) and goodness of fit (R\textsuperscript{2}). 

We also use all observations to evaluate the relative importance of each feature with the help of SHAP-values \autocite{Lundberg.2017}: Expressed in the unit of the outcome variable, SHAP-values represent the contribution of each feature to each individual prediction. SHAP-values have been proposed as a more suitable means to interpret machine learning models compared to other approaches because of improved accuracy, consistency and interpretability \autocite{Lundberg.2020}. Building on SHAP-values for all features and individual predictions we proceed by calculating the average absolute SHAP-value for each feature across all predictions, which allows for interpretation as feature importance. Higher average SHAP-values indicate that differences in a feature contribute more to predicting outcomes. We express feature importance as share of contribution (in \% of total mean absolute SHAP-values) to allow for better comparability of feature importance across countries. In addition, we visualize the distribution of SHAP-values for the most important features in each country over feature values with the help of partial dependence plots in figure \ref{fig:5b}. 

\paragraph{Logit-model} For supplementary robustness analyses, we also fit a logit-model to identify households whose consumption is relatively more carbon-intensive compared to the entire population. We construct a binary variable $e_{i}^{80^{th}}$ for each household \textit{i} indicating whether the household is among the most carbon-intensive 20\% of households in each country:

\begin{equation}\label{eq:logit}
    e_{i}^{80^{th}} =
    \begin{cases}
    1, & \text{for }  e_{i} \geq e^{80^{th}} \\
    0, & \text{for }  e_{i} < e^{80^{th}}
    \end{cases}
\end{equation}

With $P_{e_{i}^{80^{th}}}$ representing the probability of household \textit{i} to consume more carbon-intensively than 80\% of the population in each country, we are interested in the coefficients $\beta^{'}$ of the following logit-model:

\begin{equation} \label{logit}
    log \left( \frac{P_{e_{i}^{80}}}{1 - P_{e_{i}^{80}}} \right) = \alpha_{0} + \beta^{'} X_{i}^{'} + \varepsilon_{i}
\end{equation}

Estimating a logit-model serves as a robustness check for results from BRT-models. It also allows for investigation of characteristics associated with "hardship-case" households including a more accessible interpretation of results and parameters. For the purpose of informative comparison across countries, we show results from logit-models with the help of average marginal effects for each independent variable.

\paragraph{Identifying country clusters} Country-level analyses can be meaningful to identify country-specific household characteristics associated with higher levels of carbon intensity of consumption. To investigate similarities and differences with respect to feature importance between many countries, we seek to identify clusters of countries. In fact, the relationship between household-level characteristics and carbon intensity of consumption is unique for each country, but also contingent on the availability of granular data. We proceed adjusting individual feature importance by multiplying individual feature importance and country-level R\textsuperscript{2} to account for differences in available features across countries. This approach also helps to account for the aggregate performance of country-level models and allows for better comparison of feature importance across countries.

Building on (adjusted) feature importance for each country, we use the k-means algorithm for clustering. If features are missing in the data, we assume their share of contribution is zero. We normalize all feature values to allow for comparison across features. If one feature is more (less) important in one country compared to all other countries, the processed feature value will be relatively high (low). If one feature is equally important across all countries, processed feature values will be close to zero.

K-means clustering is an unsupervised machine-learning method and helps to analyze clusters of observations that are most similar in many variables within each cluster and least similar in many variables across clusters \autocite{MacQueen.1967}. We inspect the optimal number of clusters ($\{k \in \mathbb{N} \mid 2  \leq k \leq 20 \}$) with the help of average silhouette widths \autocite{Rousseeuw.1987} for each cluster \textit{k}. The silhouette width $s_{i}$ accounts for the average Euclidean distance of each observation \textit{i} to all other observations within its cluster and for the average distance to observations from the nearest neighbouring cluster. Silhouettes closer to $1$ indicate a good fit of an observation to its cluster and silhouettes closer to $-1$ indicate a poor fit. The average silhouette width $\overline{s_{k}}$ for each cluster \textit{k} expresses how well all observations fit on average to each cluster. Our approach yields $k = 6$ to be the number of clusters maximizing average silhouette width\footnote{See figures \subref{fig:G3_silhouette_2} and \subref{fig:G4_silhouette_2} for visualization.}. We also show the optimal number of clusters ($k = 13$) for k-means clustering building on non-adjusted feature importance in the Appendix\footnote{See figures \subref{fig:G1_silhouette} and \subref{fig:G2_silhouette} for visualization. Using non-adjusted feature importance for clustering changes the interpretation of clusters: Features of countries within the same cluster contribute similarly to explaining variation in carbon intensity without respecting the availability of features in the data and the explanatory power of the model.}.

In countries within the same cluster single features are similarly important to predict the carbon intensity of households. For each cluster, we compute average values for each feature to allow for investigation of differences between countries in different clusters.%We also investigate which country is most likely to all other countries in each cluster maximizing silhouette width within each cluster and use it to demonstrate cluster-level characteristics. 

It is important to note that our approach to adjust feature values for total model performance reduces bias through limited feature availability in the data. Uncorrected feature importance values may be exaggerated, if only few features exist, so countries with few features may end up in wrong clusters mainly because the model cannot help explaining much of the variation in carbon intensity. Instead, our approach ensures that all \textit{observable} features contribute to clustering. Despite many structurally unobservable household-level features, our approach might be warranted under the assumption that policy formulation (for example on optimal targeting of compensation measures) can naturally only center on observable characteristics, if targeting errors should be minimized.

\subsection{Methodological limitations}

While our previously described approach may serve as a consistent method to investigate heterogeneous policy impacts, some methodological aspects pose a limitation and thus warrant attention.

For example, utilizing expenditure survey data is susceptible to many often-described inaccuracies: Such data are prone to under-reporting \autocite{Meyer.2015}, spare the upper end of the income distribution \autocite{Blanchet.2022} and reflect consumer prices and policy regimes in respective survey years. Our approach also neglects within-sector differences in carbon intensity of consumption and builds on consumer-price-dependent \textit{expenditures} to calculate household-level carbon intensity instead of quality and quantity of consumption. This implies that we systematically overlook consumption of goods and services traded on informal markets, which may be defensible, given that additional costs through climate policy are most likely to occur through formal consumption.

Household-level expenditure data may also suffer from measurement error, which may influence the analysis of horizontal heterogeneity. Fortunately, our approach can accommodate this concern, since feature importance would be negligible, if differences in expenditure shares between households were not correlated with differences in feature values, at odds with the assumption stipulated by equation \ref{eq:relationship}.

Our approach allows for a consistent, harmonized analysis across countries, but falls short of accounting for the deployment of cleaner technologies since 2017. Yet, more recent MRIO-data with broad geographical coverage are - to the best of our knowledge - unavailable. Also, our analysis may be well-suited to inform about the immediate impacts of climate policy, but neglects medium-term effects occuring in the general equilibrium. 

One important qualification is that our modelling approach is not apt to allow for causal interpretation, notably because we examine cross-sectional variation. Instead, we attempt to provide an accurate description of household characteristics correlating with households' carbon intensity, including non-linear relationships\footnote{For example, our analysis should not be understood such that improving education inevitably \textit{leads} to a lower carbon intensity of consumption, but that households, who consume less carbon-intensively, are often better educated, controlling for other important predictors and interaction effects.}.

Collecting household-level data from various datasets impedes the cross-country comparison of model outcomes, because some features are lacking in some countries. In response, we adjust feature importance for models accuracy, but it cannot be concluded from our results that carbon intensity is unpredictable \textit{per se}, if model accuracy is low. Some important features remain structurally unobserved by us, but not by governments or other actors interested in our results. For some countries, more nuanced data can therefore help to flesh out more pervasive analyses.

Clustering countries is subject to uncertainty and contingent on which criteria are included for clustering. Our approach of adjusting feature importance helps preventing that countries end up in one cluster, simply because features are lacking in the data. Nevertheless, if more information were observable to us or if different criteria were included, countries may end up in different clusters. Arguing that we include all relevant and available criteria while minimizing redundancy, clustering can be meaningful to identify similarities in divergence.

\section{Results: Determinants of heterogeneous carbon intensity of consumption} \label{sec:results}

Climate policy can lead to short-term costs, which distribute unevenly across the population depending on the heterogeneity in household-level carbon intensity of consumption. Identifying household characteristics (including total household expenditures) that correlate with households' carbon intensity serves to understand this heterogeneity. In the following, we analyze a set of household characteristics and their importance for predicting the carbon intensity of households. We compare the importance of features across countries and assign countries to clusters.

\paragraph{Model accuracy} 
Evaluating model accuracy provides an indication about whether compensation can be targeted effectively. If a model performs well in predicting households' carbon intensity based on household characteristics, it may be more likely for governments to compensate households with high precision, based on important features. 

% Sparse models: Total household expenditures only is insufficient
Variation in total household expenditures alone is often not sufficient to predict households' carbon intensity with high accuracy. On average, the goodness of fit (R\textsuperscript{2}) accumulates to 5\% for sparse BRT-models including only total household expenditures (see figure \ref{fig:comparison} or table \ref{tab:A8}). In 79 countries, such sparse models do not contribute to explaining more than 10\% of variation in carbon intensity. This implies that compensation measures based on household expenditures (such as reducing consumption taxes, but also (targeted) transfers), which would benefit all households equally, would prove ineffective to compensate households with highest additional costs.

% Rich models: Performance increases and predictive power is usually sufficient or acceptable
In contrast, including additional features increases model accuracy. On average, R\textsuperscript{2} amounts to 23\% for rich BRT-models including many features in addition to total household expenditures. Accuracy of rich models increase substantially in comparison to sparse models, for example in the case of Jordan from 3\% to 59\% (R\textsuperscript{2}). Overall, rich BRT-models helps to predict households' carbon intensity with fair precision in many countries. Rich models' R\textsuperscript{2} accumulates to 59\% for Jordan, 53\% for Peru or 50\% for Nicaragua and Niger, exceeding 30\% in 28 countries (table \ref{tab:A8}).

% In some countries, however, predictive power is low which is an interesting finding
For some countries, however, rich models' accuracy is comparably low. In 18 countries, R\textsuperscript{2} does not exceed 10\%. Model accuracy is lowest in Bulgaria (1\%), Estonia (2\%), Serbia (3\%) or Suriname (3\%). One reason is that model performance hinges critically on data granularity. In cases of low accuracy our models are restricted to drawing on few available features, such as household expenditures, sub-national area identifiers, household size or education of household head. Nevertheless, low model accuracy implies that it is difficult in some countries to infer about households' carbon intensity through observable characteristics, including total household expenditures. In Bulgaria, for example, vertical differences are small ($\widehat{V}_{r}^{1}=0.93$) and horizontal differences within expenditure quintiles are comparably large ($\widehat{H}_{r}^{1}=1.26$, see also figure \subref{fig:Quint_B}). Moreover and as our analysis confirms, within-quintile variation in total household expenditures is largely uncorrelated with variation in carbon intensity, which provides additional evidence for analyzing heterogeneity in policy impacts beyond (vertical) differences in affluence.

\paragraph{Country-level feature importance}
% Features are not equally important across countries
The importance of features for predicting variation in carbon intensity differs across countries. Figure \ref{fig:fig_4} and table \ref{tab:A10} show adjusted feature importance for all features in each country and our indicators for vertical and horizontal inequality and the mean CO\textsubscript{2}-intensity, grouped by country clusters\footnote{See table \ref{tab:A10_Uncorrected} for non-adjusted feature importance for all features in each country.}. While inspecting feature importance helps to identify features explaining heterogeneity in carbon intensity, we also consider the contribution to predicted outcomes for different feature values, as visualized for each country in supplementary partial dependence plots (see figure \ref{fig:5b}).

% Household expenditures.
Without adjustment for model accuracy, the most important feature across countries is total household expenditures, accounting for a relative contribution of 22\% on average. Household expenditures is the single most important feature for prediction in 30 countries, and in some countries, such as \hyperref[fig:5b_LUX]{Luxembourg} or \hyperref[fig:5b_HRV]{Croatia}, differences in household expenditures contribute to more than 50\% of model prediction. With adjustment for model accuracy, household expenditures contributes most to outcome prediction in \hyperref[fig:5b_PER]{Peru} (18\%), \hyperref[fig:5b_ECU]{Ecuador} (14\%) and \hyperref[fig:5b_IRQ]{Iraq} (14\%) - countries in which we also identify consistently larger carbon intensities for poorer households compared to richer households. The relationship between household expenditures and carbon intensity is non-linear, but overall declining for 53 countries, overall increasing for 14 countries, following an inverse-U-shape for 11 countries and a U-shape for 5 countries (see figure \ref{fig:5b}). We find declining relationships between household expenditures and carbon intensity for 19 of 20 countries with highest GDP per capita, which lends credibility to our descriptive analysis in section \ref{sec:descriptive}. In such countries, more carbon-intensively consuming households spend absolutely less on consumption, but relatively more on carbon-intensive goods and services.

% Motorcycle and car ownership
Motorcycle and car ownership is the most important feature in 15 and 14 countries, respectively. In \hyperref[fig:5b_BFA]{Burkina Faso}, \hyperref[fig:5b_MLI]{Mali}, \hyperref[fig:5b_NER]{Niger} or \hyperref[fig:5b_TGO]{Togo} variation in motorcycle ownership can be attributed to more than 20\% of variation in carbon intensity. Car ownership accounts for largest adjusted feature importance in \hyperref[fig:5b_JOR]{Jordan} (33\%) and \hyperref[fig:5b_TWN]{Taiwan} (25\%). On average, vehicle ownership is the most important feature across all countries and features \textit{including} adjustment for model performance. Vehicle ownership can be a meaningful predictor for costs of climate policy in some countries: Households owning motorcycles or cars are more likely to consume more carbon-intensively than households without such vehicles in every country of our sample. This links to the propensity of vehicle-owning (and -using) households to consume relatively more transport fuels than others.

% Space: urban/rural, district, province
Spatial features, such as urban or rural location, state, province or district of household, are the most important feature in 17 countries. For example, differentiating between urban and rural households contributes to more than 40\% of model prediction in countries such as \hyperref[fig:5b_CZE]{Czech Republic}, \hyperref[fig:5b_SVK]{Slovakia} or \hyperref[fig:5b_LVA]{Latvia}. We find urban households to consume less carbon-intensively compared to rural households in a majority of countries such as \hyperref[fig:5b_FRA]{France}, \hyperref[fig:5b_NOR]{Norway} or \hyperref[fig:5b_ESP]{Spain}, and more carbon-intensively in countries such as \hyperref[fig:5b_MNG]{Mongolia}, \hyperref[fig:5b_PAK]{Pakistan} or \hyperref[fig:5b_ROU]{Romania}. For \hyperref[fig:5b_IND]{India}, where state residence accounts for adjusted feature importance of 8\%, we document that households in Jharkand, West Bengal or Manipur consume more carbon-intensively compared to households from other states. In \hyperref[fig:5b_BOL]{Bolivia}, where department residence accounts for adjusted feature importance of 6\%, households in Pando or Cochabamba consume more carbon-intensively than households from other departments. Differences in carbon intensity across space hint towards the important role of access to energy and transport infrastructure. In many cities, for example, households may choose from different transport modes including public transportation, which might help to explain lower carbon intensities in urban households in relatively richer countries. In poorer countries, however, living in urban areas may be associated with more carbon-intensive lifestyles, partially explained through enhanced access to electricity and formal fuels. This may explain more carbon-intensive consumption in urban households in Mongolia, Pakistan or Romania, where features describing energy access are missing in the data.

% Energy use
Information about energy use, such as main fuels used for cooking, lighting and heating, or electricity access and appliance ownership is the most important feature in eight of those countries, where such features are available. Main cooking fuel is an important feature in \hyperref[fig:5b_PER]{Peru} and \hyperref[fig:5b_NIC]{Nicaragua} with adjusted feature importance of 18\% and 14\%, respectively. In both countries, households cooking with LPG consume substantially more carbon-intensively than households cooking predominantly with firewood, a pattern that is consistent in all countries of our sample, in which a non-negligible share of households uses firewood or charcoal for cooking. This result is consistent with our assumption of zero direct emissions for biomass, firewood or charcoal, because of informal markets and structural impediments to regulate (and tax) emissions from such sources. Using kerosene for lighting is associated with higher carbon intensity compared to electricity and other lighting sources in \hyperref[fig:5b_UGA]{Uganda}, \hyperref[fig:5b_RWA]{Rwanda} or \hyperref[fig:5b_ETH]{Ethiopia} with adjusted feature importance of 9\% for Uganda and Rwanda. Information on heating fuels exists only in few countries, but is the single most important feature in \hyperref[fig:5b_TUR]{Turkey} and \hyperref[fig:5b_ARM]{Armenia}. Here, carbon intensity is higher in households heating with coal (\hyperref[fig:5b_TUR]{Turkey}) or natural gas (\hyperref[fig:5b_ARM]{Armenia}) compared to households heating with electricity. In other countries, such as \hyperref[fig:5b_GBR]{United Kingdom}, \hyperref[fig:5b_BRA]{Brazil}, \hyperref[fig:5b_AUT]{Austria} or \hyperref[fig:5b_URY]{Uruguay}, adjusted feature importance of heating fuels accumulates to not more than 3\%.

% Electricity access and appliance ownership
Overall, electricity access is less often an important feature, contributing a maximum of 4\% of adjusted feature importance in \hyperref[fig:5b_SEN]{Senegal}. In a majority of countries, feature importance for electricity access is low, possibly because of overall high (e.g. \hyperref[fig:5b_VNM]{Vietnam} or \hyperref[fig:5b_PHL]{Philippines}) or low electricity access rates (e.g. \hyperref[fig:5b_MWI]{Malawi} or \hyperref[fig:5b_LBR]{Liberia}, see table \ref{tab:A1}) or because of a low CO\textsubscript{2}-intensity in the electricity sector (e.g. \hyperref[fig:5b_ETH]{Ethiopia} or \hyperref[fig:5b_KEN]{Kenya}, see table \ref{tab:Electricity}).

% Appliance ownership
Instead, ownership of major household appliances (such as refrigerators, washing machines or air conditioning) is the most important feature in \hyperref[fig:5b_CHE]{Switzerland} and the \hyperref[fig:5b_PHL]{Philippines}, comprising 17\% of adjusted feature performance in the Philippines. This is less surprising, because appliance ownership is a more compelling, yet incomplete proxy for electricity \textit{use} compared to electricity \textit{access}.

% Sociodemographic features
Sociodemographic features, such as education, gender, self-identified ethnicity, nationality or religion of household head, are the most important feature in three countries. In \hyperref[fig:5b_PRT]{Portugal}, where education of household accounts for 29\% of model prediction, households with tertiary education exhibit a higher carbon intensity than households with primary or secondary education. Adjusted feature importance for gender of household head is highest in \hyperref[fig:5b_BEN]{Benin} or \hyperref[fig:5b_TGO]{Togo}, where households with female household heads are found to consume less carbon-intensively. In \hyperref[fig:5b_ISR]{Israel}, households identifying themselves as Muslim are found to consume more carbon-intensively compared to households identifying as Jewish. Households reporting to live a traditional, religious or orthodox lifestyle consume more carbon-intensively compared to secular households. For 71 out of 87 countries, sociodemographic features do not exceed 3\% of adjusted feature importance, indicating their relatively low relevance across countries for predicting differences in carbon intensity.

\clearpage
\begin{figure}[ht!]
    \centering
    \caption{Feature importance across countries by cluster}\label{fig:fig_4}
    \begin{subfigure}[b]{\textwidth}
    \centering
    \caption{Feature importance across countries of cluster A}\label{fig:fig_4_1}
    \includegraphics{1_Figures/Figure 4/Figure_4_Corrected_1.jpg}
     \begin{subcaption2}
    This figure shows the importance of features (in normalized average absolute SHAP-values) for each country, grouped by country clusters. Blue (red) colors indicate that a feature is relatively less (more) important in a country compared to all other countries and features. 'Sociodemographic' comprises features such as household size, gender, self-identified ethnicity, nationality, religion or language. 'Spatial' comprises features such as state, province, district and urban/rural-identifiers. For horizontal distribution, blue (red) colors indicate a lower (higher) heterogeneity within the poorest quintile compared to the richest quintile. For vertical distribution, blue (red) colors indicate lower (higher) median carbon intensity among the poorest quintile compared to the richest quintile. For average CO\textsubscript{2}-intensity, blue (red) colors indicate a lower (higher) average carbon intensity across all countries. For goodness of fit (R\textsuperscript{2}), blue (red) colors indicate a lower (higher) predictive performance compared to other countries. R\textsuperscript{2} is not explicitly included for clustering.
    We assign countries to 6 clusters performing k-means clustering based on scaled feature importance adjusted for model accuracy. We also show all values in table \ref{tab:A10}.
    \end{subcaption2}
    \end{subfigure}
\end{figure}
\clearpage

\clearpage
\begin{figure}[ht!]\ContinuedFloat
    \centering
    \begin{subfigure}[b]{\textwidth}
    \centering
    \caption{Feature importance across countries of clusters B to F}\label{fig:fig_4_2}
    \includegraphics{Figure 4/Figure_4_Corrected_2}
    \begin{subcaption2}
    This figure shows the importance of features (in normalized average absolute SHAP-values) for each country, grouped by country clusters. Blue (red) colors indicate that a feature is relatively less (more) important in a country compared to all other countries and features. 'Sociodemographic' comprises features such as household size, gender, self-identified ethnicity, nationality, religion or language. 'Spatial' comprises features such as state, province, district and urban/rural-identifiers. For horizontal distribution, blue (red) colors indicate a lower (higher) heterogeneity within the poorest quintile compared to the richest quintile. For vertical distribution, blue (red) colors indicate lower (higher) median carbon intensity among the poorest quintile compared to the richest quintile. For average CO\textsubscript{2}-intensity, blue (red) colors indicate a lower (higher) average carbon intensity across all countries. For goodness of fit (R\textsuperscript{2}), blue (red) colors indicate a lower (higher) predictive performance compared to other countries. R\textsuperscript{2} is not explicitly included for clustering.
    We assign countries to 6 clusters performing k-means clustering based on scaled feature importance adjusted for model accuracy. We also show all values in table \ref{tab:A10}.
    \end{subcaption2}
    \end{subfigure}
    
\end{figure}

\clearpage

\paragraph{Identifying country clusters}

% What are important features to assign countries to different clusters?
% Discuss country clusters in general and data issues, idiosyncracy. For each country cluster, one could recommend specific policy actions. Taken up again in discussion section.
% Discuss how well do countries fit into clusters?
Countries are comparable with respect to features that are important to predict differences in carbon intensity. Building on the adjusted importance of all features, the average carbon intensity of households and the vertical and horizontal distribution coefficient of countries, we identify six distinct clusters of countries. Within clusters, countries are more similar to each other compared to countries in other clusters. 

It is worth noting that not all countries fit well into their clusters, as demonstrated through an average silhouette width of 0.27 (see figures \subref{fig:G3_silhouette_2}, \subref{fig:G4_silhouette_2} and tables \ref{tab:A9} and \ref{tab:A10}). In cluster C, for example, silhouette width is negative for seven countries, which points to a larger heterogeneity between countries or, more generally, idiosyncratic patterns. Under such circumstances, it may be more difficult to draw conclusions about the distributional impacts of climate policy from the experience of other countries. 

% Introducing Figure 3
Clusters differ from each other in the importance of single features. In figure \ref{fig:fig_3}, we exhibit the relative importance of different features across clusters, after ordering clusters by cluster size. 

% Cluster A
The largest cluster A encompasses 46 countries (including \hyperref[fig:5b_USA]{USA}, \hyperref[fig:5b_CAN]{Canada}, \hyperref[fig:5b_BRA]{Brazil} or \hyperref[fig:5b_DEU]{Germany}). In countries of this cluster, our analysis yields more carbon-intensive consumption among relatively poorer households and a larger heterogeneity in carbon intensity among poorer households compared to richer households. In comparison to other clusters, most features contribute relatively little to explaining variation in carbon intensity. For example, average adjusted feature importance is 3\% for household expenditures, which is the most important feature across all 87 countries. More general, countries in cluster A have in common that it is difficult to predict carbon intensity of households with available data. One reason is that we adjust country-level feature importance for models' performance, which influences the identification of country clusters. For example, 18 out of 18 countries with relatively low model performance (R$^{2}<$10\%) appear in cluster A. In particular, data resolution may be insufficient and does not cover variables describing energy use (35 countries in this cluster lack information on major cooking fuels and 29 countries lack information on car ownership). Nevertheless, these results also allude to highly peculiar characteristics of heterogeneous carbon intensity with important implications for policy design, because attempts to compensate based on characteristics that are observable in our dataset (such as total household expenditures) will not be effective to compensate the households most affected by climate policy. This also holds true in countries in which more granular information is available, e.g. in Brazil, Colombia, Israel, Kenya or United Kingdom. Yet, impacts of climate policy can be large, as indicated by a more carbon-intensive consumption (0.74 kgCO\textsubscript{2}/USD on average) compared to countries in other clusters. 

% Cluster B
In contrast, cluster B includes 19 countries (such as \hyperref[fig:5b_BGD]{Bangladesh}, \hyperref[fig:5b_NGA]{Nigeria} or \hyperref[fig:5b_PAK]{Pakistan}) with comparably less carbon-intensive consumption (0.30 kgCO\textsubscript{2}/USD on average). In countries of this cluster, consumption is more carbon-intensive among richer households and we also find a larger heterogeneity among richer households compared to poorer households. Motorcycle ownership, main lighting fuel and sociodemographic characteristics are relatively more important compared to other clusters, while adjusted feature importance for household expenditures is 3.7\%, on average. It is striking to observe that 14 out 19 countries in our sample with lowest GDP per capita and 13 out of 17 countries of Sub-Sahara Africa belong to cluster B, even though such information was not used for clustering. Instead, clusters indicate heterogeneous patterns of energy use. One implication is that it may be inaccurate to infer about the distributional impacts of climate policy in one country from the experience of other countries, in which patterns of energy use may differ substantially.

% Cluster C
Cluster C comprises 13 countries (such as \hyperref[fig:5b_IND]{India}, \hyperref[fig:5b_IDN]{Indonesia}, \hyperref[fig:5b_MEX]{Mexico} and \hyperref[fig:5b_ZAF]{South Africa}) with comparably high average carbon intensity (1.05 kgCO\textsubscript{2}/USD on average). Within countries, differences in carbon intensity are comparably small between poorer and richer households, but richer households consume more carbon-intensively in all countries, but Vietnam. Countries have in common that spatial information, appliance and car ownership are comparably important features. Compared to other clusters, countries are less similar to each other, expressed through an average silhouette width of -0.01. 

% Cluster D
Cluster D consists of four countries in Latin America (\hyperref[fig:5b_BOL]{Bolivia}, \hyperref[fig:5b_ECU]{Ecuador}, \hyperref[fig:5b_PER]{Peru}, \hyperref[fig:5b_SLV]{El Salvador}) and \hyperref[fig:5b_IRQ]{Iraq}. Countries from this cluster can be characterized by a larger heterogeneity in carbon intensity between poorer households compared to richer households. Household expenditures and main cooking fuel stand out as important features compared to other clusters. 

% Cluster E
Countries of cluster E (\hyperref[fig:5b_ARM]{Armenia} and \hyperref[fig:5b_TUR]{Turkey}) differ from all other countries because the variation in the main heating fuel is comparably relevant for predicting carbon intensity. In contrast, households' main lighting fuel and spatial information are relatively important features in the two countries of cluster F (\hyperref[fig:5b_RWA]{Rwanda} and \hyperref[fig:5b_UGA]{Uganda}). Additionally, we observe that clusters E and F both include geographically neighbouring countries.

% Offer synthesis
While being stylized by nature our clustering approach helps emphasizing country-specific characteristics correlating with (and contributing to) heterogeneous impacts of climate policy on households. Importantly, we provide evidence for differences in household expenditures being less decisive for households' carbon intensity than often suggested. Features describing households' energy use can be helpful predictors in some countries (for example in clusters B, C, D, E and F), but not necessarily in every country. For example, main energy fuels contribute relatively little to prediction in countries of cluster A such as \hyperref[fig:5b_URY]{Uruguay}, \hyperref[fig:5b_URY]{Argentina} or \hyperref[fig:5b_CRI]{Costa Rica}, for which models' predictive power is comparably high.

\begin{figure}[ht!]
    \centering
    \includegraphics{1_Figures/Figure 3/Figure_3_Corrected.jpg}
    \caption{Average feature importance across country clusters}
    \label{fig:fig_3}
    \begin{subcaption2}
    This figure shows the average importance of features (in normalized absolute average SHAP-values) across all countries from each cluster A to F. Colors express the average importance of features in a cluster in comparison to other clusters. Blue (red) colors indicate that a feature is relatively less (more) important on average in a cluster compared to all other clusters. 'Sociodemographic' comprises normalized absolute average SHAP values for features such as education, gender, self-identified ethnicity, nationality, religion or language. 'Spatial' comprises normalized absolute SHAP-values for province, district and urban/rural-identifiers. For horizontal distribution, blue (red) colors indicate a lower (higher) heterogeneity within the poorest quintile compared to the richest quintile. For vertical distribution, blue (red) colors indicate lower (higher) median carbon intensity among the poorest quintile compared to the richest quintile. For average CO\textsubscript{2}-intensity, blue (red) colors indicate a lower (higher) average carbon intensity across clusters. For goodness of fit (R\textsuperscript{2}), blue (red) colors indicate a lower (higher) predictive performance compared to other countries. R\textsuperscript{2} is not explicitly included for clustering. We assign countries to 6 clusters performing k-means clustering. We also show all values in table \ref{tab:A9}.
    \end{subcaption2}
\end{figure}

\paragraph{Robustness check: Direction of effects}
Results from BRT-models can help understanding the contribution of individual features for predicting carbon intensity. Moreover, such model results also indicate the (non-linear) relationships between feature values and carbon intensity as visualized in partial dependence plots (figure \ref{fig:5b}). Here, we build on supplementary analyses based on a logit-model (see figure \ref{fig:Logit_ME} and equation \ref{eq:logit}) to discuss and corroborate our findings about the direction of effects and to allow for a more accessible comparison across countries.

In a majority of countries, increasing expenditures is associated with a lower probability of being a hardship-case with estimates being smaller than and statistically different from zero for 62 countries (figure \subref{fig:Logit_ME_exp}). Estimates are positive and significant ($p\leq 5\%$) for eight countries. In comparison, our analysis of vertical heterogeneity (figure \ref{fig:fig_2}) yields progressive results in 44 countries. Therefore, in some countries poorer households may be more prone to consuming more carbon-intensively than 80\% of the population, even though the distribution is overall regressive, which again supports our claim that a focus on vertical heterogeneity can be misleading.

Across countries, we document that owning (and using) a car (figure \subref{fig:Logit_ME_car}) or motorcycle (figure \subref{fig:Logit_ME_motorcycle}) is associated with a significant increase in the likelihood to consume more carbon-intensively than 80\% of the population\footnote{For car ownership, one exception is Ethiopia, where car ownership is associated with a \textit{decrease} of 14\% ($p=0.045$) in the probability to be in the most-carbon intensive quintile. Yet, our BRT-model yields non-adjusted feature importance of 0.1\% for car ownership in \hyperref[fig:5b_ETH]{Ethiopia}.}. In eight countries, our estimates show a significantly ($p\leq 5\%$) greater probability to be in the most carbon-intensive quintile for urban households, but a lower probability in 32 countries, compared to rural households (figure \subref{fig:Logit_ME_urban}).

While findings from logit-models are generally in line with our results from BRT-models, both models answer a slightly different question. In particular, models with binary dependent variables can be useful for the analysis of distributional impacts, because they help describing how parts of the population (e.g. the most carbon-intensive quintile) differs from other parts of the population\footnote{Models building on supervised machine learning are also well-suited for analyzing variation in a binary dependent variable, i.e. classification problems.}. For example, for Mexico we find that differences in cooking fuel use account for 2\% in adjusted feature importance for predicting carbon intensity, but cooking with coal instead of electricity is associated, on average, to a 44\% increased probability to consume more carbon-intensively than 80\% of the Mexican population. Under such circumstances, addressing the use of (specific) cooking fuels could need to be warranted even though adjusted feature importance is comparably low.

\section{Discussion: Unpacking the policy toolbox} \label{sec:discussion}

% State findings
Our findings provide evidence for country-specific household characteristics associated with higher levels of carbon intensity of consumption. These results can help in ex-ante assessments of climate policy to identify especially affected household profiles and thus promising means to compensate them. Both, consistently prevalent horizontal heterogeneity and the identification of different country-clusters lead to the notion that frequently proposed compensation options, such as lump-sum transfers, may not be effective in compensating the most carbon-intensive households in every country context. Most discourses around complementing compensation measures centers on industrialized countries, which pertain to cluster A and in which, as we show, household characteristics associated to carbon-intensive consumption can differ substantially from countries in other clusters.

% Qualification: Why our discussion does not give concrete policy advice, but rather a guardrail
We refrain from proposing specific compensation measures for specific countries and acknowledge that preferences for one measure over the other can be subject to normative considerations. Admittedly, compensation becomes more feasible, if climate policy centers on price-based interventions, such as carbon pricing or fossil fuel subsidy removal, thereby increasing fiscal space for reimbursing households. Moreover, a thorough comparison of alternative compensation measures should consider existing institutions, potentially constrained governmental capacity and limited information available to policymakers. In light of such qualifications, we examine which options in the policy toolbox could be more effective in supporting those households that would bear the highest additional costs, thereby reducing horizontal heterogeneity\footnote{Our analysis can also shed light on existing compensation measures for climate policy. Austria, for example, has introduced a carbon price in 2022. Revenues are distributed back to the population as a lump sum transfer. The size of the transfer is however differentiated across regions, with higher transfers paid in regions with lower transport and health infrastructure \autocite{BMK.2023}. Our analysis for \hyperref[fig:5b_AUT]{Austria} shows that total household expenditures and spatial features account on average for 35\% and 18\% of predicted values (R\textsuperscript{2}=0.19). Despite some remaining degree of unobserved heterogeneity and no explicit differentiation of transfers with respect to primary heating fuels and car ownership, the compensation measures put forward by the Austrian government are likely to reduce horizontal heterogeneity.

In contrast, Canada has introduced a nation-wide carbon tax in 2019 and channels proceedings back to households through quarterly tax returns. Canadian households from rural regions receive additional 10\% of transfers to account for higher dependence on fossil fuels for transportation \autocite{GovernmentofCanada.2023}. Recently, the Canadian government has announced to exempt heating oil from carbon pricing for three years as a means to reduce additional costs for poorer households in the Atlantic provinces, which are more likely to use oil for heating \autocite{Reuters.2023}. Our analysis for \hyperref[fig:5b_CAN]{Canada} shows that Canadian provinces account for 39\% of predicted values (R\textsuperscript{2}=0.17), but that households from Atlantic provinces are less likely to consume carbon-intensively compared to households from Saskatchewan and Ontario. Exempting heating oil from carbon pricing can be effective to ease the costs on carbon-intensively consuming households, if heating fuels, which are not included in our sample, contribute substantially to unexplained heterogeneity. Nevertheless, our analysis illustrates that Canadian provinces appear to be a poor proxy for heating fuels, implying that households from Atlantic provinces may perceive halting the carbon tax as less relieving than the government may presumably have expected.}

% Lump-sum transfers
Lump-sum transfers, potentially distributed equally per capita, are the showcase example of many economists. Indeed, such transfers would be applicable, if governments had a strong taste for reducing vertical inequality, avoiding regressive effects\footnote{In this case, \textcite{Stiglitz.2019} proposes sectorally differentiated regulation, depending on whether richer households disproportionately consume respective goods and services. This could for instance imply comparably stricter intervention in the aviation section, albeit with aggregate losses of efficiency.} and ensuring high salience of compensation \autocite{Chetty.2009}. In contrast, research calls for attention to relatively low public acceptance for such transfers and the 'equity-pollution-dilemma'\footnote{Hypothetically, reimbursing households in proportion to their costs would minimize distributional effects on aggregate, albeit partially setting off demand-side effects of the policy instrument because households would use (parts of) their reimbursement to consume more carbon-intensive products \autocite{Stiglitz.2019}.}\autocite{Sager.2019}.

Our analysis reveals that lump-sum transfers would be effective in reducing distributional impacts in countries, in which total household expenditures are an important feature and where disproportionately large costs would fall on relatively poorer households. Such transfers could prove comparatively effective in countries of cluster D, including Bolivia, Ecuador or Peru. % Horizontal heterogeneity is large in some cases, possibly reducing precision.

% Targeted transfers to low-income households OR proportional transfers
Many governments have established cash transfer programs targeted at low-income households. Seizing such existing institutions could be advantageous, even if targeting errors of cash transfer programs can be sizeable \autocite{Banerjee.2022} and associated with lower public acceptability \autocite{Bah.2019}. Our results indicate that transfers targeted to low-income households can be helpful where poorer households consume more carbon-intensively than richer households and where horizontal heterogeneity is comparably small or where household expenditures are not an important feature. In our sample, this applies to some countries from cluster A, e.g. Suriname, Morocco or Poland.

% Tax breaks - labor income tax
The discipline has also popularized reducing distortionary taxes to reap a 'double dividend' \autocite{Bovenberg.1996}. In addition, lowering income or consumption taxes provide a leverage to counteract vertical heterogeneity. For example, reducing the labor tax can be effective in compensation, if richer households consume more carbon-intensively and if household expenditures are an important feature. Indeed, in some countries of cluster B, e.g. Pakistan or Mozambique, cutting labor taxes may be useful for effective compensation while at the same time propelling formalization \autocite{Jessen.2021,Rocha.2018} and economic activity \autocite{Ulyssea.2018}.

% Tax breaks - consumption tax
Beyond benefits for aggregate efficiency, reducing excise taxes on consumption can be effective in countries where poorer households consume more carbon-intensively and where total expenditures are an important feature, e.g. in countries of cluster D. In addition, differentiated tax reductions (e.g. through VAT) could steer consumption towards less carbon-intensive products \autocite{Klenert.2023}. Reducing excise taxes on basic consumption goods including food or some forms of energy may reduce vertical heterogeneity, because poorer households typically spend a larger expenditure share on such goods\footnote{In case of more widespread informal consumption, however, reducing consumption taxes may be less progressive \autocite{Bachas.2020}.}. 

% Addressing energy use through exemption of transition fuels, infrastructure access
Uniform lump-sum transfers and (income or consumption) tax cuts would likely fall short of compensating the most carbon-intensive households in countries with large horizontal heterogeneity and low predictive power for total household expenditures. Under such circumstances, it may be important to enable access to low-carbon technologies, thereby increasing the price elasticity of households and facilitating households to consume less carbon-intensive goods and services. Where vehicle and appliance ownership are important features, lowering technological barriers can be effective, for example through incentives for energy efficiency improvements, improved public transport systems or investments in green mobility infrastructure. Such measures may prove helpful in countries of cluster C, such as Indonesia or Mexico. % large initial investments required in transport and heating sector
Main cooking fuel is an important feature in some countries of cluster B and D. Here, subsidizing clean cookstoves or subsidizing 'transition fuels' (such as LPG) may be effective. Addressing the heating sector through improvements in buildings can be helpful in Turkey or Armenia (cluster E), where main heating fuels are important, while exempting kerosene from regulation might be reasonable in Rwanda or Uganda (cluster F). % Sketch examples?

% Low predictive power is an important result, feature not a bug
One important concern for effective compensation emerges from low model accuracy, as identified for some countries. It implies that any transfer based on characteristics observable in our dataset will be ineffective to compensate the most carbon-intensive households. In some countries, households' carbon intensity is difficult to predict, which underpins the relevance of additional country- and policy-specific investigations, especially when governments face information problems \autocite{Mirrlees.1971}. In this case, cutting excise taxes on comparably carbon-intensive goods could contain increasing heterogeneity while preserving incentives for supply-side abatement \autocite{Goulder.2008}.

% Policy design - different policy instruments
Admittedly, addressing distributional impacts does not necessarily require considering different options for compensation. Instead, policymakers may also turn to different types of regulation. Our flexible analytical framework allows for investigating the heterogeneous impacts of policies with different sectoral or regional coverage. For example, figure \ref{fig:comparison_policies} compares vertical and horizontal distribution coefficients for a national climate policy instrument (as in figure \ref{fig:fig_2}) to an international climate policy instrument, which increases consumer prices in equivalence to embedded \textit{global} CO\textsubscript{2}-emissions. We also show such coefficients for policy instruments targeted at the transport and electricity sector, respectively. For example, policy instruments addressing international CO\textsubscript{2}-emissions  \autocite[such as border carbon adjustment, e.g][]{Mehling.2019,Cosbey.2019} would lead to more heterogeneous impacts in richer households in 23 countries, because richer households usually spend relatively more on imported goods and services. For transport sector policies, we document more carbon-intensive consumption among richer households compared to poorer households in 59 countries, while differences in horizontal heterogeneity exceed vertical differences in 77 countries. In contrast, electricity sector policies would likely affect poorer households more heavily in 62 countries with larger horizontal heterogeneity among poorer households in 64 countries\footnote{Those findings are in contrast to national climate policy instruments across all sectors: Here, consumption is more carbon-intensive in poorer households compared to richer households in 43 countries. Within-quintile heterogeneity is larger among poorer households in 58 countries and horizontal heterogeneity exceeds vertical heterogeneity in 66 countries. In essence, international climate policy instruments are likely to increase heterogeneity among richer households compared to poorer households; Transport sector policies would lead to more progressive results, but also larger horizontal heterogeneity in general; Electricity sector policies would lead to more regressive results with larger heterogeneity among poorer households.}. While we refrain from investigating the importance of household characteristics for predicting outcomes of such policies for now, our results underpin that distributional impacts of climate policy are country- and policy-specific. Presuming that distributional impacts are of concern, some policies may be preferred over others irrespective of possible compensation measures, although with repercussions for aggregate efficiency and revenue collection. 

% Design of non-price-based instruments
The interpretation of our findings is comparably straight-forward for price-based policies. Nevertheless, our approach can also inform the design of standards, mandates or subsidies, depending on how such policies affect the marginal costs of CO\textsubscript{2}-emissions. Distributional impacts may be less salient for such instruments, but potential compensation would also become more difficult to finance because of foregone revenues. %Non-pricing interventions can be helpful, in case they change preferences or address technological path-dependencies.

% Methodological considerations
Our analysis is mainly informative and can only provide the foundation for more comprehensive analyses exploiting more nuanced data. Such additional investigations may explicitly address inaccuracies in our modelling approach, including uncertainties about supply-side pass-through of cost increases, technological path dependencies and informational frictions. Admittedly, our work also remains calm about heterogeneous impacts of climate policy with respect to potential co-benefits \autocite[e.g.][]{Holland.2019,Karlsson.2020}, co-costs \autocite[e.g.][]{Fuje.2019,Greve.2022}, wealth \autocite[e.g.][]{Fullerton.2011} and labor \autocite[e.g.][]{Castellanos.2024}. Instead, this study provides information about the first-order distributional impacts of climate policy on consumption costs, which can however be meaningful to identify a potential demand for compensation. Clustering countries according to how the costs of climate policy distribute across the population demonstrates that some compensation measures would work more effectively in some countries than in others, potentially restricting leeway for cross-country learning. % No impacts, no adapation, no investments. Labour may require additional compensation measures, such as subsidzing low-skilled labour in carbon-intensive sectors.

% Further research: Value chain of climate policy implementation. Welfare function of policymakers or winning coalition not clear.
The distributional impacts of welfare-enhancing policy proposals do not only matter for welfare analyses, but also for understanding the political economy of climate policy. While this study provides a comprehensive assessment of such distributional impacts for climate policy, it is less clear how the distribution of costs is \textit{perceived} by the public. Often, it is argued that people prefer progressive outcomes out of justice considerations, but large horizontal heterogeneity, subjective beliefs \autocite{Douenne.2020} and scattered perceptions of fairness \autocite{MaestreAndres.2019,Povitkina.2021} cast doubt on this assumption. Future research could contribute to better understand how the (expected) distribution of costs shapes public acceptability of climate policy. Likewise, some policy instruments and complementary compensation measures may be more acceptable to the population than others, but research providing theory and empirical evidence remains scarce \autocite[e.g.][]{Sommer.2022,Valencia.2023}, at least in comparison to the literature quantifying distributional impacts. 

\section{Conclusion} \label{sec:conclusion}

% Methodological contribution
This study is the first to provide a detailed analysis of heterogeneous impacts of climate policy on households across a wide range of countries. Our flexible framework integrating multi-regional input-output data with detailed household expenditure data allows for analyzing country- and policy-specific impacts.

% Core contribution: Horizontal inequality is neglected.
Our results show that differences in total household expenditures can be important for explaining differences in policy outcomes. Nevertheless, solely focusing on such differences misses important parts of the picture. Rather, horizontal heterogeneity outweighs vertical heterogeneity and models building on household expenditures only exhibit comparably low accuracy. Analyzing heterogeneous outcomes of climate policy requires to factor in further, often-neglected household characteristics, such as information on energy use, vehicle and appliance ownership, space or sociodemography.

% Empirical analysis - what we show instead.
For each country, we quantify the contribution of single features and show that their relative importance varies in comparison to other countries. Seizing k-means clustering, we identify six country clusters with comparable distributional characteristics. % Clusters show remarkable regional homogeneity, possibly allowing for investigation of effective compensation measures beyond specific country contexts. 
Lastly, we identify complementary compensation policies that may help governments to circumvent unintended distributional effects of climate policy, which can be an important prerequisite for efficient, yet politically acceptable climate change mitigation. 

\clearpage

\begin{refcontext}[sorting=nyt]
\printbibliography
\end{refcontext}

\clearpage

\appendix 
\begin{refsection}
{\Huge Appendix} \label{sec:appendix}

Global heterogeneity in carbon intensity of consumption

\clearpage
\section{Data cleaning} \label{sec:cleaning}

We describe our approach to collecting, cleaning and harmonizing microdata and to feature engineering for machine learning modeling.

\subsection{Collecting household data}

We collect household budget survey data and extract several information before cleaning and harmonizing. Household budget survey data are often publicly available, but sometimes subject to a considerable fee. Table \ref{tab:datasets} provides publishing organizations, names of surveys and links to datasets used in this study.

\begin{itemize}
    \item For each household, we include sociodemographic information about household members where available. In all survey, households are represented through 'household heads', i.e., persons who often contribute the largest share of household income or are responsible for purchase decisions. We use information on the 'household head' as a proxy for the entire household and collect information on education, occupation, gender, self-identified ethnicity, nationality or religion of the 'household head'. We standardize information on education by using the International Standard Classification of Education (ISCED) to facilitate comparison across countries.
    \item We include spatial information where available, for example identifier for sub-national areas (provinces or states), sub-sub-national areas (districts) or villages. Often, surveys include an indicator for whether households live in urban or rural areas. Definitions of \textit{urban} and \textit{rural} may not be consistent across countries, but certainly within countries.
    \item We include information on energy use, such as on primary fuels used for cooking, lighting and heating. We harmonize information on fuels across countries to account for different names and levels of detail across countries. For example, cooking fuels include charcoal, coal, electricity, firewood, gas, kerosene, liquid fuel, LPG, other biomass or unknown fuels.
    \item We capture information on electricity access and create a binary variable indicating if households have access to electricity through electricity grids, but also through generators or solar panels. 
    \item We collect information about ownership of major transport vehicles (such as cars, motorcycles and trucks) and major household appliances (such as refrigerators, air conditioning, washing machines and television). For each country, we only include information about ownership, not about the precise number of owned vehicles and appliances to improve consistency across countries.
    \item We collect all available information on household-level expenditures, integrating information from household-level and individual-level diary entries. We do not include consumption information from home production, received as gifts or as remuneration for labor. Our rationale is that it would be difficult for a climate policy instrument to cover self-produced goods and services that are not purchased on markets. We include all expenditures on the item-level and extrapolate expenditures to yearly values. Often, households track expenditures over the course of a few weeks, but provide details on less frequent purchases in the past months or year. For more frequently purchased items, mostly food, this approach neglects seasonal consumption patterns, but resulting bias should be sufficiently small, since households are surveyed throughout the year.
    \item We do not include imputed expenditures, e.g., for hypothetical rental payments, since including them would inaccurately overestimate total household expenditures and bias expenditure shares towards less carbon-intensive services.
\end{itemize}

Code written for each country-level dataset can be found in a stable online repository (see \ref{code}).

\subsection{Cleaning and harmonizing household data}

Building on collected microdata from household budget surveys we perform several cleaning steps in order to harmonize datasets across countries as far as possible. Conditional on country-level data availability, we ensure that we clean all available data consistently. 

\begin{itemize}
    \item We remove households from the sample with missing information for key variables such as household size, sampling weights or total expenditures.
    \item We code and treat missing information about other variables as missing or 'unknown' and remove variables for each country, if missing information are dominant.
    \item For each country, we address outliers of household expenditures at the item-level. We consider any observation an outlier if it is in the 99\textsuperscript{th} percentile of all non-zero expenditures. We replace this observation with item-level median expenditures, thereby assuming that expenditure shares on such items are non-zero, but absolute values might have been exaggerated because of misreporting.
    \item We remove observations if expenditures are negative, for example, because households sell items.
    \item We remove duplicates from our sample. We check separately for duplicates at the level of household-level information and at the level of item expenditures: We consider households spending the same amount of money on the same items duplicates.
    \item We remove all households from our sample if aggregate expenditures exceed mean expenditures by five standard deviations ($z>5$).
    \item We use inflation rates from \textcite{IMF.2020} and exchange rates from the \textcite{WorldBankGroup.2023} to convert all local currencies to USD for the year 2017. Expenditures from surveys conducted before 2017 are inflated; expenditures from surveys after 2017 are deflated. This ensures consistency with calculated sectoral CO\textsubscript{2}-intensities as they refer to the year 2017. This approach does however neglect that expenditure shares may change with rising incomes and inflation.
    \item We create matching tables to assign country-level expenditure items to 65 aggregate sectors and to four broad expenditure categories (energy, food, goods, services). Items, that are difficult to match to a specific sector or to a specific category, e.g., 'other expenses', are matched to artificial sectors and categories labeled 'other'. Admittedly, we assume a carbon intensity of zero for such items in absence of more detailed information, but expenditure shares are generally low (0.7\% on average across country-level averages). 
    \item We delete observations for items indicating aggregate categories, if this would lead to double-counting of single expenditures. We delete observations for items indicating taxes (e.g. 'property tax'), since including them would prove inaccurate to calculate expenditure shares and because items indicating tax payments are not available in each country.
    \item We match items pertaining to fuels such as firewood, charcoal and other biomass to the sector \textit{lumber} to account for indirect emissions attributable to production, transportation and retail of these goods. However, we treat direct CO\textsubscript{2}-emissions of such fuels as zero, in line with assumptions by the IPCC \autocite{Grad.2023}, but also because direct emissions of such fuels are often difficult to regulate.
    \item We also identify items indicating energy use and create separate columns listing expenditures for different energy items, such as electricity, gasoline, diesel, kerosene, LPG, natural gas, charcoal, hard coal, firewood and other biomass. All matching tables are available through a separate stable online repository (see \ref{code}).
\end{itemize}

This procedure helps ensuring that sectoral expenditure shares are comparable across countries, albeit not all surveys including information on the same number and detail of consumption items. We proceed with assigning households to expenditure quintiles based on total household expenditures per capita to account for differing expenditure shares in larger households. We use expenditure quintiles for the analysing in figures \ref{fig:fig_1} and \ref{fig:Quint}.

Tables \ref{tab:A1}, \ref{tab:A2}, \ref{tab:A3}, \ref{tab:A4_CF}, \ref{tab:A5_LF} and \ref{tab:A6} show summary statistics for our final harmonized dataset, grouped by country and by country and expenditure quintile in tables \ref{tab:A2} and \ref{tab:A3}. 

% tracking and documenting removals

\subsection{Feature engineering} \label{sec:featureengineering}

Building on our harmonized dataset, we perform feature engineering on our variables (features) with the \texttt{R}-package \texttt{recipes} before performing analyses with BRT.

\begin{itemize}
    \item We exclude any feature with missing variation (for four countries).
    \item We exclude categorical feature with extremely high granularity (such as district-level identifiers).
    \item We exclude any feature with missing values.
    \item We remove the minimum number of features necessary to avoid high levels of correlation ($r>0.9$) between all features.
    \item We code observations as "other" for each feature (except province-level, district-level and urban-/rural-identifiers) that account for less than 5\% of all observations.
    \item All country-level feature sets include total household expenditures (in USD 2017) and household size. The minimum number of included features (including binary, categorical and continuous features) is 4 (for Sweden) and the maximum number of included feature is 17 (for Benin, Burkina Faso, Côte d'Ivoire, Guinea-Bissau, Senegal, Togo).
\end{itemize}

\subsection{Policy simulation}\label{sec:policysimulation}

We show that heterogeneity in household-level carbon intensity of consumption is equivalent to heterogeneity in household-level costs of climate policy, assuming that such instruments increase marginal costs of emitting CO\textsubscript{2} and that producers pass-on such costs to consumers. This analysis thus disregards general-equilibrium-effects on both the supply- and the demand-side. In general, any climate policy instrument is conceivable that leads to increasing costs in equivalence to embedded direct and indirect emissions, including (but not limited to) carbon pricing, fossil fuel subsidy removal or subsidies for low-carbon fuels.

The carbon intensity of consumption $e_{i}$ consists of sectoral carbon intensities and household-level sectoral expenditure shares as shown in equation \ref{eq:ei}. 

For example, consider the case of carbon pricing, which can be thought as a tax $\tau$ in USD/tCO\textsubscript{2}. The total absolute costs from carbon pricing equals direct and indirect carbon emissions embedded in household consumption E$_{i}$ multiplied with $\tau$. Computing total relative costs CPI$_{i}$ requires division by total household expenditures C$_{i}$:

\begin{equation}
    CPI_{i} = \frac{E_{i}*\tau}{C_{i}}
\end{equation}

Relative additional costs, i.e., the carbon pricing incidence CPI$_{i}$ can be expressed in \% ($\frac{USD_{\tau}}{USD_{i}}$). CPI$_{i}$ is equivalent to our expression for carbon intensity of consumption $e_{i}$, scaled by a proportional factor $\tau$. If $e_{A}=2*e_{B}$, then $CPI_{A}=2*CPI_{B}$, assuming that $e_{A}$ and $e_{B}$ express the carbon intensity covering all nationally released CO\textsubscript{2}-emissions for households A and B and that CPI$_{A}$ and CPI$_{B}$ refer to the relative carbon pricing incidence for a carbon price levied on all nationally released emissions in households \textit{A} and \textit{B}, respectively. In essence, heterogeneity in carbon intensity is equivalent to heterogeneity in household-level costs of climate policy instruments, under assumptions about how such instruments affect the marginal costs of emitting CO\textsubscript{2}.

In general, our modelling framework also allows for the simulation of other (sectoral) policies. Consider a carbon-tax-equivalent policy intervention in a specific sector, e.g., in the transport sector, here denoted as $\tau_{s^{*}}$. Such a sector-specific tax would cover all direct and indirect emissions released in this sector $s^{*}$, but not emissions released in other sectors. Nevertheless, customer prices of goods and services from sectors other than transport would still increase because of embedded emissions from the transport sector.

Calculating additional sets of sectoral carbon intensities $e_{s^{*}}$ including direct and indirect emissions of different sectors can help to simulate the impact of sectoral policies. Effectively, we only include direct and indirect CO\textsubscript{2}-emissions released in sectors $s^{*}$.

It is also possible to investigate the distribution of regional policies, for example of carbon border taxes covering CO\textsubscript{2}-emissions for imported goods and services. 

Supplementary figure \ref{fig:comparison_policies} shows vertical and horizontal distribution coefficients for the national carbon intensity in all sectors, in the transport sector, in the electricity sector and for the international carbon intensity in all sectors. 

\clearpage

\renewcommand\thefigure{\thesection.\arabic{figure}}
\renewcommand\thetable{\thesection.\arabic{table}}
\setcounter{figure}{0}
\setcounter{table}{0}

\section{Supplementary figures} \label{sec:figures}

% Engel-curves
\begin{figure}[ht!]
  \centering
  \caption{Engel curves: expenditure shares over total household expenditures} \label{fig:Engel}
  \begin{subfigure}[b]{\textwidth}
  \centering
  \includegraphics{1_Figures/Analysis_Parametric_Engel_Curves/Parametric_EC_0_A.pdf}
  \caption{Engel curves: expenditure shares over total household expenditures - Part A} \label{fig:Engel_1}
  \begin{subcaption2}
    This figure shows fitted lines for parametric and quadratic Engel curves for each consumption category in 30 countries of our sample. Black vertical lines indicate average household expenditures per capita for each expenditure quintile and country. Grey bars and secondary y-axis indicate the distribution of households.
  \end{subcaption2}
  \end{subfigure}
\end{figure}

\clearpage

\begin{figure}[ht!]\ContinuedFloat
   \begin{subfigure}[b]{\textwidth}
  \centering
  \includegraphics{1_Figures/Analysis_Parametric_Engel_Curves/Parametric_EC_0_B.pdf}
  \caption{Engel curves: expenditure shares over total household expenditures - Part B} \label{fig:Engel_2}
  \begin{subcaption2}
    This figure shows fitted lines for parametric and quadratic Engel curves for each consumption category in 30 countries of our sample. Black vertical lines indicate average household expenditures per capita for each expenditure quintile and country. Grey bars and secondary y-axis indicate the distribution of households.
  \end{subcaption2}
\end{subfigure}
\end{figure}

\clearpage

\begin{figure}[ht!]\ContinuedFloat
   \begin{subfigure}[b]{\textwidth}
  \centering
  \includegraphics{1_Figures/Analysis_Parametric_Engel_Curves/Parametric_EC_0_C.pdf}
  \caption{Engel curves: expenditure shares over total household expenditures - Part C} \label{fig:Engel_3}
  \begin{subcaption2}
    This figure shows fitted lines for parametric and quadratic Engel curves for each consumption category in 27 countries of our sample. Black vertical lines indicate average household expenditures per capita for each expenditure quintile and country. Grey bars and secondary y-axis indicate the distribution of households.
  \end{subcaption2}
\end{subfigure}
\end{figure}



\clearpage

% Carbon intensities
\begin{figure}[ht!]
  \centering
  \caption{Sectoral carbon intensities from GTAP} \label{fig:GTAP}\label{fig:B}
  \begin{subfigure}[b]{\textwidth}
  \centering
  \caption{Sectoral carbon intensities from GTAP - Part A} \label{fig:B1}  \includegraphics{Analysis_Carbon_Intensities_GTAP/Figure_2.1.1_A_2017B}
  \begin{subcaption2}
    This figure displays sectoral carbon intensities in kgCO$_{2}$ per USD of output for 16 sectors. We plot sectoral carbon intensities if household budget surveys in respective countries include consumption items which correspond to each sector. See our online repository for all country- and sector-level carbon intensities. We include labels with country codes if sector outputs are relatively carbon-intensive compared to other countries. Note that sectors \textit{other mining extraction (oxt)} and \textit{extraction of crude petroleum (oil)} are not matched to any item in any country.
  \end{subcaption2}
\end{subfigure}
\end{figure}

\clearpage

\begin{figure}[ht!]\ContinuedFloat
\begin{subfigure}[b]{\textwidth}
  \centering
  \caption{Sectoral carbon intensities from GTAP - Part B} \label{fig:B2}  \includegraphics{Analysis_Carbon_Intensities_GTAP/Figure_2.1.1_B_2017B}
  \begin{subcaption2}
    This figure displays sectoral carbon intensities in kgCO$_{2}$ per USD of output for 16 sectors. We plot sectoral carbon intensities if household budget surveys in respective countries include consumption items which correspond to each sector. See our online repository for all country- and sector-level carbon intensities. We include labels with country codes if sector outputs are relatively carbon-intensive compared to other countries. Note that sectors \textit{other mining extraction (oxt)} and \textit{extraction of crude petroleum (oil)} are not matched to any item in any country.
  \end{subcaption2}
\end{subfigure}
\end{figure}

\clearpage

\begin{figure}[ht!]\ContinuedFloat
\begin{subfigure}[b]{\textwidth}
  \centering
  \caption{Sectoral carbon intensities from GTAP - Part C} \label{fig:B3}  \includegraphics{Analysis_Carbon_Intensities_GTAP/Figure_2.1.1_C_2017B}
  \begin{subcaption2}
    This figure displays sectoral carbon intensities in kgCO$_{2}$ per USD of output for 16 sectors. We plot sectoral carbon intensities if household budget surveys in respective countries include consumption items which correspond to each sector. See our online repository for all country- and sector-level carbon intensities. We include labels with country codes if sector outputs are relatively carbon-intensive compared to other countries. Note that sectors \textit{other mining extraction (oxt)} and \textit{extraction of crude petroleum (oil)} are not matched to any item in any country.
  \end{subcaption2}
\end{subfigure}
\end{figure}

\clearpage

\begin{figure}[ht!]\ContinuedFloat
\begin{subfigure}[b]{\textwidth}
  \centering
  \caption{Sectoral carbon intensities from GTAP - Part D} \label{fig:B4}  \includegraphics{Analysis_Carbon_Intensities_GTAP/Figure_2.1.1_D_2017B}
  \begin{subcaption2}
    This figure displays sectoral carbon intensities in kgCO$_{2}$ per USD of output for 14 sectors. We plot sectoral carbon intensities if household budget surveys in respective countries include consumption items which correspond to each sector. See our online repository for all country- and sector-level carbon intensities. We include labels with country codes if sector outputs are relatively carbon-intensive compared to other countries. Note that sectors \textit{other mining extraction (oxt)} and \textit{extraction of crude petroleum (oil)} are not matched to any item in any country.
  \end{subcaption2}
\end{subfigure}
\end{figure}

\clearpage
\clearpage

% Boxplots over quintiles
\begin{figure}[ht!]
  \centering
  \caption{Distribution of carbon intensities over expenditure quintiles} \label{fig:Quint}
  \begin{subfigure}[b]{\textwidth}
  \centering
    \caption{Distribution of carbon intensities over expenditure quintiles - Part A} \label{fig:Quint_A}
  \includegraphics{1_Figures/Figures_Appendix/Figure_1_2017_Appendix_1.pdf}
  \begin{subcaption2}
    This figure displays the distribution of carbon intensity of consumption in kgCO$_{2}$/USD (x-axis) over expenditure quintiles (y-axis) for 30 countries. The first expenditure quintile comprises those 20\% of all households with least total expenditures per capita. The fifth expenditure quintile comprises those 20\% of all households with largest expenditures per capita. Within quintiles, boxes display the 25\textsuperscript{th} and the 75\textsuperscript{th} percentile; whiskers display the 5\textsuperscript{th} and 95\textsuperscript{th} percentile; rhombuses indicate the within-quintile average. Vertical coloured bands indicate the difference between the highest and the lowest quintile-level median carbon intensity of consumption. Blue bands indicate higher carbon intensities among richer households; red bands indicate higher carbon intensities among poorer households. See also Tables \ref{tab:A3} and \ref{tab:A7}.
  \end{subcaption2}
  \end{subfigure}
\end{figure}

\clearpage

\begin{figure}[ht!]\ContinuedFloat
   \begin{subfigure}[b]{\textwidth}
  \centering
      \caption{Distribution of carbon intensities over expenditure quintiles - Part B} \label{fig:Quint_B}
  \includegraphics{1_Figures/Figures_Appendix/Figure_1_2017_Appendix_2.pdf}
  \begin{subcaption2}
    This figure displays the distribution of carbon intensity of consumption in kgCO$_{2}$/USD (x-axis) over expenditure quintiles (y-axis) for 30 countries. The first expenditure quintile comprises those 20\% of all households with least total expenditures per capita. The fifth expenditure quintile comprises those 20\% of all households with largest expenditures per capita. Within quintiles, boxes display the 25\textsuperscript{th} and the 75\textsuperscript{th} percentile; whiskers display the 5\textsuperscript{th} and 95\textsuperscript{th} percentile; rhombuses indicate the within-quintile average. Vertical coloured bands indicate the difference between the highest and the lowest quintile-level median carbon intensity of consumption. Blue bands indicate higher carbon intensities among richer households; red bands indicate higher carbon intensities among poorer households. See also Tables \ref{tab:A3} and \ref{tab:A7}.
  \end{subcaption2}
\end{subfigure}
\end{figure}

\clearpage

\begin{figure}[ht!]\ContinuedFloat
   \begin{subfigure}[b]{\textwidth}
  \centering
    \caption{Distribution of carbon intensities over expenditure quintiles - Part C} \label{fig:Quint_C}
  \includegraphics{1_Figures/Figures_Appendix/Figure_1_2017_Appendix_3.pdf}
  \begin{subcaption2}
    This figure displays the distribution of carbon intensity of consumption in kgCO$_{2}$/USD (x-axis) over expenditure quintiles (y-axis) for 27 countries. The first expenditure quintile comprises those 20\% of all households with least total expenditures per capita. The fifth expenditure quintile comprises those 20\% of all households with largest expenditures per capita. Within quintiles, boxes display the 25\textsuperscript{th} and the 75\textsuperscript{th} percentile; whiskers display the 5\textsuperscript{th} and 95\textsuperscript{th} percentile; rhombuses indicate the within-quintile average. Vertical coloured bands indicate the difference between the highest and the lowest quintile-level median carbon intensity of consumption. Blue bands indicate higher carbon intensities among richer households; red bands indicate higher carbon intensities among poorer households. See also Tables \ref{tab:A3} and \ref{tab:A7}.
  \end{subcaption2}
\end{subfigure}
\end{figure}
\clearpage

% Silhouette plots
\begin{figure}[ht!]
\centering
  \caption{Silhouette analysis}\label{fig:Silhouette}
   \begin{subfigure}[b]{\textwidth}
   \centering
   \includegraphics{Figures_Appendix/Figure_Silhouette_2.pdf}
   \caption{Average silhouette width for different numbers of clusters \textit{k}} \label{fig:G3_silhouette_2}
   \begin{subcaption2}
     This figure displays the average silhouette width across all clusters for different numbers of clusters \textit{k}. We perform k-means clustering on a dataset with 88 country-level observations. Observations include information on \textit{adjusted} feature importance, i.e., we adjust feature importance for country-level model performance. We also include information about the vertical distribution. Vertical line and red point indicate the number of clusters that maximizes average silhouette width across all number of clusters with $k>2$.
   \end{subcaption2}
   \end{subfigure}
 \end{figure}
 \clearpage

 \begin{figure}[ht!]\ContinuedFloat
   \centering
   \begin{subfigure}[b]{\textwidth}
   \centering
   \includegraphics{Figures_Appendix/Figure_Silhouette_Clusters_2.pdf}
   \caption{Average silhouette width for each country per cluster \textit{k}} \label{fig:G4_silhouette_2}
   \begin{subcaption2}
     This figure displays the silhouette for each country for six clusters. We perform k-means clustering on a dataset with 88 country-level observations. Observations include information on \textit{adjusted} feature importance, i.e. we adjust feature importance for country-level model performance. We also include information about the vertical distribution. We order observations (y-axis) by clusters with most observations and by silhouette width. Silhouette width expresses how well each observation fits in its cluster, also in comparison to the observations from the least distant, but different cluster.
   \end{subcaption2}
   \end{subfigure}
 \end{figure}
 \clearpage

\begin{figure}[ht!]\ContinuedFloat
   \centering
   \begin{subfigure}[b]{\textwidth}
   \centering
   \includegraphics{Figures_Appendix/Figure_Silhouette_1.pdf}
   \caption{Average silhouette width for different numbers of clusters \textit{k}} \label{fig:G1_silhouette}
   \begin{subcaption2}
     This figure displays the average silhouette width across all clusters for different numbers of clusters \textit{k}. We perform k-means clustering on a dataset with 88 country-level observations. Observations include information on feature importance and the vertical distribution. In contrast to Figure \subref{fig:G3_silhouette_2}, we do not adjust feature importance for country-level model performance. Vertical line and red point indicate the number of clusters that maximizes average silhouette width across all clusters with $k>2$.
   \end{subcaption2}
   \end{subfigure}
 \end{figure}

 \clearpage

 \begin{figure}[ht!]\ContinuedFloat
   \centering
   \begin{subfigure}[b]{\textwidth}
   \centering
   \includegraphics{Figures_Appendix/Figure_Silhouette_Clusters_1.pdf}
   \caption{Average silhouette width for each country per cluster \textit{k}} \label{fig:G2_silhouette}
   \begin{subcaption2}
     This figure displays the silhouette for each country for 11 clusters. We perform k-means clustering on a dataset with 88 country-level observations. Observations include information on feature importance and the vertical distribution. In contrast to Figure \subref{fig:G2_silhouette}, we do not adjust feature importance for country-level model performance. We order observations (y-axis) by clusters with most observations and by silhouette width. Silhouette width expresses how well each observation fits in its cluster, also in comparison to the observations from the least distant, but different cluster.
   \end{subcaption2}
   \end{subfigure}
 \end{figure}
 \clearpage

 \begin{figure}[ht!]\ContinuedFloat
   \centering
   \begin{subfigure}[b]{\textwidth}
   \centering
   \includegraphics{Figures_Appendix/Figure_Silhouette_3.pdf}
   \caption{Average silhouette width for different numbers of clusters \textit{k}} \label{fig:G1_silhouette_3}
   \begin{subcaption2}
     This figure displays the average silhouette width across all clusters for different numbers of clusters \textit{k}. We perform k-means clustering on a dataset with 88 country-level observations. Observations include information on feature importance and the vertical distribution. In contrast to Figure \subref{fig:G3_silhouette_2}, we impute missing values for unobserved features with the average feature importance for each feature. We adjust feature importance for country-level model performance. Vertical line and red point indicate the number of clusters that maximizes average silhouette width across all clusters with $k>2$.
   \end{subcaption2}
   \end{subfigure}
 \end{figure}

 \clearpage

 \begin{figure}[ht!]\ContinuedFloat
   \centering
   \begin{subfigure}[b]{\textwidth}
   \centering
   \includegraphics{Figures_Appendix/Figure_Silhouette_Clusters_3.pdf}
   \caption{Average silhouette width for each country per cluster \textit{k}} \label{fig:G2_silhouette_3}
   \begin{subcaption2}
     This figure displays the silhouette for each country for 9 clusters. We perform k-means clustering on a dataset with 88 country-level observations. Observations include information feature importance and the vertical distribution. In contrast to Figure \subref{fig:G2_silhouette}, we impute missing values for unobserved features with the average feature importance for each feature. We adjust feature importance for country-level model performance.  We order observations (y-axis) by clusters with most observations and by silhouette width. Silhouette width expresses how well each observation fits in its cluster, also in comparison to the observations from the least distant, but different cluster.
   \end{subcaption2}
   \end{subfigure}
 \end{figure}
 \clearpage
\clearpage

% Comparison of R2
\begin{figure}[ht!]
    \centering
    \caption{Goodness of fit (R\textsuperscript{2}) for rich and sparse BRT-models}\label{fig:comparison}
    \includegraphics[width=\textwidth]{1_Figures/Figures_Appendix/Figure_Comparison_Models.jpg}
    \label{fig:comparison_models}
    \begin{subcaption2}
    This figure shows goodness of fit (R\textsuperscript{2}) for sparse and rich boosted regression tree models. The sparse models include household expenditures as feature (blue point, 'Household expenditures') and the rich models include all available features (red point, 'All features'), including household expenditures. We tune hyperparameters for each country and set of features and use five-fold cross-validation for evaluating model performance. See also table \ref{tab:A8} for country-level MAE and RMSE.
    \end{subcaption2}
\end{figure}

\clearpage

% SHAP-plots
\begin{figure}[ht!]
     \centering
    \caption{Partial dependence plot (SHAP) for 87 countries and nine clusters}
    \label{fig:5b}
     \begin{subfigure}[b]{\textwidth}
         \centering
         \caption{Partial dependence plot (SHAP) for Estonia (cluster A)}
         \label{fig:5b_EST}
         \includegraphics[width=\textwidth]{Figure 5b/Figure_5b_EST}      
     \end{subfigure}
    \\
    \vspace{0.5cm}
     \begin{subfigure}[b]{\textwidth}
         \centering
         \caption{Partial dependence plot (SHAP) for Bulgaria (cluster A)}
         \label{fig:5b_BGR}
         \includegraphics[width=\textwidth]{Figure 5b/Figure_5b_BGR}
     \end{subfigure}
    \\
    \vspace{0.5cm}
     \begin{subfigure}[b]{1\textwidth}
         \centering
         \caption{Partial dependence plot (SHAP) for Hungary (cluster A)}
         \label{fig:5b_HUN}
         \includegraphics[width=\textwidth]{Figure 5b/Figure_5b_HUN}
     \end{subfigure}
     \\
     \vspace{0.5cm}
        \begin{subcaption2}
     This figure shows SHAP-values for predicting carbon intensity over feature values for 87 countries in order of nine country-clusters and silhouette width. The bar chart displays normalized average absolute SHAP-values for all features. Features with less than 3\% of normalized SHAP-values are subsumed as "Other features (Sum)". Charts show SHAP-values over total household expenditures for all countries and for the three most important features in each country besides total household expenditures. Colors represent household expenditures with blue (red) colors indicating lower (higher) household expenditures.
     \end{subcaption2}
     \end{figure}
     
\begin{figure}[ht!]\ContinuedFloat
    \centering
   \begin{subfigure}[b]{\textwidth}
         \centering
         \caption{Partial dependence plot (SHAP) for Suriname (cluster A)}
         \label{fig:5b_SUR}
         \includegraphics[width=\textwidth]{Figure 5b/Figure_5b_SUR}         
     \end{subfigure}
    \\
    \vspace{0.5cm}
   \begin{subfigure}[b]{\textwidth}
         \centering
         \caption{Partial dependence plot (SHAP) for Morocco (cluster A)}
         \label{fig:5b_MAR}
         \includegraphics[width=\textwidth]{Figure 5b/Figure_5b_MAR}         
     \end{subfigure}
    \\
    \vspace{0.5cm}
   \begin{subfigure}[b]{\textwidth}
         \centering
         \caption{Partial dependence plot (SHAP) for Greece (cluster A)}
         \label{fig:5b_GRC}
         \includegraphics[width=\textwidth]{Figure 5b/Figure_5b_GRC}
    \end{subfigure}
    \\
    \vspace{0.5cm}
    \begin{subcaption2}
     This figure shows SHAP-values for predicting carbon intensity over feature values for 87 countries in order of nine country-clusters and silhouette width. The bar chart displays normalized average absolute SHAP-values for all features. Features with less than 3\% of normalized SHAP-values are subsumed as "Other features (Sum)". Charts show SHAP-values over total household expenditures for all countries and for the three most important features in each country besides total household expenditures. Colors represent household expenditures with blue (red) colors indicating lower (higher) household expenditures.
     \end{subcaption2}
\end{figure}

\begin{figure}[ht!]\ContinuedFloat
    \centering
   \begin{subfigure}[b]{\textwidth}
         \centering
         \caption{Partial dependence plot (SHAP) for Sweden (cluster A)}
         \label{fig:5b_SWE}
         \includegraphics[width=\textwidth]{Figure 5b/Figure_5b_SWE}         
     \end{subfigure}
    \\
    \vspace{0.5cm}
   \begin{subfigure}[b]{\textwidth}
         \centering
         \caption{Partial dependence plot (SHAP) for Serbia (cluster A)}
         \label{fig:5b_SRB}
         \includegraphics[width=\textwidth]{Figure 5b/Figure_5b_SRB}         
     \end{subfigure}
    \\
    \vspace{0.5cm}
   \begin{subfigure}[b]{\textwidth}
         \centering
         \caption{Partial dependence plot (SHAP) for Croatia (cluster A)}
         \label{fig:5b_HRV}
         \includegraphics[width=\textwidth]{Figure 5b/Figure_5b_HRV}
    \end{subfigure}
    \\
    \vspace{0.5cm}
    \begin{subcaption2}
     This figure shows SHAP-values for predicting carbon intensity over feature values for 87 countries in order of nine country-clusters and silhouette width. The bar chart displays normalized average absolute SHAP-values for all features. Features with less than 3\% of normalized SHAP-values are subsumed as "Other features (Sum)". Charts show SHAP-values over total household expenditures for all countries and for the three most important features in each country besides total household expenditures. Colors represent household expenditures with blue (red) colors indicating lower (higher) household expenditures.
     \end{subcaption2}
\end{figure}

\begin{figure}[ht!]\ContinuedFloat
    \centering
   \begin{subfigure}[b]{\textwidth}
         \centering
         \caption{Partial dependence plot (SHAP) for Romania (cluster A)}
         \label{fig:5b_ROU}
         \includegraphics[width=\textwidth]{Figure 5b/Figure_5b_ROU}         
     \end{subfigure}
    \\
    \vspace{0.5cm}
   \begin{subfigure}[b]{\textwidth}
         \centering
         \caption{Partial dependence plot (SHAP) for Switzerland (cluster A)}
         \label{fig:5b_CHE}
         \includegraphics[width=\textwidth]{Figure 5b/Figure_5b_CHE}         
     \end{subfigure}
    \\
    \vspace{0.5cm}
   \begin{subfigure}[b]{\textwidth}
         \centering
         \caption{Partial dependence plot (SHAP) for Poland (cluster A)}
         \label{fig:5b_POL}
         \includegraphics[width=\textwidth]{Figure 5b/Figure_5b_POL}
    \end{subfigure}
    \\
    \vspace{0.5cm}
    \begin{subcaption2}
     This figure shows SHAP-values for predicting carbon intensity over feature values for 87 countries in order of nine country-clusters and silhouette width. The bar chart displays normalized average absolute SHAP-values for all features. Features with less than 3\% of normalized SHAP-values are subsumed as "Other features (Sum)". Charts show SHAP-values over total household expenditures for all countries and for the three most important features in each country besides total household expenditures. Colors represent household expenditures with blue (red) colors indicating lower (higher) household expenditures.
     \end{subcaption2}
\end{figure}

\begin{figure}[ht!]\ContinuedFloat
    \centering
   \begin{subfigure}[b]{\textwidth}
         \centering
         \caption{Partial dependence plot (SHAP) for Brazil (cluster A)}
         \label{fig:5b_BRA}
         \includegraphics[width=\textwidth]{Figure 5b/Figure_5b_BRA}         
     \end{subfigure}
    \\
    \vspace{0.5cm}
   \begin{subfigure}[b]{\textwidth}
         \centering
         \caption{Partial dependence plot (SHAP) for France (cluster A)}
         \label{fig:5b_FRA}
         \includegraphics[width=\textwidth]{Figure 5b/Figure_5b_FRA}         
     \end{subfigure}
    \\
    \vspace{0.5cm}
   \begin{subfigure}[b]{\textwidth}
         \centering
         \caption{Partial dependence plot (SHAP) for Austria (cluster A)}
         \label{fig:5b_AUT}
         \includegraphics[width=\textwidth]{Figure 5b/Figure_5b_AUT}
    \end{subfigure}
    \\
    \vspace{0.5cm}
    \begin{subcaption2}
     This figure shows SHAP-values for predicting carbon intensity over feature values for 87 countries in order of nine country-clusters and silhouette width. The bar chart displays normalized average absolute SHAP-values for all features. Features with less than 3\% of normalized SHAP-values are subsumed as "Other features (Sum)". Charts show SHAP-values over total household expenditures for all countries and for the three most important features in each country besides total household expenditures. Colors represent household expenditures with blue (red) colors indicating lower (higher) household expenditures.
     \end{subcaption2}
\end{figure}

\begin{figure}[ht!]\ContinuedFloat
    \centering
   \begin{subfigure}[b]{\textwidth}
         \centering
         \caption{Partial dependence plot (SHAP) for Denmark (cluster A)}
         \label{fig:5b_DNK}
         \includegraphics[width=\textwidth]{Figure 5b/Figure_5b_DNK}         
     \end{subfigure}
    \\
    \vspace{0.5cm}
   \begin{subfigure}[b]{\textwidth}
         \centering
         \caption{Partial dependence plot (SHAP) for Lithuania (cluster A)}
         \label{fig:5b_LTU}
         \includegraphics[width=\textwidth]{Figure 5b/Figure_5b_LTU}         
     \end{subfigure}
    \\
    \vspace{0.5cm}
   \begin{subfigure}[b]{\textwidth}
         \centering
         \caption{Partial dependence plot (SHAP) for Cambodia (cluster A)}
         \label{fig:5b_KHM}
         \includegraphics[width=\textwidth]{Figure 5b/Figure_5b_KHM}
    \end{subfigure}
    \\
    \vspace{0.5cm}
    \begin{subcaption2}
     This figure shows SHAP-values for predicting carbon intensity over feature values for 87 countries in order of nine country-clusters and silhouette width. The bar chart displays normalized average absolute SHAP-values for all features. Features with less than 3\% of normalized SHAP-values are subsumed as "Other features (Sum)". Charts show SHAP-values over total household expenditures for all countries and for the three most important features in each country besides total household expenditures. Colors represent household expenditures with blue (red) colors indicating lower (higher) household expenditures.
     \end{subcaption2}
\end{figure}

\begin{figure}[ht!]\ContinuedFloat
    \centering
   \begin{subfigure}[b]{\textwidth}
         \centering
         \caption{Partial dependence plot (SHAP) for Finland (cluster A)}
         \label{fig:5b_FIN}
         \includegraphics[width=\textwidth]{Figure 5b/Figure_5b_FIN}         
     \end{subfigure}
    \\
    \vspace{0.5cm}
   \begin{subfigure}[b]{\textwidth}
         \centering
         \caption{Partial dependence plot (SHAP) for Colombia (cluster A)}
         \label{fig:5b_COL}
         \includegraphics[width=\textwidth]{Figure 5b/Figure_5b_COL}         
     \end{subfigure}
    \\
    \vspace{0.5cm}
   \begin{subfigure}[b]{\textwidth}
         \centering
         \caption{Partial dependence plot (SHAP) for Spain (cluster A)}
         \label{fig:5b_ESP}
         \includegraphics[width=\textwidth]{Figure 5b/Figure_5b_ESP}
    \end{subfigure}
    \\
    \vspace{0.5cm}
    \begin{subcaption2}
     This figure shows SHAP-values for predicting carbon intensity over feature values for 87 countries in order of nine country-clusters and silhouette width. The bar chart displays normalized average absolute SHAP-values for all features. Features with less than 3\% of normalized SHAP-values are subsumed as "Other features (Sum)". Charts show SHAP-values over total household expenditures for all countries and for the three most important features in each country besides total household expenditures. Colors represent household expenditures with blue (red) colors indicating lower (higher) household expenditures.
     \end{subcaption2}
\end{figure}

\begin{figure}[ht!]\ContinuedFloat
    \centering
   \begin{subfigure}[b]{\textwidth}
         \centering
         \caption{Partial dependence plot (SHAP) for USA (cluster A)}
         \label{fig:5b_USA}
         \includegraphics[width=\textwidth]{Figure 5b/Figure_5b_USA}         
     \end{subfigure}
    \\
    \vspace{0.5cm}
   \begin{subfigure}[b]{\textwidth}
         \centering
         \caption{Partial dependence plot (SHAP) for Belgium (cluster A)}
         \label{fig:5b_BEL}
         \includegraphics[width=\textwidth]{Figure 5b/Figure_5b_BEL}         
     \end{subfigure}
    \\
    \vspace{0.5cm}
   \begin{subfigure}[b]{\textwidth}
         \centering
         \caption{Partial dependence plot (SHAP) for United Kingdom (cluster A)}
         \label{fig:5b_GBR}
         \includegraphics[width=\textwidth]{Figure 5b/Figure_5b_GBR}
    \end{subfigure}
    \\
    \vspace{0.5cm}
    \begin{subcaption2}
     This figure shows SHAP-values for predicting carbon intensity over feature values for 87 countries in order of nine country-clusters and silhouette width. The bar chart displays normalized average absolute SHAP-values for all features. Features with less than 3\% of normalized SHAP-values are subsumed as "Other features (Sum)". Charts show SHAP-values over total household expenditures for all countries and for the three most important features in each country besides total household expenditures. Colors represent household expenditures with blue (red) colors indicating lower (higher) household expenditures.
     \end{subcaption2}
\end{figure}

\begin{figure}[ht!]\ContinuedFloat
    \centering
   \begin{subfigure}[b]{\textwidth}
         \centering
         \caption{Partial dependence plot (SHAP) for Cyprus (cluster A)}
         \label{fig:5b_CYP}
         \includegraphics[width=\textwidth]{Figure 5b/Figure_5b_CYP}         
     \end{subfigure}
    \\
    \vspace{0.5cm}
   \begin{subfigure}[b]{\textwidth}
         \centering
         \caption{Partial dependence plot (SHAP) for Maldives (cluster A)}
         \label{fig:5b_MDV}
         \includegraphics[width=\textwidth]{Figure 5b/Figure_5b_MDV}         
     \end{subfigure}
    \\
    \vspace{0.5cm}
   \begin{subfigure}[b]{\textwidth}
         \centering
         \caption{Partial dependence plot (SHAP) for Germany (cluster A)}
         \label{fig:5b_DEU}
         \includegraphics[width=\textwidth]{Figure 5b/Figure_5b_DEU}
    \end{subfigure}
    \\
    \vspace{0.5cm}
    \begin{subcaption2}
     This figure shows SHAP-values for predicting carbon intensity over feature values for 87 countries in order of nine country-clusters and silhouette width. The bar chart displays normalized average absolute SHAP-values for all features. Features with less than 3\% of normalized SHAP-values are subsumed as "Other features (Sum)". Charts show SHAP-values over total household expenditures for all countries and for the three most important features in each country besides total household expenditures. Colors represent household expenditures with blue (red) colors indicating lower (higher) household expenditures.
     \end{subcaption2}
\end{figure}

\begin{figure}[ht!]\ContinuedFloat
    \centering
   \begin{subfigure}[b]{\textwidth}
         \centering
         \caption{Partial dependence plot (SHAP) for Canada (cluster A)}
         \label{fig:5b_CAN}
         \includegraphics[width=\textwidth]{Figure 5b/Figure_5b_CAN}         
     \end{subfigure}
    \\
    \vspace{0.5cm}
   \begin{subfigure}[b]{\textwidth}
         \centering
         \caption{Partial dependence plot (SHAP) for the Netherlands (cluster A)}
         \label{fig:5b_NLD}
         \includegraphics[width=\textwidth]{Figure 5b/Figure_5b_NLD}         
     \end{subfigure}
    \\
    \vspace{0.5cm}
   \begin{subfigure}[b]{\textwidth}
         \centering
         \caption{Partial dependence plot (SHAP) for Uruguay (cluster A)}
         \label{fig:5b_URY}
         \includegraphics[width=\textwidth]{Figure 5b/Figure_5b_URY}
    \end{subfigure}
    \\
    \vspace{0.5cm}
    \begin{subcaption2}
     This figure shows SHAP-values for predicting carbon intensity over feature values for 87 countries in order of nine country-clusters and silhouette width. The bar chart displays normalized average absolute SHAP-values for all features. Features with less than 3\% of normalized SHAP-values are subsumed as "Other features (Sum)". Charts show SHAP-values over total household expenditures for all countries and for the three most important features in each country besides total household expenditures. Colors represent household expenditures with blue (red) colors indicating lower (higher) household expenditures.
     \end{subcaption2}
\end{figure}

\begin{figure}[ht!]\ContinuedFloat
    \centering
   \begin{subfigure}[b]{\textwidth}
         \centering
         \caption{Partial dependence plot (SHAP) for Mongolia (cluster A)}
         \label{fig:5b_MNG}
         \includegraphics[width=\textwidth]{Figure 5b/Figure_5b_MNG}         
     \end{subfigure}
    \\
    \vspace{0.5cm}
   \begin{subfigure}[b]{\textwidth}
         \centering
         \caption{Partial dependence plot (SHAP) for Barbados (cluster A)}
         \label{fig:5b_BRB}
         \includegraphics[width=\textwidth]{Figure 5b/Figure_5b_BRB}         
     \end{subfigure}
    \\
    \vspace{0.5cm}
   \begin{subfigure}[b]{\textwidth}
         \centering
         \caption{Partial dependence plot (SHAP) for Argentina (cluster A)}
         \label{fig:5b_ARG}
         \includegraphics[width=\textwidth]{Figure 5b/Figure_5b_ARG}
    \end{subfigure}
    \\
    \vspace{0.5cm}
    \begin{subcaption2}
     This figure shows SHAP-values for predicting carbon intensity over feature values for 87 countries in order of nine country-clusters and silhouette width. The bar chart displays normalized average absolute SHAP-values for all features. Features with less than 3\% of normalized SHAP-values are subsumed as "Other features (Sum)". Charts show SHAP-values over total household expenditures for all countries and for the three most important features in each country besides total household expenditures. Colors represent household expenditures with blue (red) colors indicating lower (higher) household expenditures.
     \end{subcaption2}
\end{figure}

\begin{figure}[ht!]\ContinuedFloat
    \centering
   \begin{subfigure}[b]{\textwidth}
         \centering
         \caption{Partial dependence plot (SHAP) for Costa Rica (cluster A)}
         \label{fig:5b_CRI}
         \includegraphics[width=\textwidth]{Figure 5b/Figure_5b_CRI}         
     \end{subfigure}
    \\
    \vspace{0.5cm}
   \begin{subfigure}[b]{\textwidth}
         \centering
         \caption{Partial dependence plot (SHAP) for Italy (cluster A)}
         \label{fig:5b_ITA}
         \includegraphics[width=\textwidth]{Figure 5b/Figure_5b_ITA}         
     \end{subfigure}
    \\
    \vspace{0.5cm}
   \begin{subfigure}[b]{\textwidth}
         \centering
         \caption{Partial dependence plot (SHAP) for Israel (cluster A)}
         \label{fig:5b_ISR}
         \includegraphics[width=\textwidth]{Figure 5b/Figure_5b_ISR}
    \end{subfigure}
    \\
    \vspace{0.5cm}
    \begin{subcaption2}
     This figure shows SHAP-values for predicting carbon intensity over feature values for 87 countries in order of nine country-clusters and silhouette width. The bar chart displays normalized average absolute SHAP-values for all features. Features with less than 3\% of normalized SHAP-values are subsumed as "Other features (Sum)". Charts show SHAP-values over total household expenditures for all countries and for the three most important features in each country besides total household expenditures. Colors represent household expenditures with blue (red) colors indicating lower (higher) household expenditures.
     \end{subcaption2}
\end{figure}

\begin{figure}[ht!]\ContinuedFloat
    \centering
   \begin{subfigure}[b]{\textwidth}
         \centering
         \caption{Partial dependence plot (SHAP) for Senegal (cluster B)}
         \label{fig:5b_SEN}
         \includegraphics[width=\textwidth]{Figure 5b/Figure_5b_SEN}         
     \end{subfigure}
    \\
    \vspace{0.5cm}
   \begin{subfigure}[b]{\textwidth}
         \centering
         \caption{Partial dependence plot (SHAP) for Ghana (cluster B)}
         \label{fig:5b_GHA}
         \includegraphics[width=\textwidth]{Figure 5b/Figure_5b_GHA}         
     \end{subfigure}
    \\
    \vspace{0.5cm}
   \begin{subfigure}[b]{\textwidth}
         \centering
         \caption{Partial dependence plot (SHAP) for Nigeria (cluster B)}
         \label{fig:5b_NGA}
         \includegraphics[width=\textwidth]{Figure 5b/Figure_5b_NGA}
    \end{subfigure}
    \\
    \vspace{0.5cm}
    \begin{subcaption2}
     This figure shows SHAP-values for predicting carbon intensity over feature values for 87 countries in order of nine country-clusters and silhouette width. The bar chart displays normalized average absolute SHAP-values for all features. Features with less than 3\% of normalized SHAP-values are subsumed as "Other features (Sum)". Charts show SHAP-values over total household expenditures for all countries and for the three most important features in each country besides total household expenditures. Colors represent household expenditures with blue (red) colors indicating lower (higher) household expenditures.
     \end{subcaption2}
\end{figure}

\begin{figure}[ht!]\ContinuedFloat
    \centering
   \begin{subfigure}[b]{\textwidth}
         \centering
         \caption{Partial dependence plot (SHAP) for Nicaragua (cluster B)}
         \label{fig:5b_NIC}
         \includegraphics[width=\textwidth]{Figure 5b/Figure_5b_NIC}         
     \end{subfigure}
    \\
    \vspace{0.5cm}
   \begin{subfigure}[b]{\textwidth}
         \centering
         \caption{Partial dependence plot (SHAP) for Malawi (cluster B)}
         \label{fig:5b_MWI}
         \includegraphics[width=\textwidth]{Figure 5b/Figure_5b_MWI}         
     \end{subfigure}
    \\
    \vspace{0.5cm}
   \begin{subfigure}[b]{\textwidth}
         \centering
         \caption{Partial dependence plot (SHAP) for Guinea-Bissau (cluster B)}
         \label{fig:5b_GNB}
         \includegraphics[width=\textwidth]{Figure 5b/Figure_5b_GNB}
    \end{subfigure}
    \\
    \vspace{0.5cm}
    \begin{subcaption2}
     This figure shows SHAP-values for predicting carbon intensity over feature values for 87 countries in order of nine country-clusters and silhouette width. The bar chart displays normalized average absolute SHAP-values for all features. Features with less than 3\% of normalized SHAP-values are subsumed as "Other features (Sum)". Charts show SHAP-values over total household expenditures for all countries and for the three most important features in each country besides total household expenditures. Colors represent household expenditures with blue (red) colors indicating lower (higher) household expenditures.
     \end{subcaption2}
\end{figure}

\begin{figure}[ht!]\ContinuedFloat
    \centering
   \begin{subfigure}[b]{\textwidth}
         \centering
         \caption{Partial dependence plot (SHAP) for Mozambique (cluster B)}
         \label{fig:5b_MOZ}
         \includegraphics[width=\textwidth]{Figure 5b/Figure_5b_MOZ}         
     \end{subfigure}
    \\
    \vspace{0.5cm}
   \begin{subfigure}[b]{\textwidth}
         \centering
         \caption{Partial dependence plot (SHAP) for Niger (cluster B)}
         \label{fig:5b_NER}
         \includegraphics[width=\textwidth]{Figure 5b/Figure_5b_NER}         
     \end{subfigure}
    \\
    \vspace{0.5cm}
   \begin{subfigure}[b]{\textwidth}
         \centering
         \caption{Partial dependence plot (SHAP) for Guatemala (cluster B)}
         \label{fig:5b_GTM}
         \includegraphics[width=\textwidth]{Figure 5b/Figure_5b_GTM}
    \end{subfigure}
    \\
    \vspace{0.5cm}
    \begin{subcaption2}
     This figure shows SHAP-values for predicting carbon intensity over feature values for 87 countries in order of nine country-clusters and silhouette width. The bar chart displays normalized average absolute SHAP-values for all features. Features with less than 3\% of normalized SHAP-values are subsumed as "Other features (Sum)". Charts show SHAP-values over total household expenditures for all countries and for the three most important features in each country besides total household expenditures. Colors represent household expenditures with blue (red) colors indicating lower (higher) household expenditures.
     \end{subcaption2}
\end{figure}

\begin{figure}[ht!]\ContinuedFloat
    \centering
   \begin{subfigure}[b]{\textwidth}
         \centering
         \caption{Partial dependence plot (SHAP) for India (cluster B)}
         \label{fig:5b_IND}
         \includegraphics[width=\textwidth]{Figure 5b/Figure_5b_IND}         
     \end{subfigure}
    \\
    \vspace{0.5cm}
   \begin{subfigure}[b]{\textwidth}
         \centering
         \caption{Partial dependence plot (SHAP) for Bangladesh (cluster B)}
         \label{fig:5b_BGD}
         \includegraphics[width=\textwidth]{Figure 5b/Figure_5b_BGD}         
     \end{subfigure}
    \\
    \vspace{0.5cm}
   \begin{subfigure}[b]{\textwidth}
         \centering
         \caption{Partial dependence plot (SHAP) for Liberia (cluster B)}
         \label{fig:5b_LBR}
         \includegraphics[width=\textwidth]{Figure 5b/Figure_5b_LBR}
    \end{subfigure}
    \\
    \vspace{0.5cm}
    \begin{subcaption2}
     This figure shows SHAP-values for predicting carbon intensity over feature values for 87 countries in order of nine country-clusters and silhouette width. The bar chart displays normalized average absolute SHAP-values for all features. Features with less than 3\% of normalized SHAP-values are subsumed as "Other features (Sum)". Charts show SHAP-values over total household expenditures for all countries and for the three most important features in each country besides total household expenditures. Colors represent household expenditures with blue (red) colors indicating lower (higher) household expenditures.
     \end{subcaption2}
\end{figure}

\begin{figure}[ht!]\ContinuedFloat
    \centering
   \begin{subfigure}[b]{\textwidth}
         \centering
         \caption{Partial dependence plot (SHAP) for Mali (cluster B)}
         \label{fig:5b_MLI}
         \includegraphics[width=\textwidth]{Figure 5b/Figure_5b_MLI}         
     \end{subfigure}
    \\
    \vspace{0.5cm}
   \begin{subfigure}[b]{\textwidth}
         \centering
         \caption{Partial dependence plot (SHAP) for Myanmar (cluster B)}
         \label{fig:5b_MMR}
         \includegraphics[width=\textwidth]{Figure 5b/Figure_5b_MMR}         
     \end{subfigure}
    \\
    \vspace{0.5cm}
   \begin{subfigure}[b]{\textwidth}
         \centering
         \caption{Partial dependence plot (SHAP) for Paraguay (cluster B)}
         \label{fig:5b_PRY}
         \includegraphics[width=\textwidth]{Figure 5b/Figure_5b_PRY}
    \end{subfigure}
    \\
    \vspace{0.5cm}
    \begin{subcaption2}
     This figure shows SHAP-values for predicting carbon intensity over feature values for 87 countries in order of nine country-clusters and silhouette width. The bar chart displays normalized average absolute SHAP-values for all features. Features with less than 3\% of normalized SHAP-values are subsumed as "Other features (Sum)". Charts show SHAP-values over total household expenditures for all countries and for the three most important features in each country besides total household expenditures. Colors represent household expenditures with blue (red) colors indicating lower (higher) household expenditures.
     \end{subcaption2}
\end{figure}

\begin{figure}[ht!]\ContinuedFloat
    \centering
   \begin{subfigure}[b]{\textwidth}
         \centering
         \caption{Partial dependence plot (SHAP) for Kenya (cluster B)}
         \label{fig:5b_KEN}
         \includegraphics[width=\textwidth]{Figure 5b/Figure_5b_KEN}         
     \end{subfigure}
    \\
    \vspace{0.5cm}
   \begin{subfigure}[b]{\textwidth}
         \centering
         \caption{Partial dependence plot (SHAP) for Mexico (cluster C)}
         \label{fig:5b_MEX}
         \includegraphics[width=\textwidth]{Figure 5b/Figure_5b_MEX}         
     \end{subfigure}
    \\
    \vspace{0.5cm}
   \begin{subfigure}[b]{\textwidth}
         \centering
         \caption{Partial dependence plot (SHAP) for Indonesia (cluster C)}
         \label{fig:5b_IDN}
         \includegraphics[width=\textwidth]{Figure 5b/Figure_5b_IDN}
    \end{subfigure}
    \\
    \vspace{0.5cm}
    \begin{subcaption2}
     This figure shows SHAP-values for predicting carbon intensity over feature values for 87 countries in order of nine country-clusters and silhouette width. The bar chart displays normalized average absolute SHAP-values for all features. Features with less than 3\% of normalized SHAP-values are subsumed as "Other features (Sum)". Charts show SHAP-values over total household expenditures for all countries and for the three most important features in each country besides total household expenditures. Colors represent household expenditures with blue (red) colors indicating lower (higher) household expenditures.
     \end{subcaption2}
\end{figure}

\begin{figure}[ht!]\ContinuedFloat
    \centering
   \begin{subfigure}[b]{\textwidth}
         \centering
         \caption{Partial dependence plot (SHAP) for Philippines (cluster C)}
         \label{fig:5b_PHL}
         \includegraphics[width=\textwidth]{Figure 5b/Figure_5b_PHL}         
     \end{subfigure}
    \\
    \vspace{0.5cm}
   \begin{subfigure}[b]{\textwidth}
         \centering
         \caption{Partial dependence plot (SHAP) for South Africa (cluster C)}
         \label{fig:5b_ZAF}
         \includegraphics[width=\textwidth]{Figure 5b/Figure_5b_ZAF}         
     \end{subfigure}
    \\
    \vspace{0.5cm}
   \begin{subfigure}[b]{\textwidth}
         \centering
         \caption{Partial dependence plot (SHAP) for Thailand (cluster C)}
         \label{fig:5b_THA}
         \includegraphics[width=\textwidth]{Figure 5b/Figure_5b_THA}
    \end{subfigure}
    \\
    \vspace{0.5cm}
    \begin{subcaption2}
     This figure shows SHAP-values for predicting carbon intensity over feature values for 87 countries in order of nine country-clusters and silhouette width. The bar chart displays normalized average absolute SHAP-values for all features. Features with less than 3\% of normalized SHAP-values are subsumed as "Other features (Sum)". Charts show SHAP-values over total household expenditures for all countries and for the three most important features in each country besides total household expenditures. Colors represent household expenditures with blue (red) colors indicating lower (higher) household expenditures.
     \end{subcaption2}
\end{figure}

\begin{figure}[ht!]\ContinuedFloat
    \centering
   \begin{subfigure}[b]{\textwidth}
         \centering
         \caption{Partial dependence plot (SHAP) for Vietnam (cluster C)}
         \label{fig:5b_VNM}
         \includegraphics[width=\textwidth]{Figure 5b/Figure_5b_VNM}         
     \end{subfigure}
    \\
    \vspace{0.5cm}
   \begin{subfigure}[b]{\textwidth}
         \centering
         \caption{Partial dependence plot (SHAP) for Russian Federation (cluster C)}
         \label{fig:5b_RUS}
         \includegraphics[width=\textwidth]{Figure 5b/Figure_5b_RUS}         
     \end{subfigure}
    \\
    \vspace{0.5cm}
   \begin{subfigure}[b]{\textwidth}
         \centering
         \caption{Partial dependence plot (SHAP) for Dominican Republic (cluster C)}
         \label{fig:5b_DOM}
         \includegraphics[width=\textwidth]{Figure 5b/Figure_5b_DOM}
    \end{subfigure}
    \\
    \vspace{0.5cm}
    \begin{subcaption2}
     This figure shows SHAP-values for predicting carbon intensity over feature values for 87 countries in order of nine country-clusters and silhouette width. The bar chart displays normalized average absolute SHAP-values for all features. Features with less than 3\% of normalized SHAP-values are subsumed as "Other features (Sum)". Charts show SHAP-values over total household expenditures for all countries and for the three most important features in each country besides total household expenditures. Colors represent household expenditures with blue (red) colors indicating lower (higher) household expenditures.
     \end{subcaption2}
\end{figure}

\begin{figure}[ht!]\ContinuedFloat
    \centering
   \begin{subfigure}[b]{\textwidth}
         \centering
         \caption{Partial dependence plot (SHAP) for Georgia (cluster C)}
         \label{fig:5b_GEO}
         \includegraphics[width=\textwidth]{Figure 5b/Figure_5b_GEO}         
     \end{subfigure}
    \\
    \vspace{0.5cm}
   \begin{subfigure}[b]{\textwidth}
         \centering
         \caption{Partial dependence plot (SHAP) for Egypt (cluster C)}
         \label{fig:5b_EGY}
         \includegraphics[width=\textwidth]{Figure 5b/Figure_5b_EGY}         
     \end{subfigure}
    \\
    \vspace{0.5cm}
   \begin{subfigure}[b]{\textwidth}
         \centering
         \caption{Partial dependence plot (SHAP) for Slovakia (cluster D)}
         \label{fig:5b_SVK}
         \includegraphics[width=\textwidth]{Figure 5b/Figure_5b_SVK}
    \end{subfigure}
    \\
    \vspace{0.5cm}
    \begin{subcaption2}
     This figure shows SHAP-values for predicting carbon intensity over feature values for 87 countries in order of nine country-clusters and silhouette width. The bar chart displays normalized average absolute SHAP-values for all features. Features with less than 3\% of normalized SHAP-values are subsumed as "Other features (Sum)". Charts show SHAP-values over total household expenditures for all countries and for the three most important features in each country besides total household expenditures. Colors represent household expenditures with blue (red) colors indicating lower (higher) household expenditures.
     \end{subcaption2}
\end{figure}

\begin{figure}[ht!]\ContinuedFloat
    \centering
   \begin{subfigure}[b]{\textwidth}
         \centering
         \caption{Partial dependence plot (SHAP) for Ireland (cluster D)}
         \label{fig:5b_IRL}
         \includegraphics[width=\textwidth]{Figure 5b/Figure_5b_IRL}         
     \end{subfigure}
    \\
    \vspace{0.5cm}
   \begin{subfigure}[b]{\textwidth}
         \centering
         \caption{Partial dependence plot (SHAP) for Portugal (cluster D)}
         \label{fig:5b_PRT}
         \includegraphics[width=\textwidth]{Figure 5b/Figure_5b_PRT}         
     \end{subfigure}
    \\
    \vspace{0.5cm}
   \begin{subfigure}[b]{\textwidth}
         \centering
         \caption{Partial dependence plot (SHAP) for Norway (cluster D)}
         \label{fig:5b_NOR}
         \includegraphics[width=\textwidth]{Figure 5b/Figure_5b_NOR}
    \end{subfigure}
    \\
    \vspace{0.5cm}
    \begin{subcaption2}
     This figure shows SHAP-values for predicting carbon intensity over feature values for 87 countries in order of nine country-clusters and silhouette width. The bar chart displays normalized average absolute SHAP-values for all features. Features with less than 3\% of normalized SHAP-values are subsumed as "Other features (Sum)". Charts show SHAP-values over total household expenditures for all countries and for the three most important features in each country besides total household expenditures. Colors represent household expenditures with blue (red) colors indicating lower (higher) household expenditures.
     \end{subcaption2}
\end{figure}

\begin{figure}[ht!]\ContinuedFloat
    \centering
   \begin{subfigure}[b]{\textwidth}
         \centering
         \caption{Partial dependence plot (SHAP) for Pakistan (cluster D)}
         \label{fig:5b_PAK}
         \includegraphics[width=\textwidth]{Figure 5b/Figure_5b_PAK}         
     \end{subfigure}
    \\
    \vspace{0.5cm}
   \begin{subfigure}[b]{\textwidth}
         \centering
         \caption{Partial dependence plot (SHAP) for Latvia (cluster D)}
         \label{fig:5b_LVA}
         \includegraphics[width=\textwidth]{Figure 5b/Figure_5b_LVA}         
     \end{subfigure}
    \\
    \vspace{0.5cm}
   \begin{subfigure}[b]{\textwidth}
         \centering
         \caption{Partial dependence plot (SHAP) for Luxembourg (cluster D)}
         \label{fig:5b_LUX}
         \includegraphics[width=\textwidth]{Figure 5b/Figure_5b_LUX}
    \end{subfigure}
    \\
    \vspace{0.5cm}
    \begin{subcaption2}
     This figure shows SHAP-values for predicting carbon intensity over feature values for 87 countries in order of nine country-clusters and silhouette width. The bar chart displays normalized average absolute SHAP-values for all features. Features with less than 3\% of normalized SHAP-values are subsumed as "Other features (Sum)". Charts show SHAP-values over total household expenditures for all countries and for the three most important features in each country besides total household expenditures. Colors represent household expenditures with blue (red) colors indicating lower (higher) household expenditures.
     \end{subcaption2}
\end{figure}

\begin{figure}[ht!]\ContinuedFloat
    \centering
   \begin{subfigure}[b]{\textwidth}
         \centering
         \caption{Partial dependence plot (SHAP) for Czech Republic (cluster D)}
         \label{fig:5b_CZE}
         \includegraphics[width=\textwidth]{Figure 5b/Figure_5b_CZE}         
     \end{subfigure}
    \\
    \vspace{0.5cm}
   \begin{subfigure}[b]{\textwidth}
         \centering
         \caption{Partial dependence plot (SHAP) for Chile (cluster D)}
         \label{fig:5b_CHL}
         \includegraphics[width=\textwidth]{Figure 5b/Figure_5b_CHL}         
     \end{subfigure}
    \\
    \vspace{0.5cm}
   \begin{subfigure}[b]{\textwidth}
         \centering
         \caption{Partial dependence plot (SHAP) for Peru (cluster E)}
         \label{fig:5b_PER}
         \includegraphics[width=\textwidth]{Figure 5b/Figure_5b_PER}
    \end{subfigure}
    \\
    \vspace{0.5cm}
    \begin{subcaption2}
     This figure shows SHAP-values for predicting carbon intensity over feature values for 87 countries in order of nine country-clusters and silhouette width. The bar chart displays normalized average absolute SHAP-values for all features. Features with less than 3\% of normalized SHAP-values are subsumed as "Other features (Sum)". Charts show SHAP-values over total household expenditures for all countries and for the three most important features in each country besides total household expenditures. Colors represent household expenditures with blue (red) colors indicating lower (higher) household expenditures.
     \end{subcaption2}
\end{figure}

\begin{figure}[ht!]\ContinuedFloat
    \centering
   \begin{subfigure}[b]{\textwidth}
         \centering
         \caption{Partial dependence plot (SHAP) for Ecuador (cluster E)}
         \label{fig:5b_ECU}
         \includegraphics[width=\textwidth]{Figure 5b/Figure_5b_ECU}         
     \end{subfigure}
    \\
    \vspace{0.5cm}
   \begin{subfigure}[b]{\textwidth}
         \centering
         \caption{Partial dependence plot (SHAP) for El Salvador (cluster E)}
         \label{fig:5b_SLV}
         \includegraphics[width=\textwidth]{Figure 5b/Figure_5b_SLV}         
     \end{subfigure}
    \\
    \vspace{0.5cm}
   \begin{subfigure}[b]{\textwidth}
         \centering
         \caption{Partial dependence plot (SHAP) for Bolivia (cluster E)}
         \label{fig:5b_BOL}
         \includegraphics[width=\textwidth]{Figure 5b/Figure_5b_BOL}
    \end{subfigure}
    \\
    \vspace{0.5cm}
    \begin{subcaption2}
     This figure shows SHAP-values for predicting carbon intensity over feature values for 87 countries in order of nine country-clusters and silhouette width. The bar chart displays normalized average absolute SHAP-values for all features. Features with less than 3\% of normalized SHAP-values are subsumed as "Other features (Sum)". Charts show SHAP-values over total household expenditures for all countries and for the three most important features in each country besides total household expenditures. Colors represent household expenditures with blue (red) colors indicating lower (higher) household expenditures.
     \end{subcaption2}
\end{figure}

\begin{figure}[ht!]\ContinuedFloat
    \centering
   \begin{subfigure}[b]{\textwidth}
         \centering
         \caption{Partial dependence plot (SHAP) for Iraq (cluster E)}
         \label{fig:5b_IRQ}
         \includegraphics[width=\textwidth]{Figure 5b/Figure_5b_IRQ}         
     \end{subfigure}
    \\
    \vspace{0.5cm}
   \begin{subfigure}[b]{\textwidth}
         \centering
         \caption{Partial dependence plot (SHAP) for Togo (cluster F)}
         \label{fig:5b_TGO}
         \includegraphics[width=\textwidth]{Figure 5b/Figure_5b_TGO}         
     \end{subfigure}
    \\
    \vspace{0.5cm}
   \begin{subfigure}[b]{\textwidth}
         \centering
         \caption{Partial dependence plot (SHAP) for Benin (cluster F)}
         \label{fig:5b_BEN}
         \includegraphics[width=\textwidth]{Figure 5b/Figure_5b_BEN}
    \end{subfigure}
    \\
    \vspace{0.5cm}
    \begin{subcaption2}
     This figure shows SHAP-values for predicting carbon intensity over feature values for 87 countries in order of nine country-clusters and silhouette width. The bar chart displays normalized average absolute SHAP-values for all features. Features with less than 3\% of normalized SHAP-values are subsumed as "Other features (Sum)". Charts show SHAP-values over total household expenditures for all countries and for the three most important features in each country besides total household expenditures. Colors represent household expenditures with blue (red) colors indicating lower (higher) household expenditures.
     \end{subcaption2}
\end{figure}

\begin{figure}[ht!]\ContinuedFloat
    \centering
   \begin{subfigure}[b]{\textwidth}
         \centering
         \caption{Partial dependence plot (SHAP) for Côte d'Ivoire (cluster F)}
         \label{fig:5b_CIV}
         \includegraphics[width=\textwidth]{Figure 5b/Figure_5b_CIV}         
     \end{subfigure}
    \\
    \vspace{0.5cm}
   \begin{subfigure}[b]{\textwidth}
         \centering
         \caption{Partial dependence plot (SHAP) for Burkina Faso (cluster F)}
         \label{fig:5b_BFA}
         \includegraphics[width=\textwidth]{Figure 5b/Figure_5b_BFA}         
     \end{subfigure}
    \\
    \vspace{0.5cm}
   \begin{subfigure}[b]{\textwidth}
         \centering
         \caption{Partial dependence plot (SHAP) for Uganda (cluster G)}
         \label{fig:5b_UGA}
         \includegraphics[width=\textwidth]{Figure 5b/Figure_5b_UGA}
    \end{subfigure}
    \\
    \vspace{0.5cm}
    \begin{subcaption2}
     This figure shows SHAP-values for predicting carbon intensity over feature values for 87 countries in order of nine country-clusters and silhouette width. The bar chart displays normalized average absolute SHAP-values for all features. Features with less than 3\% of normalized SHAP-values are subsumed as "Other features (Sum)". Charts show SHAP-values over total household expenditures for all countries and for the three most important features in each country besides total household expenditures. Colors represent household expenditures with blue (red) colors indicating lower (higher) household expenditures.
     \end{subcaption2}
\end{figure}


\begin{figure}[ht!]\ContinuedFloat
    \centering
   \begin{subfigure}[b]{\textwidth}
         \centering
         \caption{Partial dependence plot (SHAP) for Rwanda (cluster G)}
         \label{fig:5b_RWA}
         \includegraphics[width=\textwidth]{Figure 5b/Figure_5b_RWA}         
     \end{subfigure}
    \\
    \vspace{0.5cm}
   \begin{subfigure}[b]{\textwidth}
         \centering
         \caption{Partial dependence plot (SHAP) for Ethiopia (cluster G)}
         \label{fig:5b_ETH}
         \includegraphics[width=\textwidth]{Figure 5b/Figure_5b_ETH}         
     \end{subfigure}
    \\
    \vspace{0.5cm}
   \begin{subfigure}[b]{\textwidth}
         \centering
         \caption{Partial dependence plot (SHAP) for Jordan (cluster H)}
         \label{fig:5b_JOR}
         \includegraphics[width=\textwidth]{Figure 5b/Figure_5b_JOR}
    \end{subfigure}
    \\
    \vspace{0.5cm}
    \begin{subcaption2}
     This figure shows SHAP-values for predicting carbon intensity over feature values for 87 countries in order of nine country-clusters and silhouette width. The bar chart displays normalized average absolute SHAP-values for all features. Features with less than 3\% of normalized SHAP-values are subsumed as "Other features (Sum)". Charts show SHAP-values over total household expenditures for all countries and for the three most important features in each country besides total household expenditures. Colors represent household expenditures with blue (red) colors indicating lower (higher) household expenditures.
     \end{subcaption2}
\end{figure}

\begin{figure}[ht!]\ContinuedFloat
    \centering
   \begin{subfigure}[b]{\textwidth}
         \centering
         \caption{Partial dependence plot (SHAP) for Taiwan (cluster H)}
         \label{fig:5b_TWN}
         \includegraphics[width=\textwidth]{Figure 5b/Figure_5b_TWN}         
     \end{subfigure}
    \\
    \vspace{0.5cm}
   \begin{subfigure}[b]{\textwidth}
         \centering
         \caption{Partial dependence plot (SHAP) for Turkey (cluster I)}
         \label{fig:5b_TUR}
         \includegraphics[width=\textwidth]{Figure 5b/Figure_5b_TUR}         
     \end{subfigure}
    \\
    \vspace{0.5cm}
   \begin{subfigure}[b]{\textwidth}
         \centering
         \caption{Partial dependence plot (SHAP) for Armenia (cluster I)}
         \label{fig:5b_ARM}
         \includegraphics[width=\textwidth]{Figure 5b/Figure_5b_ARM}
    \end{subfigure}
    \\
    \vspace{0.5cm}

    \begin{subcaption2}
     This figure shows SHAP-values for predicting carbon intensity over feature values for 87 countries in order of nine country-clusters and silhouette width. The bar chart displays normalized average absolute SHAP-values for all features. Features with less than 3\% of normalized SHAP-values are subsumed as "Other features (Sum)". Charts show SHAP-values over total household expenditures for all countries and for the three most important features in each country besides total household expenditures. Colors represent household expenditures with blue (red) colors indicating lower (higher) household expenditures.
     \end{subcaption2}
\end{figure}
\clearpage

% Marginal effects-plots from Logit models

% \begin{figure}[ht!]
%   \centering
%  \caption{Average marginal effects of car ownership} \label{fig:D1_Car}
%   \includegraphics{Analysis_OLS_ME_Carbon_Footprint/AME_OLS_FP_car.01}
%   \begin{subcaption2}
%     This figure displays ...
%   \end{subcaption2}

% \end{figure}

% \clearpage

% \clearpage

% \begin{figure}[ht!]
%   \centering
%  \caption{Average marginal effects of cooking fuel choice - Part A} \label{fig:D6_Electricity_A}
%   \includegraphics{Analysis_OLS_ME_Carbon_Footprint/AME_OLS_FP_CF_Electricity A}
%   \begin{subcaption2}
%     This figure displays ...
%   \end{subcaption2}

% \end{figure}

% \clearpage

% \begin{figure}[ht!]
%   \centering
%  \caption{Average marginal effects of car ownership} \label{fig:E1_Car}
%   \includegraphics{Analysis_OLS_ME_Carbon_Intensity/AME_OLS_CI_car.01}
%   \begin{subcaption2}
%     This figure displays ...
%   \end{subcaption2}

% \end{figure}

% \clearpage

%  \begin{figure}[ht!]
%    \centering
%   \caption{Average marginal effects of car ownership} \label{fig:F1_Car}
%    \includegraphics{Analysis_Logit_Models_Marginal_Effects/Average_Marginal_Effects_affected_upper_80_car.01}
%    \begin{subcaption2}
%      This figure displays ...
%    \end{subcaption2}

%  \end{figure}

%  \clearpage
\clearpage

% Non-adjusted clustering
\clearpage
\begin{figure}[ht!]
    \centering
    \caption{Feature importance across countries by cluster - Alternative clustering}\label{fig:fig_4_uncorrected}
    \begin{subfigure}[b]{\textwidth}
    \centering
    \includegraphics{1_Figures/Figure 4/Figure_4_Uncorrected_1.pdf}
    \caption{Feature importance across countries of cluster A to C - non-adjusted}\label{fig:fig_4_1_uncorrected}
     \begin{subcaption2}
    This figure shows the importance of features (in normalized average absolute SHAP-values) for each country, grouped by country clusters. Blue (red) colors indicate that a feature is relatively less (more) important in a country compared to all other countries and features. 'Sociodemographic' comprises features such as household size, gender, self-identified ethnicity, nationality, religion or language. 'Spatial' comprises features such as state, province, district and urban/rural-identifiers. For vertical distribution, blue (red) colors indicate lower (higher) median carbon intensity among the poorest quintile compared to the richest quintile. For average CO$_{2}$-intensity, blue (red) colors indicate a lower (higher) average carbon intensity across all countries. For goodness of fit (R\textsuperscript{2}), blue (red) colors indicate a lower (higher) predictive performance compared to other countries. Average carbon intensity and R\textsuperscript{2} are not explicitly included for clustering.
    We assign countries to 5 clusters performing k-means clustering based on \textit{non-adjusted} feature importance values across all features. We also show all values in Table \ref{tab:A10_Uncorrected}.
    \end{subcaption2}
    \end{subfigure}
\end{figure}
\clearpage

\clearpage
\begin{figure}[ht!]\ContinuedFloat
    \centering
    \begin{subfigure}[b]{\textwidth}
    \centering
    \includegraphics{Figure 4/Figure_4_Uncorrected_2.pdf}
    \caption{Feature importance across countries of clusters D to L - non-adjusted}\label{fig:fig_4_2_uncorrected}
    \begin{subcaption2}
    This figure shows the importance of features (in normalized average absolute SHAP-values) for each country, grouped by country clusters. Blue (red) colors indicate that a feature is relatively less (more) important in a country compared to all other countries and features. 'Sociodemographic' comprises features such as household size, gender, self-identified ethnicity, nationality, religion or language. 'Spatial' comprises features such as state, province, district and urban/rural-identifiers. For vertical distribution, blue (red) colors indicate lower (higher) median carbon intensity among the poorest quintile compared to the richest quintile. For average carbon intensity, blue (red) colors indicate a lower (higher) average carbon intensity across all countries. For goodness of fit (R\textsuperscript{2}), blue (red) colors indicate a lower (higher) predictive performance compared to other countries. Average carbon intensity and R\textsuperscript{2} are not explicitly included for clustering.
    We assign countries to 5 clusters performing k-means clustering based on \textit{non-adjusted} feature importance values across all features. We also show all values in Table \ref{tab:A10_Uncorrected}.
    \end{subcaption2}
    \end{subfigure}
    
\end{figure}
\clearpage

\clearpage
\begin{figure}[ht!]\ContinuedFloat
    \centering
    \begin{subfigure}[b]{\textwidth}
    \centering
    \includegraphics{Figure 4/Figure_4_Corrected_Imputed_1.pdf}
    \caption{Feature importance across countries of clusters A to B - imputed}\label{fig:fig_4_1_imputed}\label{fig:fig_4_imputed}
    \begin{subcaption2}
    This figure shows the importance of features (in normalized average absolute SHAP-values) for each country, grouped by country clusters. In contrast to figure \ref{fig:fig_4}, we impute missing values for unobserved features with the average feature importance for each feature. We adjust feature importance for country-level model performance. Blue (red) colors indicate that a feature is relatively less (more) important in a country compared to all other countries and features. 'Sociodemographic' comprises features such as household size, gender, self-identified ethnicity, nationality, religion or language. 'Spatial' comprises features such as state, province, district and urban/rural-identifiers. For vertical distribution, blue (red) colors indicate lower (higher) median carbon intensity among the poorest quintile compared to the richest quintile. For average carbon intensity, blue (red) colors indicate a lower (higher) average carbon intensity across all countries. For goodness of fit (R\textsuperscript{2}), blue (red) colors indicate a lower (higher) predictive performance compared to other countries. Average carbon intensity and R\textsuperscript{2} are not explicitly included for clustering.
    We assign countries to 9 clusters performing k-means clustering based on \textit{adjusted} and \textit{imputed} feature importance values across all features. We also show all values in Table \ref{tab:A10_Imputed}.
    \end{subcaption2}
    \end{subfigure}
    
\end{figure}
\clearpage

\clearpage
\begin{figure}[ht!]\ContinuedFloat
    \centering
    \begin{subfigure}[b]{\textwidth}
    \centering
    \includegraphics{Figure 4/Figure_4_Corrected_Imputed_2.pdf}
    \caption{Feature importance across countries of clusters C to K - imputed}\label{fig:fig_4_2_imputed}
    \begin{subcaption2}
    This figure shows the importance of features (in normalized average absolute SHAP-values) for each country, grouped by country clusters. In contrast to figure \ref{fig:fig_4}, we impute missing values for unobserved features with the average feature importance for each feature. We adjust feature importance for country-level model performance. Blue (red) colors indicate that a feature is relatively less (more) important in a country compared to all other countries and features. 'Sociodemographic' comprises features such as household size, gender, self-identified ethnicity, nationality, religion or language. 'Spatial' comprises features such as state, province, district and urban/rural-identifiers. For vertical distribution, blue (red) colors indicate lower (higher) median carbon intensity among the poorest quintile compared to the richest quintile. For average carbon intensity, blue (red) colors indicate a lower (higher) average carbon intensity across all countries. For goodness of fit (R\textsuperscript{2}), blue (red) colors indicate a lower (higher) predictive performance compared to other countries. Average carbon intensity and R\textsuperscript{2} are not explicitly included for clustering.
    We assign countries to 9 clusters performing k-means clustering based on \textit{adjusted} and \textit{imputed} feature importance values across all features. We also show all values in Table \ref{tab:A10_Imputed}.
    \end{subcaption2}
    \end{subfigure}
    
\end{figure}
\clearpage
\clearpage

% Comparison of policies
\begin{figure}[ht!]
    \centering
    \caption{Vertical and horizontal distribution coefficients for different policies}
    \includegraphics[width=\textwidth]{1_Figures/Figure 2/Figure_2_2017_Policy.jpg}
    \label{fig:comparison_policies}
    \begin{subcaption2}
    This figure displays the vertical distribution coefficient comparing the median carbon intensity of the richest and the poorest quintile. The horizontal distribution coefficient compares the within-quintile differences (5\textsuperscript{th} to 95\textsuperscript{th} percentile within quintiles) of the richest and the poorest quintile. Rectangles (A) and (B) indicate higher carbon intensity (at the median) among the poorest quintile compared to the richest quintile; rectangles (C) and (D) indicate lower carbon intensity (at the median) among the poorest quintile compared to the richest quintile. Rectangles (A) and (C) indicate smaller within-quintile differences of carbon intensity among the richest quintile compared to the poorest quintile; rectangles (B) and (D) indicate larger within-quintile differences of carbon intensity among the richest quintile compared to the poorest quintile. Colors of points indicate GDP per capita for 2018 (in log-transformed constant 2010 USD).
    
    Panel "National climate policy" shows the same values as figure \ref{fig:fig_2}, i.e. distribution coefficients for carbon intensities accounting for all nationally released CO\textsubscript{2}-emissions across all sectors. Panel "International climate policy" shows distribution coefficients for carbon intensities accounting for globally released CO\textsubscript{2}-emissions embedded in national consumption. Panels "Transport sector policy" and "Electricity sector policy" display distribution coefficients for carbon intensities accounting for nationally released CO\textsubscript{2}-emissions in the transport sector and electricity sector, respectively.
    \end{subcaption2}
\end{figure}

\clearpage

\section{Supplementary tables} \label{sec:tables}

\begingroup\fontsize{8}{10}\selectfont

\begin{ThreePartTable}
\begin{TableNotes}
\item \textit{Note: } 
\item This table shows all household budget surveys used in this study. Column 'Year' refers to the year(s) when each survey was conducted. Column 'Sample size' refers to the number of individually-surveyed households in our final dataset, i.e., after data cleaning (see section \ref{sec:cleaning}). Column 'Link' refers do additional online resources and information on data access for each dataset. Note that authors do not take any responsibility for changes on linked webpages. 
\end{TableNotes}

\begin{longtable}[t]{l|p{8cm}|r|r|c}
\caption{\label{tab:datasets}Household budget surveys}\\
\toprule

\multicolumn{1}{c}{Country} & \multicolumn{1}{c}{Survey name} & \multicolumn{1}{c}{Year} & \multicolumn{1}{c}{Sample size} & \multicolumn{1}{c}{Link} \\ \hline 
    \endfirsthead
    
\caption[]{Household budget surveys \textit{(continued)}}\\
\hline \multicolumn{1}{c}{Country} & \multicolumn{1}{c}{Survey name} & \multicolumn{1}{c}{Year} & \multicolumn{1}{c}{Sample size} & \multicolumn{1}{c}{Link} \\ \hline 
\endhead

\endfoot
\bottomrule
\insertTableNotes
\endlastfoot
        
        Argentina & Encuesta Nacional de Gastos de los Hogares & 2017-2018 &  21,540  & \href{https://www.indec.gob.ar/indec/web/Nivel4-Tema-4-45-151}{Link} \\ 
        Armenia & Integrated Living Conditions Survey & 2017 &  7,776  & \href{https://microdata.worldbank.org/index.php/catalog/3591}{Link} \\ 
        Austria & Konsumerhebung & 2019-2020 &  7,162  & \href{https://www.statistik.at/ueber-uns/erhebungen/personen-und-haushaltserhebungen/konsumerhebung}{Link} \\ 
        Australia & Household Expenditure, Income and Housing Survey & 2015-2016 & 10,046 & \href{https://www.abs.gov.au/AUSSTATS/abs@.nsf/Lookup/6503.0Main+Features12015-16?OpenDocument}{Link}  \\
        Bangladesh & Household Income and Expenditure Survey & 2010 &  12,240  & \href{http://data.bbs.gov.bd/index.php/catalog/67}{Link} \\ 
        Barbados & Survey of Living Conditions & 2016 &  2,434  & \href{https://publications.iadb.org/en/barbados-survey-living-conditions-2016}{Link} \\ 
        Benin & Enquête Harmonisée sur le Conditions de Vie des Ménages & 2018-2019 &  8,012  & \href{https://microdata.worldbank.org/index.php/catalog/4291}{Link} \\ 
        Bolivia & Encuesta de Hogares & 2019 &  11,859  & \href{https://www.ine.gob.bo/index.php/estadisticas-sociales/vivienda-y-servicios-basicos/encuestas-de-hogares-vivienda/}{Link} \\ 
        Brazil & Pesquisa de orcamentos familiares & 2017-2018 &  57,889  & \href{https://www.ibge.gov.br/en/statistics/social/population/25610-pof-2017-2018-pof-en.html?=\&t=downloads}{Link} \\ 
        Burkina Faso & Enquête Harmonisée sur le Conditions de Vie des Ménages & 2018-2019 &  7,010  & \href{https://microdata.worldbank.org/index.php/catalog/4290}{Link} \\ 
        Cambodia & Living Standards Measurement Study - Plus & 2019-2020 &  1,206  & \href{https://microdata.worldbank.org/index.php/catalog/study/KHM\_2019\_LSMS-PLUS\_v02\_M}{Link} \\ 
        Canada & Survey of Household Spending & 2017 &  4,012  & \href{https://www150.statcan.gc.ca/n1/en/catalogue/62M0004X}{Link} \\ 
        Chile & Encuesta de presupuestos familiares & 2016-2017 &  15,237  & \href{https://www.ine.cl/estadisticas/sociales/ingresos-y-gastos/encuesta-de-presupuestos-familiares}{Link} \\ 
        Colombia & Encuesta Nacional de Presupuestos de los Hogares & 2016-2017 &  86,866  & \href{https://www.dane.gov.co/index.php/estadisticas-por-tema/pobreza-y-condiciones-de-vida/encuesta-nacional-de-presupuestos-de-los-hogares-enph}{Link} \\ 
        Costa Rica & Encuesta Nacional de Ingresos y Gastos de los Hogares & 2018 &  7,046  & \href{https://inec.cr/estadisticas-fuentes/encuestas/encuesta-nacional-ingresos-gastos-los-hogares}{Link} \\ 
        Côte d'Ivoire & Enquête Harmonisée sur le Conditions de Vie des Ménages & 2018-2019 &  12,992  & \href{https://microdata.worldbank.org/index.php/catalog/4292/study-description}{Link} \\ 
        Dominican Republic & Encuesta Nacional de Gastos e Ingresos de los Hogares & 2018 &  8,884  & \href{https://archivo.one.gob.do/encuestas/enigh}{Link} \\ 
        Ecuador & Encuesta Condiciones de Vida & 2013-2014 &  28,263  & \href{https://aplicaciones3.ecuadorencifras.gob.ec/BIINEC-war/index.xhtml}{Link} \\ 
        Egypt & Household Income, Expenditure and Consumption Survey & 2017-2018 &  12,485  & \href{http://www.erfdataportal.com/index.php/catalog/129}{Link} \\ 
        El Salvador & Encuesta de Hogares de Propósitos Múltiples & 2015 &  23,622  & \href{http://www.digestyc.gob.sv/index.php/temas/des/ehpm.html}{Link} \\ 
        Ethiopia & Socioeconomic Survey & 2018-2019 &  6,767  & \href{https://microdata.worldbank.org/index.php/catalog/3823}{Link} \\ 
        EU & Household Budget Survey & 2015 &  275,427  & \href{https://ec.europa.eu/eurostat/web/microdata/household-budget-survey}{Link} \\ 
        Georgia & Monitoring of Households & 2019 &  13,247  & \href{https://www.geostat.ge/en/modules/categories/128/databases-of-2009-2016-integrated-household-survey-and-2017-households-income-and-expenditure-survey}{Link} \\ 
        Ghana & Living Standards Survey 7 & 2016-2017 &  13,521  & \href{https://www2.statsghana.gov.gh/nada/index.php/catalog/97/study-description}{Link} \\ 
        Guatemala & Encuesta Nacional de Condiciones de Vida & 2014 &  11,535  & \href{https://www.proyectoencovi.com/}{Link} \\ 
        Guinea-Bissau & Enquête Harmonisée sur le Conditions de Vie des Ménages & 2018-2019 &  5,351  & \href{https://microdata.worldbank.org/index.php/catalog/4293}{Link} \\ 
        India & Socio-Economic Survey Sixty-Eighths round & 2012 &  101,581  & \href{https://catalog.ihsn.org/index.php/catalog/3281}{Link} \\ 
        Indonesia & Social Economic National Survey & 2018 &  295,116  & \href{https://www.bps.go.id/index.php/subjek/81}{Link} \\ 
        Iraq & Household Socio Economic Survey & 2012 &  24,994  & \href{https://microdata.worldbank.org/index.php/catalog/2334}{Link} \\ 
        Israel & Household Budget Survey & 2018 &  8,786  & \href{https://www.cbs.gov.il/en/publications/Pages/2022/Household-Income-and-Expenditure%E2%80%93Data-From-the-2019-Survey-and-2018-Tables-Using-a-New-Estimation-Method.aspx}{Link} \\ 
        Jordan & Household's Expenditures and Income Survey & 2013 &  4,850  & \href{https://dosweb.dos.gov.jo/products/household-income2013-2014/}{Link} \\ 
        Kenya & Integrated Household Budget Survey & 2015-2016 &  21,714  & \href{https://statistics.knbs.or.ke/nada/index.php/catalog/13}{Link} \\ 
        Liberia & Household Income Expenditure Survey & 2016 &  8,332  & \href{https://www.ilo.org/surveyLib/index.php/catalog/6955}{Link} \\ 
        Malawi & Fifth Integrated Household Survey & 2019-2020 &  11,374  & \href{https://microdata.worldbank.org/index.php/catalog/3819}{Link} \\ 
        Maldives & Household Income and Expenditure Survey & 2019 &  4,749  & \href{https://www.ilo.org/surveyLib/index.php/catalog/7598}{Link} \\ 
        Mali & Enquête Harmonisée sur le Conditions de Vie des Ménages & 2018-2019 &  6,602  & \href{https://microdata.worldbank.org/index.php/catalog/4295}{Link} \\ 
        Mexico & Encuesta Nacional de Ingresos y Gastos de los Hogares & 2020 &  74,158  & \href{https://www.inegi.org.mx/rnm/index.php/catalog/685}{Link} \\ 
        Mongolia & Household Socio-Economic Survey & 2016 &  11,197  & \href{http://web.nso.mn/nada/index.php/catalog/HSES/dataset}{Link} \\ 
        Morocco & Enquête Nationale sur la Consommation et les Dépenses des ménages & 2013-2014 &  15,970  & \href{https://www.hcp.ma/Enquete-nationale-sur-la-consommation-et-les-depenses-des-menages\_a95.html}{Link} \\ 
        Myanmar & Poverty and Living Conditions Survey & 2015 &  3,648  & \href{http://hdl.handle.net/10986/29037}{Link} \\ 
        Nicaragua & Encuesta de Medicion de Nivel de Vida & 2014 &  6,850  & \href{https://www.inide.gob.ni/Home/enmv}{Link} \\ 
        Niger & Enquête Harmonisée sur le Conditions de Vie des Ménages & 2018-2019 &  6,024  & \href{https://microdata.worldbank.org/index.php/catalog/4296}{Link} \\ 
        Nigeria & Living Standards Survey & 2018-2019 &  22,110  & \href{https://microdata.worldbank.org/index.php/catalog/5835}{Link} \\ 
        Norway & Forbruksundersøkelsen & 2012 &  3,363  & \href{https://www.ssb.no/inntekt-og-forbruk/artikler-og-publikasjoner/forbruksundersokelsen-2012}{Link} \\ 
        Pakistan & Household Integrated Economic Survey & 2013-2014 &  23,886  & \href{https://www.pbs.gov.pk/publication/household-integrated-economic-survey-hies-2018-19}{Link} \\ 
        Paraguay & Encuesta de Ingresos y Gastos y de Condiciones de Vida & 2011-2012 &  5,410  & \href{https://www.ine.gov.py/microdatos/microdatos.php}{Link} \\ 
        Peru & Encuesta Nacional de Hogares & 2019 &  34,542  & \href{https://www.datosabiertos.gob.pe/dataset/encuesta-nacional-de-hogares-enaho-2019-instituto-nacional-de-estad\%C3\%ADstica-e-inform\%C3\%A1tica-inei}{Link} \\ 
        Philippines & Family Income and Expenditure Survey & 2015 &  41,540  & \href{https://rssoncr.psa.gov.ph/fies}{Link} \\ 
        Russia & Longitudinal Monitoring survey & 2015 &  4,831  & \href{https://www.hse.ru/en/rlms/availability }{Link} \\ 
        Rwanda & Integrated Household Living Conditions Survey & 2016-2017 &  14,577  & \href{https://www.statistics.gov.rw/datasource/integrated-household-living-conditions-survey-eicv}{Link} \\ 
        Senegal & Enquête Harmonisée sur le Conditions de Vie des Ménages & 2018-2019 &  7,156  & \href{https://microdata.worldbank.org/index.php/catalog/4297}{Link} \\ 
        Serbia & Household Buget Survey & 2019 &  6,350  & \href{https://data.stat.gov.rs/?caller=0101\&languageCode=en-US}{Link} \\ 
        South Africa & Living Conditions Survey & 2014-2015 &  22,964  & \href{https://microdata.worldbank.org/index.php/catalog/2882}{Link} \\ 
        Suriname & Survey of Living Conditions & 2016-2017 &  2,025  & \href{https://www.ilo.org/surveyLib/index.php/catalog/7499}{Link} \\ 
        Switzerland & Haushaltsbudgeterhebung & 2015–2017 &  9,955  & \href{https://www.bfs.admin.ch/bfs/de/home/statistiken/wirtschaftliche-soziale-situation-bevoelkerung/erhebungen/habe.html}{Link} \\ 
        Taiwan & Survey of Family Income and Expenditure & 2019 &  16,528  & \href{https://eng.stat.gov.tw/cl.aspx?n=2357}{Link} \\ 
        Thailand & Household Socio-Economic Survey & 2013 &  42,711  & \href{https://www.ilo.org/surveyLib/index.php/catalog/1148}{Link} \\ 
        Togo & Enquête Harmonisée sur le Conditions de Vie des Ménages & 2018-2019 &  6,171  & \href{https://microdata.worldbank.org/index.php/catalog/4298}{Link} \\ 
        Turkey & Household Budget Survey & 2015 &  10,060  & \href{https://data.tuik.gov.tr/Kategori/GetKategori?p=gelir-yasam-tuketim-ve-yoksulluk-107\&dil=2}{Link} \\ 
        Uganda & National Household Survey & 2016-2017 &  15,627  & \href{https://ghdx.healthdata.org/record/uganda-national-household-survey-2016-2017}{Link} \\ 
        United Kingdom & Living Costs and Food Survey & 2018-2019 &  5,425  & \href{https://www.ons.gov.uk/peoplepopulationandcommunity/personalandhouseholdfinances/expenditure/bulletins/familyspendingintheuk/april2018tomarch2019}{Link} \\ 
        Uruguay & Encuesta Nacional de Gastos e Ingresos de los Hogares & 2016-2017 &  6,888  & \href{https://www.ine.gub.uy/web/guest/encuesta-de-gastos-e-ingresos-de-los-hogares-2016}{Link} \\ 
        USA & Consumer Expenditure Surveys & 2019 &  5,588  & \href{https://www.bls.gov/cex/pumd-getting-started-guide.htm}{Link} \\ 
        Vietnam & Household Living Standards Survey & 2012 &  9,378  & \href{https://www.niengiamthongke.net/kh%E1%BA%A3o-s%C3%A1t/vhlss-2020}{Link} \\ 
    \end{longtable}
\end{ThreePartTable}
\endgroup{}
\clearpage

\begingroup\fontsize{9}{11}\selectfont

\begin{ThreePartTable}
\begin{TableNotes}
\item \textit{Note: } 
\item This table provides summary statistics for households in our sample. All values (except observations) are household-weighted averages.
\end{TableNotes}
\begin{longtable}[t]{>{\raggedright\arraybackslash}p{1.5 cm}|>{\raggedleft\arraybackslash}p{1.5 cm}>{\centering\arraybackslash}p{1.5 cm}>{\centering\arraybackslash}p{1.5 cm}>{\centering\arraybackslash}p{1.5 cm}>{\centering\arraybackslash}p{1.5 cm}>{\centering\arraybackslash}p{1.5 cm}>{\centering\arraybackslash}p{1.5 cm}}
\caption{\label{tab:A1}Summary statistics}\\
\toprule
Country & Observations & Average 
Household Size & Urban 
Population & Electricity 
Access & Average 
Household 
Expenditures [USD] & Car 
Ownership & Share of 
Firewood or 
 Charcoal Cons.\\
\midrule
\endfirsthead
\caption[]{Summary statistics \textit{(continued)}}\\
\toprule
Country & Observations & Average 
Household Size & Urban 
Population & Electricity 
Access & Average 
Household 
Expenditures [USD] & Car 
Ownership & Share of 
Firewood or 
 Charcoal Cons.\\
\midrule
\endhead

\endfoot
\bottomrule
\insertTableNotes
\endlastfoot
Argentina & 21,540 & 3.19 &  & 99.9\% & 15,810 & 49\% & 5\%\\
Armenia & 7,776 & 3.63 & 66\% & 99.8\% & 4,779 & 32\% & 1\%\\
Austria & 7,162 & 2.23 &  &  & 38,002 & 77\% & 28\%\\
Bangladesh & 12,240 & 4.50 & 27\% & 55.2\% & 2,438 & 1\% & 39\%\\
Barbados & 2,434 & 2.62 &  & 94.7\% & 17,652 & 52\% & 0\%\\
Belgium & 6,133 & 2.31 & 96\% &  & 32,310 &  & 9\%\\
Benin & 8,012 & 5.21 & 47\% & 33.1\% & 2,690 & 3\% & 97\%\\
Bolivia & 11,859 & 3.34 & 69\% & 94.7\% & 4,089 & 17\% & 12\%\\
Brazil & 57,889 & 3.01 & 86\% & 99.5\% & 11,075 & 46\% & 3\%\\
Bulgaria & 2,964 & 2.37 & 71\% &  & 5,357 &  & 37\%\\
Burkina Faso & 7,010 & 6.51 & 31\% & 24.4\% & 2,660 & 4\% & 92\%\\
Cambodia & 1,206 & 4.34 & 27\% &  & 5,630 & 11\% & 73\%\\
Canada & 4,012 & 2.32 &  &  & 48,762 & 86\% & 0\%\\
Chile & 15,237 & 3.29 &  &  & 19,014 &  & 11\%\\
Colombia & 86,866 & 3.35 & 79\% & 98.3\% & 6,856 & 14\% & 9\%\\
Costa Rica & 7,046 & 3.24 & 71\% & 99.7\% & 11,830 & 45\% & 5\%\\
Côte d’Ivoire & 12,992 & 4.48 & 52\% & 64.1\% & 3,247 & 3\% & 77\%\\
Croatia & 2,029 & 2.89 & 59\% &  & 11,890 &  & 51\%\\
Cyprus & 2,876 & 2.70 & 74\% &  & 26,575 &  & 21\%\\
Czechia & 2,905 & 2.22 & 67\% &  & 11,098 &  & 22\%\\
Denmark & 2,205 & 2.12 & 67\% &  & 37,759 &  & 21\%\\
Dominican Republic & 8,884 & 3.21 & 81\% & 97.5\% & 7,549 & 21\% & 7\%\\
Ecuador & 28,263 & 3.68 & 69\% & 90.5\% & 6,831 & 19\% & 5\%\\
Egypt & 12,485 & 4.17 & 46\% & 99.5\% & 2,449 & 7\% & 0\%\\
El Salvador & 23,622 & 3.67 & 64\% & 95.7\% & 5,758 & 15\% & 12\%\\
Estonia & 3,395 & 2.24 & 51\% &  & 11,994 &  & 33\%\\
Ethiopia & 6,767 & 4.48 & 32\% & 55.9\% & 1,167 & 1\% & 96\%\\
Finland & 3,673 & 2.02 & 71\% &  & 31,618 &  & 43\%\\
France & 16,978 & 2.23 & 69\% &  & 26,865 &  & 0\%\\
Georgia & 13,247 & 2.44 & 61\% & 100\% & 2,436 & 29\% & 5\%\\
Germany & 52,388 & 2.00 & 90\% &  & 28,683 &  & 0\%\\
Ghana & 13,521 & 3.91 & 56\% & 83.1\% & 2,380 & 4\% & 83\%\\
Greece & 6,140 & 2.58 & 72\% &  & 19,219 &  & 28\%\\
Guatemala & 11,535 & 4.77 & 54\% & 81\% & 5,677 & 17\% & 70\%\\
Guinea-Bissau & 5,351 & 8.18 & 47\% & 21.7\% & 3,691 & 3\% & 99\%\\
Hungary & 7,183 & 2.34 & 56\% &  & 8,385 &  & 42\%\\
India & 101,581 & 4.43 & 31\% & 79.9\% & 1,612 & 4\% & 63\%\\
Indonesia & 295,116 & 3.77 & 55\% & 98.5\% & 2,838 & 11\% & 29\%\\
Iraq & 24,994 & 6.73 & 72\% & 99.3\% & 14,006 & 35\% & 3\%\\
Ireland & 6,837 & 2.73 & 65\% &  & 33,816 &  & 31\%\\
Israel & 8,786 & 3.28 & 90\% &  & 39,035 & 72\% & 0\%\\
Italy & 14,636 & 2.37 & 82\% &  & 23,955 &  & 15\%\\
Jordan & 4,850 & 5.11 & 83\% &  & 11,973 & 51\% & 0\%\\
Kenya & 21,714 & 3.98 & 44\% & 56.4\% & 2,468 &  & 82\%\\
Latvia & 3,844 & 2.37 & 56\% &  & 10,195 &  & 0\%\\
Liberia & 8,332 & 4.27 & 52\% & 16.7\% & 2,568 & 2\% & 99\%\\
Lithuania & 3,441 & 2.15 & 47\% &  & 8,884 &  & 33\%\\
Luxembourg & 3,163 & 2.42 & 81\% &  & 50,165 &  & 0\%\\
Malawi & 11,374 & 4.40 & 16\% & 10.7\% & 707 & 2\% & 99\%\\
Maldives & 4,749 & 5.19 &  &  & 20,199 & 5\% & 0\%\\
Mali & 6,602 & 7.14 & 28\% & 27.5\% & 3,458 & 4\% & 99\%\\
Mexico & 74,158 & 3.61 & 77\% & 99.5\% & 5,945 & 38\% & 16\%\\
Mongolia & 11,197 & 3.58 & 66\% &  & 5,939 &  & 44\%\\
Morocco & 15,970 & 4.74 & 65\% &  & 7,374 &  & 21\%\\
Mozambique & 11,335 & 5.01 & 31\% & 25.3\% & 2,872 & 1\% & 96\%\\
Myanmar (Burma) & 3,648 & 4.53 & 29\% & 63\% & 2,347 & 4\% & 88\%\\
Netherlands & 14,407 & 2.19 & 90\% &  & 34,292 &  & 1\%\\
Nicaragua & 6,850 & 4.38 & 60\% & 86.8\% & 4,799 & 8\% & 51\%\\
Niger & 6,024 & 5.96 & 17\% & 15.7\% & 1,901 & 2\% & 97\%\\
Nigeria & 22,110 & 5.08 & 40\% & 63.4\% & 3,013 & 8\% & 70\%\\
Norway & 3,363 & 2.77 & 82\% &  & 53,131 & 88\% & 0\%\\
Pakistan & 23,886 & 6.32 & 37\% & 90.1\% & 3,491 &  & 25\%\\
Paraguay & 5,410 & 3.90 & 61\% & 97.8\% & 7,393 & 25\% & 29\%\\
Peru & 34,542 & 3.56 & 77\% & 95.6\% & 4,673 & 12\% & 15\%\\
Philippines & 41,540 & 4.60 & 44\% & 91.1\% & 4,468 & 7\% & 45\%\\
Poland & 37,115 & 2.80 & 64\% &  & 12,779 &  & 6\%\\
Portugal & 11,392 & 2.53 & 73\% &  & 17,731 &  & 9\%\\
Romania & 30,605 & 2.66 & 58\% &  & 5,094 &  & 9\%\\
Russia & 4,831 & 2.60 &  &  & 7,511 & 41\% & 3\%\\
Rwanda & 14,577 & 4.39 & 19\% &  & 1,262 & 1\% & 41\%\\
Senegal & 7,156 & 8.91 & 53\% & 63.7\% & 6,705 & 5\% & 86\%\\
Serbia & 6,350 & 2.68 & 62\% & 99.9\% & 7,608 & 91\% & 14\%\\
Slovakia & 4,785 & 2.93 & 71\% &  & 12,839 &  & 19\%\\
South Africa & 22,964 & 3.53 & 70\% & 92.7\% & 6,958 & 27\% & 10\%\\
Spain & 22,127 & 2.50 & 75\% &  & 22,569 &  & 0\%\\
Suriname & 2,025 & 3.39 & 72\% &  & 7,589 & 38\% & 0\%\\
Sweden & 2,871 & 2.13 & 45\% &  & 29,741 &  & 0\%\\
Switzerland & 9,955 & 2.14 &  &  & 76,279 & 77\% & 0\%\\
Taiwan & 16,528 & 3.02 &  &  & 20,687 & 61\% & 0\%\\
Thailand & 42,711 & 3.04 & 36\% & 99.8\% & 3,747 & 14\% & 26\%\\
Togo & 6,171 & 4.23 & 47\% & 51.8\% & 2,381 & 3\% & 92\%\\
Turkey & 10,060 & 3.64 & 70\% &  & 9,986 & 39\% & 4\%\\
Uganda & 15,627 & 4.82 & 28\% & 39.2\% & 1,262 & 3\% & 95\%\\
United Kingdom & 5,425 & 2.37 & 77\% &  & 35,305 & 75\% & 1\%\\
United States & 5,588 & 2.44 & 94\% &  & 43,740 &  & 0\%\\
Uruguay & 6,888 & 2.82 & 83\% & 99.7\% & 21,058 & 46\% & 13\%\\
Vietnam & 9,378 & 3.84 & 30\% & 97.8\% & 2,362 & 1\% & 15\%\\*
\end{longtable}
\end{ThreePartTable}
\endgroup{}

\clearpage

\begin{table}[H]

\caption{Average expenditures and average energy expenditure shares per expenditure quintile}
\centering
\resizebox{\linewidth}{!}{
\begin{threeparttable}
\begin{tabular}[t]{l|rrrrrr|rrrrrrl|rrrrrr|rrrrrrl|rrrrrr|rrrrrrl|rrrrrr|rrrrrrl|rrrrrr|rrrrrrl|rrrrrr|rrrrrrl|rrrrrr|rrrrrrl|rrrrrr|rrrrrrl|rrrrrr|rrrrrrl|rrrrrr|rrrrrrl|rrrrrr|rrrrrrl|rrrrrr|rrrrrrl|rrrrrr|rrrrrr}
\toprule
\multicolumn{1}{c}{ } & \multicolumn{6}{c}{Average household expenditures [USD]} & \multicolumn{6}{c}{Average energy expenditure shares} \\
\cmidrule(l{3pt}r{3pt}){2-7} \cmidrule(l{3pt}r{3pt}){8-13}
\multicolumn{2}{c}{ } & \multicolumn{5}{c}{Expenditure quintile} & \multicolumn{1}{c}{ } & \multicolumn{5}{c}{Expenditure quintile} \\
\cmidrule(l{3pt}r{3pt}){3-7} \cmidrule(l{3pt}r{3pt}){9-13}
Country & All & EQ1 & EQ2 & EQ3 & EQ4 & EQ5 & All & EQ1 & EQ2 & EQ3 & EQ4 & EQ5\\
\midrule
ARG & 14,437 & 5,485 & 9,224 & 12,236 & 17,668 & 27,586 & 13.6\% & 17.1\% & 15\% & 13.7\% & 12.5\% & 9.9\%\\
ARM & 3,265 & 933 & 1,601 & 2,149 & 2,864 & 8,780 & 0.2\% & 0.6\% & 0\% & 0.1\% & 0\% & 0\%\\
BEL & 36,432 & 25,561 & 32,483 & 33,973 & 38,374 & 51,782 & 11.9\% & 14.7\% & 12.8\% & 12.5\% & 11\% & 8.5\%\\
BEN & 3,127 & 1,153 & 2,034 & 2,861 & 3,939 & 5,652 & 8.2\% & 6.1\% & 7.3\% & 8.2\% & 8.8\% & 10.5\%\\
BFA & 3,095 & 997 & 1,722 & 2,424 & 3,728 & 6,615 & 6.9\% & 4.3\% & 5.1\% & 5.9\% & 8.1\% & 11\%\\
BGD & 2,125 & 943 & 1,394 & 1,790 & 2,428 & 4,072 & 4.1\% & 4\% & 3.9\% & 4.2\% & 4.3\% & 4.1\%\\
BGR & 6,376 & 3,802 & 4,741 & 5,552 & 7,559 & 10,228 & 18\% & 19.6\% & 18.7\% & 18.7\% & 17.9\% & 15\%\\
BOL & 3,688 & 1,743 & 2,860 & 3,630 & 4,383 & 5,822 & 6.2\% & 6.7\% & 6.3\% & 6.2\% & 6.4\% & 5.7\%\\
BRA & 12,212 & 2,886 & 5,748 & 8,714 & 13,351 & 30,544 & 14.3\% & 21.7\% & 15.3\% & 13.5\% & 11.8\% & 9.3\%\\
BRB & 16,842 & 6,877 & 12,169 & 16,180 & 18,957 & 29,988 & 12.8\% & 12.5\% & 12.8\% & 14\% & 13.4\% & 11.1\%\\
CHL & 19,547 & 7,224 & 12,118 & 16,290 & 22,404 & 39,721 & 8.9\% & 12.7\% & 9.6\% & 8.7\% & 7.7\% & 5.9\%\\
CIV & 3,718 & 1,636 & 2,736 & 3,695 & 4,567 & 5,959 & 5.8\% & 4.9\% & 6\% & 6\% & 5.6\% & 6.6\%\\
COL & 9,732 & 1,974 & 3,812 & 5,633 & 9,012 & 28,230 & 8.5\% & 12.2\% & 10.1\% & 8.7\% & 7\% & 4.5\%\\
CRI & 12,177 & 4,900 & 7,525 & 9,901 & 13,675 & 24,893 & 10.3\% & 12.9\% & 11.2\% & 10.2\% & 9.7\% & 7.7\%\\
CYP & 31,916 & 18,208 & 26,426 & 31,222 & 38,459 & 45,293 & 13.5\% & 16.1\% & 14.9\% & 13.2\% & 12.2\% & 10.9\%\\
CZE & 12,621 & 9,982 & 11,706 & 11,875 & 12,891 & 16,652 & 17.9\% & 19.7\% & 19.2\% & 18.9\% & 16.9\% & 14.9\%\\
DEU & 32,797 & 24,304 & 27,630 & 30,644 & 34,326 & 47,085 & 12.6\% & 15.2\% & 13.5\% & 12.8\% & 11.9\% & 9.8\%\\
DNK & 43,833 & 35,673 & 40,781 & 38,588 & 44,774 & 59,372 & 11.6\% & 13.1\% & 12.2\% & 12\% & 11.2\% & 9.5\%\\
DOM & 7,786 & 4,154 & 5,899 & 7,159 & 8,574 & 13,146 & 9.8\% & 9.4\% & 9.1\% & 9.5\% & 9.2\% & 11.8\%\\
ECU & 10,224 & 2,561 & 4,370 & 6,060 & 8,754 & 29,378 & 5.8\% & 7.5\% & 6\% & 5.6\% & 5.8\% & 4\%\\
ESP & 26,230 & 13,606 & 20,512 & 25,673 & 31,719 & 39,646 & 12\% & 14.3\% & 13.1\% & 12.2\% & 11.1\% & 9.2\%\\
EST & 13,508 & 6,189 & 9,161 & 12,098 & 15,224 & 24,900 & 15.4\% & 19.2\% & 17.1\% & 15.5\% & 13.8\% & 11.2\%\\
ETH & 1,100 & 297 & 600 & 842 & 1,420 & 2,341 & 2.8\% & 1.2\% & 1.1\% & 2.1\% & 4.6\% & 4.7\%\\
FIN & 36,784 & 26,609 & 30,756 & 34,229 & 37,958 & 54,384 & 8.1\% & 10.1\% & 8.9\% & 8.4\% & 7.3\% & 6\%\\
FRA & 31,144 & 19,309 & 26,515 & 30,680 & 34,281 & 44,950 & 10.9\% & 13.4\% & 11.8\% & 11.3\% & 10.1\% & 8.1\%\\
GHA & 2,380 & 1,120 & 1,887 & 2,349 & 2,868 & 3,679 & 8\% & 6\% & 7.9\% & 8.2\% & 9\% & 8.7\%\\
GNB & 4,172 & 1,706 & 2,838 & 3,781 & 4,990 & 7,551 & 4\% & 1.6\% & 1.9\% & 3.5\% & 5.5\% & 7.6\%\\
GRC & 22,591 & 13,041 & 16,819 & 20,443 & 24,340 & 38,320 & 13.8\% & 16.7\% & 15.6\% & 14.2\% & 12.3\% & 10\%\\
GTM & 4,830 & 2,190 & 3,401 & 4,321 & 5,513 & 8,732 & 16\% & 20\% & 16.3\% & 15\% & 14.6\% & 14.3\%\\
HRV & 14,049 & 8,835 & 11,506 & 13,362 & 16,029 & 20,535 & 18.2\% & 20.6\% & 19.7\% & 18.3\% & 17\% & 15.4\%\\
HUN & 9,596 & 6,305 & 8,156 & 9,190 & 10,778 & 13,553 & 20.3\% & 22.2\% & 21.2\% & 21.1\% & 19.6\% & 17.2\%\\
IDN & 2,799 & 1,084 & 1,789 & 2,450 & 3,359 & 5,317 & 12\% & 13.5\% & 12.3\% & 11.7\% & 11.4\% & 10.9\%\\
IND & 1,514 & 719 & 976 & 1,244 & 1,722 & 2,909 & 8.5\% & 6.9\% & 8.1\% & 8.8\% & 9.6\% & 9.1\%\\
IRL & 39,756 & 24,619 & 32,964 & 39,836 & 46,065 & 55,302 & 13.3\% & 16\% & 14.9\% & 13.1\% & 12.5\% & 10\%\\
IRQ & 14,489 & 5,795 & 9,063 & 11,788 & 15,771 & 30,022 & 9.1\% & 11.9\% & 10\% & 9.3\% & 8.1\% & 6.3\%\\
ISR & 39,641 & 20,252 & 30,396 & 38,556 & 46,804 & 62,217 & 7.6\% & 10\% & 8.4\% & 7.3\% & 7.1\% & 5.5\%\\
ITA & 27,761 & 15,011 & 22,139 & 26,943 & 32,672 & 42,044 & 14.4\% & 19\% & 15.9\% & 13.9\% & 12.7\% & 10.4\%\\
KEN & 2,372 & 653 & 1,338 & 2,009 & 2,801 & 5,060 & 6.3\% & 6\% & 6.5\% & 6.8\% & 6.4\% & 5.9\%\\
KHM & 5,426 & 2,165 & 3,449 & 4,515 & 6,341 & 10,674 & 10\% & 12.1\% & 11\% & 9.7\% & 8.7\% & 8.3\%\\
LBR & 16,611 & 899 & 1,731 & 2,570 & 3,460 & 58,528 & 3.4\% & 2.6\% & 2.4\% & 3.3\% & 3.9\% & 4.6\%\\
LTU & 10,073 & 6,009 & 7,382 & 8,790 & 11,923 & 16,266 & 18\% & 18.1\% & 18.3\% & 18.9\% & 19.1\% & 15.8\%\\
LUX & 57,716 & 37,957 & 47,098 & 57,618 & 66,726 & 79,202 & 8.6\% & 11.8\% & 9.4\% & 8.2\% & 7.5\% & 6.1\%\\
LVA & 11,726 & 5,939 & 7,891 & 9,919 & 12,974 & 21,912 & 17.4\% & 19.4\% & 19.5\% & 17.4\% & 17.1\% & 13.6\%\\
MAR & 8,194 & 4,348 & 5,959 & 7,177 & 9,066 & 14,425 & 7.8\% & 10.3\% & 8\% & 7.4\% & 6.9\% & 6.5\%\\
MDV & 19,238 & 10,074 & 15,158 & 18,870 & 23,676 & 28,443 & 6.8\% & 9.8\% & 8\% & 6.5\% & 5.4\% & 4.2\%\\
MEX & 6,846 & 3,038 & 4,878 & 6,181 & 7,814 & 12,319 & 11.2\% & 10.3\% & 11.1\% & 11.8\% & 12\% & 10.8\%\\
MLI & 4,011 & 1,388 & 2,361 & 3,470 & 5,136 & 7,703 & 6.3\% & 4.5\% & 5.5\% & 5.8\% & 7.5\% & 7.9\%\\
MMR & 2,541 & 1,166 & 1,723 & 2,249 & 2,951 & 4,619 & 5.3\% & 4.6\% & 5.1\% & 4.9\% & 5.5\% & 6.1\%\\
MNG & 7,174 & 3,576 & 5,052 & 6,198 & 7,767 & 13,280 & 9.8\% & 10.4\% & 11\% & 10.4\% & 9.8\% & 7.3\%\\
MWI & 790 & 171 & 372 & 553 & 849 & 2,004 & 2.8\% & 0.3\% & 0.7\% & 1.6\% & 3.6\% & 7.6\%\\
NER & 2,206 & 720 & 1,287 & 1,747 & 2,445 & 4,833 & 2.9\% & 0.6\% & 1.4\% & 1.9\% & 3.3\% & 7.1\%\\
NGA & 3,955 & 1,821 & 3,059 & 3,974 & 5,027 & 5,894 & 5\% & 3.6\% & 4.4\% & 5.1\% & 5.7\% & 6\%\\
NIC & 5,581 & 1,463 & 2,647 & 3,739 & 5,472 & 14,591 & 6\% & 4.3\% & 5.2\% & 6.2\% & 6.8\% & 7.5\%\\
NLD & 39,679 & 32,670 & 37,108 & 36,702 & 39,800 & 52,116 & 10\% & 12.5\% & 10.9\% & 9.9\% & 8.9\% & 7.8\%\\
NOR & 64,706 & 35,240 & 51,003 & 62,112 & 73,374 & 101,851 & 10.3\% & 13.6\% & 11.7\% & 10.2\% & 9\% & 7.1\%\\
PAK & 862 & 328 & 508 & 695 & 968 & 1,812 & 2.9\% & 2.6\% & 3\% & 3.2\% & 3.1\% & 2.7\%\\
PER & 4,866 & 1,668 & 3,251 & 4,532 & 5,848 & 9,033 & 8\% & 9\% & 8.7\% & 8\% & 7.6\% & 6.8\%\\
PHL & 4,838 & 1,946 & 2,951 & 4,143 & 5,790 & 9,360 & 5.7\% & 3.6\% & 5\% & 6.1\% & 6.9\% & 7.1\%\\
POL & 14,963 & 8,256 & 10,588 & 12,296 & 15,514 & 28,165 & 14.6\% & 16.1\% & 16.8\% & 16\% & 14.3\% & 9.9\%\\
PRT & 20,299 & 10,263 & 14,940 & 18,552 & 23,271 & 34,476 & 17.2\% & 22.4\% & 19.1\% & 17.2\% & 15.3\% & 12.1\%\\
PRY & 8,371 & 2,793 & 5,437 & 7,872 & 10,284 & 15,473 & 10.4\% & 9.7\% & 11\% & 10.3\% & 10.5\% & 10.5\%\\
ROU & 6,040 & 4,014 & 5,023 & 5,883 & 6,641 & 8,640 & 16.6\% & 13.5\% & 16.6\% & 17.9\% & 18.1\% & 17\%\\
RWA & 1,353 & 439 & 723 & 988 & 1,468 & 3,147 & 3.2\% & 1.2\% & 1.8\% & 2.6\% & 4.2\% & 6\%\\
SEN & 7,639 & 3,495 & 5,748 & 7,795 & 9,351 & 11,806 & 4.9\% & 2.5\% & 4\% & 5.5\% & 5.8\% & 6.5\%\\
SLV & 5,707 & 1,277 & 2,951 & 4,699 & 6,885 & 12,724 & 20\% & 25.9\% & 23\% & 20.4\% & 16.9\% & 13.9\%\\
SUR & 8,490 & 3,295 & 5,660 & 7,658 & 10,050 & 15,804 & 6\% & 8.3\% & 6.7\% & 5.8\% & 5.4\% & 3.9\%\\
SVK & 15,012 & 10,277 & 12,861 & 14,025 & 15,774 & 22,129 & 19.6\% & 23\% & 21.1\% & 20.8\% & 18.5\% & 14.5\%\\
SWE & 33,803 & 23,812 & 29,946 & 33,343 & 35,248 & 46,716 & 10.7\% & 13.1\% & 12.3\% & 10.9\% & 9.1\% & 8.1\%\\
TGO & 2,733 & 939 & 1,766 & 2,619 & 3,620 & 4,725 & 7.6\% & 3.6\% & 6.5\% & 8.2\% & 9.3\% & 10.3\%\\
THA & 3,917 & 1,084 & 1,957 & 3,133 & 4,961 & 8,451 & 19.8\% & 20.4\% & 23\% & 22.6\% & 18.8\% & 14.4\%\\
TUR & 12,906 & 6,400 & 9,001 & 11,595 & 14,389 & 23,145 & 11.4\% & 10.8\% & 12.2\% & 12.1\% & 11.8\% & 10.2\%\\
UGA & 1,494 & 341 & 776 & 1,225 & 1,900 & 3,223 & 5.2\% & 3.9\% & 3.4\% & 4.6\% & 6.4\% & 7.5\%\\
URY & 20,528 & 7,939 & 13,025 & 17,923 & 24,282 & 39,484 & 9.7\% & 13.5\% & 10.8\% & 9.5\% & 8.3\% & 6.6\%\\
ZAF & 7,223 & 1,826 & 2,979 & 4,125 & 6,966 & 20,224 & 11\% & 10.8\% & 10\% & 10.6\% & 11.9\% & 11.6\%\\
\bottomrule
\end{tabular}
\begin{tablenotes}
\item \textit{Note: } 
\item This table shows average household expenditures and average energy expenditure shares for households in our sample. We estimate household-weighted averages for the whole population and per expenditure quintile.
\end{tablenotes}
\end{threeparttable}}
\end{table}

\clearpage

\begin{table}[H]

\caption{Average carbon footprint and average USD/tCO$_{2}$ carbon price incidence per expenditure quintile}
\centering
\resizebox{\linewidth}{!}{
\begin{threeparttable}
\begin{tabular}[t]{l|rrrrrr|rrrrrrl|rrrrrr|rrrrrrl|rrrrrr|rrrrrrl|rrrrrr|rrrrrrl|rrrrrr|rrrrrrl|rrrrrr|rrrrrrl|rrrrrr|rrrrrrl|rrrrrr|rrrrrrl|rrrrrr|rrrrrrl|rrrrrr|rrrrrrl|rrrrrr|rrrrrrl|rrrrrr|rrrrrrl|rrrrrr|rrrrrr}
\toprule
\multicolumn{1}{c}{ } & \multicolumn{6}{c}{Average carbon footprint [tCO$_{2}$]} & \multicolumn{6}{c}{Average incidence from USD 40/tCO$_{2}$ carbon price} \\
\cmidrule(l{3pt}r{3pt}){2-7} \cmidrule(l{3pt}r{3pt}){8-13}
\multicolumn{2}{c}{ } & \multicolumn{5}{c}{Expenditure quintile} & \multicolumn{1}{c}{ } & \multicolumn{5}{c}{Expenditure quintile} \\
\cmidrule(l{3pt}r{3pt}){3-7} \cmidrule(l{3pt}r{3pt}){9-13}
Country & All & EQ1 & EQ2 & EQ3 & EQ4 & EQ5 & All & EQ1 & EQ2 & EQ3 & EQ4 & EQ5\\
\midrule
ARG & 10.4 & 5.0 & 7.7 & 9.6 & 12.8 & 16.6 & 3.19\% & 3.93\% & 3.44\% & 3.18\% & 2.93\% & 2.45\%\\
ARM & 1.2 & 0.2 & 0.3 & 0.5 & 0.7 & 4.1 & 0.92\% & 0.71\% & 0.74\% & 0.81\% & 0.89\% & 1.44\%\\
BEL & 13.4 & 11.4 & 13.3 & 13.4 & 13.9 & 15.1 & 1.63\% & 1.85\% & 1.75\% & 1.71\% & 1.55\% & 1.29\%\\
BEN & 1.3 & 0.4 & 0.7 & 1.0 & 1.4 & 3.1 & 1.47\% & 1.26\% & 1.34\% & 1.37\% & 1.43\% & 1.95\%\\
BFA & 1.9 & 0.5 & 0.9 & 1.3 & 2.1 & 4.7 & 2.16\% & 1.98\% & 2.02\% & 2.06\% & 2.17\% & 2.56\%\\
BGD & 0.9 & 0.3 & 0.4 & 0.6 & 1.0 & 1.9 & 1.48\% & 1.2\% & 1.24\% & 1.38\% & 1.63\% & 1.93\%\\
BGR & 4.7 & 2.8 & 3.4 & 4.5 & 5.7 & 7.1 & 2.94\% & 2.83\% & 2.84\% & 3.09\% & 3.05\% & 2.88\%\\
BOL & 2.3 & 1.2 & 1.9 & 2.4 & 2.8 & 3.3 & 2.64\% & 2.84\% & 2.72\% & 2.67\% & 2.62\% & 2.36\%\\
BRA & 5.7 & 1.8 & 3.1 & 4.6 & 6.7 & 12.4 & 2.17\% & 2.78\% & 2.23\% & 2.11\% & 1.98\% & 1.73\%\\
BRB & 9.9 & 4.4 & 7.6 & 10.6 & 12.0 & 14.8 & 2.49\% & 2.65\% & 2.58\% & 2.66\% & 2.5\% & 2.09\%\\
CHL & 7.9 & 4.1 & 5.8 & 7.2 & 9.2 & 13.3 & 1.85\% & 2.41\% & 2\% & 1.82\% & 1.65\% & 1.37\%\\
CIV & 1.8 & 0.8 & 1.3 & 1.7 & 2.0 & 3.0 & 1.8\% & 1.89\% & 1.84\% & 1.77\% & 1.69\% & 1.79\%\\
COL & 3.8 & 1.2 & 2.2 & 2.9 & 4.0 & 8.5 & 2.04\% & 2.52\% & 2.32\% & 2.11\% & 1.82\% & 1.44\%\\
CRI & 3.5 & 1.4 & 2.4 & 3.0 & 4.3 & 6.2 & 1.16\% & 1.14\% & 1.24\% & 1.19\% & 1.2\% & 1.04\%\\
CYP & 17.2 & 11.8 & 16.3 & 17.2 & 19.8 & 21.0 & 2.32\% & 2.61\% & 2.51\% & 2.29\% & 2.18\% & 2.01\%\\
CZE & 10.8 & 9.8 & 10.6 & 10.9 & 10.6 & 12.0 & 3.65\% & 4.11\% & 3.76\% & 3.87\% & 3.48\% & 3.04\%\\
DEU & 11.9 & 10.6 & 10.9 & 11.5 & 12.3 & 14.4 & 1.46\% & 1.68\% & 1.49\% & 1.45\% & 1.41\% & 1.26\%\\
DNK & 15.2 & 14.8 & 15.0 & 13.7 & 14.9 & 17.7 & 1.47\% & 1.73\% & 1.54\% & 1.46\% & 1.36\% & 1.25\%\\
DOM & 4.1 & 1.8 & 2.7 & 3.5 & 4.2 & 8.2 & 1.92\% & 1.78\% & 1.8\% & 1.88\% & 1.86\% & 2.29\%\\
ECU & 3.2 & 1.4 & 2.1 & 2.7 & 3.7 & 6.0 & 1.87\% & 2.5\% & 2\% & 1.84\% & 1.77\% & 1.22\%\\
ESP & 10.5 & 6.1 & 9.2 & 11.0 & 12.7 & 13.7 & 1.67\% & 1.8\% & 1.79\% & 1.73\% & 1.6\% & 1.41\%\\
EST & 8.5 & 4.6 & 6.5 & 8.2 & 9.5 & 13.9 & 2.72\% & 3\% & 2.94\% & 2.72\% & 2.56\% & 2.39\%\\
ETH & 0.1 & 0.0 & 0.1 & 0.1 & 0.1 & 0.2 & 0.4\% & 0.46\% & 0.4\% & 0.37\% & 0.38\% & 0.38\%\\
FIN & 12.0 & 9.3 & 10.7 & 12.2 & 12.4 & 15.3 & 1.32\% & 1.4\% & 1.36\% & 1.4\% & 1.28\% & 1.16\%\\
FRA & 10.6 & 7.9 & 10.3 & 11.5 & 11.3 & 12.2 & 1.46\% & 1.68\% & 1.57\% & 1.52\% & 1.37\% & 1.15\%\\
GHA & 0.7 & 0.3 & 0.5 & 0.7 & 1.0 & 1.3 & 1.11\% & 0.86\% & 0.99\% & 1.08\% & 1.25\% & 1.35\%\\
GNB & 1.2 & 0.3 & 0.6 & 0.9 & 1.4 & 2.9 & 0.98\% & 0.73\% & 0.76\% & 0.92\% & 1.09\% & 1.4\%\\
GRC & 14.5 & 9.9 & 12.1 & 14.0 & 15.5 & 20.8 & 2.75\% & 3.11\% & 2.94\% & 2.8\% & 2.6\% & 2.3\%\\
GTM & 2.3 & 0.5 & 1.1 & 1.8 & 2.7 & 5.2 & 1.59\% & 0.96\% & 1.22\% & 1.59\% & 1.92\% & 2.25\%\\
HRV & 8.4 & 5.0 & 7.3 & 8.2 & 9.7 & 11.8 & 2.31\% & 2.05\% & 2.4\% & 2.35\% & 2.37\% & 2.37\%\\
HUN & 6.2 & 3.9 & 5.5 & 6.3 & 7.1 & 8.1 & 2.56\% & 2.44\% & 2.64\% & 2.72\% & 2.6\% & 2.4\%\\
IDN & 2.6 & 0.9 & 1.6 & 2.3 & 3.2 & 5.2 & 3.79\% & 3.67\% & 3.62\% & 3.74\% & 3.89\% & 4.01\%\\
IND & 1.5 & 0.7 & 1.0 & 1.3 & 1.8 & 2.7 & 4.08\% & 4.2\% & 4.2\% & 4.16\% & 4.07\% & 3.77\%\\
IRL & 20.1 & 15.2 & 19.1 & 20.7 & 23.3 & 22.2 & 2.3\% & 2.79\% & 2.55\% & 2.24\% & 2.18\% & 1.71\%\\
IRQ & 8.2 & 3.9 & 5.8 & 7.4 & 9.4 & 14.3 & 2.53\% & 2.83\% & 2.63\% & 2.57\% & 2.45\% & 2.15\%\\
ISR & 17.2 & 11.8 & 15.6 & 17.6 & 19.7 & 21.4 & 1.92\% & 2.54\% & 2.08\% & 1.82\% & 1.73\% & 1.42\%\\
ITA & 13.6 & 9.3 & 12.4 & 13.7 & 15.6 & 17.4 & 2.12\% & 2.52\% & 2.28\% & 2.07\% & 1.97\% & 1.73\%\\
KEN & 1.4 & 0.3 & 0.7 & 1.1 & 1.7 & 3.5 & 2.08\% & 1.59\% & 1.92\% & 2.06\% & 2.23\% & 2.59\%\\
KHM & 1.9 & 0.8 & 1.3 & 1.6 & 2.2 & 3.6 & 1.42\% & 1.43\% & 1.5\% & 1.4\% & 1.39\% & 1.36\%\\
LBR & 2.6 & 0.1 & 0.3 & 0.6 & 0.9 & 8.7 & 0.83\% & 0.57\% & 0.68\% & 0.86\% & 0.92\% & 1.05\%\\
LTU & 3.6 & 2.1 & 2.7 & 3.2 & 4.4 & 5.4 & 1.4\% & 1.33\% & 1.37\% & 1.47\% & 1.47\% & 1.35\%\\
LUX & 17.0 & 14.7 & 15.5 & 16.9 & 18.5 & 19.2 & 1.32\% & 1.68\% & 1.42\% & 1.25\% & 1.21\% & 1.04\%\\
LVA & 8.8 & 6.6 & 7.1 & 7.6 & 10.0 & 12.8 & 3.51\% & 4.99\% & 3.83\% & 3.07\% & 3.2\% & 2.47\%\\
MAR & 3.5 & 1.9 & 2.5 & 3.0 & 3.8 & 6.3 & 1.68\% & 1.79\% & 1.67\% & 1.65\% & 1.63\% & 1.68\%\\
MDV & 7.2 & 4.8 & 6.7 & 7.4 & 8.2 & 8.7 & 1.61\% & 1.95\% & 1.8\% & 1.6\% & 1.44\% & 1.25\%\\
MEX & 4.6 & 2.0 & 3.4 & 4.4 & 5.5 & 7.6 & 2.75\% & 2.65\% & 2.79\% & 2.88\% & 2.85\% & 2.56\%\\
MLI & 1.5 & 0.5 & 0.8 & 1.2 & 1.9 & 3.0 & 1.37\% & 1.32\% & 1.34\% & 1.3\% & 1.4\% & 1.48\%\\
MMR & 1.1 & 0.4 & 0.6 & 0.9 & 1.3 & 2.4 & 1.54\% & 1.27\% & 1.41\% & 1.46\% & 1.59\% & 1.99\%\\
MNG & 11.8 & 7.1 & 9.5 & 10.9 & 12.7 & 18.9 & 7.25\% & 8.13\% & 7.82\% & 7.33\% & 6.92\% & 6.05\%\\
MWI & 0.2 & 0.0 & 0.0 & 0.1 & 0.1 & 0.7 & 0.65\% & 0.53\% & 0.52\% & 0.57\% & 0.63\% & 1.01\%\\
NER & 0.7 & 0.2 & 0.3 & 0.4 & 0.6 & 2.0 & 0.99\% & 0.9\% & 0.84\% & 0.88\% & 0.96\% & 1.38\%\\
NGA & 1.5 & 0.4 & 0.9 & 1.4 & 2.1 & 2.6 & 1.37\% & 0.96\% & 1.17\% & 1.41\% & 1.6\% & 1.71\%\\
NIC & 2.5 & 0.4 & 0.9 & 1.5 & 2.7 & 7.3 & 1.57\% & 1\% & 1.28\% & 1.52\% & 1.84\% & 2.24\%\\
NLD & 17.1 & 16.9 & 17.3 & 16.0 & 16.3 & 19.1 & 1.83\% & 2.16\% & 1.95\% & 1.82\% & 1.68\% & 1.53\%\\
NOR & 15.9 & 10.2 & 14.4 & 16.6 & 18.1 & 20.5 & 1.06\% & 1.11\% & 1.14\% & 1.13\% & 1.03\% & 0.88\%\\
PAK & 0.4 & 0.1 & 0.2 & 0.3 & 0.4 & 0.9 & 1.56\% & 1.26\% & 1.42\% & 1.59\% & 1.67\% & 1.85\%\\
PER & 2.2 & 1.0 & 1.8 & 2.2 & 2.6 & 3.5 & 2.16\% & 2.56\% & 2.43\% & 2.17\% & 1.95\% & 1.67\%\\
PHL & 2.2 & 0.6 & 1.1 & 1.8 & 2.8 & 4.8 & 1.64\% & 1.17\% & 1.44\% & 1.7\% & 1.9\% & 2.01\%\\
POL & 17.2 & 10.8 & 15.1 & 16.9 & 19.1 & 23.9 & 5.15\% & 5.05\% & 5.65\% & 5.66\% & 5.35\% & 4.05\%\\
PRT & 11.0 & 7.3 & 9.4 & 10.8 & 12.4 & 15.0 & 2.3\% & 2.81\% & 2.48\% & 2.3\% & 2.12\% & 1.81\%\\
PRY & 3.3 & 1.3 & 2.7 & 3.3 & 3.8 & 5.4 & 1.7\% & 1.77\% & 2.06\% & 1.75\% & 1.53\% & 1.39\%\\
ROU & 3.8 & 1.9 & 3.0 & 3.9 & 4.5 & 5.7 & 2.48\% & 1.93\% & 2.4\% & 2.63\% & 2.73\% & 2.7\%\\
RWA & 0.3 & 0.0 & 0.1 & 0.1 & 0.2 & 1.0 & 0.57\% & 0.43\% & 0.44\% & 0.5\% & 0.58\% & 0.92\%\\
SEN & 2.6 & 0.8 & 1.4 & 2.5 & 3.3 & 4.8 & 1.19\% & 0.82\% & 0.95\% & 1.23\% & 1.38\% & 1.56\%\\
SLV & 2.7 & 0.9 & 1.8 & 2.5 & 3.1 & 5.0 & 2.09\% & 2.75\% & 2.4\% & 2.04\% & 1.75\% & 1.52\%\\
SUR & 3.4 & 1.5 & 2.4 & 3.2 & 4.3 & 5.8 & 1.68\% & 1.8\% & 1.77\% & 1.7\% & 1.66\% & 1.46\%\\
SVK & 7.5 & 6.8 & 7.0 & 7.9 & 7.7 & 8.4 & 2.2\% & 2.66\% & 2.29\% & 2.36\% & 2.06\% & 1.65\%\\
SWE & 7.3 & 6.1 & 7.3 & 7.7 & 7.0 & 8.6 & 0.87\% & 0.97\% & 0.93\% & 0.9\% & 0.79\% & 0.78\%\\
TGO & 0.9 & 0.2 & 0.5 & 0.7 & 1.1 & 1.8 & 1.06\% & 0.76\% & 0.98\% & 1.01\% & 1.13\% & 1.41\%\\
THA & 3.8 & 1.2 & 2.2 & 3.5 & 5.0 & 7.2 & 4.06\% & 4.06\% & 4.47\% & 4.46\% & 3.96\% & 3.36\%\\
TUR & 11.5 & 7.2 & 10.2 & 11.8 & 12.7 & 15.5 & 4.04\% & 4.43\% & 4.74\% & 4.32\% & 3.76\% & 2.97\%\\
UGA & 0.4 & 0.1 & 0.1 & 0.2 & 0.4 & 1.1 & 0.91\% & 1.03\% & 0.75\% & 0.74\% & 0.83\% & 1.2\%\\
URY & 3.7 & 1.8 & 2.6 & 3.4 & 4.5 & 6.4 & 0.78\% & 0.92\% & 0.81\% & 0.77\% & 0.72\% & 0.66\%\\
ZAF & 13.0 & 4.0 & 6.2 & 8.5 & 14.0 & 32.3 & 8.51\% & 9.67\% & 8.79\% & 8.67\% & 8.36\% & 7.03\%\\
\bottomrule
\end{tabular}
\begin{tablenotes}
\item \textit{Note: } 
\item This table shows average carbon footprints in tCO$_{2}$ and average levels of carbon price incidence for households in all countries of our sample. We estimate household-weighted averages for the whole population and per expenditure quintile.
\end{tablenotes}
\end{threeparttable}}
\end{table}

\clearpage

\begin{table}[H]

\caption{\label{tab:A4_CF}Share of households using cooking fuels}
\centering
\resizebox{\linewidth}{!}{
\begin{threeparttable}
\begin{tabular}[t]{l|rrrrr|rrrrr|rrrrrl|rrrrr|rrrrr|rrrrrl|rrrrr|rrrrr|rrrrrl|rrrrr|rrrrr|rrrrrl|rrrrr|rrrrr|rrrrrl|rrrrr|rrrrr|rrrrrl|rrrrr|rrrrr|rrrrrl|rrrrr|rrrrr|rrrrrl|rrrrr|rrrrr|rrrrrl|rrrrr|rrrrr|rrrrrl|rrrrr|rrrrr|rrrrrl|rrrrr|rrrrr|rrrrrl|rrrrr|rrrrr|rrrrrl|rrrrr|rrrrr|rrrrrl|rrrrr|rrrrr|rrrrrl|rrrrr|rrrrr|rrrrr}
\toprule
\multicolumn{1}{c}{ } & \multicolumn{5}{c}{Solid fuels} & \multicolumn{5}{c}{Liquid or gaseous fuels} & \multicolumn{5}{c}{Electricity} \\
\cmidrule(l{3pt}r{3pt}){2-6} \cmidrule(l{3pt}r{3pt}){7-11} \cmidrule(l{3pt}r{3pt}){12-16}
\multicolumn{1}{c}{ } & \multicolumn{5}{c}{Expenditure quintile} & \multicolumn{5}{c}{Expenditure quintile} & \multicolumn{5}{c}{Expenditure quintile} \\
\cmidrule(l{3pt}r{3pt}){2-6} \cmidrule(l{3pt}r{3pt}){7-11} \cmidrule(l{3pt}r{3pt}){12-16}
Country & EQ1 & EQ2 & EQ3 & EQ4 & EQ5 & EQ1 & EQ2 & EQ3 & EQ4 & EQ5 & EQ1 & EQ2 & EQ3 & EQ4 & EQ5\\
\midrule
Argentina & - & - & - & - & - & 99\% & 99\% & 99\% & 98\% & 96\% & 1\% & 0\% & 1\% & 2\% & 4\%\\
Barbados & 0\% & 0\% & - & - & - & 89\% & 95\% & 94\% & 94\% & 88\% & 4\% & 4\% & 5\% & 5\% & 11\%\\
Benin & 100\% & 100\% & 99\% & 96\% & 77\% & - & 0\% & 1\% & 3\% & 23\% & - & - & - & - & -\\
Bolivia & 36\% & 12\% & 6\% & 3\% & 2\% & 63\% & 87\% & 92\% & 93\% & 89\% & - & 0\% & 0\% & 0\% & 1\%\\
Brazil & 3\% & 1\% & 0\% & 0\% & 0\% & 95\% & 98\% & 98\% & 99\% & 98\% & 0\% & 1\% & 1\% & 1\% & 1\%\\
Burkina Faso & 99\% & 100\% & 98\% & 89\% & 43\% & 0\% & 0\% & 1\% & 11\% & 56\% & - & - & - & - & -\\
Cambodia & 82\% & 59\% & 59\% & 44\% & 24\% & 17\% & 41\% & 41\% & 54\% & 74\% & 1\% & 0\% & 1\% & 0\% & 2\%\\
Colombia & 28\% & 10\% & 4\% & 3\% & 1\% & 68\% & 86\% & 92\% & 92\% & 92\% & 3\% & 3\% & 3\% & 3\% & 5\%\\
Costa Rica & 11\% & 4\% & 3\% & 2\% & 1\% & 52\% & 54\% & 47\% & 44\% & 29\% & 36\% & 41\% & 50\% & 54\% & 69\%\\
Côte d’Ivoire & 97\% & 92\% & 73\% & 49\% & 27\% & 2\% & 8\% & 26\% & 49\% & 68\% & - & - & - & - & 0\%\\
Dominican Republic & 10\% & 4\% & 3\% & 2\% & 1\% & 89\% & 94\% & 93\% & 92\% & 91\% & 0\% & - & 0\% & 0\% & 0\%\\
Ecuador & 15\% & 4\% & 2\% & 1\% & 0\% & 80\% & 94\% & 95\% & 96\% & 95\% & 0\% & 0\% & 0\% & 0\% & 1\%\\
Egypt & 0\% & 0\% & 0\% & 0\% & - & 100\% & 100\% & 100\% & 100\% & 100\% & 0\% & 0\% & - & 0\% & 0\%\\
El Salvador & 32\% & 12\% & 7\% & 3\% & 2\% & 62\% & 87\% & 91\% & 95\% & 88\% & 0\% & 0\% & 1\% & 1\% & 4\%\\
Ethiopia & 99\% & 99\% & 98\% & 90\% & 64\% & 0\% & 1\% & 0\% & 1\% & 2\% & 0\% & 0\% & 1\% & 8\% & 29\%\\
Georgia & - & - & - & - & - & 95\% & 97\% & 98\% & 98\% & 99\% & - & - & - & - & -\\
Ghana & 97\% & 87\% & 70\% & 55\% & 31\% & 2\% & 11\% & 25\% & 35\% & 51\% & - & 0\% & 0\% & 0\% & 1\%\\
Guatemala & 98\% & 92\% & 75\% & 58\% & 28\% & 1\% & 7\% & 23\% & 41\% & 68\% & - & - & - & - & -\\
Guinea-Bissau & 100\% & 99\% & 98\% & 99\% & 93\% & - & 0\% & 0\% & 1\% & 6\% & - & - & - & - & -\\
India & 92\% & 84\% & 70\% & 41\% & 9\% & 2\% & 9\% & 25\% & 56\% & 79\% & 0\% & 0\% & 0\% & 0\% & 0\%\\
Indonesia & 42\% & 21\% & 12\% & 6\% & 2\% & 57\% & 78\% & 87\% & 92\% & 92\% & 0\% & 0\% & 0\% & 1\% & 1\%\\
Iraq & 2\% & 0\% & 0\% & 0\% & 0\% & 98\% & 99\% & 100\% & 99\% & 99\% & 1\% & 1\% & 0\% & 1\% & 0\%\\
Jordan & 0\% & 0\% & 0\% & - & - & 100\% & 100\% & 100\% & 100\% & 100\% & - & - & - & - & -\\
Kenya & 98\% & 94\% & 79\% & 52\% & 24\% & 1\% & 5\% & 18\% & 44\% & 70\% & 0\% & 0\% & 1\% & 2\% & 2\%\\
Liberia & 100\% & 99\% & 99\% & 99\% & 98\% & 0\% & 0\% & - & 0\% & 0\% & 0\% & - & - & 0\% & 0\%\\
Malawi & 100\% & 100\% & 100\% & 100\% & 95\% & - & - & - & - & - & - & - & 0\% & 0\% & 5\%\\
Maldives & 2\% & 0\% & 0\% & - & - & 96\% & 96\% & 98\% & 97\% & 95\% & 0\% & 1\% & 1\% & 1\% & 2\%\\
Mali & 100\% & 100\% & 100\% & 99\% & 94\% & - & - & - & 1\% & 5\% & - & - & - & - & -\\
Mexico & 43\% & 17\% & 9\% & 4\% & 2\% & 56\% & 81\% & 90\% & 94\% & 95\% & 1\% & 1\% & 1\% & 1\% & 2\%\\
Mozambique & 100\% & 100\% & 99\% & 99\% & 85\% & 0\% & 0\% & 0\% & 1\% & 11\% & - & 0\% & 0\% & 1\% & 4\%\\
Myanmar (Burma) & 95\% & 90\% & 85\% & 78\% & 66\% & 1\% & 0\% & 1\% & 1\% & 3\% & 3\% & 10\% & 14\% & 19\% & 30\%\\
Nicaragua & 94\% & 75\% & 49\% & 28\% & 10\% & 5\% & 24\% & 50\% & 70\% & 88\% & 0\% & 0\% & 1\% & 1\% & 0\%\\
Niger & 98\% & 99\% & 99\% & 98\% & 81\% & - & - & 0\% & 1\% & 18\% & - & - & - & - & -\\
Nigeria & 98\% & 91\% & 72\% & 47\% & 19\% & 1\% & 9\% & 27\% & 52\% & 77\% & - & - & - & - & -\\
Paraguay & 83\% & 56\% & 28\% & 17\% & 5\% & 12\% & 38\% & 65\% & 74\% & 81\% & 2\% & 4\% & 5\% & 8\% & 10\%\\
Peru & 31\% & 10\% & 4\% & 2\% & 0\% & 60\% & 85\% & 89\% & 87\% & 76\% & 1\% & 3\% & 5\% & 11\% & 21\%\\
Rwanda & - & - & - & - & 0\% & - & - & - & 0\% & 5\% & 99\% & 99\% & 99\% & 100\% & 94\%\\
Senegal & 98\% & 90\% & 71\% & 48\% & 18\% & 2\% & 10\% & 29\% & 51\% & 79\% & - & - & - & 0\% & 0\%\\
South Africa & 28\% & 13\% & 6\% & 2\% & 0\% & 8\% & 9\% & 9\% & 6\% & 8\% & 63\% & 77\% & 85\% & 91\% & 92\%\\
Suriname & - & - & - & - & - & 99\% & 98\% & 99\% & 97\% & 96\% & 0\% & 2\% & 0\% & 2\% & 2\%\\
Thailand & 56\% & 33\% & 16\% & 8\% & 4\% & 38\% & 63\% & 77\% & 76\% & 67\% & 1\% & 1\% & 2\% & 4\% & 7\%\\
Togo & 100\% & 99\% & 96\% & 90\% & 62\% & - & 0\% & 3\% & 9\% & 36\% & - & - & - & - & -\\
Turkey & 16\% & 3\% & 1\% & 1\% & 0\% & 80\% & 96\% & 98\% & 98\% & 98\% & 3\% & 1\% & 0\% & 1\% & 2\%\\
Uganda & 96\% & 98\% & 97\% & 95\% & 85\% & 0\% & 0\% & 0\% & 1\% & 6\% & 0\% & 0\% & 0\% & 1\% & 2\%\\
Uruguay & 3\% & 1\% & 1\% & 1\% & 0\% & 93\% & 96\% & 96\% & 94\% & 90\% & 3\% & 3\% & 3\% & 6\% & 10\%\\
\bottomrule
\end{tabular}
\begin{tablenotes}
\item \textit{Note: } 
\item This table shows the share of households using different cooking fuels, such as solid fuels (e.g., firewood, charcoal, coal, biomass), liquid fuels (e.g., LPG, natural gas, kerosene), or electricity over expenditure quintiles.
\end{tablenotes}
\end{threeparttable}}
\end{table}

\clearpage

\begin{table}[H]

\caption{\label{tab:A5_LF}Share of households using lighting fuels}
\centering
\resizebox{\linewidth}{!}{
\begin{threeparttable}
\begin{tabular}[t]{l|rrrrr|rrrrr|rrrrrl|rrrrr|rrrrr|rrrrrl|rrrrr|rrrrr|rrrrrl|rrrrr|rrrrr|rrrrrl|rrrrr|rrrrr|rrrrrl|rrrrr|rrrrr|rrrrrl|rrrrr|rrrrr|rrrrrl|rrrrr|rrrrr|rrrrrl|rrrrr|rrrrr|rrrrrl|rrrrr|rrrrr|rrrrrl|rrrrr|rrrrr|rrrrrl|rrrrr|rrrrr|rrrrrl|rrrrr|rrrrr|rrrrrl|rrrrr|rrrrr|rrrrrl|rrrrr|rrrrr|rrrrrl|rrrrr|rrrrr|rrrrr}
\toprule
\multicolumn{1}{c}{ } & \multicolumn{5}{c}{Kerosene} & \multicolumn{5}{c}{Electricity} & \multicolumn{5}{c}{Other lighting fuels} \\
\cmidrule(l{3pt}r{3pt}){2-6} \cmidrule(l{3pt}r{3pt}){7-11} \cmidrule(l{3pt}r{3pt}){12-16}
\multicolumn{1}{c}{ } & \multicolumn{5}{c}{Expenditure quintile} & \multicolumn{5}{c}{Expenditure quintile} & \multicolumn{5}{c}{Expenditure quintile} \\
\cmidrule(l{3pt}r{3pt}){2-6} \cmidrule(l{3pt}r{3pt}){7-11} \cmidrule(l{3pt}r{3pt}){12-16}
Country & EQ1 & EQ2 & EQ3 & EQ4 & EQ5 & EQ1 & EQ2 & EQ3 & EQ4 & EQ5 & EQ1 & EQ2 & EQ3 & EQ4 & EQ5\\
\midrule
Barbados & 1\% & 1\% & 1\% & 0\% & - & 88\% & 95\% & 97\% & 97\% & 97\% & 3\% & 3\% & 2\% & 2\% & 1\%\\
Benin & 1\% & 0\% & 1\% & 0\% & 1\% & 20\% & 30\% & 42\% & 60\% & 74\% & 80\% & 70\% & 58\% & 40\% & 25\%\\
Burkina Faso & 0\% & 0\% & 0\% & 0\% & 0\% & 29\% & 38\% & 44\% & 66\% & 91\% & 65\% & 59\% & 52\% & 30\% & 8\%\\
Cambodia & 2\% & 1\% & - & - & 1\% & 85\% & 94\% & 96\% & 96\% & 98\% & 12\% & 5\% & 4\% & 4\% & 1\%\\
Costa Rica & - & - & - & - & - & 99\% & 100\% & 100\% & 100\% & 100\% & - & - & - & - & -\\
Côte d’Ivoire & 0\% & 0\% & 0\% & 0\% & 0\% & 60\% & 74\% & 84\% & 90\% & 95\% & 37\% & 24\% & 15\% & 9\% & 4\%\\
Dominican Republic & 2\% & 2\% & 1\% & 1\% & 0\% & 96\% & 97\% & 98\% & 98\% & 99\% & 2\% & 1\% & 1\% & 1\% & 0\%\\
Ecuador & - & - & - & - & - & 95\% & 99\% & 99\% & 100\% & 100\% & - & - & - & - & -\\
Egypt & 0\% & 0\% & 0\% & 0\% & 1\% & 99\% & 100\% & 99\% & 99\% & 99\% & 0\% & 0\% & 0\% & 0\% & 0\%\\
El Salvador & 4\% & 1\% & 0\% & 0\% & 0\% & 87\% & 96\% & 98\% & 99\% & 99\% & 9\% & 3\% & 2\% & 1\% & 1\%\\
Ethiopia & 30\% & 27\% & 23\% & 14\% & 3\% & 30\% & 43\% & 48\% & 68\% & 90\% & 41\% & 29\% & 29\% & 18\% & 7\%\\
Ghana & 1\% & 1\% & 1\% & 1\% & - & 60\% & 80\% & 88\% & 92\% & 96\% & 36\% & 17\% & 11\% & 7\% & 4\%\\
Guatemala & - & - & - & - & - & 58\% & 82\% & 89\% & 96\% & 97\% & 37\% & 15\% & 9\% & 4\% & 2\%\\
Guinea-Bissau & 1\% & 0\% & 0\% & 0\% & 0\% & 43\% & 46\% & 49\% & 58\% & 72\% & 48\% & 48\% & 47\% & 37\% & 25\%\\
India & 48\% & 28\% & 15\% & 6\% & 2\% & 51\% & 72\% & 85\% & 94\% & 98\% & 0\% & 0\% & 0\% & 0\% & 0\%\\
Indonesia & - & - & - & - & - & 96\% & 98\% & 99\% & 100\% & 100\% & - & - & - & - & -\\
Iraq & 1\% & 0\% & 0\% & 0\% & 0\% & 99\% & 100\% & 100\% & 100\% & 100\% & 0\% & - & - & - & -\\
Kenya & 56\% & 53\% & 37\% & 20\% & 9\% & 23\% & 38\% & 57\% & 75\% & 88\% & 18\% & 8\% & 5\% & 4\% & 2\%\\
Liberia & - & 0\% & 0\% & - & - & 0\% & 3\% & 9\% & 20\% & 38\% & 98\% & 96\% & 90\% & 78\% & 59\%\\
Malawi & 1\% & 1\% & 0\% & 0\% & 0\% & 0\% & 1\% & 3\% & 10\% & 39\% & 97\% & 97\% & 95\% & 88\% & 58\%\\
Mali & 1\% & 1\% & 0\% & 0\% & 0\% & 61\% & 66\% & 68\% & 80\% & 94\% & 27\% & 26\% & 26\% & 18\% & 5\%\\
Mozambique & 11\% & 14\% & 17\% & 17\% & 8\% & 2\% & 5\% & 14\% & 39\% & 74\% & 87\% & 82\% & 68\% & 43\% & 18\%\\
Myanmar (Burma) & 13\% & 5\% & 4\% & 5\% & 2\% & 46\% & 55\% & 61\% & 69\% & 77\% & 41\% & 39\% & 35\% & 27\% & 21\%\\
Nicaragua & 14\% & 4\% & 3\% & 2\% & 0\% & 62\% & 85\% & 92\% & 96\% & 99\% & - & - & - & - & -\\
Niger & 1\% & 0\% & 0\% & 0\% & 0\% & 3\% & 6\% & 13\% & 25\% & 58\% & 95\% & 94\% & 87\% & 74\% & 41\%\\
Peru & 1\% & 0\% & 0\% & 0\% & 0\% & 86\% & 96\% & 98\% & 99\% & 99\% & - & - & - & - & -\\
Rwanda & - & - & - & - & - & 79\% & 83\% & 83\% & 85\% & 92\% & 20\% & 16\% & 16\% & 14\% & 8\%\\
Senegal & 1\% & 1\% & 0\% & 0\% & 0\% & 40\% & 61\% & 83\% & 91\% & 96\% & 55\% & 35\% & 14\% & 8\% & 3\%\\
South Africa & 3\% & 2\% & 2\% & 1\% & 0\% & 85\% & 89\% & 92\% & 96\% & 99\% & 12\% & 8\% & 6\% & 3\% & 0\%\\
Suriname & - & - & - & - & - & 89\% & 96\% & 99\% & 99\% & 99\% & 6\% & 2\% & 1\% & 0\% & 1\%\\
Togo & 0\% & 0\% & 1\% & 0\% & 0\% & 13\% & 36\% & 62\% & 79\% & 89\% & 85\% & 63\% & 37\% & 19\% & 10\%\\
Uganda & 44\% & 50\% & 40\% & 24\% & 10\% & 14\% & 21\% & 33\% & 52\% & 76\% & 8\% & 3\% & 3\% & 5\% & 4\%\\
Uruguay & 0\% & 0\% & - & - & - & 99\% & 100\% & 100\% & 100\% & 100\% & 1\% & 0\% & 0\% & 0\% & 0\%\\
Vietnam & 5\% & 1\% & 0\% & 0\% & 0\% & 94\% & 99\% & 100\% & 100\% & 100\% & - & - & - & - & -\\
\bottomrule
\end{tabular}
\begin{tablenotes}
\item \textit{Note: } 
\item This table shows the share of households using different lighting fuels over expenditure quintiles.
\end{tablenotes}
\end{threeparttable}}
\end{table}

\clearpage

\begin{table}[H]

\caption{Share of households possessing different assets}
\centering
\resizebox{\linewidth}{!}{
\begin{threeparttable}
\begin{tabular}[t]{l|rrr|rrr|rrr|rrr|rrrl|rrr|rrr|rrr|rrr|rrrl|rrr|rrr|rrr|rrr|rrrl|rrr|rrr|rrr|rrr|rrrl|rrr|rrr|rrr|rrr|rrrl|rrr|rrr|rrr|rrr|rrrl|rrr|rrr|rrr|rrr|rrrl|rrr|rrr|rrr|rrr|rrrl|rrr|rrr|rrr|rrr|rrrl|rrr|rrr|rrr|rrr|rrrl|rrr|rrr|rrr|rrr|rrrl|rrr|rrr|rrr|rrr|rrrl|rrr|rrr|rrr|rrr|rrrl|rrr|rrr|rrr|rrr|rrrl|rrr|rrr|rrr|rrr|rrrl|rrr|rrr|rrr|rrr|rrr}
\toprule
\multicolumn{1}{c}{ } & \multicolumn{3}{c}{Car} & \multicolumn{3}{c}{TV} & \multicolumn{3}{c}{Refrigerator} & \multicolumn{3}{c}{AC} & \multicolumn{3}{c}{Washing machine} \\
\cmidrule(l{3pt}r{3pt}){2-4} \cmidrule(l{3pt}r{3pt}){5-7} \cmidrule(l{3pt}r{3pt}){8-10} \cmidrule(l{3pt}r{3pt}){11-13} \cmidrule(l{3pt}r{3pt}){14-16}
Country & All & EQ1 & EQ5 & All & EQ1 & EQ5 & All & EQ1 & EQ5 & All & EQ1 & EQ5 & All & EQ1 & EQ5\\
\midrule
ARG & 26\% & 66\% & 49\% & 96\% & 97\% & 97\% & 95\% & 99\% & 98\% & 33\% & 72\% & 53\% & 81\% & 87\% & 87\%\\
ARM & 35\% & 35\% & 32\% & 99\% & 99\% & 99\% & 95\% & 97\% & 96\% & 5\% & 12\% & 8\% & 93\% & 94\% & 92\%\\
BEN & 0\% & 12\% & 3\% & 3\% & 52\% & 23\% & 0\% & 14\% & 4\% & 0\% & 1\% & 0\% & 0\% & 1\% & 0\%\\
BFA & 0\% & 17\% & 4\% & 3\% & 78\% & 30\% & 0\% & 38\% & 9\% & 0\% & 8\% & 2\% & 0\% & 0\% & 0\%\\
BGD & 0\% & 2\% & 1\% & 9\% & 71\% & 36\% & 0\% & 44\% & 12\% & - & - & - & 0\% & 1\% & 0\%\\
BOL & 5\% & 31\% & 17\% & 61\% & 92\% & 84\% & 28\% & 77\% & 61\% & 2\% & 22\% & 10\% & 2\% & 40\% & 18\%\\
BRA & 17\% & 76\% & 46\% & 94\% & 98\% & 96\% & 96\% & 99\% & 98\% & 6\% & 42\% & 20\% & 38\% & 87\% & 65\%\\
BRB & 21\% & 75\% & 52\% & 34\% & 61\% & 49\% & 84\% & 97\% & 94\% & 2\% & 18\% & 8\% & 60\% & 86\% & 75\%\\
CIV & 0\% & 10\% & 3\% & 15\% & 70\% & 45\% & 1\% & 35\% & 15\% & 0\% & 9\% & 2\% & 1\% & 5\% & 2\%\\
COL & 1\% & 39\% & 14\% & 81\% & 97\% & 92\% & 66\% & 92\% & 83\% & 1\% & 7\% & 4\% & 34\% & 82\% & 61\%\\
CRI & 19\% & 74\% & 45\% & 95\% & 98\% & 97\% & 92\% & 98\% & 96\% & - & - & - & - & - & -\\
DOM & 6\% & 45\% & 21\% & 83\% & 89\% & 87\% & 74\% & 87\% & 83\% & 2\% & 37\% & 14\% & 72\% & 84\% & 80\%\\
ECU & 2\% & 52\% & 19\% & 78\% & 98\% & 91\% & 56\% & 93\% & 80\% & 0\% & 17\% & 6\% & 15\% & 71\% & 45\%\\
ETH & 0\% & 4\% & 1\% & 1\% & 51\% & 18\% & 0\% & 25\% & 7\% & - & - & - & - & - & -\\
GHA & 1\% & 9\% & 4\% & 31\% & 85\% & 64\% & 7\% & 57\% & 36\% & 0\% & 3\% & 1\% & 0\% & 3\% & 1\%\\
GNB & 0\% & 12\% & 3\% & 5\% & 59\% & 26\% & 0\% & 40\% & 13\% & 0\% & 2\% & 1\% & 0\% & 1\% & 0\%\\
GTM & 2\% & 44\% & 17\% & 34\% & 92\% & 71\% & 0\% & 16\% & 5\% & - & - & - & 0\% & 36\% & 11\%\\
IDN & 1\% & 36\% & 11\% & 2\% & 38\% & 14\% & 25\% & 80\% & 57\% & 0\% & 29\% & 8\% & - & - & -\\
IND & 1\% & 15\% & 4\% & 23\% & 82\% & 59\% & 1\% & 58\% & 20\% & 2\% & 30\% & 12\% & 0\% & 32\% & 9\%\\
IRQ & 17\% & 62\% & 35\% & - & - & - & 83\% & 98\% & 92\% & 21\% & 59\% & 41\% & 41\% & 89\% & 69\%\\
ISR & 53\% & 82\% & 72\% & 76\% & 93\% & 88\% & 100\% & 100\% & 100\% & 89\% & 97\% & 93\% & 97\% & 94\% & 96\%\\
KHM & 2\% & 34\% & 11\% & - & - & - & - & - & - & - & - & - & - & - & -\\
LBR & 0\% & 6\% & 2\% & 1\% & 43\% & 18\% & 0\% & 15\% & 4\% & 0\% & 1\% & 0\% & - & - & -\\
MDV & 2\% & 8\% & 5\% & 86\% & 81\% & 87\% & 92\% & 82\% & 90\% & 58\% & 65\% & 68\% & 92\% & 82\% & 90\%\\
MEX & 21\% & 57\% & 40\% & 75\% & 54\% & 67\% & 74\% & 94\% & 88\% & 100\% & 100\% & 100\% & 50\% & 82\% & 71\%\\
MLI & 0\% & 17\% & 4\% & 13\% & 73\% & 37\% & 0\% & 34\% & 10\% & 0\% & 10\% & 2\% & 0\% & 0\% & 0\%\\
MMR & 0\% & 11\% & 4\% & 26\% & 72\% & 49\% & 1\% & 34\% & 14\% & 0\% & 11\% & 3\% & 0\% & 12\% & 4\%\\
MNG & - & - & - & 94\% & 99\% & 97\% & - & - & - & - & - & - & - & - & -\\
MWI & 0\% & 6\% & 2\% & 0\% & 38\% & 11\% & 0\% & 19\% & 4\% & 0\% & 0\% & 0\% & 0\% & 0\% & 0\%\\
NER & 0\% & 9\% & 2\% & 0\% & 41\% & 10\% & 0\% & 18\% & 4\% & 0\% & 4\% & 1\% & 0\% & 0\% & 0\%\\
NGA & 1\% & 19\% & 8\% & 11\% & 76\% & 48\% & 2\% & 49\% & 24\% & 0\% & 9\% & 3\% & 0\% & 8\% & 2\%\\
NIC & 0\% & 29\% & 8\% & 39\% & 95\% & 75\% & 7\% & 79\% & 40\% & 0\% & 6\% & 1\% & 0\% & 31\% & 10\%\\
NOR & 85\% & 93\% & 88\% & 96\% & 98\% & 97\% & 96\% & 97\% & 96\% & - & - & - & 93\% & 96\% & 94\%\\
PAK & 0\% & 16\% & 4\% & 26\% & 83\% & 56\% & 9\% & 79\% & 43\% & 0\% & 18\% & 5\% & 14\% & 79\% & 47\%\\
PER & 2\% & 29\% & 12\% & 52\% & 93\% & 81\% & 15\% & 80\% & 53\% & - & - & - & 3\% & 61\% & 30\%\\
PHL & 0\% & 27\% & 7\% & 45\% & 95\% & 77\% & 6\% & 81\% & 41\% & 0\% & 40\% & 12\% & 4\% & 72\% & 36\%\\
PRY & 2\% & 57\% & 25\% & 71\% & 93\% & 87\% & 59\% & 90\% & 80\% & 2\% & 60\% & 25\% & 40\% & 77\% & 66\%\\
RWA & 0\% & 5\% & 1\% & 0\% & 37\% & 10\% & 0\% & 8\% & 2\% & - & - & - & 0\% & 0\% & 0\%\\
SEN & 0\% & 20\% & 5\% & 17\% & 85\% & 58\% & 4\% & 65\% & 32\% & 0\% & 11\% & 2\% & 0\% & 2\% & 0\%\\
SLV & 1\% & 40\% & 15\% & 68\% & 95\% & 87\% & 36\% & 84\% & 67\% & 0\% & 5\% & 1\% & 2\% & 44\% & 17\%\\
SUR & 29\% & 44\% & 38\% & 66\% & 58\% & 66\% & 67\% & 84\% & 80\% & 10\% & 54\% & 31\% & 69\% & 88\% & 83\%\\
TGO & 0\% & 10\% & 3\% & 3\% & 70\% & 36\% & 0\% & 21\% & 6\% & 0\% & 3\% & 1\% & 0\% & 1\% & 0\%\\
THA & 1\% & 39\% & 14\% & 93\% & 97\% & 97\% & 82\% & 90\% & 90\% & 1\% & 45\% & 18\% & 39\% & 72\% & 63\%\\
TUR & 17\% & 65\% & 39\% & 23\% & 64\% & 41\% & 97\% & 100\% & 99\% & 13\% & 36\% & 21\% & 91\% & 98\% & 96\%\\
UGA & 0\% & 11\% & 3\% & 0\% & 52\% & 17\% & 0\% & 19\% & 5\% & - & - & - & - & - & -\\
URY & 26\% & 67\% & 46\% & 96\% & 97\% & 97\% & 97\% & 99\% & 99\% & 20\% & 60\% & 42\% & 74\% & 90\% & 85\%\\
ZAF & 3\% & 75\% & 27\% & 70\% & 91\% & 79\% & 54\% & 90\% & 69\% & - & - & - & 12\% & 69\% & 34\%\\
\bottomrule
\end{tabular}
\begin{tablenotes}
\item \textit{Note: } 
\item This table shows the share of households possessing differents assets for all households (first and fifth expenditure quintile, respectively) in different countries.
\end{tablenotes}
\end{threeparttable}}
\end{table}

\clearpage

% \foreach \country in {ARG, ARM, BEL, BEN, BFA,
% BGD, BGR, BOL, BRA, BRB, CHL, %CIV,
%  COL, CRI, CYP, CZE, DEU, DNK, DOM, ECU, ESP, EST, ETH, FIN, FRA, GHA, GRC, GTM,
%  HRV, HUN, IDN, IND, IRL, IRQ, ISR, ITA, KEN, KHM, LBR, LTU, LUX, LVA, MAR, MDV, MEX, MLI, MMR, MNG, MWI, NER, NGA,  NIC, NLD, NOR, PAK, PER, PHL, POL, PRT, PRY, ROU, RWA, SEN, SLV, SUR, SVK, SWE, TGO, THA, TUR, UGA, URY, ZAF}{
%   \input{../2_Tables/3c_OLS_Logit_combined/Table_CF_CI_UL20_\country}
% }

% \clearpage

\begingroup\fontsize{9}{11}\selectfont

\begin{ThreePartTable}
\begin{TableNotes}
\item \textit{Note: } 
\item This table shows the median carbon intensity in the first expenditure quintile ($\overline{e}_{r}^{1}$) and in the fifth quintile ($\overline{e}_{r}^{5}$). It displays the difference between the 5\textsuperscript{th} (20\textsuperscript{th}) and 95\textsuperscript{th} (80\textsuperscript{th}) within-quintile percentile for the first ($\overline{H}_{r}^{1}$ and $\overline{H}_{r}^{1*}$) and the fifth quintile ($\overline{H}_{r}^{5}$ and $\overline{H}_{r}^{5*}$). It also compares median carbon intensity in the first income quintile to that in the fifth quintile ($\widehat{V}$$_{r}^{1}$). Lastly it displays our comparison index facilitating the comparison of within-quintile variation between the first and fifth quintile ($\widehat{H}_{r}^{1}$ and $\widehat{H}_{r}^{1*}$, respectively).
\end{TableNotes}
\begin{longtable}[t]{l|cc|cccc|cccl|cc|cccc|cccl|cc|cccc|cccl|cc|cccc|cccl|cc|cccc|cccl|cc|cccc|cccl|cc|cccc|cccl|cc|cccc|cccl|cc|cccc|cccl|cc|cccc|ccc}
\caption{\label{tab:A7}Comparing median carbon intensity and horizontal heterogeneity between first and fifth expenditure quintile}\\
\toprule
\multicolumn{1}{c}{Country} & \multicolumn{1}{c}{$\overline{e}_{r}^{1}$} & \multicolumn{1}{c}{$\overline{e}_{r}^{5}$} & \multicolumn{1}{c}{$\overline{H}_{r}^{1}$} & \multicolumn{1}{c}{$\overline{H}_{r}^{5}$} & \multicolumn{1}{c}{$\overline{H}_{r}^{1*}$} & \multicolumn{1}{c}{$\overline{H}_{r}^{5*}$} & \multicolumn{1}{c}{$\widehat{V}_{r}^{1}$} & \multicolumn{1}{c}{$\widehat{H}_{r}^{1}$} & \multicolumn{1}{c}{$\widehat{H}_{r}^{1*}$} \\
\cmidrule(l{3pt}r{3pt}){1-1} \cmidrule(l{3pt}r{3pt}){2-2} \cmidrule(l{3pt}r{3pt}){3-3} \cmidrule(l{3pt}r{3pt}){4-4} \cmidrule(l{3pt}r{3pt}){5-5} \cmidrule(l{3pt}r{3pt}){6-6} \cmidrule(l{3pt}r{3pt}){7-7} \cmidrule(l{3pt}r{3pt}){8-8} \cmidrule(l{3pt}r{3pt}){9-9} \cmidrule(l{3pt}r{3pt}){10-10}
\endfirsthead
\caption[]{Comparing median carbon intensity and horizontal heterogeneity between first and fifth expenditure quintile \textit{(continued)}}\\
\toprule
\endhead

\endfoot
\bottomrule
\insertTableNotes
\endlastfoot
Argentina & 1.44 & 0.74 & 3.15 & 1.78 & 1.45 & 0.88 & 1.93 & 1.77 & 1.64\\
Armenia & 1.07 & 0.58 & 3.64 & 2.30 & 1.56 & 0.88 & 1.85 & 1.59 & 1.78\\
Australia & 0.91 & 0.50 & 1.41 & 0.80 & 0.64 & 0.35 & 1.83 & 1.75 & 1.80\\
Austria & 0.62 & 0.39 & 1.66 & 0.85 & 0.87 & 0.42 & 1.58 & 1.95 & 2.09\\
Bangladesh & 0.32 & 0.31 & 0.33 & 0.46 & 0.15 & 0.20 & 1.03 & 0.71 & 0.75\\
Barbados & 0.58 & 0.63 & 2.17 & 1.52 & 1.05 & 0.78 & 0.91 & 1.43 & 1.35\\
Belgium & 0.80 & 0.56 & 1.52 & 0.94 & 0.72 & 0.47 & 1.42 & 1.62 & 1.53\\
Benin & 0.18 & 0.38 & 1.26 & 1.42 & 0.71 & 0.61 & 0.49 & 0.88 & 1.17\\
Bolivia & 0.39 & 0.37 & 0.80 & 0.56 & 0.33 & 0.28 & 1.05 & 1.44 & 1.17\\
Brazil & 0.85 & 0.59 & 2.12 & 1.33 & 0.84 & 0.68 & 1.45 & 1.59 & 1.23\\
Bulgaria & 0.60 & 0.66 & 1.43 & 1.11 & 0.50 & 0.49 & 0.92 & 1.29 & 1.04\\
Burkina Faso & 0.02 & 0.46 & 1.58 & 1.41 & 0.65 & 0.52 & 0.04 & 1.12 & 1.25\\
Cambodia & 0.45 & 0.39 & 1.25 & 0.96 & 0.63 & 0.46 & 1.13 & 1.30 & 1.36\\
Canada & 0.66 & 0.56 & 1.73 & 0.77 & 0.79 & 0.36 & 1.19 & 2.24 & 2.22\\
Chile & 0.76 & 0.48 & 1.23 & 0.63 & 0.57 & 0.31 & 1.58 & 1.96 & 1.83\\
Colombia & 0.46 & 0.32 & 1.87 & 0.88 & 0.86 & 0.42 & 1.44 & 2.11 & 2.06\\
Costa Rica & 0.24 & 0.29 & 1.22 & 1.18 & 0.54 & 0.65 & 0.83 & 1.03 & 0.83\\
Côte d’Ivoire & 0.06 & 0.20 & 1.06 & 1.09 & 0.32 & 0.38 & 0.30 & 0.97 & 0.84\\
Croatia & 0.52 & 0.74 & 1.56 & 1.93 & 0.83 & 0.83 & 0.70 & 0.81 & 1.00\\
Cyprus & 0.79 & 0.61 & 1.68 & 1.04 & 0.89 & 0.54 & 1.29 & 1.61 & 1.66\\
Czechia & 1.56 & 1.41 & 2.83 & 2.14 & 1.12 & 0.90 & 1.11 & 1.32 & 1.26\\
Denmark & 0.39 & 0.34 & 1.58 & 1.09 & 0.68 & 0.40 & 1.13 & 1.45 & 1.69\\
Dominican Republic & 0.36 & 0.49 & 1.14 & 1.74 & 0.45 & 0.91 & 0.75 & 0.65 & 0.49\\
Ecuador & 0.34 & 0.27 & 1.02 & 0.78 & 0.37 & 0.37 & 1.26 & 1.31 & 1.00\\
Egypt & 0.59 & 0.61 & 0.46 & 0.73 & 0.21 & 0.30 & 0.98 & 0.63 & 0.71\\
El Salvador & 0.18 & 0.24 & 2.11 & 1.24 & 1.24 & 0.50 & 0.75 & 1.70 & 2.49\\
Estonia & 0.78 & 0.59 & 1.89 & 1.22 & 0.86 & 0.63 & 1.31 & 1.56 & 1.36\\
Ethiopia & 0.07 & 0.08 & 0.34 & 0.11 & 0.11 & 0.04 & 0.98 & 3.11 & 2.66\\
Finland & 0.45 & 0.40 & 1.25 & 0.96 & 0.63 & 0.52 & 1.12 & 1.30 & 1.23\\
France & 0.50 & 0.43 & 1.79 & 1.19 & 0.95 & 0.66 & 1.15 & 1.51 & 1.44\\
Georgia & 0.69 & 0.78 & 2.75 & 2.42 & 1.20 & 1.34 & 0.88 & 1.14 & 0.89\\
Germany & 1.29 & 0.93 & 2.18 & 1.49 & 1.02 & 0.69 & 1.38 & 1.46 & 1.48\\
Ghana & 0.07 & 0.16 & 0.49 & 1.10 & 0.11 & 0.23 & 0.42 & 0.45 & 0.49\\
Greece & 0.77 & 0.59 & 1.32 & 0.92 & 0.69 & 0.46 & 1.30 & 1.44 & 1.50\\
Guatemala & 0.06 & 0.50 & 0.54 & 1.72 & 0.13 & 0.83 & 0.13 & 0.32 & 0.16\\
Guinea-Bissau & 0.05 & 0.18 & 0.68 & 0.96 & 0.16 & 0.29 & 0.29 & 0.70 & 0.55\\
Hungary & 0.86 & 1.03 & 2.23 & 1.96 & 1.14 & 1.01 & 0.84 & 1.14 & 1.13\\
India & 0.98 & 0.99 & 0.81 & 1.11 & 0.40 & 0.50 & 0.99 & 0.73 & 0.80\\
Indonesia & 0.96 & 0.99 & 1.71 & 1.44 & 0.86 & 0.72 & 0.97 & 1.18 & 1.20\\
Iraq & 0.87 & 0.53 & 1.51 & 1.01 & 0.68 & 0.48 & 1.65 & 1.50 & 1.41\\
Ireland & 0.96 & 0.67 & 2.34 & 1.18 & 1.06 & 0.59 & 1.44 & 1.98 & 1.80\\
Israel & 0.65 & 0.38 & 1.52 & 0.93 & 0.69 & 0.46 & 1.72 & 1.62 & 1.49\\
Italy & 0.94 & 0.66 & 1.56 & 0.99 & 0.81 & 0.49 & 1.42 & 1.57 & 1.66\\
Jordan & 0.76 & 1.27 & 1.75 & 2.02 & 0.84 & 1.28 & 0.60 & 0.87 & 0.66\\
Kenya & 0.29 & 0.42 & 0.96 & 1.05 & 0.43 & 0.47 & 0.69 & 0.91 & 0.91\\
Latvia & 0.47 & 0.47 & 2.10 & 1.50 & 1.13 & 0.91 & 0.98 & 1.40 & 1.24\\
Liberia & 0.06 & 0.18 & 0.38 & 0.94 & 0.08 & 0.28 & 0.34 & 0.41 & 0.29\\
Lithuania & 0.24 & 0.45 & 1.42 & 1.39 & 0.69 & 0.70 & 0.54 & 1.02 & 0.98\\
Luxembourg & 0.61 & 0.37 & 1.33 & 0.76 & 0.66 & 0.40 & 1.65 & 1.75 & 1.67\\
Malawi & 0.01 & 0.02 & 0.02 & 0.32 & 0.01 & 0.04 & 0.37 & 0.07 & 0.15\\
Maldives & 0.29 & 0.18 & 0.50 & 0.36 & 0.22 & 0.17 & 1.57 & 1.42 & 1.24\\
Mali & 0.02 & 0.35 & 1.39 & 1.06 & 0.73 & 0.60 & 0.05 & 1.31 & 1.21\\
Mexico & 0.73 & 1.03 & 2.05 & 1.95 & 0.94 & 1.00 & 0.72 & 1.05 & 0.93\\
Mongolia & 1.20 & 0.52 & 3.84 & 1.98 & 2.29 & 0.93 & 2.30 & 1.94 & 2.46\\
Morocco & 0.63 & 0.53 & 0.75 & 0.81 & 0.32 & 0.39 & 1.19 & 0.93 & 0.82\\
Mozambique & 0.03 & 0.15 & 1.30 & 2.01 & 0.21 & 0.55 & 0.20 & 0.65 & 0.38\\
Myanmar (Burma) & 0.25 & 0.46 & 0.90 & 1.78 & 0.39 & 0.72 & 0.54 & 0.51 & 0.55\\
Netherlands & 0.93 & 0.65 & 1.18 & 0.89 & 0.59 & 0.45 & 1.42 & 1.32 & 1.32\\
Nicaragua & 0.03 & 0.26 & 0.46 & 1.33 & 0.14 & 0.47 & 0.11 & 0.34 & 0.29\\
Niger & 0.03 & 0.09 & 0.10 & 0.82 & 0.03 & 0.34 & 0.35 & 0.12 & 0.09\\
Nigeria & 0.18 & 0.54 & 0.82 & 0.94 & 0.39 & 0.43 & 0.34 & 0.87 & 0.90\\
Norway & 0.62 & 0.40 & 2.00 & 1.07 & 1.14 & 0.55 & 1.54 & 1.87 & 2.05\\
Pakistan & 0.29 & 0.70 & 0.83 & 1.05 & 0.40 & 0.54 & 0.41 & 0.79 & 0.75\\
Paraguay & 0.37 & 0.45 & 1.96 & 1.14 & 0.88 & 0.53 & 0.82 & 1.71 & 1.65\\
Peru & 0.70 & 0.47 & 2.51 & 0.84 & 1.31 & 0.39 & 1.50 & 2.97 & 3.34\\
Philippines & 0.26 & 0.52 & 0.50 & 0.71 & 0.20 & 0.33 & 0.50 & 0.71 & 0.61\\
Poland & 1.27 & 0.90 & 2.55 & 2.42 & 0.96 & 0.64 & 1.42 & 1.05 & 1.51\\
Portugal & 1.15 & 0.72 & 1.78 & 1.09 & 0.91 & 0.57 & 1.60 & 1.63 & 1.60\\
Romania & 0.49 & 0.75 & 1.37 & 1.75 & 0.61 & 0.86 & 0.66 & 0.78 & 0.71\\
Russia & 1.08 & 1.13 & 2.60 & 1.88 & 1.01 & 0.97 & 0.96 & 1.38 & 1.04\\
Rwanda & 0.00 & 0.02 & 0.04 & 0.68 & 0.00 & 0.04 & 0.17 & 0.06 & 0.10\\
Senegal & 0.07 & 0.28 & 0.51 & 0.98 & 0.15 & 0.28 & 0.27 & 0.53 & 0.53\\
Serbia & 0.73 & 0.78 & 1.26 & 2.29 & 0.55 & 0.54 & 0.94 & 0.55 & 1.03\\
Slovakia & 0.83 & 0.67 & 3.06 & 1.74 & 1.54 & 0.79 & 1.24 & 1.76 & 1.95\\
South Africa & 1.86 & 1.99 & 2.49 & 2.38 & 1.09 & 1.11 & 0.93 & 1.05 & 0.98\\
Spain & 0.62 & 0.59 & 1.57 & 1.04 & 0.76 & 0.55 & 1.05 & 1.50 & 1.38\\
Suriname & 0.18 & 0.12 & 0.83 & 0.54 & 0.31 & 0.16 & 1.53 & 1.54 & 1.94\\
Sweden & 0.45 & 0.38 & 1.65 & 0.98 & 0.91 & 0.58 & 1.19 & 1.69 & 1.56\\
Switzerland & 0.23 & 0.19 & 0.89 & 0.62 & 0.38 & 0.23 & 1.23 & 1.42 & 1.62\\
Taiwan & 1.13 & 1.16 & 0.93 & 1.12 & 0.47 & 0.63 & 0.98 & 0.84 & 0.74\\
Thailand & 1.45 & 1.51 & 2.23 & 2.10 & 1.20 & 1.28 & 0.96 & 1.06 & 0.93\\
Togo & 0.00 & 0.14 & 1.17 & 1.64 & 0.03 & 0.74 & 0.02 & 0.71 & 0.04\\
Turkey & 1.33 & 1.25 & 4.52 & 2.36 & 2.18 & 0.95 & 1.06 & 1.91 & 2.29\\
Uganda & 0.04 & 0.09 & 1.17 & 1.12 & 0.40 & 0.28 & 0.41 & 1.05 & 1.45\\
United Kingdom & 0.81 & 0.54 & 2.27 & 1.11 & 1.28 & 0.54 & 1.51 & 2.05 & 2.35\\
United States & 1.12 & 0.73 & 1.94 & 1.02 & 0.96 & 0.46 & 1.54 & 1.89 & 2.07\\
Uruguay & 0.24 & 0.20 & 0.88 & 0.56 & 0.35 & 0.29 & 1.21 & 1.57 & 1.24\\
Vietnam & 0.56 & 0.46 & 0.50 & 0.41 & 0.23 & 0.21 & 1.20 & 1.23 & 1.12\\*
\end{longtable}
\end{ThreePartTable}
\endgroup{}

\clearpage

\begingroup\fontsize{8}{10}\selectfont

\begin{ThreePartTable}
\begin{TableNotes}
\item \textit{Note: } 
\item This table shows performance metrics for boosted regression tree models including exclusively household expenditures ('Sparse model') and including all features available features ('Rich model') and . MAE is the mean absolute error of predictions; RMSE is the root mean squared error of predictions; R\textsuperscript{2} is the squared correlation of prediction errors. Unit of MAE and RMSE is kgCO\textsubscript{2} per US-\$. We show MAE, RMSE and R\textsuperscript{2} for five-fold cross validation on the entire dataset. 
\end{TableNotes}
\begin{longtable}[t]{l|r|rrr|rrrl|r|rrr|rrrl|r|rrr|rrrl|r|rrr|rrrl|r|rrr|rrrl|r|rrr|rrrl|r|rrr|rrrl|r|rrr|rrr}
\caption{\label{tab:A8}Evaluation of boosted regression tree models}\\
\toprule
\multicolumn{2}{c}{ } & \multicolumn{3}{c}{Sparse model} & \multicolumn{3}{c}{Rich model} \\
\cmidrule(l{3pt}r{3pt}){3-5} \cmidrule(l{3pt}r{3pt}){6-8}
Country & Mean & MAE & RMSE & R\\textsuperscript{2} & MAE & RMSE & R\\textsuperscript{2}\\
\midrule
\endfirsthead
\caption[]{Evaluation of boosted regression tree models \textit{(continued)}}\\
\toprule
Country & Mean & MAE & RMSE & R\\textsuperscript{2} & MAE & RMSE & R\\textsuperscript{2}\\
\midrule
\endhead

\endfoot
\bottomrule
\insertTableNotes
\endlastfoot
Argentina & 1.28 & 0.57 & 0.82 & 0.14 & 0.50 & 0.73 & 0.27\\
Armenia & 1.22 & 0.70 & 1.09 & 0.04 & 0.60 & 0.97 & 0.26\\
Australia & 0.76 & 0.26 & 0.35 & 0.17 & 0.25 & 0.34 & 0.21\\
Austria & 0.60 & 0.29 & 0.40 & 0.07 & 0.26 & 0.37 & 0.21\\
Bangladesh & 0.32 & 0.09 & 0.14 & 0.02 & 0.08 & 0.13 & 0.16\\
Barbados & 0.86 & 0.48 & 0.68 & 0.02 & 0.40 & 0.60 & 0.24\\
Belgium & 0.75 & 0.28 & 0.37 & 0.09 & 0.27 & 0.36 & 0.13\\
Benin & 0.40 & 0.30 & 0.42 & 0.04 & 0.22 & 0.34 & 0.35\\
Bolivia & 0.43 & 0.16 & 0.22 & 0.03 & 0.12 & 0.17 & 0.41\\
Brazil & 0.84 & 0.42 & 0.65 & 0.05 & 0.38 & 0.60 & 0.19\\
Bulgaria & 0.81 & 0.37 & 0.82 & 0.01 & 0.37 & 0.83 & 0.01\\
Burkina Faso & 0.40 & 0.30 & 0.42 & 0.05 & 0.19 & 0.31 & 0.48\\
Cambodia & 0.50 & 0.26 & 0.37 & 0.01 & 0.22 & 0.33 & 0.22\\
Canada & 0.67 & 0.27 & 0.38 & 0.02 & 0.24 & 0.35 & 0.16\\
Chile & 0.69 & 0.24 & 0.36 & 0.06 & 0.24 & 0.35 & 0.12\\
Colombia & 0.57 & 0.31 & 0.48 & 0.02 & 0.27 & 0.43 & 0.20\\
Costa Rica & 0.43 & 0.35 & 0.47 & 0.02 & 0.27 & 0.41 & 0.26\\
Côte d’Ivoire & 0.27 & 0.26 & 0.38 & 0.01 & 0.17 & 0.29 & 0.43\\
Croatia & 0.78 & 0.42 & 0.61 & 0.06 & 0.42 & 0.61 & 0.06\\
Cyprus & 0.75 & 0.32 & 0.43 & 0.05 & 0.31 & 0.42 & 0.08\\
Czechia & 1.72 & 0.57 & 0.94 & 0.01 & 0.54 & 0.89 & 0.11\\
Denmark & 0.50 & 0.33 & 0.46 & 0.00 & 0.31 & 0.44 & 0.05\\
Dominican Republic & 0.54 & 0.32 & 0.45 & 0.04 & 0.22 & 0.35 & 0.42\\
Ecuador & 0.36 & 0.19 & 0.36 & 0.03 & 0.12 & 0.26 & 0.46\\
Egypt & 0.62 & 0.12 & 0.18 & 0.03 & 0.11 & 0.15 & 0.27\\
El Salvador & 0.51 & 0.37 & 0.54 & 0.02 & 0.26 & 0.45 & 0.31\\
Estonia & 0.80 & 0.37 & 0.49 & 0.01 & 0.36 & 0.48 & 0.03\\
Ethiopia & 0.10 & 0.04 & 0.08 & 0.01 & 0.03 & 0.07 & 0.19\\
Finland & 0.56 & 0.30 & 0.39 & 0.01 & 0.28 & 0.38 & 0.08\\
France & 0.65 & 0.41 & 0.56 & 0.02 & 0.39 & 0.54 & 0.08\\
Georgia & 1.04 & 0.66 & 0.93 & 0.02 & 0.51 & 0.76 & 0.32\\
Germany & 1.21 & 0.43 & 0.61 & 0.02 & 0.41 & 0.57 & 0.10\\
Ghana & 0.20 & 0.16 & 0.30 & 0.04 & 0.12 & 0.24 & 0.36\\
Greece & 0.77 & 0.29 & 0.39 & 0.02 & 0.27 & 0.37 & 0.08\\
Guatemala & 0.40 & 0.28 & 0.42 & 0.13 & 0.19 & 0.32 & 0.48\\
Guinea-Bissau & 0.20 & 0.16 & 0.27 & 0.06 & 0.14 & 0.24 & 0.22\\
Hungary & 1.16 & 0.52 & 0.67 & 0.00 & 0.51 & 0.68 & 0.03\\
India & 1.07 & 0.25 & 0.38 & 0.01 & 0.19 & 0.29 & 0.41\\
Indonesia & 1.05 & 0.36 & 0.49 & 0.02 & 0.28 & 0.39 & 0.37\\
Iraq & 0.80 & 0.29 & 0.44 & 0.20 & 0.28 & 0.42 & 0.29\\
Ireland & 0.95 & 0.42 & 0.65 & 0.06 & 0.40 & 0.61 & 0.14\\
Israel & 0.62 & 0.30 & 0.42 & 0.03 & 0.27 & 0.37 & 0.23\\
Italy & 0.85 & 0.32 & 0.42 & 0.03 & 0.31 & 0.40 & 0.10\\
Jordan & 1.13 & 0.50 & 0.64 & 0.03 & 0.30 & 0.41 & 0.59\\
Kenya & 0.42 & 0.24 & 0.39 & 0.02 & 0.21 & 0.36 & 0.15\\
Latvia & 0.68 & 0.49 & 0.63 & 0.03 & 0.46 & 0.60 & 0.13\\
Liberia & 0.20 & 0.15 & 0.26 & 0.09 & 0.15 & 0.25 & 0.13\\
Lithuania & 0.47 & 0.37 & 0.53 & 0.05 & 0.36 & 0.52 & 0.07\\
Luxembourg & 0.54 & 0.22 & 0.30 & 0.13 & 0.22 & 0.29 & 0.16\\
Malawi & 0.03 & 0.04 & 0.13 & 0.04 & 0.03 & 0.12 & 0.20\\
Maldives & 0.26 & 0.11 & 0.15 & 0.01 & 0.10 & 0.14 & 0.15\\
Mali & 0.36 & 0.30 & 0.39 & 0.02 & 0.20 & 0.30 & 0.41\\
Mexico & 1.10 & 0.54 & 0.76 & 0.01 & 0.42 & 0.62 & 0.31\\
Mongolia & 1.17 & 0.77 & 1.09 & 0.05 & 0.70 & 0.99 & 0.19\\
Morocco & 0.62 & 0.17 & 0.25 & 0.01 & 0.16 & 0.24 & 0.07\\
Mozambique & 0.25 & 0.28 & 0.51 & 0.14 & 0.27 & 0.50 & 0.18\\
Myanmar (Burma) & 0.46 & 0.27 & 0.41 & 0.05 & 0.25 & 0.38 & 0.17\\
Netherlands & 0.84 & 0.25 & 0.33 & 0.09 & 0.23 & 0.31 & 0.16\\
Nicaragua & 0.26 & 0.23 & 0.36 & 0.03 & 0.12 & 0.25 & 0.51\\
Niger & 0.10 & 0.11 & 0.23 & 0.17 & 0.07 & 0.17 & 0.52\\
Nigeria & 0.44 & 0.23 & 0.31 & 0.05 & 0.18 & 0.26 & 0.35\\
Norway & 0.65 & 0.37 & 0.53 & 0.04 & 0.33 & 0.47 & 0.23\\
Pakistan & 0.57 & 0.24 & 0.32 & 0.09 & 0.22 & 0.29 & 0.25\\
Paraguay & 0.59 & 0.35 & 0.55 & 0.00 & 0.29 & 0.48 & 0.23\\
Peru & 0.72 & 0.37 & 0.60 & 0.08 & 0.24 & 0.43 & 0.53\\
Philippines & 0.44 & 0.15 & 0.20 & 0.16 & 0.12 & 0.17 & 0.45\\
Poland & 1.60 & 0.81 & 2.04 & 0.01 & 0.84 & 2.00 & 0.04\\
Portugal & 1.00 & 0.37 & 0.48 & 0.04 & 0.34 & 0.45 & 0.17\\
Romania & 0.80 & 0.42 & 0.67 & 0.00 & 0.39 & 0.65 & 0.07\\
Russia & 1.29 & 0.51 & 0.83 & 0.01 & 0.41 & 0.71 & 0.28\\
Rwanda & 0.04 & 0.05 & 0.14 & 0.17 & 0.03 & 0.11 & 0.48\\
Senegal & 0.25 & 0.15 & 0.24 & 0.05 & 0.11 & 0.20 & 0.34\\
Serbia & 0.97 & 0.46 & 1.23 & 0.00 & 0.47 & 1.19 & 0.05\\
Slovakia & 1.06 & 0.66 & 1.04 & 0.02 & 0.61 & 0.99 & 0.14\\
South Africa & 2.04 & 0.58 & 0.86 & 0.02 & 0.47 & 0.69 & 0.36\\
Spain & 0.74 & 0.34 & 0.48 & 0.00 & 0.32 & 0.46 & 0.09\\
Suriname & 0.23 & 0.17 & 0.28 & 0.02 & 0.16 & 0.28 & 0.03\\
Sweden & 0.48 & 0.32 & 0.43 & 0.01 & 0.31 & 0.41 & 0.06\\
Switzerland & 0.29 & 0.20 & 0.32 & 0.00 & 0.17 & 0.28 & 0.14\\
Taiwan & 1.17 & 0.25 & 0.31 & 0.01 & 0.20 & 0.25 & 0.36\\
Thailand & 1.58 & 0.51 & 0.66 & 0.08 & 0.42 & 0.56 & 0.33\\
Togo & 0.26 & 0.31 & 0.48 & 0.04 & 0.18 & 0.37 & 0.43\\
Turkey & 1.75 & 0.88 & 1.30 & 0.02 & 0.76 & 1.15 & 0.22\\
Uganda & 0.20 & 0.20 & 0.38 & 0.04 & 0.15 & 0.33 & 0.29\\
United Kingdom & 0.83 & 0.40 & 0.58 & 0.05 & 0.37 & 0.55 & 0.16\\
United States & 0.99 & 0.34 & 0.47 & 0.07 & 0.33 & 0.45 & 0.13\\
Uruguay & 0.28 & 0.18 & 0.27 & 0.00 & 0.13 & 0.21 & 0.35\\
Vietnam & 0.53 & 0.11 & 0.15 & 0.08 & 0.10 & 0.13 & 0.28\\*
\end{longtable}
\end{ThreePartTable}
\endgroup{}

\clearpage

\begin{table}[H]

\caption{Average feature importance across country clusters}
\centering
\resizebox{\linewidth}{!}{
\begin{threeparttable}
\begin{tabular}[t]{lrrrrrrrrrrrrrrrrrrrr}
\toprule
\rotatebox{90}{Cluster} & \rotatebox{90}{Number} & \rotatebox{90}{Average silhouette width} & \rotatebox{90}{Mean carbon intensity} & \rotatebox{90}{Horizontal inequality} & \rotatebox{90}{Vertical inequality} & \rotatebox{90}{HH expenditures} & \rotatebox{90}{HH size} & \rotatebox{90}{Education} & \rotatebox{90}{Gender HHH} & \rotatebox{90}{Sociodemographic} & \rotatebox{90}{Urban} & \rotatebox{90}{Province} & \rotatebox{90}{District} & \rotatebox{90}{Electricity access} & \rotatebox{90}{Cooking fuel} & \rotatebox{90}{Heating fuel} & \rotatebox{90}{Lighting fuel} & \rotatebox{90}{Car own.} & \rotatebox{90}{Motorcycle own.} & \rotatebox{90}{Appliance own.}\\
\midrule
A & 47 & 0.30 & 0.54 & 1.56 & 1.26 & 0.04 & 0.01 & 0.01 & 0.01 & 0.00 & 0.01 & 0.01 & 0.01 & 0.00 & 0.00 & 0.01 & 0.00 & 0.02 & 0.00 & 0.00\\
B & 23 & 0.09 & 0.34 & 0.72 & 0.55 & 0.04 & 0.02 & 0.01 & 0.01 & 0.01 & 0.01 & 0.04 & 0.00 & 0.02 & 0.02 & 0.00 & 0.01 & 0.03 & 0.05 & 0.03\\
C & 17 & -0.04 & 0.81 & 1.51 & 1.21 & 0.09 & 0.02 & 0.01 & 0.00 & 0.01 & 0.02 & 0.05 & 0.01 & 0.01 & 0.04 & 0.00 & 0.00 & 0.06 & 0.02 & 0.04\\
\bottomrule
\end{tabular}
\begin{tablenotes}
\item \textit{Note: } 
\item This table shows the average importance of features in percent (based on absolute average SHAP-values per feature) across all countries from each cluster A to N. Columns 'Mean carbon intensity', 'Horizontal inequality' and 'Vertical inequality' show average values. Column 'number' refers to the number of countries assigned to this cluster.
\end{tablenotes}
\end{threeparttable}}
\end{table}

\clearpage

\begingroup\fontsize{9}{11}\selectfont

\begin{ThreePartTable}
\begin{TableNotes}
\item \textit{Note: } 
\item This table shows feature importance in percent (based on absolute average SHAP-values per feature) across all countries and per cluster. Feature importance is adjusted for model accuracy. Column 'Vertical distribution' shows average values. Column 'number' refers to the number of countries assigned to this cluster.
\end{TableNotes}
\begin{longtable}[t]{>{\raggedright\arraybackslash}p{0.35 cm}>{\raggedright\arraybackslash}p{3 cm}>{\raggedleft\arraybackslash}p{0.8 cm}>{\raggedleft\arraybackslash}p{0.35 cm}>{\raggedleft\arraybackslash}p{0.35 cm}>{\raggedleft\arraybackslash}p{0.35 cm}>{\raggedleft\arraybackslash}p{0.35 cm}>{\raggedleft\arraybackslash}p{0.35 cm}>{\raggedleft\arraybackslash}p{0.35 cm}>{\raggedleft\arraybackslash}p{0.35 cm}>{\raggedleft\arraybackslash}p{0.35 cm}>{\raggedleft\arraybackslash}p{0.35 cm}>{\raggedleft\arraybackslash}p{0.35 cm}>{\raggedleft\arraybackslash}p{0.35 cm}}
\caption{\label{tab:A10}Feature importance across countries by cluster}\\
\toprule
\rotatebox{90}{Cluster} & \rotatebox{90}{Country} & \rotatebox{90}{Silhouette width} & \rotatebox{90}{Vertical distribution} & \rotatebox{90}{HH expenditures} & \rotatebox{90}{Sociodemographic} & \rotatebox{90}{Spatial} & \rotatebox{90}{Electricity access} & \rotatebox{90}{Cooking fuel} & \rotatebox{90}{Heating fuel} & \rotatebox{90}{Lighting fuel} & \rotatebox{90}{Car own.} & \rotatebox{90}{Motorcycle own.} & \rotatebox{90}{Appliance own.}\\
\midrule
\endfirsthead
\caption[]{Feature importance across countries by cluster \textit{(continued)}}\\
\toprule
\rotatebox{90}{Cluster} & \rotatebox{90}{Country} & \rotatebox{90}{Silhouette width} & \rotatebox{90}{Vertical distribution} & \rotatebox{90}{HH expenditures} & \rotatebox{90}{Sociodemographic} & \rotatebox{90}{Spatial} & \rotatebox{90}{Electricity access} & \rotatebox{90}{Cooking fuel} & \rotatebox{90}{Heating fuel} & \rotatebox{90}{Lighting fuel} & \rotatebox{90}{Car own.} & \rotatebox{90}{Motorcycle own.} & \rotatebox{90}{Appliance own.}\\
\midrule
\endhead

\endfoot
\bottomrule
\insertTableNotes
\endlastfoot
\begingroup\fontsize{8}{10}\selectfont A\endgroup & \begingroup\fontsize{8}{10}\selectfont Greece\endgroup & \begingroup\fontsize{8}{10}\selectfont 0.55\endgroup & \begingroup\fontsize{8}{10}\selectfont 1.30\endgroup & \begingroup\fontsize{8}{10}\selectfont 0.02\endgroup & \begingroup\fontsize{8}{10}\selectfont 0.03\endgroup & \begingroup\fontsize{8}{10}\selectfont 0.03\endgroup & \begingroup\fontsize{8}{10}\selectfont \endgroup & \begingroup\fontsize{8}{10}\selectfont \endgroup & \begingroup\fontsize{8}{10}\selectfont \endgroup & \begingroup\fontsize{8}{10}\selectfont \endgroup & \begingroup\fontsize{8}{10}\selectfont \endgroup & \begingroup\fontsize{8}{10}\selectfont \endgroup & \begingroup\fontsize{8}{10}\selectfont \endgroup\\
\begingroup\fontsize{8}{10}\selectfont A\endgroup & \begingroup\fontsize{8}{10}\selectfont Morocco\endgroup & \begingroup\fontsize{8}{10}\selectfont 0.54\endgroup & \begingroup\fontsize{8}{10}\selectfont 1.19\endgroup & \begingroup\fontsize{8}{10}\selectfont 0.02\endgroup & \begingroup\fontsize{8}{10}\selectfont 0.01\endgroup & \begingroup\fontsize{8}{10}\selectfont 0.04\endgroup & \begingroup\fontsize{8}{10}\selectfont \endgroup & \begingroup\fontsize{8}{10}\selectfont \endgroup & \begingroup\fontsize{8}{10}\selectfont \endgroup & \begingroup\fontsize{8}{10}\selectfont \endgroup & \begingroup\fontsize{8}{10}\selectfont \endgroup & \begingroup\fontsize{8}{10}\selectfont \endgroup & \begingroup\fontsize{8}{10}\selectfont \endgroup\\
\begingroup\fontsize{8}{10}\selectfont A\endgroup & \begingroup\fontsize{8}{10}\selectfont Finland\endgroup & \begingroup\fontsize{8}{10}\selectfont 0.54\endgroup & \begingroup\fontsize{8}{10}\selectfont 1.12\endgroup & \begingroup\fontsize{8}{10}\selectfont 0.01\endgroup & \begingroup\fontsize{8}{10}\selectfont 0.03\endgroup & \begingroup\fontsize{8}{10}\selectfont 0.03\endgroup & \begingroup\fontsize{8}{10}\selectfont \endgroup & \begingroup\fontsize{8}{10}\selectfont \endgroup & \begingroup\fontsize{8}{10}\selectfont \endgroup & \begingroup\fontsize{8}{10}\selectfont \endgroup & \begingroup\fontsize{8}{10}\selectfont \endgroup & \begingroup\fontsize{8}{10}\selectfont \endgroup & \begingroup\fontsize{8}{10}\selectfont \endgroup\\
\begingroup\fontsize{8}{10}\selectfont A\endgroup & \begingroup\fontsize{8}{10}\selectfont France\endgroup & \begingroup\fontsize{8}{10}\selectfont 0.53\endgroup & \begingroup\fontsize{8}{10}\selectfont 1.15\endgroup & \begingroup\fontsize{8}{10}\selectfont 0.01\endgroup & \begingroup\fontsize{8}{10}\selectfont 0.02\endgroup & \begingroup\fontsize{8}{10}\selectfont 0.04\endgroup & \begingroup\fontsize{8}{10}\selectfont \endgroup & \begingroup\fontsize{8}{10}\selectfont \endgroup & \begingroup\fontsize{8}{10}\selectfont \endgroup & \begingroup\fontsize{8}{10}\selectfont \endgroup & \begingroup\fontsize{8}{10}\selectfont \endgroup & \begingroup\fontsize{8}{10}\selectfont \endgroup & \begingroup\fontsize{8}{10}\selectfont \endgroup\\
\begingroup\fontsize{8}{10}\selectfont A\endgroup & \begingroup\fontsize{8}{10}\selectfont Sweden\endgroup & \begingroup\fontsize{8}{10}\selectfont 0.53\endgroup & \begingroup\fontsize{8}{10}\selectfont 1.19\endgroup & \begingroup\fontsize{8}{10}\selectfont 0.02\endgroup & \begingroup\fontsize{8}{10}\selectfont 0.01\endgroup & \begingroup\fontsize{8}{10}\selectfont 0.03\endgroup & \begingroup\fontsize{8}{10}\selectfont \endgroup & \begingroup\fontsize{8}{10}\selectfont \endgroup & \begingroup\fontsize{8}{10}\selectfont \endgroup & \begingroup\fontsize{8}{10}\selectfont \endgroup & \begingroup\fontsize{8}{10}\selectfont \endgroup & \begingroup\fontsize{8}{10}\selectfont \endgroup & \begingroup\fontsize{8}{10}\selectfont \endgroup\\
\begingroup\fontsize{8}{10}\selectfont A\endgroup & \begingroup\fontsize{8}{10}\selectfont Denmark\endgroup & \begingroup\fontsize{8}{10}\selectfont 0.52\endgroup & \begingroup\fontsize{8}{10}\selectfont 1.13\endgroup & \begingroup\fontsize{8}{10}\selectfont 0.01\endgroup & \begingroup\fontsize{8}{10}\selectfont 0.03\endgroup & \begingroup\fontsize{8}{10}\selectfont 0.01\endgroup & \begingroup\fontsize{8}{10}\selectfont \endgroup & \begingroup\fontsize{8}{10}\selectfont \endgroup & \begingroup\fontsize{8}{10}\selectfont \endgroup & \begingroup\fontsize{8}{10}\selectfont \endgroup & \begingroup\fontsize{8}{10}\selectfont \endgroup & \begingroup\fontsize{8}{10}\selectfont \endgroup & \begingroup\fontsize{8}{10}\selectfont \endgroup\\
\begingroup\fontsize{8}{10}\selectfont A\endgroup & \begingroup\fontsize{8}{10}\selectfont Cyprus\endgroup & \begingroup\fontsize{8}{10}\selectfont 0.52\endgroup & \begingroup\fontsize{8}{10}\selectfont 1.29\endgroup & \begingroup\fontsize{8}{10}\selectfont 0.03\endgroup & \begingroup\fontsize{8}{10}\selectfont 0.03\endgroup & \begingroup\fontsize{8}{10}\selectfont 0.01\endgroup & \begingroup\fontsize{8}{10}\selectfont \endgroup & \begingroup\fontsize{8}{10}\selectfont \endgroup & \begingroup\fontsize{8}{10}\selectfont \endgroup & \begingroup\fontsize{8}{10}\selectfont \endgroup & \begingroup\fontsize{8}{10}\selectfont \endgroup & \begingroup\fontsize{8}{10}\selectfont \endgroup & \begingroup\fontsize{8}{10}\selectfont \endgroup\\
\begingroup\fontsize{8}{10}\selectfont A\endgroup & \begingroup\fontsize{8}{10}\selectfont Poland\endgroup & \begingroup\fontsize{8}{10}\selectfont 0.52\endgroup & \begingroup\fontsize{8}{10}\selectfont 1.42\endgroup & \begingroup\fontsize{8}{10}\selectfont 0.01\endgroup & \begingroup\fontsize{8}{10}\selectfont 0.01\endgroup & \begingroup\fontsize{8}{10}\selectfont 0.02\endgroup & \begingroup\fontsize{8}{10}\selectfont \endgroup & \begingroup\fontsize{8}{10}\selectfont \endgroup & \begingroup\fontsize{8}{10}\selectfont \endgroup & \begingroup\fontsize{8}{10}\selectfont \endgroup & \begingroup\fontsize{8}{10}\selectfont \endgroup & \begingroup\fontsize{8}{10}\selectfont \endgroup & \begingroup\fontsize{8}{10}\selectfont \endgroup\\
\begingroup\fontsize{8}{10}\selectfont A\endgroup & \begingroup\fontsize{8}{10}\selectfont Italy\endgroup & \begingroup\fontsize{8}{10}\selectfont 0.52\endgroup & \begingroup\fontsize{8}{10}\selectfont 1.42\endgroup & \begingroup\fontsize{8}{10}\selectfont 0.03\endgroup & \begingroup\fontsize{8}{10}\selectfont 0.05\endgroup & \begingroup\fontsize{8}{10}\selectfont 0.03\endgroup & \begingroup\fontsize{8}{10}\selectfont \endgroup & \begingroup\fontsize{8}{10}\selectfont \endgroup & \begingroup\fontsize{8}{10}\selectfont \endgroup & \begingroup\fontsize{8}{10}\selectfont \endgroup & \begingroup\fontsize{8}{10}\selectfont \endgroup & \begingroup\fontsize{8}{10}\selectfont \endgroup & \begingroup\fontsize{8}{10}\selectfont \endgroup\\
\begingroup\fontsize{8}{10}\selectfont A\endgroup & \begingroup\fontsize{8}{10}\selectfont Belgium\endgroup & \begingroup\fontsize{8}{10}\selectfont 0.52\endgroup & \begingroup\fontsize{8}{10}\selectfont 1.42\endgroup & \begingroup\fontsize{8}{10}\selectfont 0.05\endgroup & \begingroup\fontsize{8}{10}\selectfont 0.03\endgroup & \begingroup\fontsize{8}{10}\selectfont 0.05\endgroup & \begingroup\fontsize{8}{10}\selectfont \endgroup & \begingroup\fontsize{8}{10}\selectfont \endgroup & \begingroup\fontsize{8}{10}\selectfont \endgroup & \begingroup\fontsize{8}{10}\selectfont \endgroup & \begingroup\fontsize{8}{10}\selectfont \endgroup & \begingroup\fontsize{8}{10}\selectfont \endgroup & \begingroup\fontsize{8}{10}\selectfont \endgroup\\
\begingroup\fontsize{8}{10}\selectfont A\endgroup & \begingroup\fontsize{8}{10}\selectfont Germany\endgroup & \begingroup\fontsize{8}{10}\selectfont 0.52\endgroup & \begingroup\fontsize{8}{10}\selectfont 1.38\endgroup & \begingroup\fontsize{8}{10}\selectfont 0.02\endgroup & \begingroup\fontsize{8}{10}\selectfont 0.05\endgroup & \begingroup\fontsize{8}{10}\selectfont 0.03\endgroup & \begingroup\fontsize{8}{10}\selectfont \endgroup & \begingroup\fontsize{8}{10}\selectfont \endgroup & \begingroup\fontsize{8}{10}\selectfont \endgroup & \begingroup\fontsize{8}{10}\selectfont \endgroup & \begingroup\fontsize{8}{10}\selectfont \endgroup & \begingroup\fontsize{8}{10}\selectfont \endgroup & \begingroup\fontsize{8}{10}\selectfont \endgroup\\
\begingroup\fontsize{8}{10}\selectfont A\endgroup & \begingroup\fontsize{8}{10}\selectfont Spain\endgroup & \begingroup\fontsize{8}{10}\selectfont 0.50\endgroup & \begingroup\fontsize{8}{10}\selectfont 1.05\endgroup & \begingroup\fontsize{8}{10}\selectfont 0.01\endgroup & \begingroup\fontsize{8}{10}\selectfont 0.03\endgroup & \begingroup\fontsize{8}{10}\selectfont 0.05\endgroup & \begingroup\fontsize{8}{10}\selectfont \endgroup & \begingroup\fontsize{8}{10}\selectfont \endgroup & \begingroup\fontsize{8}{10}\selectfont \endgroup & \begingroup\fontsize{8}{10}\selectfont \endgroup & \begingroup\fontsize{8}{10}\selectfont \endgroup & \begingroup\fontsize{8}{10}\selectfont \endgroup & \begingroup\fontsize{8}{10}\selectfont \endgroup\\
\begingroup\fontsize{8}{10}\selectfont A\endgroup & \begingroup\fontsize{8}{10}\selectfont Ireland\endgroup & \begingroup\fontsize{8}{10}\selectfont 0.49\endgroup & \begingroup\fontsize{8}{10}\selectfont 1.44\endgroup & \begingroup\fontsize{8}{10}\selectfont 0.06\endgroup & \begingroup\fontsize{8}{10}\selectfont 0.04\endgroup & \begingroup\fontsize{8}{10}\selectfont 0.05\endgroup & \begingroup\fontsize{8}{10}\selectfont \endgroup & \begingroup\fontsize{8}{10}\selectfont \endgroup & \begingroup\fontsize{8}{10}\selectfont \endgroup & \begingroup\fontsize{8}{10}\selectfont \endgroup & \begingroup\fontsize{8}{10}\selectfont \endgroup & \begingroup\fontsize{8}{10}\selectfont \endgroup & \begingroup\fontsize{8}{10}\selectfont \endgroup\\
\begingroup\fontsize{8}{10}\selectfont A\endgroup & \begingroup\fontsize{8}{10}\selectfont Czechia\endgroup & \begingroup\fontsize{8}{10}\selectfont 0.49\endgroup & \begingroup\fontsize{8}{10}\selectfont 1.11\endgroup & \begingroup\fontsize{8}{10}\selectfont 0.02\endgroup & \begingroup\fontsize{8}{10}\selectfont 0.02\endgroup & \begingroup\fontsize{8}{10}\selectfont 0.06\endgroup & \begingroup\fontsize{8}{10}\selectfont \endgroup & \begingroup\fontsize{8}{10}\selectfont \endgroup & \begingroup\fontsize{8}{10}\selectfont \endgroup & \begingroup\fontsize{8}{10}\selectfont \endgroup & \begingroup\fontsize{8}{10}\selectfont \endgroup & \begingroup\fontsize{8}{10}\selectfont \endgroup & \begingroup\fontsize{8}{10}\selectfont \endgroup\\
\begingroup\fontsize{8}{10}\selectfont A\endgroup & \begingroup\fontsize{8}{10}\selectfont Estonia\endgroup & \begingroup\fontsize{8}{10}\selectfont 0.49\endgroup & \begingroup\fontsize{8}{10}\selectfont 1.31\endgroup & \begingroup\fontsize{8}{10}\selectfont 0.01\endgroup & \begingroup\fontsize{8}{10}\selectfont 0.01\endgroup & \begingroup\fontsize{8}{10}\selectfont 0.00\endgroup & \begingroup\fontsize{8}{10}\selectfont \endgroup & \begingroup\fontsize{8}{10}\selectfont \endgroup & \begingroup\fontsize{8}{10}\selectfont \endgroup & \begingroup\fontsize{8}{10}\selectfont \endgroup & \begingroup\fontsize{8}{10}\selectfont \endgroup & \begingroup\fontsize{8}{10}\selectfont \endgroup & \begingroup\fontsize{8}{10}\selectfont \endgroup\\
\begingroup\fontsize{8}{10}\selectfont A\endgroup & \begingroup\fontsize{8}{10}\selectfont Suriname\endgroup & \begingroup\fontsize{8}{10}\selectfont 0.48\endgroup & \begingroup\fontsize{8}{10}\selectfont 1.53\endgroup & \begingroup\fontsize{8}{10}\selectfont 0.00\endgroup & \begingroup\fontsize{8}{10}\selectfont 0.01\endgroup & \begingroup\fontsize{8}{10}\selectfont 0.01\endgroup & \begingroup\fontsize{8}{10}\selectfont \endgroup & \begingroup\fontsize{8}{10}\selectfont 0.00\endgroup & \begingroup\fontsize{8}{10}\selectfont \endgroup & \begingroup\fontsize{8}{10}\selectfont 0.00\endgroup & \begingroup\fontsize{8}{10}\selectfont 0.00\endgroup & \begingroup\fontsize{8}{10}\selectfont \endgroup & \begingroup\fontsize{8}{10}\selectfont 0.00\endgroup\\
\begingroup\fontsize{8}{10}\selectfont A\endgroup & \begingroup\fontsize{8}{10}\selectfont Hungary\endgroup & \begingroup\fontsize{8}{10}\selectfont 0.48\endgroup & \begingroup\fontsize{8}{10}\selectfont 0.84\endgroup & \begingroup\fontsize{8}{10}\selectfont 0.01\endgroup & \begingroup\fontsize{8}{10}\selectfont 0.02\endgroup & \begingroup\fontsize{8}{10}\selectfont 0.01\endgroup & \begingroup\fontsize{8}{10}\selectfont \endgroup & \begingroup\fontsize{8}{10}\selectfont \endgroup & \begingroup\fontsize{8}{10}\selectfont \endgroup & \begingroup\fontsize{8}{10}\selectfont \endgroup & \begingroup\fontsize{8}{10}\selectfont \endgroup & \begingroup\fontsize{8}{10}\selectfont \endgroup & \begingroup\fontsize{8}{10}\selectfont \endgroup\\
\begingroup\fontsize{8}{10}\selectfont A\endgroup & \begingroup\fontsize{8}{10}\selectfont United States\endgroup & \begingroup\fontsize{8}{10}\selectfont 0.48\endgroup & \begingroup\fontsize{8}{10}\selectfont 1.54\endgroup & \begingroup\fontsize{8}{10}\selectfont 0.05\endgroup & \begingroup\fontsize{8}{10}\selectfont 0.05\endgroup & \begingroup\fontsize{8}{10}\selectfont 0.03\endgroup & \begingroup\fontsize{8}{10}\selectfont \endgroup & \begingroup\fontsize{8}{10}\selectfont \endgroup & \begingroup\fontsize{8}{10}\selectfont \endgroup & \begingroup\fontsize{8}{10}\selectfont \endgroup & \begingroup\fontsize{8}{10}\selectfont \endgroup & \begingroup\fontsize{8}{10}\selectfont \endgroup & \begingroup\fontsize{8}{10}\selectfont \endgroup\\
\begingroup\fontsize{8}{10}\selectfont A\endgroup & \begingroup\fontsize{8}{10}\selectfont Serbia\endgroup & \begingroup\fontsize{8}{10}\selectfont 0.47\endgroup & \begingroup\fontsize{8}{10}\selectfont 0.94\endgroup & \begingroup\fontsize{8}{10}\selectfont 0.02\endgroup & \begingroup\fontsize{8}{10}\selectfont 0.01\endgroup & \begingroup\fontsize{8}{10}\selectfont 0.01\endgroup & \begingroup\fontsize{8}{10}\selectfont 0.00\endgroup & \begingroup\fontsize{8}{10}\selectfont \endgroup & \begingroup\fontsize{8}{10}\selectfont 0.00\endgroup & \begingroup\fontsize{8}{10}\selectfont \endgroup & \begingroup\fontsize{8}{10}\selectfont 0.00\endgroup & \begingroup\fontsize{8}{10}\selectfont \endgroup & \begingroup\fontsize{8}{10}\selectfont 0.01\endgroup\\
\begingroup\fontsize{8}{10}\selectfont A\endgroup & \begingroup\fontsize{8}{10}\selectfont Romania\endgroup & \begingroup\fontsize{8}{10}\selectfont 0.47\endgroup & \begingroup\fontsize{8}{10}\selectfont 0.66\endgroup & \begingroup\fontsize{8}{10}\selectfont 0.01\endgroup & \begingroup\fontsize{8}{10}\selectfont 0.02\endgroup & \begingroup\fontsize{8}{10}\selectfont 0.04\endgroup & \begingroup\fontsize{8}{10}\selectfont \endgroup & \begingroup\fontsize{8}{10}\selectfont \endgroup & \begingroup\fontsize{8}{10}\selectfont \endgroup & \begingroup\fontsize{8}{10}\selectfont \endgroup & \begingroup\fontsize{8}{10}\selectfont \endgroup & \begingroup\fontsize{8}{10}\selectfont \endgroup & \begingroup\fontsize{8}{10}\selectfont \endgroup\\
\begingroup\fontsize{8}{10}\selectfont A\endgroup & \begingroup\fontsize{8}{10}\selectfont Croatia\endgroup & \begingroup\fontsize{8}{10}\selectfont 0.47\endgroup & \begingroup\fontsize{8}{10}\selectfont 0.70\endgroup & \begingroup\fontsize{8}{10}\selectfont 0.03\endgroup & \begingroup\fontsize{8}{10}\selectfont 0.02\endgroup & \begingroup\fontsize{8}{10}\selectfont 0.01\endgroup & \begingroup\fontsize{8}{10}\selectfont \endgroup & \begingroup\fontsize{8}{10}\selectfont \endgroup & \begingroup\fontsize{8}{10}\selectfont \endgroup & \begingroup\fontsize{8}{10}\selectfont \endgroup & \begingroup\fontsize{8}{10}\selectfont \endgroup & \begingroup\fontsize{8}{10}\selectfont \endgroup & \begingroup\fontsize{8}{10}\selectfont \endgroup\\
\begingroup\fontsize{8}{10}\selectfont A\endgroup & \begingroup\fontsize{8}{10}\selectfont Slovakia\endgroup & \begingroup\fontsize{8}{10}\selectfont 0.46\endgroup & \begingroup\fontsize{8}{10}\selectfont 1.24\endgroup & \begingroup\fontsize{8}{10}\selectfont 0.03\endgroup & \begingroup\fontsize{8}{10}\selectfont 0.05\endgroup & \begingroup\fontsize{8}{10}\selectfont 0.06\endgroup & \begingroup\fontsize{8}{10}\selectfont \endgroup & \begingroup\fontsize{8}{10}\selectfont \endgroup & \begingroup\fontsize{8}{10}\selectfont \endgroup & \begingroup\fontsize{8}{10}\selectfont \endgroup & \begingroup\fontsize{8}{10}\selectfont \endgroup & \begingroup\fontsize{8}{10}\selectfont \endgroup & \begingroup\fontsize{8}{10}\selectfont \endgroup\\
\begingroup\fontsize{8}{10}\selectfont A\endgroup & \begingroup\fontsize{8}{10}\selectfont Latvia\endgroup & \begingroup\fontsize{8}{10}\selectfont 0.45\endgroup & \begingroup\fontsize{8}{10}\selectfont 0.98\endgroup & \begingroup\fontsize{8}{10}\selectfont 0.02\endgroup & \begingroup\fontsize{8}{10}\selectfont 0.04\endgroup & \begingroup\fontsize{8}{10}\selectfont 0.07\endgroup & \begingroup\fontsize{8}{10}\selectfont \endgroup & \begingroup\fontsize{8}{10}\selectfont \endgroup & \begingroup\fontsize{8}{10}\selectfont \endgroup & \begingroup\fontsize{8}{10}\selectfont \endgroup & \begingroup\fontsize{8}{10}\selectfont \endgroup & \begingroup\fontsize{8}{10}\selectfont \endgroup & \begingroup\fontsize{8}{10}\selectfont \endgroup\\
\begingroup\fontsize{8}{10}\selectfont A\endgroup & \begingroup\fontsize{8}{10}\selectfont Netherlands\endgroup & \begingroup\fontsize{8}{10}\selectfont 0.45\endgroup & \begingroup\fontsize{8}{10}\selectfont 1.42\endgroup & \begingroup\fontsize{8}{10}\selectfont 0.07\endgroup & \begingroup\fontsize{8}{10}\selectfont 0.04\endgroup & \begingroup\fontsize{8}{10}\selectfont 0.05\endgroup & \begingroup\fontsize{8}{10}\selectfont \endgroup & \begingroup\fontsize{8}{10}\selectfont \endgroup & \begingroup\fontsize{8}{10}\selectfont \endgroup & \begingroup\fontsize{8}{10}\selectfont \endgroup & \begingroup\fontsize{8}{10}\selectfont \endgroup & \begingroup\fontsize{8}{10}\selectfont \endgroup & \begingroup\fontsize{8}{10}\selectfont \endgroup\\
\begingroup\fontsize{8}{10}\selectfont A\endgroup & \begingroup\fontsize{8}{10}\selectfont Canada\endgroup & \begingroup\fontsize{8}{10}\selectfont 0.44\endgroup & \begingroup\fontsize{8}{10}\selectfont 1.19\endgroup & \begingroup\fontsize{8}{10}\selectfont 0.04\endgroup & \begingroup\fontsize{8}{10}\selectfont 0.03\endgroup & \begingroup\fontsize{8}{10}\selectfont 0.06\endgroup & \begingroup\fontsize{8}{10}\selectfont \endgroup & \begingroup\fontsize{8}{10}\selectfont \endgroup & \begingroup\fontsize{8}{10}\selectfont \endgroup & \begingroup\fontsize{8}{10}\selectfont \endgroup & \begingroup\fontsize{8}{10}\selectfont 0.02\endgroup & \begingroup\fontsize{8}{10}\selectfont \endgroup & \begingroup\fontsize{8}{10}\selectfont 0.01\endgroup\\
\begingroup\fontsize{8}{10}\selectfont A\endgroup & \begingroup\fontsize{8}{10}\selectfont Bulgaria\endgroup & \begingroup\fontsize{8}{10}\selectfont 0.44\endgroup & \begingroup\fontsize{8}{10}\selectfont 0.92\endgroup & \begingroup\fontsize{8}{10}\selectfont 0.00\endgroup & \begingroup\fontsize{8}{10}\selectfont 0.00\endgroup & \begingroup\fontsize{8}{10}\selectfont 0.00\endgroup & \begingroup\fontsize{8}{10}\selectfont \endgroup & \begingroup\fontsize{8}{10}\selectfont \endgroup & \begingroup\fontsize{8}{10}\selectfont \endgroup & \begingroup\fontsize{8}{10}\selectfont \endgroup & \begingroup\fontsize{8}{10}\selectfont \endgroup & \begingroup\fontsize{8}{10}\selectfont \endgroup & \begingroup\fontsize{8}{10}\selectfont \endgroup\\
\begingroup\fontsize{8}{10}\selectfont A\endgroup & \begingroup\fontsize{8}{10}\selectfont Lithuania\endgroup & \begingroup\fontsize{8}{10}\selectfont 0.44\endgroup & \begingroup\fontsize{8}{10}\selectfont 0.54\endgroup & \begingroup\fontsize{8}{10}\selectfont 0.03\endgroup & \begingroup\fontsize{8}{10}\selectfont 0.03\endgroup & \begingroup\fontsize{8}{10}\selectfont 0.01\endgroup & \begingroup\fontsize{8}{10}\selectfont \endgroup & \begingroup\fontsize{8}{10}\selectfont \endgroup & \begingroup\fontsize{8}{10}\selectfont \endgroup & \begingroup\fontsize{8}{10}\selectfont \endgroup & \begingroup\fontsize{8}{10}\selectfont \endgroup & \begingroup\fontsize{8}{10}\selectfont \endgroup & \begingroup\fontsize{8}{10}\selectfont \endgroup\\
\begingroup\fontsize{8}{10}\selectfont A\endgroup & \begingroup\fontsize{8}{10}\selectfont Brazil\endgroup & \begingroup\fontsize{8}{10}\selectfont 0.41\endgroup & \begingroup\fontsize{8}{10}\selectfont 1.45\endgroup & \begingroup\fontsize{8}{10}\selectfont 0.04\endgroup & \begingroup\fontsize{8}{10}\selectfont 0.03\endgroup & \begingroup\fontsize{8}{10}\selectfont 0.04\endgroup & \begingroup\fontsize{8}{10}\selectfont 0.00\endgroup & \begingroup\fontsize{8}{10}\selectfont 0.00\endgroup & \begingroup\fontsize{8}{10}\selectfont 0.01\endgroup & \begingroup\fontsize{8}{10}\selectfont \endgroup & \begingroup\fontsize{8}{10}\selectfont 0.05\endgroup & \begingroup\fontsize{8}{10}\selectfont 0.02\endgroup & \begingroup\fontsize{8}{10}\selectfont 0.01\endgroup\\
\begingroup\fontsize{8}{10}\selectfont A\endgroup & \begingroup\fontsize{8}{10}\selectfont Maldives\endgroup & \begingroup\fontsize{8}{10}\selectfont 0.41\endgroup & \begingroup\fontsize{8}{10}\selectfont 1.57\endgroup & \begingroup\fontsize{8}{10}\selectfont 0.02\endgroup & \begingroup\fontsize{8}{10}\selectfont 0.02\endgroup & \begingroup\fontsize{8}{10}\selectfont 0.07\endgroup & \begingroup\fontsize{8}{10}\selectfont \endgroup & \begingroup\fontsize{8}{10}\selectfont 0.00\endgroup & \begingroup\fontsize{8}{10}\selectfont \endgroup & \begingroup\fontsize{8}{10}\selectfont \endgroup & \begingroup\fontsize{8}{10}\selectfont 0.00\endgroup & \begingroup\fontsize{8}{10}\selectfont 0.02\endgroup & \begingroup\fontsize{8}{10}\selectfont 0.02\endgroup\\
\begingroup\fontsize{8}{10}\selectfont A\endgroup & \begingroup\fontsize{8}{10}\selectfont Colombia\endgroup & \begingroup\fontsize{8}{10}\selectfont 0.41\endgroup & \begingroup\fontsize{8}{10}\selectfont 1.44\endgroup & \begingroup\fontsize{8}{10}\selectfont 0.05\endgroup & \begingroup\fontsize{8}{10}\selectfont 0.04\endgroup & \begingroup\fontsize{8}{10}\selectfont 0.03\endgroup & \begingroup\fontsize{8}{10}\selectfont 0.00\endgroup & \begingroup\fontsize{8}{10}\selectfont 0.03\endgroup & \begingroup\fontsize{8}{10}\selectfont \endgroup & \begingroup\fontsize{8}{10}\selectfont \endgroup & \begingroup\fontsize{8}{10}\selectfont 0.03\endgroup & \begingroup\fontsize{8}{10}\selectfont 0.03\endgroup & \begingroup\fontsize{8}{10}\selectfont 0.01\endgroup\\
\begingroup\fontsize{8}{10}\selectfont A\endgroup & \begingroup\fontsize{8}{10}\selectfont Cambodia\endgroup & \begingroup\fontsize{8}{10}\selectfont 0.39\endgroup & \begingroup\fontsize{8}{10}\selectfont 1.13\endgroup & \begingroup\fontsize{8}{10}\selectfont 0.04\endgroup & \begingroup\fontsize{8}{10}\selectfont 0.03\endgroup & \begingroup\fontsize{8}{10}\selectfont 0.05\endgroup & \begingroup\fontsize{8}{10}\selectfont \endgroup & \begingroup\fontsize{8}{10}\selectfont 0.03\endgroup & \begingroup\fontsize{8}{10}\selectfont \endgroup & \begingroup\fontsize{8}{10}\selectfont 0.00\endgroup & \begingroup\fontsize{8}{10}\selectfont 0.01\endgroup & \begingroup\fontsize{8}{10}\selectfont 0.05\endgroup & \begingroup\fontsize{8}{10}\selectfont \endgroup\\
\begingroup\fontsize{8}{10}\selectfont A\endgroup & \begingroup\fontsize{8}{10}\selectfont Liberia\endgroup & \begingroup\fontsize{8}{10}\selectfont 0.37\endgroup & \begingroup\fontsize{8}{10}\selectfont 0.34\endgroup & \begingroup\fontsize{8}{10}\selectfont 0.05\endgroup & \begingroup\fontsize{8}{10}\selectfont 0.03\endgroup & \begingroup\fontsize{8}{10}\selectfont 0.03\endgroup & \begingroup\fontsize{8}{10}\selectfont 0.00\endgroup & \begingroup\fontsize{8}{10}\selectfont 0.01\endgroup & \begingroup\fontsize{8}{10}\selectfont \endgroup & \begingroup\fontsize{8}{10}\selectfont 0.00\endgroup & \begingroup\fontsize{8}{10}\selectfont 0.00\endgroup & \begingroup\fontsize{8}{10}\selectfont 0.01\endgroup & \begingroup\fontsize{8}{10}\selectfont 0.00\endgroup\\
\begingroup\fontsize{8}{10}\selectfont A\endgroup & \begingroup\fontsize{8}{10}\selectfont Chile\endgroup & \begingroup\fontsize{8}{10}\selectfont 0.37\endgroup & \begingroup\fontsize{8}{10}\selectfont 1.58\endgroup & \begingroup\fontsize{8}{10}\selectfont 0.05\endgroup & \begingroup\fontsize{8}{10}\selectfont 0.07\endgroup & \begingroup\fontsize{8}{10}\selectfont 0.00\endgroup & \begingroup\fontsize{8}{10}\selectfont \endgroup & \begingroup\fontsize{8}{10}\selectfont \endgroup & \begingroup\fontsize{8}{10}\selectfont \endgroup & \begingroup\fontsize{8}{10}\selectfont \endgroup & \begingroup\fontsize{8}{10}\selectfont \endgroup & \begingroup\fontsize{8}{10}\selectfont \endgroup & \begingroup\fontsize{8}{10}\selectfont \endgroup\\
\begingroup\fontsize{8}{10}\selectfont A\endgroup & \begingroup\fontsize{8}{10}\selectfont Austria\endgroup & \begingroup\fontsize{8}{10}\selectfont 0.35\endgroup & \begingroup\fontsize{8}{10}\selectfont 1.58\endgroup & \begingroup\fontsize{8}{10}\selectfont 0.07\endgroup & \begingroup\fontsize{8}{10}\selectfont 0.03\endgroup & \begingroup\fontsize{8}{10}\selectfont 0.04\endgroup & \begingroup\fontsize{8}{10}\selectfont \endgroup & \begingroup\fontsize{8}{10}\selectfont \endgroup & \begingroup\fontsize{8}{10}\selectfont 0.01\endgroup & \begingroup\fontsize{8}{10}\selectfont \endgroup & \begingroup\fontsize{8}{10}\selectfont 0.05\endgroup & \begingroup\fontsize{8}{10}\selectfont 0.00\endgroup & \begingroup\fontsize{8}{10}\selectfont 0.00\endgroup\\
\begingroup\fontsize{8}{10}\selectfont A\endgroup & \begingroup\fontsize{8}{10}\selectfont Luxembourg\endgroup & \begingroup\fontsize{8}{10}\selectfont 0.34\endgroup & \begingroup\fontsize{8}{10}\selectfont 1.65\endgroup & \begingroup\fontsize{8}{10}\selectfont 0.09\endgroup & \begingroup\fontsize{8}{10}\selectfont 0.05\endgroup & \begingroup\fontsize{8}{10}\selectfont 0.02\endgroup & \begingroup\fontsize{8}{10}\selectfont \endgroup & \begingroup\fontsize{8}{10}\selectfont \endgroup & \begingroup\fontsize{8}{10}\selectfont \endgroup & \begingroup\fontsize{8}{10}\selectfont \endgroup & \begingroup\fontsize{8}{10}\selectfont \endgroup & \begingroup\fontsize{8}{10}\selectfont \endgroup & \begingroup\fontsize{8}{10}\selectfont \endgroup\\
\begingroup\fontsize{8}{10}\selectfont A\endgroup & \begingroup\fontsize{8}{10}\selectfont Kenya\endgroup & \begingroup\fontsize{8}{10}\selectfont 0.34\endgroup & \begingroup\fontsize{8}{10}\selectfont 0.69\endgroup & \begingroup\fontsize{8}{10}\selectfont 0.02\endgroup & \begingroup\fontsize{8}{10}\selectfont 0.03\endgroup & \begingroup\fontsize{8}{10}\selectfont 0.05\endgroup & \begingroup\fontsize{8}{10}\selectfont 0.00\endgroup & \begingroup\fontsize{8}{10}\selectfont 0.02\endgroup & \begingroup\fontsize{8}{10}\selectfont \endgroup & \begingroup\fontsize{8}{10}\selectfont 0.03\endgroup & \begingroup\fontsize{8}{10}\selectfont \endgroup & \begingroup\fontsize{8}{10}\selectfont \endgroup & \begingroup\fontsize{8}{10}\selectfont \endgroup\\
\begingroup\fontsize{8}{10}\selectfont A\endgroup & \begingroup\fontsize{8}{10}\selectfont Myanmar (Burma)\endgroup & \begingroup\fontsize{8}{10}\selectfont 0.33\endgroup & \begingroup\fontsize{8}{10}\selectfont 0.54\endgroup & \begingroup\fontsize{8}{10}\selectfont 0.03\endgroup & \begingroup\fontsize{8}{10}\selectfont 0.02\endgroup & \begingroup\fontsize{8}{10}\selectfont 0.02\endgroup & \begingroup\fontsize{8}{10}\selectfont 0.00\endgroup & \begingroup\fontsize{8}{10}\selectfont 0.01\endgroup & \begingroup\fontsize{8}{10}\selectfont \endgroup & \begingroup\fontsize{8}{10}\selectfont 0.01\endgroup & \begingroup\fontsize{8}{10}\selectfont 0.02\endgroup & \begingroup\fontsize{8}{10}\selectfont 0.04\endgroup & \begingroup\fontsize{8}{10}\selectfont 0.02\endgroup\\
\begingroup\fontsize{8}{10}\selectfont A\endgroup & \begingroup\fontsize{8}{10}\selectfont Norway\endgroup & \begingroup\fontsize{8}{10}\selectfont 0.29\endgroup & \begingroup\fontsize{8}{10}\selectfont 1.54\endgroup & \begingroup\fontsize{8}{10}\selectfont 0.07\endgroup & \begingroup\fontsize{8}{10}\selectfont 0.03\endgroup & \begingroup\fontsize{8}{10}\selectfont 0.09\endgroup & \begingroup\fontsize{8}{10}\selectfont \endgroup & \begingroup\fontsize{8}{10}\selectfont \endgroup & \begingroup\fontsize{8}{10}\selectfont \endgroup & \begingroup\fontsize{8}{10}\selectfont \endgroup & \begingroup\fontsize{8}{10}\selectfont 0.03\endgroup & \begingroup\fontsize{8}{10}\selectfont 0.01\endgroup & \begingroup\fontsize{8}{10}\selectfont 0.00\endgroup\\
\begingroup\fontsize{8}{10}\selectfont A\endgroup & \begingroup\fontsize{8}{10}\selectfont Mozambique\endgroup & \begingroup\fontsize{8}{10}\selectfont 0.29\endgroup & \begingroup\fontsize{8}{10}\selectfont 0.20\endgroup & \begingroup\fontsize{8}{10}\selectfont 0.06\endgroup & \begingroup\fontsize{8}{10}\selectfont 0.03\endgroup & \begingroup\fontsize{8}{10}\selectfont 0.05\endgroup & \begingroup\fontsize{8}{10}\selectfont 0.00\endgroup & \begingroup\fontsize{8}{10}\selectfont 0.02\endgroup & \begingroup\fontsize{8}{10}\selectfont \endgroup & \begingroup\fontsize{8}{10}\selectfont 0.01\endgroup & \begingroup\fontsize{8}{10}\selectfont 0.00\endgroup & \begingroup\fontsize{8}{10}\selectfont 0.01\endgroup & \begingroup\fontsize{8}{10}\selectfont 0.00\endgroup\\
\begingroup\fontsize{8}{10}\selectfont A\endgroup & \begingroup\fontsize{8}{10}\selectfont Israel\endgroup & \begingroup\fontsize{8}{10}\selectfont 0.28\endgroup & \begingroup\fontsize{8}{10}\selectfont 1.72\endgroup & \begingroup\fontsize{8}{10}\selectfont 0.06\endgroup & \begingroup\fontsize{8}{10}\selectfont 0.06\endgroup & \begingroup\fontsize{8}{10}\selectfont 0.05\endgroup & \begingroup\fontsize{8}{10}\selectfont \endgroup & \begingroup\fontsize{8}{10}\selectfont \endgroup & \begingroup\fontsize{8}{10}\selectfont \endgroup & \begingroup\fontsize{8}{10}\selectfont \endgroup & \begingroup\fontsize{8}{10}\selectfont 0.06\endgroup & \begingroup\fontsize{8}{10}\selectfont 0.00\endgroup & \begingroup\fontsize{8}{10}\selectfont 0.01\endgroup\\
\begingroup\fontsize{8}{10}\selectfont A\endgroup & \begingroup\fontsize{8}{10}\selectfont Switzerland\endgroup & \begingroup\fontsize{8}{10}\selectfont 0.28\endgroup & \begingroup\fontsize{8}{10}\selectfont 1.23\endgroup & \begingroup\fontsize{8}{10}\selectfont 0.02\endgroup & \begingroup\fontsize{8}{10}\selectfont 0.01\endgroup & \begingroup\fontsize{8}{10}\selectfont 0.03\endgroup & \begingroup\fontsize{8}{10}\selectfont \endgroup & \begingroup\fontsize{8}{10}\selectfont \endgroup & \begingroup\fontsize{8}{10}\selectfont \endgroup & \begingroup\fontsize{8}{10}\selectfont \endgroup & \begingroup\fontsize{8}{10}\selectfont 0.03\endgroup & \begingroup\fontsize{8}{10}\selectfont 0.01\endgroup & \begingroup\fontsize{8}{10}\selectfont 0.04\endgroup\\
\begingroup\fontsize{8}{10}\selectfont A\endgroup & \begingroup\fontsize{8}{10}\selectfont Mongolia\endgroup & \begingroup\fontsize{8}{10}\selectfont 0.28\endgroup & \begingroup\fontsize{8}{10}\selectfont 2.30\endgroup & \begingroup\fontsize{8}{10}\selectfont 0.06\endgroup & \begingroup\fontsize{8}{10}\selectfont 0.02\endgroup & \begingroup\fontsize{8}{10}\selectfont 0.11\endgroup & \begingroup\fontsize{8}{10}\selectfont \endgroup & \begingroup\fontsize{8}{10}\selectfont \endgroup & \begingroup\fontsize{8}{10}\selectfont \endgroup & \begingroup\fontsize{8}{10}\selectfont \endgroup & \begingroup\fontsize{8}{10}\selectfont \endgroup & \begingroup\fontsize{8}{10}\selectfont \endgroup & \begingroup\fontsize{8}{10}\selectfont 0.00\endgroup\\
\begingroup\fontsize{8}{10}\selectfont A\endgroup & \begingroup\fontsize{8}{10}\selectfont Australia\endgroup & \begingroup\fontsize{8}{10}\selectfont 0.27\endgroup & \begingroup\fontsize{8}{10}\selectfont 1.83\endgroup & \begingroup\fontsize{8}{10}\selectfont 0.10\endgroup & \begingroup\fontsize{8}{10}\selectfont 0.07\endgroup & \begingroup\fontsize{8}{10}\selectfont 0.04\endgroup & \begingroup\fontsize{8}{10}\selectfont \endgroup & \begingroup\fontsize{8}{10}\selectfont \endgroup & \begingroup\fontsize{8}{10}\selectfont \endgroup & \begingroup\fontsize{8}{10}\selectfont \endgroup & \begingroup\fontsize{8}{10}\selectfont \endgroup & \begingroup\fontsize{8}{10}\selectfont \endgroup & \begingroup\fontsize{8}{10}\selectfont \endgroup\\
\begingroup\fontsize{8}{10}\selectfont A\endgroup & \begingroup\fontsize{8}{10}\selectfont Argentina\endgroup & \begingroup\fontsize{8}{10}\selectfont 0.25\endgroup & \begingroup\fontsize{8}{10}\selectfont 1.93\endgroup & \begingroup\fontsize{8}{10}\selectfont 0.08\endgroup & \begingroup\fontsize{8}{10}\selectfont 0.04\endgroup & \begingroup\fontsize{8}{10}\selectfont 0.07\endgroup & \begingroup\fontsize{8}{10}\selectfont 0.00\endgroup & \begingroup\fontsize{8}{10}\selectfont 0.01\endgroup & \begingroup\fontsize{8}{10}\selectfont 0.01\endgroup & \begingroup\fontsize{8}{10}\selectfont \endgroup & \begingroup\fontsize{8}{10}\selectfont 0.05\endgroup & \begingroup\fontsize{8}{10}\selectfont 0.01\endgroup & \begingroup\fontsize{8}{10}\selectfont 0.01\endgroup\\
\begingroup\fontsize{8}{10}\selectfont A\endgroup & \begingroup\fontsize{8}{10}\selectfont Portugal\endgroup & \begingroup\fontsize{8}{10}\selectfont 0.24\endgroup & \begingroup\fontsize{8}{10}\selectfont 1.60\endgroup & \begingroup\fontsize{8}{10}\selectfont 0.05\endgroup & \begingroup\fontsize{8}{10}\selectfont 0.10\endgroup & \begingroup\fontsize{8}{10}\selectfont 0.03\endgroup & \begingroup\fontsize{8}{10}\selectfont \endgroup & \begingroup\fontsize{8}{10}\selectfont \endgroup & \begingroup\fontsize{8}{10}\selectfont \endgroup & \begingroup\fontsize{8}{10}\selectfont \endgroup & \begingroup\fontsize{8}{10}\selectfont \endgroup & \begingroup\fontsize{8}{10}\selectfont \endgroup & \begingroup\fontsize{8}{10}\selectfont \endgroup\\
\begingroup\fontsize{8}{10}\selectfont A\endgroup & \begingroup\fontsize{8}{10}\selectfont Guinea-Bissau\endgroup & \begingroup\fontsize{8}{10}\selectfont 0.24\endgroup & \begingroup\fontsize{8}{10}\selectfont 0.29\endgroup & \begingroup\fontsize{8}{10}\selectfont 0.03\endgroup & \begingroup\fontsize{8}{10}\selectfont 0.05\endgroup & \begingroup\fontsize{8}{10}\selectfont 0.05\endgroup & \begingroup\fontsize{8}{10}\selectfont 0.00\endgroup & \begingroup\fontsize{8}{10}\selectfont 0.01\endgroup & \begingroup\fontsize{8}{10}\selectfont \endgroup & \begingroup\fontsize{8}{10}\selectfont 0.01\endgroup & \begingroup\fontsize{8}{10}\selectfont 0.02\endgroup & \begingroup\fontsize{8}{10}\selectfont 0.05\endgroup & \begingroup\fontsize{8}{10}\selectfont 0.00\endgroup\\
\begingroup\fontsize{8}{10}\selectfont A\endgroup & \begingroup\fontsize{8}{10}\selectfont Malawi\endgroup & \begingroup\fontsize{8}{10}\selectfont 0.23\endgroup & \begingroup\fontsize{8}{10}\selectfont 0.37\endgroup & \begingroup\fontsize{8}{10}\selectfont 0.04\endgroup & \begingroup\fontsize{8}{10}\selectfont 0.04\endgroup & \begingroup\fontsize{8}{10}\selectfont 0.01\endgroup & \begingroup\fontsize{8}{10}\selectfont 0.00\endgroup & \begingroup\fontsize{8}{10}\selectfont 0.01\endgroup & \begingroup\fontsize{8}{10}\selectfont \endgroup & \begingroup\fontsize{8}{10}\selectfont 0.00\endgroup & \begingroup\fontsize{8}{10}\selectfont 0.04\endgroup & \begingroup\fontsize{8}{10}\selectfont 0.04\endgroup & \begingroup\fontsize{8}{10}\selectfont 0.01\endgroup\\
\begingroup\fontsize{8}{10}\selectfont A\endgroup & \begingroup\fontsize{8}{10}\selectfont Bangladesh\endgroup & \begingroup\fontsize{8}{10}\selectfont 0.19\endgroup & \begingroup\fontsize{8}{10}\selectfont 1.03\endgroup & \begingroup\fontsize{8}{10}\selectfont 0.03\endgroup & \begingroup\fontsize{8}{10}\selectfont 0.03\endgroup & \begingroup\fontsize{8}{10}\selectfont 0.03\endgroup & \begingroup\fontsize{8}{10}\selectfont 0.02\endgroup & \begingroup\fontsize{8}{10}\selectfont \endgroup & \begingroup\fontsize{8}{10}\selectfont \endgroup & \begingroup\fontsize{8}{10}\selectfont \endgroup & \begingroup\fontsize{8}{10}\selectfont 0.00\endgroup & \begingroup\fontsize{8}{10}\selectfont 0.02\endgroup & \begingroup\fontsize{8}{10}\selectfont 0.03\endgroup\\
\begingroup\fontsize{8}{10}\selectfont A\endgroup & \begingroup\fontsize{8}{10}\selectfont Ethiopia\endgroup & \begingroup\fontsize{8}{10}\selectfont 0.17\endgroup & \begingroup\fontsize{8}{10}\selectfont 0.98\endgroup & \begingroup\fontsize{8}{10}\selectfont 0.03\endgroup & \begingroup\fontsize{8}{10}\selectfont 0.03\endgroup & \begingroup\fontsize{8}{10}\selectfont 0.07\endgroup & \begingroup\fontsize{8}{10}\selectfont 0.00\endgroup & \begingroup\fontsize{8}{10}\selectfont 0.02\endgroup & \begingroup\fontsize{8}{10}\selectfont \endgroup & \begingroup\fontsize{8}{10}\selectfont 0.04\endgroup & \begingroup\fontsize{8}{10}\selectfont 0.00\endgroup & \begingroup\fontsize{8}{10}\selectfont 0.00\endgroup & \begingroup\fontsize{8}{10}\selectfont 0.00\endgroup\\
\begingroup\fontsize{8}{10}\selectfont A\endgroup & \begingroup\fontsize{8}{10}\selectfont United Kingdom\endgroup & \begingroup\fontsize{8}{10}\selectfont 0.17\endgroup & \begingroup\fontsize{8}{10}\selectfont 1.51\endgroup & \begingroup\fontsize{8}{10}\selectfont 0.06\endgroup & \begingroup\fontsize{8}{10}\selectfont 0.01\endgroup & \begingroup\fontsize{8}{10}\selectfont 0.02\endgroup & \begingroup\fontsize{8}{10}\selectfont \endgroup & \begingroup\fontsize{8}{10}\selectfont \endgroup & \begingroup\fontsize{8}{10}\selectfont 0.03\endgroup & \begingroup\fontsize{8}{10}\selectfont \endgroup & \begingroup\fontsize{8}{10}\selectfont 0.03\endgroup & \begingroup\fontsize{8}{10}\selectfont \endgroup & \begingroup\fontsize{8}{10}\selectfont 0.00\endgroup\\
\midrule
\begingroup\fontsize{8}{10}\selectfont B\endgroup & \begingroup\fontsize{8}{10}\selectfont Jordan\endgroup & \begingroup\fontsize{8}{10}\selectfont 0.15\endgroup & \begingroup\fontsize{8}{10}\selectfont 0.60\endgroup & \begingroup\fontsize{8}{10}\selectfont 0.04\endgroup & \begingroup\fontsize{8}{10}\selectfont 0.06\endgroup & \begingroup\fontsize{8}{10}\selectfont 0.15\endgroup & \begingroup\fontsize{8}{10}\selectfont \endgroup & \begingroup\fontsize{8}{10}\selectfont 0.00\endgroup & \begingroup\fontsize{8}{10}\selectfont \endgroup & \begingroup\fontsize{8}{10}\selectfont \endgroup & \begingroup\fontsize{8}{10}\selectfont 0.32\endgroup & \begingroup\fontsize{8}{10}\selectfont \endgroup & \begingroup\fontsize{8}{10}\selectfont 0.01\endgroup\\
\begingroup\fontsize{8}{10}\selectfont B\endgroup & \begingroup\fontsize{8}{10}\selectfont Mexico\endgroup & \begingroup\fontsize{8}{10}\selectfont 0.14\endgroup & \begingroup\fontsize{8}{10}\selectfont 0.72\endgroup & \begingroup\fontsize{8}{10}\selectfont 0.03\endgroup & \begingroup\fontsize{8}{10}\selectfont 0.03\endgroup & \begingroup\fontsize{8}{10}\selectfont 0.05\endgroup & \begingroup\fontsize{8}{10}\selectfont 0.00\endgroup & \begingroup\fontsize{8}{10}\selectfont 0.02\endgroup & \begingroup\fontsize{8}{10}\selectfont \endgroup & \begingroup\fontsize{8}{10}\selectfont \endgroup & \begingroup\fontsize{8}{10}\selectfont 0.11\endgroup & \begingroup\fontsize{8}{10}\selectfont 0.01\endgroup & \begingroup\fontsize{8}{10}\selectfont 0.06\endgroup\\
\begingroup\fontsize{8}{10}\selectfont B\endgroup & \begingroup\fontsize{8}{10}\selectfont Dominican Republic\endgroup & \begingroup\fontsize{8}{10}\selectfont 0.14\endgroup & \begingroup\fontsize{8}{10}\selectfont 0.75\endgroup & \begingroup\fontsize{8}{10}\selectfont 0.03\endgroup & \begingroup\fontsize{8}{10}\selectfont 0.05\endgroup & \begingroup\fontsize{8}{10}\selectfont 0.07\endgroup & \begingroup\fontsize{8}{10}\selectfont 0.00\endgroup & \begingroup\fontsize{8}{10}\selectfont 0.02\endgroup & \begingroup\fontsize{8}{10}\selectfont \endgroup & \begingroup\fontsize{8}{10}\selectfont 0.00\endgroup & \begingroup\fontsize{8}{10}\selectfont 0.12\endgroup & \begingroup\fontsize{8}{10}\selectfont 0.11\endgroup & \begingroup\fontsize{8}{10}\selectfont 0.03\endgroup\\
\begingroup\fontsize{8}{10}\selectfont B\endgroup & \begingroup\fontsize{8}{10}\selectfont Guatemala\endgroup & \begingroup\fontsize{8}{10}\selectfont 0.11\endgroup & \begingroup\fontsize{8}{10}\selectfont 0.13\endgroup & \begingroup\fontsize{8}{10}\selectfont 0.03\endgroup & \begingroup\fontsize{8}{10}\selectfont 0.07\endgroup & \begingroup\fontsize{8}{10}\selectfont 0.04\endgroup & \begingroup\fontsize{8}{10}\selectfont 0.00\endgroup & \begingroup\fontsize{8}{10}\selectfont 0.08\endgroup & \begingroup\fontsize{8}{10}\selectfont \endgroup & \begingroup\fontsize{8}{10}\selectfont 0.00\endgroup & \begingroup\fontsize{8}{10}\selectfont 0.15\endgroup & \begingroup\fontsize{8}{10}\selectfont 0.07\endgroup & \begingroup\fontsize{8}{10}\selectfont 0.03\endgroup\\
\begingroup\fontsize{8}{10}\selectfont B\endgroup & \begingroup\fontsize{8}{10}\selectfont Philippines\endgroup & \begingroup\fontsize{8}{10}\selectfont 0.06\endgroup & \begingroup\fontsize{8}{10}\selectfont 0.50\endgroup & \begingroup\fontsize{8}{10}\selectfont 0.03\endgroup & \begingroup\fontsize{8}{10}\selectfont 0.04\endgroup & \begingroup\fontsize{8}{10}\selectfont 0.08\endgroup & \begingroup\fontsize{8}{10}\selectfont 0.01\endgroup & \begingroup\fontsize{8}{10}\selectfont \endgroup & \begingroup\fontsize{8}{10}\selectfont \endgroup & \begingroup\fontsize{8}{10}\selectfont \endgroup & \begingroup\fontsize{8}{10}\selectfont 0.03\endgroup & \begingroup\fontsize{8}{10}\selectfont 0.08\endgroup & \begingroup\fontsize{8}{10}\selectfont 0.18\endgroup\\
\begingroup\fontsize{8}{10}\selectfont B\endgroup & \begingroup\fontsize{8}{10}\selectfont South Africa\endgroup & \begingroup\fontsize{8}{10}\selectfont 0.06\endgroup & \begingroup\fontsize{8}{10}\selectfont 0.93\endgroup & \begingroup\fontsize{8}{10}\selectfont 0.08\endgroup & \begingroup\fontsize{8}{10}\selectfont 0.03\endgroup & \begingroup\fontsize{8}{10}\selectfont 0.04\endgroup & \begingroup\fontsize{8}{10}\selectfont 0.01\endgroup & \begingroup\fontsize{8}{10}\selectfont 0.01\endgroup & \begingroup\fontsize{8}{10}\selectfont 0.01\endgroup & \begingroup\fontsize{8}{10}\selectfont 0.01\endgroup & \begingroup\fontsize{8}{10}\selectfont 0.15\endgroup & \begingroup\fontsize{8}{10}\selectfont 0.00\endgroup & \begingroup\fontsize{8}{10}\selectfont 0.03\endgroup\\
\begingroup\fontsize{8}{10}\selectfont B\endgroup & \begingroup\fontsize{8}{10}\selectfont Taiwan\endgroup & \begingroup\fontsize{8}{10}\selectfont 0.05\endgroup & \begingroup\fontsize{8}{10}\selectfont 0.98\endgroup & \begingroup\fontsize{8}{10}\selectfont 0.06\endgroup & \begingroup\fontsize{8}{10}\selectfont 0.11\endgroup & \begingroup\fontsize{8}{10}\selectfont 0.00\endgroup & \begingroup\fontsize{8}{10}\selectfont \endgroup & \begingroup\fontsize{8}{10}\selectfont \endgroup & \begingroup\fontsize{8}{10}\selectfont \endgroup & \begingroup\fontsize{8}{10}\selectfont \endgroup & \begingroup\fontsize{8}{10}\selectfont 0.18\endgroup & \begingroup\fontsize{8}{10}\selectfont 0.01\endgroup & \begingroup\fontsize{8}{10}\selectfont 0.00\endgroup\\
\begingroup\fontsize{8}{10}\selectfont B\endgroup & \begingroup\fontsize{8}{10}\selectfont Georgia\endgroup & \begingroup\fontsize{8}{10}\selectfont 0.04\endgroup & \begingroup\fontsize{8}{10}\selectfont 0.88\endgroup & \begingroup\fontsize{8}{10}\selectfont 0.03\endgroup & \begingroup\fontsize{8}{10}\selectfont 0.04\endgroup & \begingroup\fontsize{8}{10}\selectfont 0.06\endgroup & \begingroup\fontsize{8}{10}\selectfont \endgroup & \begingroup\fontsize{8}{10}\selectfont 0.06\endgroup & \begingroup\fontsize{8}{10}\selectfont \endgroup & \begingroup\fontsize{8}{10}\selectfont \endgroup & \begingroup\fontsize{8}{10}\selectfont 0.12\endgroup & \begingroup\fontsize{8}{10}\selectfont 0.00\endgroup & \begingroup\fontsize{8}{10}\selectfont 0.02\endgroup\\
\begingroup\fontsize{8}{10}\selectfont B\endgroup & \begingroup\fontsize{8}{10}\selectfont Russia\endgroup & \begingroup\fontsize{8}{10}\selectfont 0.03\endgroup & \begingroup\fontsize{8}{10}\selectfont 0.96\endgroup & \begingroup\fontsize{8}{10}\selectfont 0.02\endgroup & \begingroup\fontsize{8}{10}\selectfont 0.04\endgroup & \begingroup\fontsize{8}{10}\selectfont 0.06\endgroup & \begingroup\fontsize{8}{10}\selectfont \endgroup & \begingroup\fontsize{8}{10}\selectfont \endgroup & \begingroup\fontsize{8}{10}\selectfont \endgroup & \begingroup\fontsize{8}{10}\selectfont \endgroup & \begingroup\fontsize{8}{10}\selectfont 0.14\endgroup & \begingroup\fontsize{8}{10}\selectfont 0.00\endgroup & \begingroup\fontsize{8}{10}\selectfont 0.02\endgroup\\
\begingroup\fontsize{8}{10}\selectfont B\endgroup & \begingroup\fontsize{8}{10}\selectfont Thailand\endgroup & \begingroup\fontsize{8}{10}\selectfont 0.03\endgroup & \begingroup\fontsize{8}{10}\selectfont 0.96\endgroup & \begingroup\fontsize{8}{10}\selectfont 0.03\endgroup & \begingroup\fontsize{8}{10}\selectfont 0.08\endgroup & \begingroup\fontsize{8}{10}\selectfont 0.03\endgroup & \begingroup\fontsize{8}{10}\selectfont 0.00\endgroup & \begingroup\fontsize{8}{10}\selectfont 0.03\endgroup & \begingroup\fontsize{8}{10}\selectfont \endgroup & \begingroup\fontsize{8}{10}\selectfont \endgroup & \begingroup\fontsize{8}{10}\selectfont 0.05\endgroup & \begingroup\fontsize{8}{10}\selectfont 0.07\endgroup & \begingroup\fontsize{8}{10}\selectfont 0.05\endgroup\\
\begingroup\fontsize{8}{10}\selectfont B\endgroup & \begingroup\fontsize{8}{10}\selectfont Indonesia\endgroup & \begingroup\fontsize{8}{10}\selectfont 0.01\endgroup & \begingroup\fontsize{8}{10}\selectfont 0.97\endgroup & \begingroup\fontsize{8}{10}\selectfont 0.06\endgroup & \begingroup\fontsize{8}{10}\selectfont 0.03\endgroup & \begingroup\fontsize{8}{10}\selectfont 0.05\endgroup & \begingroup\fontsize{8}{10}\selectfont 0.01\endgroup & \begingroup\fontsize{8}{10}\selectfont 0.05\endgroup & \begingroup\fontsize{8}{10}\selectfont \endgroup & \begingroup\fontsize{8}{10}\selectfont 0.00\endgroup & \begingroup\fontsize{8}{10}\selectfont 0.03\endgroup & \begingroup\fontsize{8}{10}\selectfont 0.09\endgroup & \begingroup\fontsize{8}{10}\selectfont 0.06\endgroup\\
\begingroup\fontsize{8}{10}\selectfont B\endgroup & \begingroup\fontsize{8}{10}\selectfont Uruguay\endgroup & \begingroup\fontsize{8}{10}\selectfont -0.05\endgroup & \begingroup\fontsize{8}{10}\selectfont 1.21\endgroup & \begingroup\fontsize{8}{10}\selectfont 0.06\endgroup & \begingroup\fontsize{8}{10}\selectfont 0.03\endgroup & \begingroup\fontsize{8}{10}\selectfont 0.04\endgroup & \begingroup\fontsize{8}{10}\selectfont 0.00\endgroup & \begingroup\fontsize{8}{10}\selectfont 0.00\endgroup & \begingroup\fontsize{8}{10}\selectfont 0.02\endgroup & \begingroup\fontsize{8}{10}\selectfont 0.00\endgroup & \begingroup\fontsize{8}{10}\selectfont 0.14\endgroup & \begingroup\fontsize{8}{10}\selectfont 0.05\endgroup & \begingroup\fontsize{8}{10}\selectfont 0.01\endgroup\\
\begingroup\fontsize{8}{10}\selectfont B\endgroup & \begingroup\fontsize{8}{10}\selectfont Egypt\endgroup & \begingroup\fontsize{8}{10}\selectfont -0.06\endgroup & \begingroup\fontsize{8}{10}\selectfont 0.98\endgroup & \begingroup\fontsize{8}{10}\selectfont 0.05\endgroup & \begingroup\fontsize{8}{10}\selectfont 0.05\endgroup & \begingroup\fontsize{8}{10}\selectfont 0.04\endgroup & \begingroup\fontsize{8}{10}\selectfont 0.00\endgroup & \begingroup\fontsize{8}{10}\selectfont 0.00\endgroup & \begingroup\fontsize{8}{10}\selectfont \endgroup & \begingroup\fontsize{8}{10}\selectfont 0.00\endgroup & \begingroup\fontsize{8}{10}\selectfont 0.09\endgroup & \begingroup\fontsize{8}{10}\selectfont \endgroup & \begingroup\fontsize{8}{10}\selectfont 0.03\endgroup\\
\begingroup\fontsize{8}{10}\selectfont B\endgroup & \begingroup\fontsize{8}{10}\selectfont Vietnam\endgroup & \begingroup\fontsize{8}{10}\selectfont -0.08\endgroup & \begingroup\fontsize{8}{10}\selectfont 1.20\endgroup & \begingroup\fontsize{8}{10}\selectfont 0.09\endgroup & \begingroup\fontsize{8}{10}\selectfont 0.03\endgroup & \begingroup\fontsize{8}{10}\selectfont 0.07\endgroup & \begingroup\fontsize{8}{10}\selectfont 0.01\endgroup & \begingroup\fontsize{8}{10}\selectfont \endgroup & \begingroup\fontsize{8}{10}\selectfont \endgroup & \begingroup\fontsize{8}{10}\selectfont 0.00\endgroup & \begingroup\fontsize{8}{10}\selectfont 0.00\endgroup & \begingroup\fontsize{8}{10}\selectfont 0.01\endgroup & \begingroup\fontsize{8}{10}\selectfont 0.08\endgroup\\
\begingroup\fontsize{8}{10}\selectfont B\endgroup & \begingroup\fontsize{8}{10}\selectfont Barbados\endgroup & \begingroup\fontsize{8}{10}\selectfont -0.13\endgroup & \begingroup\fontsize{8}{10}\selectfont 0.91\endgroup & \begingroup\fontsize{8}{10}\selectfont 0.04\endgroup & \begingroup\fontsize{8}{10}\selectfont 0.04\endgroup & \begingroup\fontsize{8}{10}\selectfont 0.02\endgroup & \begingroup\fontsize{8}{10}\selectfont 0.00\endgroup & \begingroup\fontsize{8}{10}\selectfont 0.02\endgroup & \begingroup\fontsize{8}{10}\selectfont \endgroup & \begingroup\fontsize{8}{10}\selectfont 0.00\endgroup & \begingroup\fontsize{8}{10}\selectfont 0.10\endgroup & \begingroup\fontsize{8}{10}\selectfont \endgroup & \begingroup\fontsize{8}{10}\selectfont 0.02\endgroup\\
\begingroup\fontsize{8}{10}\selectfont B\endgroup & \begingroup\fontsize{8}{10}\selectfont Costa Rica\endgroup & \begingroup\fontsize{8}{10}\selectfont -0.14\endgroup & \begingroup\fontsize{8}{10}\selectfont 0.83\endgroup & \begingroup\fontsize{8}{10}\selectfont 0.03\endgroup & \begingroup\fontsize{8}{10}\selectfont 0.02\endgroup & \begingroup\fontsize{8}{10}\selectfont 0.03\endgroup & \begingroup\fontsize{8}{10}\selectfont 0.00\endgroup & \begingroup\fontsize{8}{10}\selectfont 0.03\endgroup & \begingroup\fontsize{8}{10}\selectfont \endgroup & \begingroup\fontsize{8}{10}\selectfont 0.00\endgroup & \begingroup\fontsize{8}{10}\selectfont 0.12\endgroup & \begingroup\fontsize{8}{10}\selectfont 0.03\endgroup & \begingroup\fontsize{8}{10}\selectfont 0.00\endgroup\\
\midrule
\begingroup\fontsize{8}{10}\selectfont C\endgroup & \begingroup\fontsize{8}{10}\selectfont Togo\endgroup & \begingroup\fontsize{8}{10}\selectfont 0.33\endgroup & \begingroup\fontsize{8}{10}\selectfont 0.02\endgroup & \begingroup\fontsize{8}{10}\selectfont 0.04\endgroup & \begingroup\fontsize{8}{10}\selectfont 0.09\endgroup & \begingroup\fontsize{8}{10}\selectfont 0.03\endgroup & \begingroup\fontsize{8}{10}\selectfont 0.01\endgroup & \begingroup\fontsize{8}{10}\selectfont 0.02\endgroup & \begingroup\fontsize{8}{10}\selectfont \endgroup & \begingroup\fontsize{8}{10}\selectfont 0.01\endgroup & \begingroup\fontsize{8}{10}\selectfont 0.02\endgroup & \begingroup\fontsize{8}{10}\selectfont 0.21\endgroup & \begingroup\fontsize{8}{10}\selectfont 0.01\endgroup\\
\begingroup\fontsize{8}{10}\selectfont C\endgroup & \begingroup\fontsize{8}{10}\selectfont Burkina Faso\endgroup & \begingroup\fontsize{8}{10}\selectfont 0.29\endgroup & \begingroup\fontsize{8}{10}\selectfont 0.04\endgroup & \begingroup\fontsize{8}{10}\selectfont 0.06\endgroup & \begingroup\fontsize{8}{10}\selectfont 0.05\endgroup & \begingroup\fontsize{8}{10}\selectfont 0.06\endgroup & \begingroup\fontsize{8}{10}\selectfont 0.01\endgroup & \begingroup\fontsize{8}{10}\selectfont 0.01\endgroup & \begingroup\fontsize{8}{10}\selectfont \endgroup & \begingroup\fontsize{8}{10}\selectfont 0.01\endgroup & \begingroup\fontsize{8}{10}\selectfont 0.04\endgroup & \begingroup\fontsize{8}{10}\selectfont 0.23\endgroup & \begingroup\fontsize{8}{10}\selectfont 0.01\endgroup\\
\begingroup\fontsize{8}{10}\selectfont C\endgroup & \begingroup\fontsize{8}{10}\selectfont Mali\endgroup & \begingroup\fontsize{8}{10}\selectfont 0.28\endgroup & \begingroup\fontsize{8}{10}\selectfont 0.05\endgroup & \begingroup\fontsize{8}{10}\selectfont 0.03\endgroup & \begingroup\fontsize{8}{10}\selectfont 0.04\endgroup & \begingroup\fontsize{8}{10}\selectfont 0.08\endgroup & \begingroup\fontsize{8}{10}\selectfont 0.01\endgroup & \begingroup\fontsize{8}{10}\selectfont 0.01\endgroup & \begingroup\fontsize{8}{10}\selectfont \endgroup & \begingroup\fontsize{8}{10}\selectfont 0.02\endgroup & \begingroup\fontsize{8}{10}\selectfont 0.02\endgroup & \begingroup\fontsize{8}{10}\selectfont 0.19\endgroup & \begingroup\fontsize{8}{10}\selectfont 0.01\endgroup\\
\begingroup\fontsize{8}{10}\selectfont C\endgroup & \begingroup\fontsize{8}{10}\selectfont Nigeria\endgroup & \begingroup\fontsize{8}{10}\selectfont 0.27\endgroup & \begingroup\fontsize{8}{10}\selectfont 0.34\endgroup & \begingroup\fontsize{8}{10}\selectfont 0.01\endgroup & \begingroup\fontsize{8}{10}\selectfont 0.05\endgroup & \begingroup\fontsize{8}{10}\selectfont 0.07\endgroup & \begingroup\fontsize{8}{10}\selectfont 0.04\endgroup & \begingroup\fontsize{8}{10}\selectfont 0.06\endgroup & \begingroup\fontsize{8}{10}\selectfont \endgroup & \begingroup\fontsize{8}{10}\selectfont \endgroup & \begingroup\fontsize{8}{10}\selectfont 0.02\endgroup & \begingroup\fontsize{8}{10}\selectfont 0.07\endgroup & \begingroup\fontsize{8}{10}\selectfont 0.03\endgroup\\
\begingroup\fontsize{8}{10}\selectfont C\endgroup & \begingroup\fontsize{8}{10}\selectfont Niger\endgroup & \begingroup\fontsize{8}{10}\selectfont 0.27\endgroup & \begingroup\fontsize{8}{10}\selectfont 0.35\endgroup & \begingroup\fontsize{8}{10}\selectfont 0.04\endgroup & \begingroup\fontsize{8}{10}\selectfont 0.04\endgroup & \begingroup\fontsize{8}{10}\selectfont 0.03\endgroup & \begingroup\fontsize{8}{10}\selectfont 0.02\endgroup & \begingroup\fontsize{8}{10}\selectfont 0.02\endgroup & \begingroup\fontsize{8}{10}\selectfont \endgroup & \begingroup\fontsize{8}{10}\selectfont 0.02\endgroup & \begingroup\fontsize{8}{10}\selectfont 0.10\endgroup & \begingroup\fontsize{8}{10}\selectfont 0.22\endgroup & \begingroup\fontsize{8}{10}\selectfont 0.02\endgroup\\
\begingroup\fontsize{8}{10}\selectfont C\endgroup & \begingroup\fontsize{8}{10}\selectfont Côte d’Ivoire\endgroup & \begingroup\fontsize{8}{10}\selectfont 0.26\endgroup & \begingroup\fontsize{8}{10}\selectfont 0.30\endgroup & \begingroup\fontsize{8}{10}\selectfont 0.03\endgroup & \begingroup\fontsize{8}{10}\selectfont 0.11\endgroup & \begingroup\fontsize{8}{10}\selectfont 0.08\endgroup & \begingroup\fontsize{8}{10}\selectfont 0.01\endgroup & \begingroup\fontsize{8}{10}\selectfont 0.02\endgroup & \begingroup\fontsize{8}{10}\selectfont \endgroup & \begingroup\fontsize{8}{10}\selectfont 0.01\endgroup & \begingroup\fontsize{8}{10}\selectfont 0.01\endgroup & \begingroup\fontsize{8}{10}\selectfont 0.16\endgroup & \begingroup\fontsize{8}{10}\selectfont 0.01\endgroup\\
\begingroup\fontsize{8}{10}\selectfont C\endgroup & \begingroup\fontsize{8}{10}\selectfont Senegal\endgroup & \begingroup\fontsize{8}{10}\selectfont 0.21\endgroup & \begingroup\fontsize{8}{10}\selectfont 0.27\endgroup & \begingroup\fontsize{8}{10}\selectfont 0.02\endgroup & \begingroup\fontsize{8}{10}\selectfont 0.03\endgroup & \begingroup\fontsize{8}{10}\selectfont 0.03\endgroup & \begingroup\fontsize{8}{10}\selectfont 0.04\endgroup & \begingroup\fontsize{8}{10}\selectfont 0.05\endgroup & \begingroup\fontsize{8}{10}\selectfont \endgroup & \begingroup\fontsize{8}{10}\selectfont 0.01\endgroup & \begingroup\fontsize{8}{10}\selectfont 0.04\endgroup & \begingroup\fontsize{8}{10}\selectfont 0.07\endgroup & \begingroup\fontsize{8}{10}\selectfont 0.05\endgroup\\
\begingroup\fontsize{8}{10}\selectfont C\endgroup & \begingroup\fontsize{8}{10}\selectfont Benin\endgroup & \begingroup\fontsize{8}{10}\selectfont 0.21\endgroup & \begingroup\fontsize{8}{10}\selectfont 0.49\endgroup & \begingroup\fontsize{8}{10}\selectfont 0.03\endgroup & \begingroup\fontsize{8}{10}\selectfont 0.09\endgroup & \begingroup\fontsize{8}{10}\selectfont 0.03\endgroup & \begingroup\fontsize{8}{10}\selectfont 0.01\endgroup & \begingroup\fontsize{8}{10}\selectfont 0.01\endgroup & \begingroup\fontsize{8}{10}\selectfont \endgroup & \begingroup\fontsize{8}{10}\selectfont 0.02\endgroup & \begingroup\fontsize{8}{10}\selectfont 0.03\endgroup & \begingroup\fontsize{8}{10}\selectfont 0.12\endgroup & \begingroup\fontsize{8}{10}\selectfont 0.01\endgroup\\
\begingroup\fontsize{8}{10}\selectfont C\endgroup & \begingroup\fontsize{8}{10}\selectfont Ghana\endgroup & \begingroup\fontsize{8}{10}\selectfont 0.13\endgroup & \begingroup\fontsize{8}{10}\selectfont 0.42\endgroup & \begingroup\fontsize{8}{10}\selectfont 0.02\endgroup & \begingroup\fontsize{8}{10}\selectfont 0.05\endgroup & \begingroup\fontsize{8}{10}\selectfont 0.07\endgroup & \begingroup\fontsize{8}{10}\selectfont 0.01\endgroup & \begingroup\fontsize{8}{10}\selectfont 0.05\endgroup & \begingroup\fontsize{8}{10}\selectfont \endgroup & \begingroup\fontsize{8}{10}\selectfont 0.01\endgroup & \begingroup\fontsize{8}{10}\selectfont 0.04\endgroup & \begingroup\fontsize{8}{10}\selectfont 0.08\endgroup & \begingroup\fontsize{8}{10}\selectfont 0.03\endgroup\\
\begingroup\fontsize{8}{10}\selectfont C\endgroup & \begingroup\fontsize{8}{10}\selectfont India\endgroup & \begingroup\fontsize{8}{10}\selectfont 0.07\endgroup & \begingroup\fontsize{8}{10}\selectfont 0.99\endgroup & \begingroup\fontsize{8}{10}\selectfont 0.03\endgroup & \begingroup\fontsize{8}{10}\selectfont 0.04\endgroup & \begingroup\fontsize{8}{10}\selectfont 0.14\endgroup & \begingroup\fontsize{8}{10}\selectfont 0.01\endgroup & \begingroup\fontsize{8}{10}\selectfont 0.06\endgroup & \begingroup\fontsize{8}{10}\selectfont \endgroup & \begingroup\fontsize{8}{10}\selectfont 0.00\endgroup & \begingroup\fontsize{8}{10}\selectfont 0.02\endgroup & \begingroup\fontsize{8}{10}\selectfont 0.08\endgroup & \begingroup\fontsize{8}{10}\selectfont 0.03\endgroup\\
\begingroup\fontsize{8}{10}\selectfont C\endgroup & \begingroup\fontsize{8}{10}\selectfont Pakistan\endgroup & \begingroup\fontsize{8}{10}\selectfont -0.09\endgroup & \begingroup\fontsize{8}{10}\selectfont 0.41\endgroup & \begingroup\fontsize{8}{10}\selectfont 0.06\endgroup & \begingroup\fontsize{8}{10}\selectfont 0.07\endgroup & \begingroup\fontsize{8}{10}\selectfont 0.10\endgroup & \begingroup\fontsize{8}{10}\selectfont 0.01\endgroup & \begingroup\fontsize{8}{10}\selectfont \endgroup & \begingroup\fontsize{8}{10}\selectfont \endgroup & \begingroup\fontsize{8}{10}\selectfont \endgroup & \begingroup\fontsize{8}{10}\selectfont \endgroup & \begingroup\fontsize{8}{10}\selectfont \endgroup & \begingroup\fontsize{8}{10}\selectfont \endgroup\\
\midrule
\begingroup\fontsize{8}{10}\selectfont D\endgroup & \begingroup\fontsize{8}{10}\selectfont Peru\endgroup & \begingroup\fontsize{8}{10}\selectfont 0.28\endgroup & \begingroup\fontsize{8}{10}\selectfont 1.50\endgroup & \begingroup\fontsize{8}{10}\selectfont 0.18\endgroup & \begingroup\fontsize{8}{10}\selectfont 0.04\endgroup & \begingroup\fontsize{8}{10}\selectfont 0.05\endgroup & \begingroup\fontsize{8}{10}\selectfont 0.00\endgroup & \begingroup\fontsize{8}{10}\selectfont 0.19\endgroup & \begingroup\fontsize{8}{10}\selectfont \endgroup & \begingroup\fontsize{8}{10}\selectfont 0.00\endgroup & \begingroup\fontsize{8}{10}\selectfont 0.02\endgroup & \begingroup\fontsize{8}{10}\selectfont 0.02\endgroup & \begingroup\fontsize{8}{10}\selectfont 0.03\endgroup\\
\begingroup\fontsize{8}{10}\selectfont D\endgroup & \begingroup\fontsize{8}{10}\selectfont Ecuador\endgroup & \begingroup\fontsize{8}{10}\selectfont 0.19\endgroup & \begingroup\fontsize{8}{10}\selectfont 1.26\endgroup & \begingroup\fontsize{8}{10}\selectfont 0.14\endgroup & \begingroup\fontsize{8}{10}\selectfont 0.03\endgroup & \begingroup\fontsize{8}{10}\selectfont 0.03\endgroup & \begingroup\fontsize{8}{10}\selectfont 0.00\endgroup & \begingroup\fontsize{8}{10}\selectfont 0.06\endgroup & \begingroup\fontsize{8}{10}\selectfont \endgroup & \begingroup\fontsize{8}{10}\selectfont 0.00\endgroup & \begingroup\fontsize{8}{10}\selectfont 0.13\endgroup & \begingroup\fontsize{8}{10}\selectfont 0.05\endgroup & \begingroup\fontsize{8}{10}\selectfont 0.02\endgroup\\
\begingroup\fontsize{8}{10}\selectfont D\endgroup & \begingroup\fontsize{8}{10}\selectfont Nicaragua\endgroup & \begingroup\fontsize{8}{10}\selectfont 0.11\endgroup & \begingroup\fontsize{8}{10}\selectfont 0.11\endgroup & \begingroup\fontsize{8}{10}\selectfont 0.05\endgroup & \begingroup\fontsize{8}{10}\selectfont 0.04\endgroup & \begingroup\fontsize{8}{10}\selectfont 0.03\endgroup & \begingroup\fontsize{8}{10}\selectfont 0.01\endgroup & \begingroup\fontsize{8}{10}\selectfont 0.15\endgroup & \begingroup\fontsize{8}{10}\selectfont \endgroup & \begingroup\fontsize{8}{10}\selectfont 0.00\endgroup & \begingroup\fontsize{8}{10}\selectfont 0.12\endgroup & \begingroup\fontsize{8}{10}\selectfont 0.10\endgroup & \begingroup\fontsize{8}{10}\selectfont 0.01\endgroup\\
\begingroup\fontsize{8}{10}\selectfont D\endgroup & \begingroup\fontsize{8}{10}\selectfont Bolivia\endgroup & \begingroup\fontsize{8}{10}\selectfont 0.08\endgroup & \begingroup\fontsize{8}{10}\selectfont 1.05\endgroup & \begingroup\fontsize{8}{10}\selectfont 0.10\endgroup & \begingroup\fontsize{8}{10}\selectfont 0.04\endgroup & \begingroup\fontsize{8}{10}\selectfont 0.06\endgroup & \begingroup\fontsize{8}{10}\selectfont 0.01\endgroup & \begingroup\fontsize{8}{10}\selectfont 0.06\endgroup & \begingroup\fontsize{8}{10}\selectfont \endgroup & \begingroup\fontsize{8}{10}\selectfont \endgroup & \begingroup\fontsize{8}{10}\selectfont 0.05\endgroup & \begingroup\fontsize{8}{10}\selectfont 0.04\endgroup & \begingroup\fontsize{8}{10}\selectfont 0.04\endgroup\\
\begingroup\fontsize{8}{10}\selectfont D\endgroup & \begingroup\fontsize{8}{10}\selectfont El Salvador\endgroup & \begingroup\fontsize{8}{10}\selectfont -0.04\endgroup & \begingroup\fontsize{8}{10}\selectfont 0.75\endgroup & \begingroup\fontsize{8}{10}\selectfont 0.08\endgroup & \begingroup\fontsize{8}{10}\selectfont 0.06\endgroup & \begingroup\fontsize{8}{10}\selectfont 0.02\endgroup & \begingroup\fontsize{8}{10}\selectfont 0.00\endgroup & \begingroup\fontsize{8}{10}\selectfont 0.06\endgroup & \begingroup\fontsize{8}{10}\selectfont \endgroup & \begingroup\fontsize{8}{10}\selectfont 0.00\endgroup & \begingroup\fontsize{8}{10}\selectfont 0.05\endgroup & \begingroup\fontsize{8}{10}\selectfont 0.01\endgroup & \begingroup\fontsize{8}{10}\selectfont 0.01\endgroup\\
\begingroup\fontsize{8}{10}\selectfont D\endgroup & \begingroup\fontsize{8}{10}\selectfont Iraq\endgroup & \begingroup\fontsize{8}{10}\selectfont -0.08\endgroup & \begingroup\fontsize{8}{10}\selectfont 1.65\endgroup & \begingroup\fontsize{8}{10}\selectfont 0.14\endgroup & \begingroup\fontsize{8}{10}\selectfont 0.03\endgroup & \begingroup\fontsize{8}{10}\selectfont 0.01\endgroup & \begingroup\fontsize{8}{10}\selectfont 0.00\endgroup & \begingroup\fontsize{8}{10}\selectfont 0.00\endgroup & \begingroup\fontsize{8}{10}\selectfont 0.01\endgroup & \begingroup\fontsize{8}{10}\selectfont 0.00\endgroup & \begingroup\fontsize{8}{10}\selectfont 0.07\endgroup & \begingroup\fontsize{8}{10}\selectfont 0.00\endgroup & \begingroup\fontsize{8}{10}\selectfont 0.02\endgroup\\
\begingroup\fontsize{8}{10}\selectfont D\endgroup & \begingroup\fontsize{8}{10}\selectfont Paraguay\endgroup & \begingroup\fontsize{8}{10}\selectfont -0.14\endgroup & \begingroup\fontsize{8}{10}\selectfont 0.82\endgroup & \begingroup\fontsize{8}{10}\selectfont 0.03\endgroup & \begingroup\fontsize{8}{10}\selectfont 0.02\endgroup & \begingroup\fontsize{8}{10}\selectfont 0.01\endgroup & \begingroup\fontsize{8}{10}\selectfont 0.00\endgroup & \begingroup\fontsize{8}{10}\selectfont 0.10\endgroup & \begingroup\fontsize{8}{10}\selectfont \endgroup & \begingroup\fontsize{8}{10}\selectfont \endgroup & \begingroup\fontsize{8}{10}\selectfont 0.03\endgroup & \begingroup\fontsize{8}{10}\selectfont 0.03\endgroup & \begingroup\fontsize{8}{10}\selectfont 0.01\endgroup\\
\midrule
\begingroup\fontsize{8}{10}\selectfont E\endgroup & \begingroup\fontsize{8}{10}\selectfont Uganda\endgroup & \begingroup\fontsize{8}{10}\selectfont 0.59\endgroup & \begingroup\fontsize{8}{10}\selectfont 0.41\endgroup & \begingroup\fontsize{8}{10}\selectfont 0.04\endgroup & \begingroup\fontsize{8}{10}\selectfont 0.03\endgroup & \begingroup\fontsize{8}{10}\selectfont 0.04\endgroup & \begingroup\fontsize{8}{10}\selectfont 0.01\endgroup & \begingroup\fontsize{8}{10}\selectfont 0.02\endgroup & \begingroup\fontsize{8}{10}\selectfont \endgroup & \begingroup\fontsize{8}{10}\selectfont 0.10\endgroup & \begingroup\fontsize{8}{10}\selectfont 0.02\endgroup & \begingroup\fontsize{8}{10}\selectfont 0.04\endgroup & \begingroup\fontsize{8}{10}\selectfont 0.00\endgroup\\
\begingroup\fontsize{8}{10}\selectfont E\endgroup & \begingroup\fontsize{8}{10}\selectfont Rwanda\endgroup & \begingroup\fontsize{8}{10}\selectfont 0.58\endgroup & \begingroup\fontsize{8}{10}\selectfont 0.17\endgroup & \begingroup\fontsize{8}{10}\selectfont 0.06\endgroup & \begingroup\fontsize{8}{10}\selectfont 0.03\endgroup & \begingroup\fontsize{8}{10}\selectfont 0.07\endgroup & \begingroup\fontsize{8}{10}\selectfont \endgroup & \begingroup\fontsize{8}{10}\selectfont 0.05\endgroup & \begingroup\fontsize{8}{10}\selectfont \endgroup & \begingroup\fontsize{8}{10}\selectfont 0.09\endgroup & \begingroup\fontsize{8}{10}\selectfont 0.09\endgroup & \begingroup\fontsize{8}{10}\selectfont 0.06\endgroup & \begingroup\fontsize{8}{10}\selectfont 0.02\endgroup\\
\midrule
\begingroup\fontsize{8}{10}\selectfont F\endgroup & \begingroup\fontsize{8}{10}\selectfont Turkey\endgroup & \begingroup\fontsize{8}{10}\selectfont 0.61\endgroup & \begingroup\fontsize{8}{10}\selectfont 1.06\endgroup & \begingroup\fontsize{8}{10}\selectfont 0.02\endgroup & \begingroup\fontsize{8}{10}\selectfont 0.02\endgroup & \begingroup\fontsize{8}{10}\selectfont 0.01\endgroup & \begingroup\fontsize{8}{10}\selectfont \endgroup & \begingroup\fontsize{8}{10}\selectfont 0.03\endgroup & \begingroup\fontsize{8}{10}\selectfont 0.09\endgroup & \begingroup\fontsize{8}{10}\selectfont \endgroup & \begingroup\fontsize{8}{10}\selectfont 0.03\endgroup & \begingroup\fontsize{8}{10}\selectfont 0.00\endgroup & \begingroup\fontsize{8}{10}\selectfont 0.02\endgroup\\
\begingroup\fontsize{8}{10}\selectfont F\endgroup & \begingroup\fontsize{8}{10}\selectfont Armenia\endgroup & \begingroup\fontsize{8}{10}\selectfont 0.55\endgroup & \begingroup\fontsize{8}{10}\selectfont 1.85\endgroup & \begingroup\fontsize{8}{10}\selectfont 0.06\endgroup & \begingroup\fontsize{8}{10}\selectfont 0.02\endgroup & \begingroup\fontsize{8}{10}\selectfont 0.05\endgroup & \begingroup\fontsize{8}{10}\selectfont 0.00\endgroup & \begingroup\fontsize{8}{10}\selectfont \endgroup & \begingroup\fontsize{8}{10}\selectfont 0.08\endgroup & \begingroup\fontsize{8}{10}\selectfont \endgroup & \begingroup\fontsize{8}{10}\selectfont 0.05\endgroup & \begingroup\fontsize{8}{10}\selectfont 0.00\endgroup & \begingroup\fontsize{8}{10}\selectfont 0.01\endgroup\\*
\end{longtable}
\end{ThreePartTable}
\endgroup{}

\clearpage

\begingroup\fontsize{9}{11}\selectfont

\begin{ThreePartTable}
\begin{TableNotes}
\item \textit{Note: } 
\item This table shows feature importance in percent (based on absolute average SHAP-values per feature) across all countries and per cluster. Feature importance is unadjusted for model accuracy. Columns 'Mean carbon intensity', 'Horizontal inequality' and 'Vertical inequality' show average values. Column 'number' refers to the number of countries assigned to this cluster.
\end{TableNotes}
\begin{longtable}[t]{>{\raggedright\arraybackslash}p{0.35 cm}>{\raggedright\arraybackslash}p{0.35 cm}>{\raggedleft\arraybackslash}p{0.9 cm}>{\raggedleft\arraybackslash}p{0.35 cm}>{\raggedleft\arraybackslash}p{0.35 cm}>{\raggedleft\arraybackslash}p{0.35 cm}>{\raggedleft\arraybackslash}p{0.35 cm}>{\raggedleft\arraybackslash}p{0.35 cm}>{\raggedleft\arraybackslash}p{0.35 cm}>{\raggedleft\arraybackslash}p{0.35 cm}>{\raggedleft\arraybackslash}p{0.35 cm}>{\raggedleft\arraybackslash}p{0.35 cm}>{\raggedleft\arraybackslash}p{0.35 cm}>{\raggedleft\arraybackslash}p{0.35 cm}>{\raggedleft\arraybackslash}p{0.35 cm}>{\raggedleft\arraybackslash}p{0.35 cm}>{}p{0.35 cm}>{}p{0.35 cm}>{}p{0.35 cm}>{}p{0.35 cm}>{}p{0.35 cm}}
\caption{\label{tab:A10_Uncorrected}Feature importance across countries by cluster}\\
\toprule
\rotatebox{90}{Cluster} & \rotatebox{90}{Country} & \rotatebox{90}{Silhouette width} & \rotatebox{90}{Horizontal inequality} & \rotatebox{90}{Vertical inequality} & \rotatebox{90}{Mean carbon intensity} & \rotatebox{90}{HH expenditures} & \rotatebox{90}{Sociodemographic} & \rotatebox{90}{Spatial} & \rotatebox{90}{Electricity access} & \rotatebox{90}{Cooking fuel} & \rotatebox{90}{Heating fuel} & \rotatebox{90}{Lighting fuel} & \rotatebox{90}{Car own.} & \rotatebox{90}{Motorcycle own.} & \rotatebox{90}{Appliance own.}\\
\midrule
\endfirsthead
\caption[]{Feature importance across countries by cluster \textit{(continued)}}\\
\toprule
\rotatebox{90}{Cluster} & \rotatebox{90}{Country} & \rotatebox{90}{Silhouette width} & \rotatebox{90}{Horizontal inequality} & \rotatebox{90}{Vertical inequality} & \rotatebox{90}{Mean carbon intensity} & \rotatebox{90}{HH expenditures} & \rotatebox{90}{Sociodemographic} & \rotatebox{90}{Spatial} & \rotatebox{90}{Electricity access} & \rotatebox{90}{Cooking fuel} & \rotatebox{90}{Heating fuel} & \rotatebox{90}{Lighting fuel} & \rotatebox{90}{Car own.} & \rotatebox{90}{Motorcycle own.} & \rotatebox{90}{Appliance own.}\\
\midrule
\endhead

\endfoot
\bottomrule
\insertTableNotes
\endlastfoot
\begingroup\fontsize{8}{10}\selectfont A\endgroup & \begingroup\fontsize{8}{10}\selectfont CYP\endgroup & \begingroup\fontsize{8}{10}\selectfont 0.49\endgroup & \begingroup\fontsize{8}{10}\selectfont 1.61\endgroup & \begingroup\fontsize{8}{10}\selectfont 1.29\endgroup & \begingroup\fontsize{8}{10}\selectfont 0.73\endgroup & \begingroup\fontsize{8}{10}\selectfont 0.41\endgroup & \begingroup\fontsize{8}{10}\selectfont 0.46\endgroup & \begingroup\fontsize{8}{10}\selectfont 0.14\endgroup & \begingroup\fontsize{8}{10}\selectfont \endgroup & \begingroup\fontsize{8}{10}\selectfont \endgroup & \begingroup\fontsize{8}{10}\selectfont \endgroup & \begingroup\fontsize{8}{10}\selectfont \endgroup & \begingroup\fontsize{8}{10}\selectfont \endgroup & \begingroup\fontsize{8}{10}\selectfont \endgroup & \begingroup\fontsize{8}{10}\selectfont \endgroup\\
\begingroup\fontsize{8}{10}\selectfont A\endgroup & \begingroup\fontsize{8}{10}\selectfont EST\endgroup & \begingroup\fontsize{8}{10}\selectfont 0.48\endgroup & \begingroup\fontsize{8}{10}\selectfont 1.54\endgroup & \begingroup\fontsize{8}{10}\selectfont 1.37\endgroup & \begingroup\fontsize{8}{10}\selectfont 0.79\endgroup & \begingroup\fontsize{8}{10}\selectfont 0.43\endgroup & \begingroup\fontsize{8}{10}\selectfont 0.49\endgroup & \begingroup\fontsize{8}{10}\selectfont 0.08\endgroup & \begingroup\fontsize{8}{10}\selectfont \endgroup & \begingroup\fontsize{8}{10}\selectfont \endgroup & \begingroup\fontsize{8}{10}\selectfont \endgroup & \begingroup\fontsize{8}{10}\selectfont \endgroup & \begingroup\fontsize{8}{10}\selectfont \endgroup & \begingroup\fontsize{8}{10}\selectfont \endgroup & \begingroup\fontsize{8}{10}\selectfont \endgroup\\
\begingroup\fontsize{8}{10}\selectfont A\endgroup & \begingroup\fontsize{8}{10}\selectfont BGR\endgroup & \begingroup\fontsize{8}{10}\selectfont 0.43\endgroup & \begingroup\fontsize{8}{10}\selectfont 1.26\endgroup & \begingroup\fontsize{8}{10}\selectfont 0.93\endgroup & \begingroup\fontsize{8}{10}\selectfont 0.80\endgroup & \begingroup\fontsize{8}{10}\selectfont 0.43\endgroup & \begingroup\fontsize{8}{10}\selectfont 0.52\endgroup & \begingroup\fontsize{8}{10}\selectfont 0.04\endgroup & \begingroup\fontsize{8}{10}\selectfont \endgroup & \begingroup\fontsize{8}{10}\selectfont \endgroup & \begingroup\fontsize{8}{10}\selectfont \endgroup & \begingroup\fontsize{8}{10}\selectfont \endgroup & \begingroup\fontsize{8}{10}\selectfont \endgroup & \begingroup\fontsize{8}{10}\selectfont \endgroup & \begingroup\fontsize{8}{10}\selectfont \endgroup\\
\begingroup\fontsize{8}{10}\selectfont A\endgroup & \begingroup\fontsize{8}{10}\selectfont CHL\endgroup & \begingroup\fontsize{8}{10}\selectfont 0.40\endgroup & \begingroup\fontsize{8}{10}\selectfont 1.99\endgroup & \begingroup\fontsize{8}{10}\selectfont 1.59\endgroup & \begingroup\fontsize{8}{10}\selectfont 0.68\endgroup & \begingroup\fontsize{8}{10}\selectfont 0.43\endgroup & \begingroup\fontsize{8}{10}\selectfont 0.55\endgroup & \begingroup\fontsize{8}{10}\selectfont 0.01\endgroup & \begingroup\fontsize{8}{10}\selectfont \endgroup & \begingroup\fontsize{8}{10}\selectfont \endgroup & \begingroup\fontsize{8}{10}\selectfont \endgroup & \begingroup\fontsize{8}{10}\selectfont \endgroup & \begingroup\fontsize{8}{10}\selectfont \endgroup & \begingroup\fontsize{8}{10}\selectfont \endgroup & \begingroup\fontsize{8}{10}\selectfont \endgroup\\
\begingroup\fontsize{8}{10}\selectfont A\endgroup & \begingroup\fontsize{8}{10}\selectfont PRT\endgroup & \begingroup\fontsize{8}{10}\selectfont 0.39\endgroup & \begingroup\fontsize{8}{10}\selectfont 1.66\endgroup & \begingroup\fontsize{8}{10}\selectfont 1.62\endgroup & \begingroup\fontsize{8}{10}\selectfont 0.99\endgroup & \begingroup\fontsize{8}{10}\selectfont 0.27\endgroup & \begingroup\fontsize{8}{10}\selectfont 0.56\endgroup & \begingroup\fontsize{8}{10}\selectfont 0.17\endgroup & \begingroup\fontsize{8}{10}\selectfont \endgroup & \begingroup\fontsize{8}{10}\selectfont \endgroup & \begingroup\fontsize{8}{10}\selectfont \endgroup & \begingroup\fontsize{8}{10}\selectfont \endgroup & \begingroup\fontsize{8}{10}\selectfont \endgroup & \begingroup\fontsize{8}{10}\selectfont \endgroup & \begingroup\fontsize{8}{10}\selectfont \endgroup\\
\begingroup\fontsize{8}{10}\selectfont A\endgroup & \begingroup\fontsize{8}{10}\selectfont USA\endgroup & \begingroup\fontsize{8}{10}\selectfont 0.34\endgroup & \begingroup\fontsize{8}{10}\selectfont 1.90\endgroup & \begingroup\fontsize{8}{10}\selectfont 1.55\endgroup & \begingroup\fontsize{8}{10}\selectfont 0.99\endgroup & \begingroup\fontsize{8}{10}\selectfont 0.36\endgroup & \begingroup\fontsize{8}{10}\selectfont 0.40\endgroup & \begingroup\fontsize{8}{10}\selectfont 0.24\endgroup & \begingroup\fontsize{8}{10}\selectfont \endgroup & \begingroup\fontsize{8}{10}\selectfont \endgroup & \begingroup\fontsize{8}{10}\selectfont \endgroup & \begingroup\fontsize{8}{10}\selectfont \endgroup & \begingroup\fontsize{8}{10}\selectfont \endgroup & \begingroup\fontsize{8}{10}\selectfont \endgroup & \begingroup\fontsize{8}{10}\selectfont \endgroup\\
\begingroup\fontsize{8}{10}\selectfont A\endgroup & \begingroup\fontsize{8}{10}\selectfont ITA\endgroup & \begingroup\fontsize{8}{10}\selectfont 0.32\endgroup & \begingroup\fontsize{8}{10}\selectfont 1.57\endgroup & \begingroup\fontsize{8}{10}\selectfont 1.43\endgroup & \begingroup\fontsize{8}{10}\selectfont 0.85\endgroup & \begingroup\fontsize{8}{10}\selectfont 0.31\endgroup & \begingroup\fontsize{8}{10}\selectfont 0.42\endgroup & \begingroup\fontsize{8}{10}\selectfont 0.27\endgroup & \begingroup\fontsize{8}{10}\selectfont \endgroup & \begingroup\fontsize{8}{10}\selectfont \endgroup & \begingroup\fontsize{8}{10}\selectfont \endgroup & \begingroup\fontsize{8}{10}\selectfont \endgroup & \begingroup\fontsize{8}{10}\selectfont \endgroup & \begingroup\fontsize{8}{10}\selectfont \endgroup & \begingroup\fontsize{8}{10}\selectfont \endgroup\\
\begingroup\fontsize{8}{10}\selectfont A\endgroup & \begingroup\fontsize{8}{10}\selectfont DNK\endgroup & \begingroup\fontsize{8}{10}\selectfont 0.25\endgroup & \begingroup\fontsize{8}{10}\selectfont 1.46\endgroup & \begingroup\fontsize{8}{10}\selectfont 1.13\endgroup & \begingroup\fontsize{8}{10}\selectfont 0.50\endgroup & \begingroup\fontsize{8}{10}\selectfont 0.19\endgroup & \begingroup\fontsize{8}{10}\selectfont 0.55\endgroup & \begingroup\fontsize{8}{10}\selectfont 0.26\endgroup & \begingroup\fontsize{8}{10}\selectfont \endgroup & \begingroup\fontsize{8}{10}\selectfont \endgroup & \begingroup\fontsize{8}{10}\selectfont \endgroup & \begingroup\fontsize{8}{10}\selectfont \endgroup & \begingroup\fontsize{8}{10}\selectfont \endgroup & \begingroup\fontsize{8}{10}\selectfont \endgroup & \begingroup\fontsize{8}{10}\selectfont \endgroup\\
\begingroup\fontsize{8}{10}\selectfont A\endgroup & \begingroup\fontsize{8}{10}\selectfont LUX\endgroup & \begingroup\fontsize{8}{10}\selectfont 0.23\endgroup & \begingroup\fontsize{8}{10}\selectfont 1.76\endgroup & \begingroup\fontsize{8}{10}\selectfont 1.66\endgroup & \begingroup\fontsize{8}{10}\selectfont 0.54\endgroup & \begingroup\fontsize{8}{10}\selectfont 0.56\endgroup & \begingroup\fontsize{8}{10}\selectfont 0.34\endgroup & \begingroup\fontsize{8}{10}\selectfont 0.10\endgroup & \begingroup\fontsize{8}{10}\selectfont \endgroup & \begingroup\fontsize{8}{10}\selectfont \endgroup & \begingroup\fontsize{8}{10}\selectfont \endgroup & \begingroup\fontsize{8}{10}\selectfont \endgroup & \begingroup\fontsize{8}{10}\selectfont \endgroup & \begingroup\fontsize{8}{10}\selectfont \endgroup & \begingroup\fontsize{8}{10}\selectfont \endgroup\\
\begingroup\fontsize{8}{10}\selectfont A\endgroup & \begingroup\fontsize{8}{10}\selectfont LTU\endgroup & \begingroup\fontsize{8}{10}\selectfont 0.16\endgroup & \begingroup\fontsize{8}{10}\selectfont 1.02\endgroup & \begingroup\fontsize{8}{10}\selectfont 0.53\endgroup & \begingroup\fontsize{8}{10}\selectfont 0.47\endgroup & \begingroup\fontsize{8}{10}\selectfont 0.35\endgroup & \begingroup\fontsize{8}{10}\selectfont 0.46\endgroup & \begingroup\fontsize{8}{10}\selectfont 0.19\endgroup & \begingroup\fontsize{8}{10}\selectfont \endgroup & \begingroup\fontsize{8}{10}\selectfont \endgroup & \begingroup\fontsize{8}{10}\selectfont \endgroup & \begingroup\fontsize{8}{10}\selectfont \endgroup & \begingroup\fontsize{8}{10}\selectfont \endgroup & \begingroup\fontsize{8}{10}\selectfont \endgroup & \begingroup\fontsize{8}{10}\selectfont \endgroup\\
\begingroup\fontsize{8}{10}\selectfont A\endgroup & \begingroup\fontsize{8}{10}\selectfont DEU\endgroup & \begingroup\fontsize{8}{10}\selectfont 0.15\endgroup & \begingroup\fontsize{8}{10}\selectfont 1.45\endgroup & \begingroup\fontsize{8}{10}\selectfont 1.37\endgroup & \begingroup\fontsize{8}{10}\selectfont 1.22\endgroup & \begingroup\fontsize{8}{10}\selectfont 0.24\endgroup & \begingroup\fontsize{8}{10}\selectfont 0.46\endgroup & \begingroup\fontsize{8}{10}\selectfont 0.30\endgroup & \begingroup\fontsize{8}{10}\selectfont \endgroup & \begingroup\fontsize{8}{10}\selectfont \endgroup & \begingroup\fontsize{8}{10}\selectfont \endgroup & \begingroup\fontsize{8}{10}\selectfont \endgroup & \begingroup\fontsize{8}{10}\selectfont \endgroup & \begingroup\fontsize{8}{10}\selectfont \endgroup & \begingroup\fontsize{8}{10}\selectfont \endgroup\\
\begingroup\fontsize{8}{10}\selectfont A\endgroup & \begingroup\fontsize{8}{10}\selectfont HRV\endgroup & \begingroup\fontsize{8}{10}\selectfont 0.14\endgroup & \begingroup\fontsize{8}{10}\selectfont 0.82\endgroup & \begingroup\fontsize{8}{10}\selectfont 0.70\endgroup & \begingroup\fontsize{8}{10}\selectfont 0.77\endgroup & \begingroup\fontsize{8}{10}\selectfont 0.55\endgroup & \begingroup\fontsize{8}{10}\selectfont 0.34\endgroup & \begingroup\fontsize{8}{10}\selectfont 0.11\endgroup & \begingroup\fontsize{8}{10}\selectfont \endgroup & \begingroup\fontsize{8}{10}\selectfont \endgroup & \begingroup\fontsize{8}{10}\selectfont \endgroup & \begingroup\fontsize{8}{10}\selectfont \endgroup & \begingroup\fontsize{8}{10}\selectfont \endgroup & \begingroup\fontsize{8}{10}\selectfont \endgroup & \begingroup\fontsize{8}{10}\selectfont \endgroup\\
\begingroup\fontsize{8}{10}\selectfont A\endgroup & \begingroup\fontsize{8}{10}\selectfont IRL\endgroup & \begingroup\fontsize{8}{10}\selectfont 0.11\endgroup & \begingroup\fontsize{8}{10}\selectfont 1.99\endgroup & \begingroup\fontsize{8}{10}\selectfont 1.44\endgroup & \begingroup\fontsize{8}{10}\selectfont 0.95\endgroup & \begingroup\fontsize{8}{10}\selectfont 0.32\endgroup & \begingroup\fontsize{8}{10}\selectfont 0.35\endgroup & \begingroup\fontsize{8}{10}\selectfont 0.33\endgroup & \begingroup\fontsize{8}{10}\selectfont \endgroup & \begingroup\fontsize{8}{10}\selectfont \endgroup & \begingroup\fontsize{8}{10}\selectfont \endgroup & \begingroup\fontsize{8}{10}\selectfont \endgroup & \begingroup\fontsize{8}{10}\selectfont \endgroup & \begingroup\fontsize{8}{10}\selectfont \endgroup & \begingroup\fontsize{8}{10}\selectfont \endgroup\\
\begingroup\fontsize{8}{10}\selectfont A\endgroup & \begingroup\fontsize{8}{10}\selectfont NLD\endgroup & \begingroup\fontsize{8}{10}\selectfont 0.10\endgroup & \begingroup\fontsize{8}{10}\selectfont 1.32\endgroup & \begingroup\fontsize{8}{10}\selectfont 1.42\endgroup & \begingroup\fontsize{8}{10}\selectfont 0.83\endgroup & \begingroup\fontsize{8}{10}\selectfont 0.43\endgroup & \begingroup\fontsize{8}{10}\selectfont 0.26\endgroup & \begingroup\fontsize{8}{10}\selectfont 0.31\endgroup & \begingroup\fontsize{8}{10}\selectfont \endgroup & \begingroup\fontsize{8}{10}\selectfont \endgroup & \begingroup\fontsize{8}{10}\selectfont \endgroup & \begingroup\fontsize{8}{10}\selectfont \endgroup & \begingroup\fontsize{8}{10}\selectfont \endgroup & \begingroup\fontsize{8}{10}\selectfont \endgroup & \begingroup\fontsize{8}{10}\selectfont \endgroup\\
\begingroup\fontsize{8}{10}\selectfont A\endgroup & \begingroup\fontsize{8}{10}\selectfont GRC\endgroup & \begingroup\fontsize{8}{10}\selectfont 0.10\endgroup & \begingroup\fontsize{8}{10}\selectfont 1.44\endgroup & \begingroup\fontsize{8}{10}\selectfont 1.31\endgroup & \begingroup\fontsize{8}{10}\selectfont 0.76\endgroup & \begingroup\fontsize{8}{10}\selectfont 0.25\endgroup & \begingroup\fontsize{8}{10}\selectfont 0.42\endgroup & \begingroup\fontsize{8}{10}\selectfont 0.33\endgroup & \begingroup\fontsize{8}{10}\selectfont \endgroup & \begingroup\fontsize{8}{10}\selectfont \endgroup & \begingroup\fontsize{8}{10}\selectfont \endgroup & \begingroup\fontsize{8}{10}\selectfont \endgroup & \begingroup\fontsize{8}{10}\selectfont \endgroup & \begingroup\fontsize{8}{10}\selectfont \endgroup & \begingroup\fontsize{8}{10}\selectfont \endgroup\\
\begingroup\fontsize{8}{10}\selectfont A\endgroup & \begingroup\fontsize{8}{10}\selectfont SUR\endgroup & \begingroup\fontsize{8}{10}\selectfont 0.07\endgroup & \begingroup\fontsize{8}{10}\selectfont 1.54\endgroup & \begingroup\fontsize{8}{10}\selectfont 1.55\endgroup & \begingroup\fontsize{8}{10}\selectfont 0.24\endgroup & \begingroup\fontsize{8}{10}\selectfont 0.12\endgroup & \begingroup\fontsize{8}{10}\selectfont 0.44\endgroup & \begingroup\fontsize{8}{10}\selectfont 0.26\endgroup & \begingroup\fontsize{8}{10}\selectfont \endgroup & \begingroup\fontsize{8}{10}\selectfont 0.03\endgroup & \begingroup\fontsize{8}{10}\selectfont \endgroup & \begingroup\fontsize{8}{10}\selectfont 0.01\endgroup & \begingroup\fontsize{8}{10}\selectfont 0.04\endgroup & \begingroup\fontsize{8}{10}\selectfont \endgroup & \begingroup\fontsize{8}{10}\selectfont 0.10\endgroup\\
\begingroup\fontsize{8}{10}\selectfont A\endgroup & \begingroup\fontsize{8}{10}\selectfont HUN\endgroup & \begingroup\fontsize{8}{10}\selectfont 0.02\endgroup & \begingroup\fontsize{8}{10}\selectfont 1.13\endgroup & \begingroup\fontsize{8}{10}\selectfont 0.83\endgroup & \begingroup\fontsize{8}{10}\selectfont 1.14\endgroup & \begingroup\fontsize{8}{10}\selectfont 0.23\endgroup & \begingroup\fontsize{8}{10}\selectfont 0.45\endgroup & \begingroup\fontsize{8}{10}\selectfont 0.33\endgroup & \begingroup\fontsize{8}{10}\selectfont \endgroup & \begingroup\fontsize{8}{10}\selectfont \endgroup & \begingroup\fontsize{8}{10}\selectfont \endgroup & \begingroup\fontsize{8}{10}\selectfont \endgroup & \begingroup\fontsize{8}{10}\selectfont \endgroup & \begingroup\fontsize{8}{10}\selectfont \endgroup & \begingroup\fontsize{8}{10}\selectfont \endgroup\\
\midrule
\begingroup\fontsize{8}{10}\selectfont B\endgroup & \begingroup\fontsize{8}{10}\selectfont FRA\endgroup & \begingroup\fontsize{8}{10}\selectfont 0.48\endgroup & \begingroup\fontsize{8}{10}\selectfont 1.51\endgroup & \begingroup\fontsize{8}{10}\selectfont 1.15\endgroup & \begingroup\fontsize{8}{10}\selectfont 0.65\endgroup & \begingroup\fontsize{8}{10}\selectfont 0.14\endgroup & \begingroup\fontsize{8}{10}\selectfont 0.28\endgroup & \begingroup\fontsize{8}{10}\selectfont 0.58\endgroup & \begingroup\fontsize{8}{10}\selectfont \endgroup & \begingroup\fontsize{8}{10}\selectfont \endgroup & \begingroup\fontsize{8}{10}\selectfont \endgroup & \begingroup\fontsize{8}{10}\selectfont \endgroup & \begingroup\fontsize{8}{10}\selectfont \endgroup & \begingroup\fontsize{8}{10}\selectfont \endgroup & \begingroup\fontsize{8}{10}\selectfont \endgroup\\
\begingroup\fontsize{8}{10}\selectfont B\endgroup & \begingroup\fontsize{8}{10}\selectfont ESP\endgroup & \begingroup\fontsize{8}{10}\selectfont 0.44\endgroup & \begingroup\fontsize{8}{10}\selectfont 1.50\endgroup & \begingroup\fontsize{8}{10}\selectfont 1.05\endgroup & \begingroup\fontsize{8}{10}\selectfont 0.74\endgroup & \begingroup\fontsize{8}{10}\selectfont 0.08\endgroup & \begingroup\fontsize{8}{10}\selectfont 0.32\endgroup & \begingroup\fontsize{8}{10}\selectfont 0.60\endgroup & \begingroup\fontsize{8}{10}\selectfont \endgroup & \begingroup\fontsize{8}{10}\selectfont \endgroup & \begingroup\fontsize{8}{10}\selectfont \endgroup & \begingroup\fontsize{8}{10}\selectfont \endgroup & \begingroup\fontsize{8}{10}\selectfont \endgroup & \begingroup\fontsize{8}{10}\selectfont \endgroup & \begingroup\fontsize{8}{10}\selectfont \endgroup\\
\begingroup\fontsize{8}{10}\selectfont B\endgroup & \begingroup\fontsize{8}{10}\selectfont LVA\endgroup & \begingroup\fontsize{8}{10}\selectfont 0.43\endgroup & \begingroup\fontsize{8}{10}\selectfont 1.40\endgroup & \begingroup\fontsize{8}{10}\selectfont 0.99\endgroup & \begingroup\fontsize{8}{10}\selectfont 0.67\endgroup & \begingroup\fontsize{8}{10}\selectfont 0.20\endgroup & \begingroup\fontsize{8}{10}\selectfont 0.29\endgroup & \begingroup\fontsize{8}{10}\selectfont 0.50\endgroup & \begingroup\fontsize{8}{10}\selectfont \endgroup & \begingroup\fontsize{8}{10}\selectfont \endgroup & \begingroup\fontsize{8}{10}\selectfont \endgroup & \begingroup\fontsize{8}{10}\selectfont \endgroup & \begingroup\fontsize{8}{10}\selectfont \endgroup & \begingroup\fontsize{8}{10}\selectfont \endgroup & \begingroup\fontsize{8}{10}\selectfont \endgroup\\
\begingroup\fontsize{8}{10}\selectfont B\endgroup & \begingroup\fontsize{8}{10}\selectfont MAR\endgroup & \begingroup\fontsize{8}{10}\selectfont 0.35\endgroup & \begingroup\fontsize{8}{10}\selectfont 0.93\endgroup & \begingroup\fontsize{8}{10}\selectfont 1.22\endgroup & \begingroup\fontsize{8}{10}\selectfont 0.62\endgroup & \begingroup\fontsize{8}{10}\selectfont 0.29\endgroup & \begingroup\fontsize{8}{10}\selectfont 0.21\endgroup & \begingroup\fontsize{8}{10}\selectfont 0.50\endgroup & \begingroup\fontsize{8}{10}\selectfont \endgroup & \begingroup\fontsize{8}{10}\selectfont \endgroup & \begingroup\fontsize{8}{10}\selectfont \endgroup & \begingroup\fontsize{8}{10}\selectfont \endgroup & \begingroup\fontsize{8}{10}\selectfont \endgroup & \begingroup\fontsize{8}{10}\selectfont \endgroup & \begingroup\fontsize{8}{10}\selectfont \endgroup\\
\begingroup\fontsize{8}{10}\selectfont B\endgroup & \begingroup\fontsize{8}{10}\selectfont SWE\endgroup & \begingroup\fontsize{8}{10}\selectfont 0.34\endgroup & \begingroup\fontsize{8}{10}\selectfont 1.66\endgroup & \begingroup\fontsize{8}{10}\selectfont 1.18\endgroup & \begingroup\fontsize{8}{10}\selectfont 0.49\endgroup & \begingroup\fontsize{8}{10}\selectfont 0.26\endgroup & \begingroup\fontsize{8}{10}\selectfont 0.16\endgroup & \begingroup\fontsize{8}{10}\selectfont 0.58\endgroup & \begingroup\fontsize{8}{10}\selectfont \endgroup & \begingroup\fontsize{8}{10}\selectfont \endgroup & \begingroup\fontsize{8}{10}\selectfont \endgroup & \begingroup\fontsize{8}{10}\selectfont \endgroup & \begingroup\fontsize{8}{10}\selectfont \endgroup & \begingroup\fontsize{8}{10}\selectfont \endgroup & \begingroup\fontsize{8}{10}\selectfont \endgroup\\
\begingroup\fontsize{8}{10}\selectfont B\endgroup & \begingroup\fontsize{8}{10}\selectfont ROU\endgroup & \begingroup\fontsize{8}{10}\selectfont 0.33\endgroup & \begingroup\fontsize{8}{10}\selectfont 0.78\endgroup & \begingroup\fontsize{8}{10}\selectfont 0.65\endgroup & \begingroup\fontsize{8}{10}\selectfont 0.79\endgroup & \begingroup\fontsize{8}{10}\selectfont 0.15\endgroup & \begingroup\fontsize{8}{10}\selectfont 0.30\endgroup & \begingroup\fontsize{8}{10}\selectfont 0.56\endgroup & \begingroup\fontsize{8}{10}\selectfont \endgroup & \begingroup\fontsize{8}{10}\selectfont \endgroup & \begingroup\fontsize{8}{10}\selectfont \endgroup & \begingroup\fontsize{8}{10}\selectfont \endgroup & \begingroup\fontsize{8}{10}\selectfont \endgroup & \begingroup\fontsize{8}{10}\selectfont \endgroup & \begingroup\fontsize{8}{10}\selectfont \endgroup\\
\begingroup\fontsize{8}{10}\selectfont B\endgroup & \begingroup\fontsize{8}{10}\selectfont CZE\endgroup & \begingroup\fontsize{8}{10}\selectfont 0.32\endgroup & \begingroup\fontsize{8}{10}\selectfont 1.39\endgroup & \begingroup\fontsize{8}{10}\selectfont 1.11\endgroup & \begingroup\fontsize{8}{10}\selectfont 1.71\endgroup & \begingroup\fontsize{8}{10}\selectfont 0.19\endgroup & \begingroup\fontsize{8}{10}\selectfont 0.27\endgroup & \begingroup\fontsize{8}{10}\selectfont 0.54\endgroup & \begingroup\fontsize{8}{10}\selectfont \endgroup & \begingroup\fontsize{8}{10}\selectfont \endgroup & \begingroup\fontsize{8}{10}\selectfont \endgroup & \begingroup\fontsize{8}{10}\selectfont \endgroup & \begingroup\fontsize{8}{10}\selectfont \endgroup & \begingroup\fontsize{8}{10}\selectfont \endgroup & \begingroup\fontsize{8}{10}\selectfont \endgroup\\
\begingroup\fontsize{8}{10}\selectfont B\endgroup & \begingroup\fontsize{8}{10}\selectfont SVK\endgroup & \begingroup\fontsize{8}{10}\selectfont 0.31\endgroup & \begingroup\fontsize{8}{10}\selectfont 1.76\endgroup & \begingroup\fontsize{8}{10}\selectfont 1.23\endgroup & \begingroup\fontsize{8}{10}\selectfont 1.06\endgroup & \begingroup\fontsize{8}{10}\selectfont 0.20\endgroup & \begingroup\fontsize{8}{10}\selectfont 0.34\endgroup & \begingroup\fontsize{8}{10}\selectfont 0.47\endgroup & \begingroup\fontsize{8}{10}\selectfont \endgroup & \begingroup\fontsize{8}{10}\selectfont \endgroup & \begingroup\fontsize{8}{10}\selectfont \endgroup & \begingroup\fontsize{8}{10}\selectfont \endgroup & \begingroup\fontsize{8}{10}\selectfont \endgroup & \begingroup\fontsize{8}{10}\selectfont \endgroup & \begingroup\fontsize{8}{10}\selectfont \endgroup\\
\begingroup\fontsize{8}{10}\selectfont B\endgroup & \begingroup\fontsize{8}{10}\selectfont FIN\endgroup & \begingroup\fontsize{8}{10}\selectfont 0.18\endgroup & \begingroup\fontsize{8}{10}\selectfont 1.30\endgroup & \begingroup\fontsize{8}{10}\selectfont 1.02\endgroup & \begingroup\fontsize{8}{10}\selectfont 0.50\endgroup & \begingroup\fontsize{8}{10}\selectfont 0.18\endgroup & \begingroup\fontsize{8}{10}\selectfont 0.40\endgroup & \begingroup\fontsize{8}{10}\selectfont 0.43\endgroup & \begingroup\fontsize{8}{10}\selectfont \endgroup & \begingroup\fontsize{8}{10}\selectfont \endgroup & \begingroup\fontsize{8}{10}\selectfont \endgroup & \begingroup\fontsize{8}{10}\selectfont \endgroup & \begingroup\fontsize{8}{10}\selectfont \endgroup & \begingroup\fontsize{8}{10}\selectfont \endgroup & \begingroup\fontsize{8}{10}\selectfont \endgroup\\
\begingroup\fontsize{8}{10}\selectfont B\endgroup & \begingroup\fontsize{8}{10}\selectfont POL\endgroup & \begingroup\fontsize{8}{10}\selectfont 0.16\endgroup & \begingroup\fontsize{8}{10}\selectfont 1.05\endgroup & \begingroup\fontsize{8}{10}\selectfont 1.47\endgroup & \begingroup\fontsize{8}{10}\selectfont 1.58\endgroup & \begingroup\fontsize{8}{10}\selectfont 0.27\endgroup & \begingroup\fontsize{8}{10}\selectfont 0.32\endgroup & \begingroup\fontsize{8}{10}\selectfont 0.42\endgroup & \begingroup\fontsize{8}{10}\selectfont \endgroup & \begingroup\fontsize{8}{10}\selectfont \endgroup & \begingroup\fontsize{8}{10}\selectfont \endgroup & \begingroup\fontsize{8}{10}\selectfont \endgroup & \begingroup\fontsize{8}{10}\selectfont \endgroup & \begingroup\fontsize{8}{10}\selectfont \endgroup & \begingroup\fontsize{8}{10}\selectfont \endgroup\\
\begingroup\fontsize{8}{10}\selectfont B\endgroup & \begingroup\fontsize{8}{10}\selectfont MNG\endgroup & \begingroup\fontsize{8}{10}\selectfont 0.07\endgroup & \begingroup\fontsize{8}{10}\selectfont 1.96\endgroup & \begingroup\fontsize{8}{10}\selectfont 2.38\endgroup & \begingroup\fontsize{8}{10}\selectfont 1.14\endgroup & \begingroup\fontsize{8}{10}\selectfont 0.32\endgroup & \begingroup\fontsize{8}{10}\selectfont 0.11\endgroup & \begingroup\fontsize{8}{10}\selectfont 0.56\endgroup & \begingroup\fontsize{8}{10}\selectfont \endgroup & \begingroup\fontsize{8}{10}\selectfont \endgroup & \begingroup\fontsize{8}{10}\selectfont \endgroup & \begingroup\fontsize{8}{10}\selectfont \endgroup & \begingroup\fontsize{8}{10}\selectfont \endgroup & \begingroup\fontsize{8}{10}\selectfont \endgroup & \begingroup\fontsize{8}{10}\selectfont 0.01\endgroup\\
\begingroup\fontsize{8}{10}\selectfont B\endgroup & \begingroup\fontsize{8}{10}\selectfont BEL\endgroup & \begingroup\fontsize{8}{10}\selectfont 0.06\endgroup & \begingroup\fontsize{8}{10}\selectfont 1.61\endgroup & \begingroup\fontsize{8}{10}\selectfont 1.43\endgroup & \begingroup\fontsize{8}{10}\selectfont 0.75\endgroup & \begingroup\fontsize{8}{10}\selectfont 0.38\endgroup & \begingroup\fontsize{8}{10}\selectfont 0.23\endgroup & \begingroup\fontsize{8}{10}\selectfont 0.39\endgroup & \begingroup\fontsize{8}{10}\selectfont \endgroup & \begingroup\fontsize{8}{10}\selectfont \endgroup & \begingroup\fontsize{8}{10}\selectfont \endgroup & \begingroup\fontsize{8}{10}\selectfont \endgroup & \begingroup\fontsize{8}{10}\selectfont \endgroup & \begingroup\fontsize{8}{10}\selectfont \endgroup & \begingroup\fontsize{8}{10}\selectfont \endgroup\\
\midrule
\begingroup\fontsize{8}{10}\selectfont C\endgroup & \begingroup\fontsize{8}{10}\selectfont TGO\endgroup & \begingroup\fontsize{8}{10}\selectfont 0.48\endgroup & \begingroup\fontsize{8}{10}\selectfont 0.71\endgroup & \begingroup\fontsize{8}{10}\selectfont 0.01\endgroup & \begingroup\fontsize{8}{10}\selectfont 0.26\endgroup & \begingroup\fontsize{8}{10}\selectfont 0.09\endgroup & \begingroup\fontsize{8}{10}\selectfont 0.23\endgroup & \begingroup\fontsize{8}{10}\selectfont 0.07\endgroup & \begingroup\fontsize{8}{10}\selectfont 0.03\endgroup & \begingroup\fontsize{8}{10}\selectfont 0.04\endgroup & \begingroup\fontsize{8}{10}\selectfont \endgroup & \begingroup\fontsize{8}{10}\selectfont 0.02\endgroup & \begingroup\fontsize{8}{10}\selectfont 0.04\endgroup & \begingroup\fontsize{8}{10}\selectfont 0.47\endgroup & \begingroup\fontsize{8}{10}\selectfont 0.02\endgroup\\
\begingroup\fontsize{8}{10}\selectfont C\endgroup & \begingroup\fontsize{8}{10}\selectfont BFA\endgroup & \begingroup\fontsize{8}{10}\selectfont 0.47\endgroup & \begingroup\fontsize{8}{10}\selectfont 1.12\endgroup & \begingroup\fontsize{8}{10}\selectfont 0.03\endgroup & \begingroup\fontsize{8}{10}\selectfont 0.39\endgroup & \begingroup\fontsize{8}{10}\selectfont 0.12\endgroup & \begingroup\fontsize{8}{10}\selectfont 0.12\endgroup & \begingroup\fontsize{8}{10}\selectfont 0.13\endgroup & \begingroup\fontsize{8}{10}\selectfont 0.02\endgroup & \begingroup\fontsize{8}{10}\selectfont 0.03\endgroup & \begingroup\fontsize{8}{10}\selectfont \endgroup & \begingroup\fontsize{8}{10}\selectfont 0.03\endgroup & \begingroup\fontsize{8}{10}\selectfont 0.07\endgroup & \begingroup\fontsize{8}{10}\selectfont 0.47\endgroup & \begingroup\fontsize{8}{10}\selectfont 0.02\endgroup\\
\begingroup\fontsize{8}{10}\selectfont C\endgroup & \begingroup\fontsize{8}{10}\selectfont CIV\endgroup & \begingroup\fontsize{8}{10}\selectfont 0.45\endgroup & \begingroup\fontsize{8}{10}\selectfont 0.97\endgroup & \begingroup\fontsize{8}{10}\selectfont 0.29\endgroup & \begingroup\fontsize{8}{10}\selectfont 0.26\endgroup & \begingroup\fontsize{8}{10}\selectfont 0.06\endgroup & \begingroup\fontsize{8}{10}\selectfont 0.25\endgroup & \begingroup\fontsize{8}{10}\selectfont 0.18\endgroup & \begingroup\fontsize{8}{10}\selectfont 0.01\endgroup & \begingroup\fontsize{8}{10}\selectfont 0.04\endgroup & \begingroup\fontsize{8}{10}\selectfont \endgroup & \begingroup\fontsize{8}{10}\selectfont 0.02\endgroup & \begingroup\fontsize{8}{10}\selectfont 0.03\endgroup & \begingroup\fontsize{8}{10}\selectfont 0.37\endgroup & \begingroup\fontsize{8}{10}\selectfont 0.03\endgroup\\
\begingroup\fontsize{8}{10}\selectfont C\endgroup & \begingroup\fontsize{8}{10}\selectfont BEN\endgroup & \begingroup\fontsize{8}{10}\selectfont 0.44\endgroup & \begingroup\fontsize{8}{10}\selectfont 0.88\endgroup & \begingroup\fontsize{8}{10}\selectfont 0.49\endgroup & \begingroup\fontsize{8}{10}\selectfont 0.40\endgroup & \begingroup\fontsize{8}{10}\selectfont 0.08\endgroup & \begingroup\fontsize{8}{10}\selectfont 0.25\endgroup & \begingroup\fontsize{8}{10}\selectfont 0.10\endgroup & \begingroup\fontsize{8}{10}\selectfont 0.03\endgroup & \begingroup\fontsize{8}{10}\selectfont 0.03\endgroup & \begingroup\fontsize{8}{10}\selectfont \endgroup & \begingroup\fontsize{8}{10}\selectfont 0.05\endgroup & \begingroup\fontsize{8}{10}\selectfont 0.08\endgroup & \begingroup\fontsize{8}{10}\selectfont 0.35\endgroup & \begingroup\fontsize{8}{10}\selectfont 0.02\endgroup\\
\begingroup\fontsize{8}{10}\selectfont C\endgroup & \begingroup\fontsize{8}{10}\selectfont MLI\endgroup & \begingroup\fontsize{8}{10}\selectfont 0.43\endgroup & \begingroup\fontsize{8}{10}\selectfont 1.32\endgroup & \begingroup\fontsize{8}{10}\selectfont 0.03\endgroup & \begingroup\fontsize{8}{10}\selectfont 0.36\endgroup & \begingroup\fontsize{8}{10}\selectfont 0.08\endgroup & \begingroup\fontsize{8}{10}\selectfont 0.07\endgroup & \begingroup\fontsize{8}{10}\selectfont 0.18\endgroup & \begingroup\fontsize{8}{10}\selectfont 0.02\endgroup & \begingroup\fontsize{8}{10}\selectfont 0.01\endgroup & \begingroup\fontsize{8}{10}\selectfont \endgroup & \begingroup\fontsize{8}{10}\selectfont 0.06\endgroup & \begingroup\fontsize{8}{10}\selectfont 0.06\endgroup & \begingroup\fontsize{8}{10}\selectfont 0.51\endgroup & \begingroup\fontsize{8}{10}\selectfont 0.02\endgroup\\
\begingroup\fontsize{8}{10}\selectfont C\endgroup & \begingroup\fontsize{8}{10}\selectfont GNB\endgroup & \begingroup\fontsize{8}{10}\selectfont 0.39\endgroup & \begingroup\fontsize{8}{10}\selectfont 0.68\endgroup & \begingroup\fontsize{8}{10}\selectfont 0.29\endgroup & \begingroup\fontsize{8}{10}\selectfont 0.19\endgroup & \begingroup\fontsize{8}{10}\selectfont 0.14\endgroup & \begingroup\fontsize{8}{10}\selectfont 0.20\endgroup & \begingroup\fontsize{8}{10}\selectfont 0.18\endgroup & \begingroup\fontsize{8}{10}\selectfont 0.01\endgroup & \begingroup\fontsize{8}{10}\selectfont 0.01\endgroup & \begingroup\fontsize{8}{10}\selectfont \endgroup & \begingroup\fontsize{8}{10}\selectfont 0.02\endgroup & \begingroup\fontsize{8}{10}\selectfont 0.11\endgroup & \begingroup\fontsize{8}{10}\selectfont 0.30\endgroup & \begingroup\fontsize{8}{10}\selectfont 0.04\endgroup\\
\begingroup\fontsize{8}{10}\selectfont C\endgroup & \begingroup\fontsize{8}{10}\selectfont NER\endgroup & \begingroup\fontsize{8}{10}\selectfont 0.32\endgroup & \begingroup\fontsize{8}{10}\selectfont 0.12\endgroup & \begingroup\fontsize{8}{10}\selectfont 0.34\endgroup & \begingroup\fontsize{8}{10}\selectfont 0.10\endgroup & \begingroup\fontsize{8}{10}\selectfont 0.09\endgroup & \begingroup\fontsize{8}{10}\selectfont 0.09\endgroup & \begingroup\fontsize{8}{10}\selectfont 0.08\endgroup & \begingroup\fontsize{8}{10}\selectfont 0.03\endgroup & \begingroup\fontsize{8}{10}\selectfont 0.02\endgroup & \begingroup\fontsize{8}{10}\selectfont \endgroup & \begingroup\fontsize{8}{10}\selectfont 0.04\endgroup & \begingroup\fontsize{8}{10}\selectfont 0.21\endgroup & \begingroup\fontsize{8}{10}\selectfont 0.40\endgroup & \begingroup\fontsize{8}{10}\selectfont 0.03\endgroup\\
\begingroup\fontsize{8}{10}\selectfont C\endgroup & \begingroup\fontsize{8}{10}\selectfont MMR\endgroup & \begingroup\fontsize{8}{10}\selectfont 0.13\endgroup & \begingroup\fontsize{8}{10}\selectfont 0.55\endgroup & \begingroup\fontsize{8}{10}\selectfont 0.57\endgroup & \begingroup\fontsize{8}{10}\selectfont 0.40\endgroup & \begingroup\fontsize{8}{10}\selectfont 0.15\endgroup & \begingroup\fontsize{8}{10}\selectfont 0.17\endgroup & \begingroup\fontsize{8}{10}\selectfont 0.19\endgroup & \begingroup\fontsize{8}{10}\selectfont 0.02\endgroup & \begingroup\fontsize{8}{10}\selectfont 0.04\endgroup & \begingroup\fontsize{8}{10}\selectfont \endgroup & \begingroup\fontsize{8}{10}\selectfont 0.05\endgroup & \begingroup\fontsize{8}{10}\selectfont 0.12\endgroup & \begingroup\fontsize{8}{10}\selectfont 0.19\endgroup & \begingroup\fontsize{8}{10}\selectfont 0.08\endgroup\\
\begingroup\fontsize{8}{10}\selectfont C\endgroup & \begingroup\fontsize{8}{10}\selectfont MWI\endgroup & \begingroup\fontsize{8}{10}\selectfont 0.13\endgroup & \begingroup\fontsize{8}{10}\selectfont 0.07\endgroup & \begingroup\fontsize{8}{10}\selectfont 0.35\endgroup & \begingroup\fontsize{8}{10}\selectfont 0.03\endgroup & \begingroup\fontsize{8}{10}\selectfont 0.17\endgroup & \begingroup\fontsize{8}{10}\selectfont 0.18\endgroup & \begingroup\fontsize{8}{10}\selectfont 0.03\endgroup & \begingroup\fontsize{8}{10}\selectfont 0.02\endgroup & \begingroup\fontsize{8}{10}\selectfont 0.04\endgroup & \begingroup\fontsize{8}{10}\selectfont \endgroup & \begingroup\fontsize{8}{10}\selectfont 0.02\endgroup & \begingroup\fontsize{8}{10}\selectfont 0.22\endgroup & \begingroup\fontsize{8}{10}\selectfont 0.25\endgroup & \begingroup\fontsize{8}{10}\selectfont 0.07\endgroup\\
\midrule
\begingroup\fontsize{8}{10}\selectfont D\endgroup & \begingroup\fontsize{8}{10}\selectfont AUT\endgroup & \begingroup\fontsize{8}{10}\selectfont 0.37\endgroup & \begingroup\fontsize{8}{10}\selectfont 1.96\endgroup & \begingroup\fontsize{8}{10}\selectfont 1.60\endgroup & \begingroup\fontsize{8}{10}\selectfont 0.60\endgroup & \begingroup\fontsize{8}{10}\selectfont 0.35\endgroup & \begingroup\fontsize{8}{10}\selectfont 0.15\endgroup & \begingroup\fontsize{8}{10}\selectfont 0.18\endgroup & \begingroup\fontsize{8}{10}\selectfont \endgroup & \begingroup\fontsize{8}{10}\selectfont \endgroup & \begingroup\fontsize{8}{10}\selectfont 0.06\endgroup & \begingroup\fontsize{8}{10}\selectfont \endgroup & \begingroup\fontsize{8}{10}\selectfont 0.23\endgroup & \begingroup\fontsize{8}{10}\selectfont 0.02\endgroup & \begingroup\fontsize{8}{10}\selectfont 0.02\endgroup\\
\begingroup\fontsize{8}{10}\selectfont D\endgroup & \begingroup\fontsize{8}{10}\selectfont BRA\endgroup & \begingroup\fontsize{8}{10}\selectfont 0.33\endgroup & \begingroup\fontsize{8}{10}\selectfont 1.59\endgroup & \begingroup\fontsize{8}{10}\selectfont 1.44\endgroup & \begingroup\fontsize{8}{10}\selectfont 0.82\endgroup & \begingroup\fontsize{8}{10}\selectfont 0.21\endgroup & \begingroup\fontsize{8}{10}\selectfont 0.15\endgroup & \begingroup\fontsize{8}{10}\selectfont 0.21\endgroup & \begingroup\fontsize{8}{10}\selectfont 0.01\endgroup & \begingroup\fontsize{8}{10}\selectfont 0.01\endgroup & \begingroup\fontsize{8}{10}\selectfont 0.04\endgroup & \begingroup\fontsize{8}{10}\selectfont \endgroup & \begingroup\fontsize{8}{10}\selectfont 0.24\endgroup & \begingroup\fontsize{8}{10}\selectfont 0.10\endgroup & \begingroup\fontsize{8}{10}\selectfont 0.03\endgroup\\
\begingroup\fontsize{8}{10}\selectfont D\endgroup & \begingroup\fontsize{8}{10}\selectfont ARG\endgroup & \begingroup\fontsize{8}{10}\selectfont 0.30\endgroup & \begingroup\fontsize{8}{10}\selectfont 1.76\endgroup & \begingroup\fontsize{8}{10}\selectfont 1.93\endgroup & \begingroup\fontsize{8}{10}\selectfont 1.28\endgroup & \begingroup\fontsize{8}{10}\selectfont 0.30\endgroup & \begingroup\fontsize{8}{10}\selectfont 0.13\endgroup & \begingroup\fontsize{8}{10}\selectfont 0.23\endgroup & \begingroup\fontsize{8}{10}\selectfont \endgroup & \begingroup\fontsize{8}{10}\selectfont 0.02\endgroup & \begingroup\fontsize{8}{10}\selectfont 0.02\endgroup & \begingroup\fontsize{8}{10}\selectfont \endgroup & \begingroup\fontsize{8}{10}\selectfont 0.22\endgroup & \begingroup\fontsize{8}{10}\selectfont 0.02\endgroup & \begingroup\fontsize{8}{10}\selectfont 0.05\endgroup\\
\begingroup\fontsize{8}{10}\selectfont D\endgroup & \begingroup\fontsize{8}{10}\selectfont ISR\endgroup & \begingroup\fontsize{8}{10}\selectfont 0.30\endgroup & \begingroup\fontsize{8}{10}\selectfont 1.63\endgroup & \begingroup\fontsize{8}{10}\selectfont 1.72\endgroup & \begingroup\fontsize{8}{10}\selectfont 0.62\endgroup & \begingroup\fontsize{8}{10}\selectfont 0.27\endgroup & \begingroup\fontsize{8}{10}\selectfont 0.24\endgroup & \begingroup\fontsize{8}{10}\selectfont 0.21\endgroup & \begingroup\fontsize{8}{10}\selectfont \endgroup & \begingroup\fontsize{8}{10}\selectfont \endgroup & \begingroup\fontsize{8}{10}\selectfont \endgroup & \begingroup\fontsize{8}{10}\selectfont \endgroup & \begingroup\fontsize{8}{10}\selectfont 0.25\endgroup & \begingroup\fontsize{8}{10}\selectfont 0.00\endgroup & \begingroup\fontsize{8}{10}\selectfont 0.03\endgroup\\
\begingroup\fontsize{8}{10}\selectfont D\endgroup & \begingroup\fontsize{8}{10}\selectfont IRQ\endgroup & \begingroup\fontsize{8}{10}\selectfont 0.21\endgroup & \begingroup\fontsize{8}{10}\selectfont 1.49\endgroup & \begingroup\fontsize{8}{10}\selectfont 1.64\endgroup & \begingroup\fontsize{8}{10}\selectfont 0.81\endgroup & \begingroup\fontsize{8}{10}\selectfont 0.49\endgroup & \begingroup\fontsize{8}{10}\selectfont 0.12\endgroup & \begingroup\fontsize{8}{10}\selectfont 0.04\endgroup & \begingroup\fontsize{8}{10}\selectfont 0.01\endgroup & \begingroup\fontsize{8}{10}\selectfont 0.01\endgroup & \begingroup\fontsize{8}{10}\selectfont 0.02\endgroup & \begingroup\fontsize{8}{10}\selectfont 0.00\endgroup & \begingroup\fontsize{8}{10}\selectfont 0.24\endgroup & \begingroup\fontsize{8}{10}\selectfont 0.00\endgroup & \begingroup\fontsize{8}{10}\selectfont 0.07\endgroup\\
\begingroup\fontsize{8}{10}\selectfont D\endgroup & \begingroup\fontsize{8}{10}\selectfont NOR\endgroup & \begingroup\fontsize{8}{10}\selectfont 0.13\endgroup & \begingroup\fontsize{8}{10}\selectfont 1.87\endgroup & \begingroup\fontsize{8}{10}\selectfont 1.54\endgroup & \begingroup\fontsize{8}{10}\selectfont 0.65\endgroup & \begingroup\fontsize{8}{10}\selectfont 0.30\endgroup & \begingroup\fontsize{8}{10}\selectfont 0.10\endgroup & \begingroup\fontsize{8}{10}\selectfont 0.41\endgroup & \begingroup\fontsize{8}{10}\selectfont \endgroup & \begingroup\fontsize{8}{10}\selectfont \endgroup & \begingroup\fontsize{8}{10}\selectfont \endgroup & \begingroup\fontsize{8}{10}\selectfont \endgroup & \begingroup\fontsize{8}{10}\selectfont 0.16\endgroup & \begingroup\fontsize{8}{10}\selectfont 0.02\endgroup & \begingroup\fontsize{8}{10}\selectfont 0.01\endgroup\\
\begingroup\fontsize{8}{10}\selectfont D\endgroup & \begingroup\fontsize{8}{10}\selectfont URY\endgroup & \begingroup\fontsize{8}{10}\selectfont 0.11\endgroup & \begingroup\fontsize{8}{10}\selectfont 1.57\endgroup & \begingroup\fontsize{8}{10}\selectfont 1.22\endgroup & \begingroup\fontsize{8}{10}\selectfont 0.28\endgroup & \begingroup\fontsize{8}{10}\selectfont 0.19\endgroup & \begingroup\fontsize{8}{10}\selectfont 0.10\endgroup & \begingroup\fontsize{8}{10}\selectfont 0.10\endgroup & \begingroup\fontsize{8}{10}\selectfont 0.00\endgroup & \begingroup\fontsize{8}{10}\selectfont 0.01\endgroup & \begingroup\fontsize{8}{10}\selectfont 0.05\endgroup & \begingroup\fontsize{8}{10}\selectfont 0.00\endgroup & \begingroup\fontsize{8}{10}\selectfont 0.40\endgroup & \begingroup\fontsize{8}{10}\selectfont 0.13\endgroup & \begingroup\fontsize{8}{10}\selectfont 0.02\endgroup\\
\begingroup\fontsize{8}{10}\selectfont D\endgroup & \begingroup\fontsize{8}{10}\selectfont CAN\endgroup & \begingroup\fontsize{8}{10}\selectfont 0.06\endgroup & \begingroup\fontsize{8}{10}\selectfont 2.20\endgroup & \begingroup\fontsize{8}{10}\selectfont 1.18\endgroup & \begingroup\fontsize{8}{10}\selectfont 0.67\endgroup & \begingroup\fontsize{8}{10}\selectfont 0.22\endgroup & \begingroup\fontsize{8}{10}\selectfont 0.19\endgroup & \begingroup\fontsize{8}{10}\selectfont 0.39\endgroup & \begingroup\fontsize{8}{10}\selectfont \endgroup & \begingroup\fontsize{8}{10}\selectfont \endgroup & \begingroup\fontsize{8}{10}\selectfont \endgroup & \begingroup\fontsize{8}{10}\selectfont \endgroup & \begingroup\fontsize{8}{10}\selectfont 0.14\endgroup & \begingroup\fontsize{8}{10}\selectfont \endgroup & \begingroup\fontsize{8}{10}\selectfont 0.07\endgroup\\
\begingroup\fontsize{8}{10}\selectfont D\endgroup & \begingroup\fontsize{8}{10}\selectfont MDV\endgroup & \begingroup\fontsize{8}{10}\selectfont 0.02\endgroup & \begingroup\fontsize{8}{10}\selectfont 1.49\endgroup & \begingroup\fontsize{8}{10}\selectfont 1.61\endgroup & \begingroup\fontsize{8}{10}\selectfont 0.25\endgroup & \begingroup\fontsize{8}{10}\selectfont 0.18\endgroup & \begingroup\fontsize{8}{10}\selectfont 0.14\endgroup & \begingroup\fontsize{8}{10}\selectfont 0.41\endgroup & \begingroup\fontsize{8}{10}\selectfont \endgroup & \begingroup\fontsize{8}{10}\selectfont 0.02\endgroup & \begingroup\fontsize{8}{10}\selectfont \endgroup & \begingroup\fontsize{8}{10}\selectfont \endgroup & \begingroup\fontsize{8}{10}\selectfont 0.02\endgroup & \begingroup\fontsize{8}{10}\selectfont 0.10\endgroup & \begingroup\fontsize{8}{10}\selectfont 0.14\endgroup\\
\midrule
\begingroup\fontsize{8}{10}\selectfont E\endgroup & \begingroup\fontsize{8}{10}\selectfont JOR\endgroup & \begingroup\fontsize{8}{10}\selectfont 0.36\endgroup & \begingroup\fontsize{8}{10}\selectfont 0.86\endgroup & \begingroup\fontsize{8}{10}\selectfont 0.60\endgroup & \begingroup\fontsize{8}{10}\selectfont 1.12\endgroup & \begingroup\fontsize{8}{10}\selectfont 0.09\endgroup & \begingroup\fontsize{8}{10}\selectfont 0.09\endgroup & \begingroup\fontsize{8}{10}\selectfont 0.25\endgroup & \begingroup\fontsize{8}{10}\selectfont \endgroup & \begingroup\fontsize{8}{10}\selectfont 0.00\endgroup & \begingroup\fontsize{8}{10}\selectfont \endgroup & \begingroup\fontsize{8}{10}\selectfont \endgroup & \begingroup\fontsize{8}{10}\selectfont 0.56\endgroup & \begingroup\fontsize{8}{10}\selectfont \endgroup & \begingroup\fontsize{8}{10}\selectfont 0.02\endgroup\\
\begingroup\fontsize{8}{10}\selectfont E\endgroup & \begingroup\fontsize{8}{10}\selectfont RUS\endgroup & \begingroup\fontsize{8}{10}\selectfont 0.32\endgroup & \begingroup\fontsize{8}{10}\selectfont 1.19\endgroup & \begingroup\fontsize{8}{10}\selectfont 0.96\endgroup & \begingroup\fontsize{8}{10}\selectfont 1.34\endgroup & \begingroup\fontsize{8}{10}\selectfont 0.09\endgroup & \begingroup\fontsize{8}{10}\selectfont 0.16\endgroup & \begingroup\fontsize{8}{10}\selectfont 0.25\endgroup & \begingroup\fontsize{8}{10}\selectfont \endgroup & \begingroup\fontsize{8}{10}\selectfont \endgroup & \begingroup\fontsize{8}{10}\selectfont \endgroup & \begingroup\fontsize{8}{10}\selectfont \endgroup & \begingroup\fontsize{8}{10}\selectfont 0.40\endgroup & \begingroup\fontsize{8}{10}\selectfont 0.00\endgroup & \begingroup\fontsize{8}{10}\selectfont 0.09\endgroup\\
\begingroup\fontsize{8}{10}\selectfont E\endgroup & \begingroup\fontsize{8}{10}\selectfont TWN\endgroup & \begingroup\fontsize{8}{10}\selectfont 0.29\endgroup & \begingroup\fontsize{8}{10}\selectfont 0.91\endgroup & \begingroup\fontsize{8}{10}\selectfont 0.98\endgroup & \begingroup\fontsize{8}{10}\selectfont 1.07\endgroup & \begingroup\fontsize{8}{10}\selectfont 0.21\endgroup & \begingroup\fontsize{8}{10}\selectfont 0.20\endgroup & \begingroup\fontsize{8}{10}\selectfont 0.00\endgroup & \begingroup\fontsize{8}{10}\selectfont \endgroup & \begingroup\fontsize{8}{10}\selectfont \endgroup & \begingroup\fontsize{8}{10}\selectfont \endgroup & \begingroup\fontsize{8}{10}\selectfont \endgroup & \begingroup\fontsize{8}{10}\selectfont 0.54\endgroup & \begingroup\fontsize{8}{10}\selectfont 0.04\endgroup & \begingroup\fontsize{8}{10}\selectfont 0.01\endgroup\\
\begingroup\fontsize{8}{10}\selectfont E\endgroup & \begingroup\fontsize{8}{10}\selectfont BRB\endgroup & \begingroup\fontsize{8}{10}\selectfont 0.29\endgroup & \begingroup\fontsize{8}{10}\selectfont 1.40\endgroup & \begingroup\fontsize{8}{10}\selectfont 0.92\endgroup & \begingroup\fontsize{8}{10}\selectfont 0.86\endgroup & \begingroup\fontsize{8}{10}\selectfont 0.18\endgroup & \begingroup\fontsize{8}{10}\selectfont 0.14\endgroup & \begingroup\fontsize{8}{10}\selectfont 0.07\endgroup & \begingroup\fontsize{8}{10}\selectfont 0.00\endgroup & \begingroup\fontsize{8}{10}\selectfont 0.08\endgroup & \begingroup\fontsize{8}{10}\selectfont \endgroup & \begingroup\fontsize{8}{10}\selectfont 0.00\endgroup & \begingroup\fontsize{8}{10}\selectfont 0.46\endgroup & \begingroup\fontsize{8}{10}\selectfont \endgroup & \begingroup\fontsize{8}{10}\selectfont 0.07\endgroup\\
\begingroup\fontsize{8}{10}\selectfont E\endgroup & \begingroup\fontsize{8}{10}\selectfont EGY\endgroup & \begingroup\fontsize{8}{10}\selectfont 0.22\endgroup & \begingroup\fontsize{8}{10}\selectfont 0.62\endgroup & \begingroup\fontsize{8}{10}\selectfont 0.98\endgroup & \begingroup\fontsize{8}{10}\selectfont 0.61\endgroup & \begingroup\fontsize{8}{10}\selectfont 0.20\endgroup & \begingroup\fontsize{8}{10}\selectfont 0.13\endgroup & \begingroup\fontsize{8}{10}\selectfont 0.13\endgroup & \begingroup\fontsize{8}{10}\selectfont 0.00\endgroup & \begingroup\fontsize{8}{10}\selectfont 0.00\endgroup & \begingroup\fontsize{8}{10}\selectfont \endgroup & \begingroup\fontsize{8}{10}\selectfont 0.00\endgroup & \begingroup\fontsize{8}{10}\selectfont 0.42\endgroup & \begingroup\fontsize{8}{10}\selectfont \endgroup & \begingroup\fontsize{8}{10}\selectfont 0.11\endgroup\\
\begingroup\fontsize{8}{10}\selectfont E\endgroup & \begingroup\fontsize{8}{10}\selectfont ZAF\endgroup & \begingroup\fontsize{8}{10}\selectfont 0.18\endgroup & \begingroup\fontsize{8}{10}\selectfont 1.05\endgroup & \begingroup\fontsize{8}{10}\selectfont 0.90\endgroup & \begingroup\fontsize{8}{10}\selectfont 2.15\endgroup & \begingroup\fontsize{8}{10}\selectfont 0.22\endgroup & \begingroup\fontsize{8}{10}\selectfont 0.08\endgroup & \begingroup\fontsize{8}{10}\selectfont 0.12\endgroup & \begingroup\fontsize{8}{10}\selectfont 0.04\endgroup & \begingroup\fontsize{8}{10}\selectfont 0.02\endgroup & \begingroup\fontsize{8}{10}\selectfont 0.01\endgroup & \begingroup\fontsize{8}{10}\selectfont 0.01\endgroup & \begingroup\fontsize{8}{10}\selectfont 0.41\endgroup & \begingroup\fontsize{8}{10}\selectfont 0.00\endgroup & \begingroup\fontsize{8}{10}\selectfont 0.09\endgroup\\
\begingroup\fontsize{8}{10}\selectfont E\endgroup & \begingroup\fontsize{8}{10}\selectfont MEX\endgroup & \begingroup\fontsize{8}{10}\selectfont 0.16\endgroup & \begingroup\fontsize{8}{10}\selectfont 1.05\endgroup & \begingroup\fontsize{8}{10}\selectfont 0.70\endgroup & \begingroup\fontsize{8}{10}\selectfont 1.12\endgroup & \begingroup\fontsize{8}{10}\selectfont 0.08\endgroup & \begingroup\fontsize{8}{10}\selectfont 0.12\endgroup & \begingroup\fontsize{8}{10}\selectfont 0.17\endgroup & \begingroup\fontsize{8}{10}\selectfont 0.00\endgroup & \begingroup\fontsize{8}{10}\selectfont 0.06\endgroup & \begingroup\fontsize{8}{10}\selectfont \endgroup & \begingroup\fontsize{8}{10}\selectfont \endgroup & \begingroup\fontsize{8}{10}\selectfont 0.36\endgroup & \begingroup\fontsize{8}{10}\selectfont 0.03\endgroup & \begingroup\fontsize{8}{10}\selectfont 0.18\endgroup\\
\begingroup\fontsize{8}{10}\selectfont E\endgroup & \begingroup\fontsize{8}{10}\selectfont GEO\endgroup & \begingroup\fontsize{8}{10}\selectfont 0.07\endgroup & \begingroup\fontsize{8}{10}\selectfont 1.14\endgroup & \begingroup\fontsize{8}{10}\selectfont 0.88\endgroup & \begingroup\fontsize{8}{10}\selectfont 1.04\endgroup & \begingroup\fontsize{8}{10}\selectfont 0.08\endgroup & \begingroup\fontsize{8}{10}\selectfont 0.12\endgroup & \begingroup\fontsize{8}{10}\selectfont 0.18\endgroup & \begingroup\fontsize{8}{10}\selectfont \endgroup & \begingroup\fontsize{8}{10}\selectfont 0.18\endgroup & \begingroup\fontsize{8}{10}\selectfont \endgroup & \begingroup\fontsize{8}{10}\selectfont \endgroup & \begingroup\fontsize{8}{10}\selectfont 0.36\endgroup & \begingroup\fontsize{8}{10}\selectfont 0.00\endgroup & \begingroup\fontsize{8}{10}\selectfont 0.07\endgroup\\
\midrule
\begingroup\fontsize{8}{10}\selectfont F\endgroup & \begingroup\fontsize{8}{10}\selectfont SLV\endgroup & \begingroup\fontsize{8}{10}\selectfont 0.30\endgroup & \begingroup\fontsize{8}{10}\selectfont 1.70\endgroup & \begingroup\fontsize{8}{10}\selectfont 0.74\endgroup & \begingroup\fontsize{8}{10}\selectfont 0.51\endgroup & \begingroup\fontsize{8}{10}\selectfont 0.29\endgroup & \begingroup\fontsize{8}{10}\selectfont 0.21\endgroup & \begingroup\fontsize{8}{10}\selectfont 0.06\endgroup & \begingroup\fontsize{8}{10}\selectfont 0.01\endgroup & \begingroup\fontsize{8}{10}\selectfont 0.19\endgroup & \begingroup\fontsize{8}{10}\selectfont \endgroup & \begingroup\fontsize{8}{10}\selectfont 0.01\endgroup & \begingroup\fontsize{8}{10}\selectfont 0.16\endgroup & \begingroup\fontsize{8}{10}\selectfont 0.03\endgroup & \begingroup\fontsize{8}{10}\selectfont 0.04\endgroup\\
\begingroup\fontsize{8}{10}\selectfont F\endgroup & \begingroup\fontsize{8}{10}\selectfont PER\endgroup & \begingroup\fontsize{8}{10}\selectfont 0.27\endgroup & \begingroup\fontsize{8}{10}\selectfont 2.97\endgroup & \begingroup\fontsize{8}{10}\selectfont 1.49\endgroup & \begingroup\fontsize{8}{10}\selectfont 0.72\endgroup & \begingroup\fontsize{8}{10}\selectfont 0.34\endgroup & \begingroup\fontsize{8}{10}\selectfont 0.07\endgroup & \begingroup\fontsize{8}{10}\selectfont 0.10\endgroup & \begingroup\fontsize{8}{10}\selectfont 0.01\endgroup & \begingroup\fontsize{8}{10}\selectfont 0.34\endgroup & \begingroup\fontsize{8}{10}\selectfont \endgroup & \begingroup\fontsize{8}{10}\selectfont 0.01\endgroup & \begingroup\fontsize{8}{10}\selectfont 0.04\endgroup & \begingroup\fontsize{8}{10}\selectfont 0.04\endgroup & \begingroup\fontsize{8}{10}\selectfont 0.06\endgroup\\
\begingroup\fontsize{8}{10}\selectfont F\endgroup & \begingroup\fontsize{8}{10}\selectfont BOL\endgroup & \begingroup\fontsize{8}{10}\selectfont 0.19\endgroup & \begingroup\fontsize{8}{10}\selectfont 1.49\endgroup & \begingroup\fontsize{8}{10}\selectfont 1.10\endgroup & \begingroup\fontsize{8}{10}\selectfont 0.45\endgroup & \begingroup\fontsize{8}{10}\selectfont 0.24\endgroup & \begingroup\fontsize{8}{10}\selectfont 0.12\endgroup & \begingroup\fontsize{8}{10}\selectfont 0.16\endgroup & \begingroup\fontsize{8}{10}\selectfont 0.04\endgroup & \begingroup\fontsize{8}{10}\selectfont 0.14\endgroup & \begingroup\fontsize{8}{10}\selectfont \endgroup & \begingroup\fontsize{8}{10}\selectfont \endgroup & \begingroup\fontsize{8}{10}\selectfont 0.12\endgroup & \begingroup\fontsize{8}{10}\selectfont 0.08\endgroup & \begingroup\fontsize{8}{10}\selectfont 0.09\endgroup\\
\begingroup\fontsize{8}{10}\selectfont F\endgroup & \begingroup\fontsize{8}{10}\selectfont ECU\endgroup & \begingroup\fontsize{8}{10}\selectfont 0.13\endgroup & \begingroup\fontsize{8}{10}\selectfont 1.33\endgroup & \begingroup\fontsize{8}{10}\selectfont 1.26\endgroup & \begingroup\fontsize{8}{10}\selectfont 0.35\endgroup & \begingroup\fontsize{8}{10}\selectfont 0.31\endgroup & \begingroup\fontsize{8}{10}\selectfont 0.07\endgroup & \begingroup\fontsize{8}{10}\selectfont 0.07\endgroup & \begingroup\fontsize{8}{10}\selectfont 0.01\endgroup & \begingroup\fontsize{8}{10}\selectfont 0.13\endgroup & \begingroup\fontsize{8}{10}\selectfont \endgroup & \begingroup\fontsize{8}{10}\selectfont 0.00\endgroup & \begingroup\fontsize{8}{10}\selectfont 0.27\endgroup & \begingroup\fontsize{8}{10}\selectfont 0.11\endgroup & \begingroup\fontsize{8}{10}\selectfont 0.03\endgroup\\
\begingroup\fontsize{8}{10}\selectfont F\endgroup & \begingroup\fontsize{8}{10}\selectfont COL\endgroup & \begingroup\fontsize{8}{10}\selectfont 0.12\endgroup & \begingroup\fontsize{8}{10}\selectfont 2.11\endgroup & \begingroup\fontsize{8}{10}\selectfont 1.45\endgroup & \begingroup\fontsize{8}{10}\selectfont 0.57\endgroup & \begingroup\fontsize{8}{10}\selectfont 0.23\endgroup & \begingroup\fontsize{8}{10}\selectfont 0.19\endgroup & \begingroup\fontsize{8}{10}\selectfont 0.15\endgroup & \begingroup\fontsize{8}{10}\selectfont 0.00\endgroup & \begingroup\fontsize{8}{10}\selectfont 0.13\endgroup & \begingroup\fontsize{8}{10}\selectfont \endgroup & \begingroup\fontsize{8}{10}\selectfont \endgroup & \begingroup\fontsize{8}{10}\selectfont 0.13\endgroup & \begingroup\fontsize{8}{10}\selectfont 0.14\endgroup & \begingroup\fontsize{8}{10}\selectfont 0.04\endgroup\\
\begingroup\fontsize{8}{10}\selectfont F\endgroup & \begingroup\fontsize{8}{10}\selectfont KHM\endgroup & \begingroup\fontsize{8}{10}\selectfont 0.11\endgroup & \begingroup\fontsize{8}{10}\selectfont 1.31\endgroup & \begingroup\fontsize{8}{10}\selectfont 1.09\endgroup & \begingroup\fontsize{8}{10}\selectfont 0.49\endgroup & \begingroup\fontsize{8}{10}\selectfont 0.21\endgroup & \begingroup\fontsize{8}{10}\selectfont 0.13\endgroup & \begingroup\fontsize{8}{10}\selectfont 0.17\endgroup & \begingroup\fontsize{8}{10}\selectfont \endgroup & \begingroup\fontsize{8}{10}\selectfont 0.16\endgroup & \begingroup\fontsize{8}{10}\selectfont \endgroup & \begingroup\fontsize{8}{10}\selectfont 0.01\endgroup & \begingroup\fontsize{8}{10}\selectfont 0.06\endgroup & \begingroup\fontsize{8}{10}\selectfont 0.26\endgroup & \begingroup\fontsize{8}{10}\selectfont \endgroup\\
\midrule
\begingroup\fontsize{8}{10}\selectfont G\endgroup & \begingroup\fontsize{8}{10}\selectfont GTM\endgroup & \begingroup\fontsize{8}{10}\selectfont 0.36\endgroup & \begingroup\fontsize{8}{10}\selectfont 0.31\endgroup & \begingroup\fontsize{8}{10}\selectfont 0.12\endgroup & \begingroup\fontsize{8}{10}\selectfont 0.40\endgroup & \begingroup\fontsize{8}{10}\selectfont 0.06\endgroup & \begingroup\fontsize{8}{10}\selectfont 0.14\endgroup & \begingroup\fontsize{8}{10}\selectfont 0.09\endgroup & \begingroup\fontsize{8}{10}\selectfont 0.00\endgroup & \begingroup\fontsize{8}{10}\selectfont 0.16\endgroup & \begingroup\fontsize{8}{10}\selectfont \endgroup & \begingroup\fontsize{8}{10}\selectfont 0.01\endgroup & \begingroup\fontsize{8}{10}\selectfont 0.31\endgroup & \begingroup\fontsize{8}{10}\selectfont 0.15\endgroup & \begingroup\fontsize{8}{10}\selectfont 0.07\endgroup\\
\begingroup\fontsize{8}{10}\selectfont G\endgroup & \begingroup\fontsize{8}{10}\selectfont NIC\endgroup & \begingroup\fontsize{8}{10}\selectfont 0.32\endgroup & \begingroup\fontsize{8}{10}\selectfont 0.34\endgroup & \begingroup\fontsize{8}{10}\selectfont 0.11\endgroup & \begingroup\fontsize{8}{10}\selectfont 0.25\endgroup & \begingroup\fontsize{8}{10}\selectfont 0.11\endgroup & \begingroup\fontsize{8}{10}\selectfont 0.07\endgroup & \begingroup\fontsize{8}{10}\selectfont 0.06\endgroup & \begingroup\fontsize{8}{10}\selectfont 0.02\endgroup & \begingroup\fontsize{8}{10}\selectfont 0.29\endgroup & \begingroup\fontsize{8}{10}\selectfont \endgroup & \begingroup\fontsize{8}{10}\selectfont 0.01\endgroup & \begingroup\fontsize{8}{10}\selectfont 0.22\endgroup & \begingroup\fontsize{8}{10}\selectfont 0.19\endgroup & \begingroup\fontsize{8}{10}\selectfont 0.03\endgroup\\
\begingroup\fontsize{8}{10}\selectfont G\endgroup & \begingroup\fontsize{8}{10}\selectfont PRY\endgroup & \begingroup\fontsize{8}{10}\selectfont 0.29\endgroup & \begingroup\fontsize{8}{10}\selectfont 0.87\endgroup & \begingroup\fontsize{8}{10}\selectfont 0.61\endgroup & \begingroup\fontsize{8}{10}\selectfont 0.43\endgroup & \begingroup\fontsize{8}{10}\selectfont 0.13\endgroup & \begingroup\fontsize{8}{10}\selectfont 0.08\endgroup & \begingroup\fontsize{8}{10}\selectfont 0.06\endgroup & \begingroup\fontsize{8}{10}\selectfont 0.00\endgroup & \begingroup\fontsize{8}{10}\selectfont 0.14\endgroup & \begingroup\fontsize{8}{10}\selectfont \endgroup & \begingroup\fontsize{8}{10}\selectfont \endgroup & \begingroup\fontsize{8}{10}\selectfont 0.27\endgroup & \begingroup\fontsize{8}{10}\selectfont 0.27\endgroup & \begingroup\fontsize{8}{10}\selectfont 0.04\endgroup\\
\begingroup\fontsize{8}{10}\selectfont G\endgroup & \begingroup\fontsize{8}{10}\selectfont CRI\endgroup & \begingroup\fontsize{8}{10}\selectfont 0.11\endgroup & \begingroup\fontsize{8}{10}\selectfont 1.03\endgroup & \begingroup\fontsize{8}{10}\selectfont 0.83\endgroup & \begingroup\fontsize{8}{10}\selectfont 0.42\endgroup & \begingroup\fontsize{8}{10}\selectfont 0.10\endgroup & \begingroup\fontsize{8}{10}\selectfont 0.08\endgroup & \begingroup\fontsize{8}{10}\selectfont 0.12\endgroup & \begingroup\fontsize{8}{10}\selectfont 0.00\endgroup & \begingroup\fontsize{8}{10}\selectfont 0.14\endgroup & \begingroup\fontsize{8}{10}\selectfont \endgroup & \begingroup\fontsize{8}{10}\selectfont 0.00\endgroup & \begingroup\fontsize{8}{10}\selectfont 0.46\endgroup & \begingroup\fontsize{8}{10}\selectfont 0.10\endgroup & \begingroup\fontsize{8}{10}\selectfont 0.00\endgroup\\
\begingroup\fontsize{8}{10}\selectfont G\endgroup & \begingroup\fontsize{8}{10}\selectfont DOM\endgroup & \begingroup\fontsize{8}{10}\selectfont 0.04\endgroup & \begingroup\fontsize{8}{10}\selectfont 0.66\endgroup & \begingroup\fontsize{8}{10}\selectfont 0.74\endgroup & \begingroup\fontsize{8}{10}\selectfont 0.54\endgroup & \begingroup\fontsize{8}{10}\selectfont 0.07\endgroup & \begingroup\fontsize{8}{10}\selectfont 0.11\endgroup & \begingroup\fontsize{8}{10}\selectfont 0.13\endgroup & \begingroup\fontsize{8}{10}\selectfont 0.00\endgroup & \begingroup\fontsize{8}{10}\selectfont 0.05\endgroup & \begingroup\fontsize{8}{10}\selectfont \endgroup & \begingroup\fontsize{8}{10}\selectfont 0.00\endgroup & \begingroup\fontsize{8}{10}\selectfont 0.31\endgroup & \begingroup\fontsize{8}{10}\selectfont 0.27\endgroup & \begingroup\fontsize{8}{10}\selectfont 0.07\endgroup\\
\begingroup\fontsize{8}{10}\selectfont G\endgroup & \begingroup\fontsize{8}{10}\selectfont GHA\endgroup & \begingroup\fontsize{8}{10}\selectfont 0.03\endgroup & \begingroup\fontsize{8}{10}\selectfont 0.45\endgroup & \begingroup\fontsize{8}{10}\selectfont 0.42\endgroup & \begingroup\fontsize{8}{10}\selectfont 0.20\endgroup & \begingroup\fontsize{8}{10}\selectfont 0.06\endgroup & \begingroup\fontsize{8}{10}\selectfont 0.14\endgroup & \begingroup\fontsize{8}{10}\selectfont 0.18\endgroup & \begingroup\fontsize{8}{10}\selectfont 0.03\endgroup & \begingroup\fontsize{8}{10}\selectfont 0.15\endgroup & \begingroup\fontsize{8}{10}\selectfont \endgroup & \begingroup\fontsize{8}{10}\selectfont 0.02\endgroup & \begingroup\fontsize{8}{10}\selectfont 0.10\endgroup & \begingroup\fontsize{8}{10}\selectfont 0.22\endgroup & \begingroup\fontsize{8}{10}\selectfont 0.09\endgroup\\
\midrule
\begingroup\fontsize{8}{10}\selectfont H\endgroup & \begingroup\fontsize{8}{10}\selectfont IDN\endgroup & \begingroup\fontsize{8}{10}\selectfont 0.24\endgroup & \begingroup\fontsize{8}{10}\selectfont 1.18\endgroup & \begingroup\fontsize{8}{10}\selectfont 0.97\endgroup & \begingroup\fontsize{8}{10}\selectfont 1.06\endgroup & \begingroup\fontsize{8}{10}\selectfont 0.17\endgroup & \begingroup\fontsize{8}{10}\selectfont 0.09\endgroup & \begingroup\fontsize{8}{10}\selectfont 0.17\endgroup & \begingroup\fontsize{8}{10}\selectfont 0.01\endgroup & \begingroup\fontsize{8}{10}\selectfont 0.12\endgroup & \begingroup\fontsize{8}{10}\selectfont \endgroup & \begingroup\fontsize{8}{10}\selectfont 0.01\endgroup & \begingroup\fontsize{8}{10}\selectfont 0.07\endgroup & \begingroup\fontsize{8}{10}\selectfont 0.21\endgroup & \begingroup\fontsize{8}{10}\selectfont 0.16\endgroup\\
\begingroup\fontsize{8}{10}\selectfont H\endgroup & \begingroup\fontsize{8}{10}\selectfont THA\endgroup & \begingroup\fontsize{8}{10}\selectfont 0.19\endgroup & \begingroup\fontsize{8}{10}\selectfont 1.06\endgroup & \begingroup\fontsize{8}{10}\selectfont 0.97\endgroup & \begingroup\fontsize{8}{10}\selectfont 1.58\endgroup & \begingroup\fontsize{8}{10}\selectfont 0.08\endgroup & \begingroup\fontsize{8}{10}\selectfont 0.23\endgroup & \begingroup\fontsize{8}{10}\selectfont 0.12\endgroup & \begingroup\fontsize{8}{10}\selectfont 0.00\endgroup & \begingroup\fontsize{8}{10}\selectfont 0.08\endgroup & \begingroup\fontsize{8}{10}\selectfont \endgroup & \begingroup\fontsize{8}{10}\selectfont \endgroup & \begingroup\fontsize{8}{10}\selectfont 0.13\endgroup & \begingroup\fontsize{8}{10}\selectfont 0.19\endgroup & \begingroup\fontsize{8}{10}\selectfont 0.15\endgroup\\
\begingroup\fontsize{8}{10}\selectfont H\endgroup & \begingroup\fontsize{8}{10}\selectfont IND\endgroup & \begingroup\fontsize{8}{10}\selectfont 0.11\endgroup & \begingroup\fontsize{8}{10}\selectfont 0.73\endgroup & \begingroup\fontsize{8}{10}\selectfont 0.99\endgroup & \begingroup\fontsize{8}{10}\selectfont 1.07\endgroup & \begingroup\fontsize{8}{10}\selectfont 0.08\endgroup & \begingroup\fontsize{8}{10}\selectfont 0.09\endgroup & \begingroup\fontsize{8}{10}\selectfont 0.35\endgroup & \begingroup\fontsize{8}{10}\selectfont 0.03\endgroup & \begingroup\fontsize{8}{10}\selectfont 0.15\endgroup & \begingroup\fontsize{8}{10}\selectfont \endgroup & \begingroup\fontsize{8}{10}\selectfont 0.02\endgroup & \begingroup\fontsize{8}{10}\selectfont 0.04\endgroup & \begingroup\fontsize{8}{10}\selectfont 0.18\endgroup & \begingroup\fontsize{8}{10}\selectfont 0.06\endgroup\\
\begingroup\fontsize{8}{10}\selectfont H\endgroup & \begingroup\fontsize{8}{10}\selectfont SRB\endgroup & \begingroup\fontsize{8}{10}\selectfont 0.05\endgroup & \begingroup\fontsize{8}{10}\selectfont 0.64\endgroup & \begingroup\fontsize{8}{10}\selectfont 0.96\endgroup & \begingroup\fontsize{8}{10}\selectfont 0.95\endgroup & \begingroup\fontsize{8}{10}\selectfont 0.28\endgroup & \begingroup\fontsize{8}{10}\selectfont 0.23\endgroup & \begingroup\fontsize{8}{10}\selectfont 0.25\endgroup & \begingroup\fontsize{8}{10}\selectfont 0.00\endgroup & \begingroup\fontsize{8}{10}\selectfont \endgroup & \begingroup\fontsize{8}{10}\selectfont 0.07\endgroup & \begingroup\fontsize{8}{10}\selectfont \endgroup & \begingroup\fontsize{8}{10}\selectfont 0.01\endgroup & \begingroup\fontsize{8}{10}\selectfont \endgroup & \begingroup\fontsize{8}{10}\selectfont 0.16\endgroup\\
\midrule
\begingroup\fontsize{8}{10}\selectfont I\endgroup & \begingroup\fontsize{8}{10}\selectfont UGA\endgroup & \begingroup\fontsize{8}{10}\selectfont 0.37\endgroup & \begingroup\fontsize{8}{10}\selectfont 1.05\endgroup & \begingroup\fontsize{8}{10}\selectfont 0.42\endgroup & \begingroup\fontsize{8}{10}\selectfont 0.20\endgroup & \begingroup\fontsize{8}{10}\selectfont 0.16\endgroup & \begingroup\fontsize{8}{10}\selectfont 0.10\endgroup & \begingroup\fontsize{8}{10}\selectfont 0.14\endgroup & \begingroup\fontsize{8}{10}\selectfont 0.03\endgroup & \begingroup\fontsize{8}{10}\selectfont 0.06\endgroup & \begingroup\fontsize{8}{10}\selectfont \endgroup & \begingroup\fontsize{8}{10}\selectfont 0.31\endgroup & \begingroup\fontsize{8}{10}\selectfont 0.07\endgroup & \begingroup\fontsize{8}{10}\selectfont 0.12\endgroup & \begingroup\fontsize{8}{10}\selectfont 0.01\endgroup\\
\begingroup\fontsize{8}{10}\selectfont I\endgroup & \begingroup\fontsize{8}{10}\selectfont ETH\endgroup & \begingroup\fontsize{8}{10}\selectfont 0.19\endgroup & \begingroup\fontsize{8}{10}\selectfont 3.11\endgroup & \begingroup\fontsize{8}{10}\selectfont 0.97\endgroup & \begingroup\fontsize{8}{10}\selectfont 0.10\endgroup & \begingroup\fontsize{8}{10}\selectfont 0.13\endgroup & \begingroup\fontsize{8}{10}\selectfont 0.14\endgroup & \begingroup\fontsize{8}{10}\selectfont 0.37\endgroup & \begingroup\fontsize{8}{10}\selectfont 0.03\endgroup & \begingroup\fontsize{8}{10}\selectfont 0.09\endgroup & \begingroup\fontsize{8}{10}\selectfont \endgroup & \begingroup\fontsize{8}{10}\selectfont 0.22\endgroup & \begingroup\fontsize{8}{10}\selectfont 0.00\endgroup & \begingroup\fontsize{8}{10}\selectfont 0.00\endgroup & \begingroup\fontsize{8}{10}\selectfont 0.02\endgroup\\
\begingroup\fontsize{8}{10}\selectfont I\endgroup & \begingroup\fontsize{8}{10}\selectfont KEN\endgroup & \begingroup\fontsize{8}{10}\selectfont 0.14\endgroup & \begingroup\fontsize{8}{10}\selectfont 0.92\endgroup & \begingroup\fontsize{8}{10}\selectfont 0.70\endgroup & \begingroup\fontsize{8}{10}\selectfont 0.41\endgroup & \begingroup\fontsize{8}{10}\selectfont 0.12\endgroup & \begingroup\fontsize{8}{10}\selectfont 0.18\endgroup & \begingroup\fontsize{8}{10}\selectfont 0.34\endgroup & \begingroup\fontsize{8}{10}\selectfont 0.02\endgroup & \begingroup\fontsize{8}{10}\selectfont 0.16\endgroup & \begingroup\fontsize{8}{10}\selectfont \endgroup & \begingroup\fontsize{8}{10}\selectfont 0.18\endgroup & \begingroup\fontsize{8}{10}\selectfont \endgroup & \begingroup\fontsize{8}{10}\selectfont \endgroup & \begingroup\fontsize{8}{10}\selectfont \endgroup\\
\begingroup\fontsize{8}{10}\selectfont I\endgroup & \begingroup\fontsize{8}{10}\selectfont RWA\endgroup & \begingroup\fontsize{8}{10}\selectfont 0.00\endgroup & \begingroup\fontsize{8}{10}\selectfont 0.06\endgroup & \begingroup\fontsize{8}{10}\selectfont 0.16\endgroup & \begingroup\fontsize{8}{10}\selectfont 0.04\endgroup & \begingroup\fontsize{8}{10}\selectfont 0.15\endgroup & \begingroup\fontsize{8}{10}\selectfont 0.08\endgroup & \begingroup\fontsize{8}{10}\selectfont 0.15\endgroup & \begingroup\fontsize{8}{10}\selectfont \endgroup & \begingroup\fontsize{8}{10}\selectfont 0.09\endgroup & \begingroup\fontsize{8}{10}\selectfont \endgroup & \begingroup\fontsize{8}{10}\selectfont 0.18\endgroup & \begingroup\fontsize{8}{10}\selectfont 0.18\endgroup & \begingroup\fontsize{8}{10}\selectfont 0.12\endgroup & \begingroup\fontsize{8}{10}\selectfont 0.05\endgroup\\
\midrule
\begingroup\fontsize{8}{10}\selectfont J\endgroup & \begingroup\fontsize{8}{10}\selectfont TUR\endgroup & \begingroup\fontsize{8}{10}\selectfont 0.41\endgroup & \begingroup\fontsize{8}{10}\selectfont 1.94\endgroup & \begingroup\fontsize{8}{10}\selectfont 1.06\endgroup & \begingroup\fontsize{8}{10}\selectfont 1.75\endgroup & \begingroup\fontsize{8}{10}\selectfont 0.11\endgroup & \begingroup\fontsize{8}{10}\selectfont 0.08\endgroup & \begingroup\fontsize{8}{10}\selectfont 0.02\endgroup & \begingroup\fontsize{8}{10}\selectfont \endgroup & \begingroup\fontsize{8}{10}\selectfont 0.12\endgroup & \begingroup\fontsize{8}{10}\selectfont 0.45\endgroup & \begingroup\fontsize{8}{10}\selectfont \endgroup & \begingroup\fontsize{8}{10}\selectfont 0.14\endgroup & \begingroup\fontsize{8}{10}\selectfont 0.00\endgroup & \begingroup\fontsize{8}{10}\selectfont 0.08\endgroup\\
\begingroup\fontsize{8}{10}\selectfont J\endgroup & \begingroup\fontsize{8}{10}\selectfont ARM\endgroup & \begingroup\fontsize{8}{10}\selectfont 0.38\endgroup & \begingroup\fontsize{8}{10}\selectfont 1.57\endgroup & \begingroup\fontsize{8}{10}\selectfont 1.94\endgroup & \begingroup\fontsize{8}{10}\selectfont 1.20\endgroup & \begingroup\fontsize{8}{10}\selectfont 0.21\endgroup & \begingroup\fontsize{8}{10}\selectfont 0.08\endgroup & \begingroup\fontsize{8}{10}\selectfont 0.18\endgroup & \begingroup\fontsize{8}{10}\selectfont 0.00\endgroup & \begingroup\fontsize{8}{10}\selectfont \endgroup & \begingroup\fontsize{8}{10}\selectfont 0.29\endgroup & \begingroup\fontsize{8}{10}\selectfont \endgroup & \begingroup\fontsize{8}{10}\selectfont 0.20\endgroup & \begingroup\fontsize{8}{10}\selectfont 0.00\endgroup & \begingroup\fontsize{8}{10}\selectfont 0.04\endgroup\\
\begingroup\fontsize{8}{10}\selectfont J\endgroup & \begingroup\fontsize{8}{10}\selectfont GBR\endgroup & \begingroup\fontsize{8}{10}\selectfont -0.04\endgroup & \begingroup\fontsize{8}{10}\selectfont 2.03\endgroup & \begingroup\fontsize{8}{10}\selectfont 1.50\endgroup & \begingroup\fontsize{8}{10}\selectfont 0.83\endgroup & \begingroup\fontsize{8}{10}\selectfont 0.31\endgroup & \begingroup\fontsize{8}{10}\selectfont 0.11\endgroup & \begingroup\fontsize{8}{10}\selectfont 0.18\endgroup & \begingroup\fontsize{8}{10}\selectfont \endgroup & \begingroup\fontsize{8}{10}\selectfont \endgroup & \begingroup\fontsize{8}{10}\selectfont 0.21\endgroup & \begingroup\fontsize{8}{10}\selectfont \endgroup & \begingroup\fontsize{8}{10}\selectfont 0.18\endgroup & \begingroup\fontsize{8}{10}\selectfont \endgroup & \begingroup\fontsize{8}{10}\selectfont 0.02\endgroup\\
\midrule
\begingroup\fontsize{8}{10}\selectfont K\endgroup & \begingroup\fontsize{8}{10}\selectfont LBR\endgroup & \begingroup\fontsize{8}{10}\selectfont 0.51\endgroup & \begingroup\fontsize{8}{10}\selectfont 0.40\endgroup & \begingroup\fontsize{8}{10}\selectfont 0.35\endgroup & \begingroup\fontsize{8}{10}\selectfont 0.18\endgroup & \begingroup\fontsize{8}{10}\selectfont 0.36\endgroup & \begingroup\fontsize{8}{10}\selectfont 0.19\endgroup & \begingroup\fontsize{8}{10}\selectfont 0.22\endgroup & \begingroup\fontsize{8}{10}\selectfont 0.04\endgroup & \begingroup\fontsize{8}{10}\selectfont 0.05\endgroup & \begingroup\fontsize{8}{10}\selectfont \endgroup & \begingroup\fontsize{8}{10}\selectfont 0.03\endgroup & \begingroup\fontsize{8}{10}\selectfont 0.01\endgroup & \begingroup\fontsize{8}{10}\selectfont 0.08\endgroup & \begingroup\fontsize{8}{10}\selectfont 0.02\endgroup\\
\begingroup\fontsize{8}{10}\selectfont K\endgroup & \begingroup\fontsize{8}{10}\selectfont MOZ\endgroup & \begingroup\fontsize{8}{10}\selectfont 0.46\endgroup & \begingroup\fontsize{8}{10}\selectfont 0.24\endgroup & \begingroup\fontsize{8}{10}\selectfont 0.15\endgroup & \begingroup\fontsize{8}{10}\selectfont 0.20\endgroup & \begingroup\fontsize{8}{10}\selectfont 0.33\endgroup & \begingroup\fontsize{8}{10}\selectfont 0.20\endgroup & \begingroup\fontsize{8}{10}\selectfont 0.24\endgroup & \begingroup\fontsize{8}{10}\selectfont 0.02\endgroup & \begingroup\fontsize{8}{10}\selectfont 0.09\endgroup & \begingroup\fontsize{8}{10}\selectfont \endgroup & \begingroup\fontsize{8}{10}\selectfont 0.06\endgroup & \begingroup\fontsize{8}{10}\selectfont 0.01\endgroup & \begingroup\fontsize{8}{10}\selectfont 0.04\endgroup & \begingroup\fontsize{8}{10}\selectfont 0.01\endgroup\\
\begingroup\fontsize{8}{10}\selectfont K\endgroup & \begingroup\fontsize{8}{10}\selectfont PAK\endgroup & \begingroup\fontsize{8}{10}\selectfont 0.24\endgroup & \begingroup\fontsize{8}{10}\selectfont 0.80\endgroup & \begingroup\fontsize{8}{10}\selectfont 0.41\endgroup & \begingroup\fontsize{8}{10}\selectfont 0.57\endgroup & \begingroup\fontsize{8}{10}\selectfont 0.26\endgroup & \begingroup\fontsize{8}{10}\selectfont 0.32\endgroup & \begingroup\fontsize{8}{10}\selectfont 0.37\endgroup & \begingroup\fontsize{8}{10}\selectfont 0.05\endgroup & \begingroup\fontsize{8}{10}\selectfont \endgroup & \begingroup\fontsize{8}{10}\selectfont \endgroup & \begingroup\fontsize{8}{10}\selectfont \endgroup & \begingroup\fontsize{8}{10}\selectfont \endgroup & \begingroup\fontsize{8}{10}\selectfont \endgroup & \begingroup\fontsize{8}{10}\selectfont \endgroup\\
\midrule
\begingroup\fontsize{8}{10}\selectfont L\endgroup & \begingroup\fontsize{8}{10}\selectfont PHL\endgroup & \begingroup\fontsize{8}{10}\selectfont 0.36\endgroup & \begingroup\fontsize{8}{10}\selectfont 0.71\endgroup & \begingroup\fontsize{8}{10}\selectfont 0.51\endgroup & \begingroup\fontsize{8}{10}\selectfont 0.44\endgroup & \begingroup\fontsize{8}{10}\selectfont 0.08\endgroup & \begingroup\fontsize{8}{10}\selectfont 0.08\endgroup & \begingroup\fontsize{8}{10}\selectfont 0.19\endgroup & \begingroup\fontsize{8}{10}\selectfont 0.02\endgroup & \begingroup\fontsize{8}{10}\selectfont \endgroup & \begingroup\fontsize{8}{10}\selectfont \endgroup & \begingroup\fontsize{8}{10}\selectfont \endgroup & \begingroup\fontsize{8}{10}\selectfont 0.07\endgroup & \begingroup\fontsize{8}{10}\selectfont 0.18\endgroup & \begingroup\fontsize{8}{10}\selectfont 0.37\endgroup\\
\begingroup\fontsize{8}{10}\selectfont L\endgroup & \begingroup\fontsize{8}{10}\selectfont CHE\endgroup & \begingroup\fontsize{8}{10}\selectfont 0.36\endgroup & \begingroup\fontsize{8}{10}\selectfont 1.38\endgroup & \begingroup\fontsize{8}{10}\selectfont 1.18\endgroup & \begingroup\fontsize{8}{10}\selectfont 0.26\endgroup & \begingroup\fontsize{8}{10}\selectfont 0.13\endgroup & \begingroup\fontsize{8}{10}\selectfont 0.09\endgroup & \begingroup\fontsize{8}{10}\selectfont 0.25\endgroup & \begingroup\fontsize{8}{10}\selectfont \endgroup & \begingroup\fontsize{8}{10}\selectfont \endgroup & \begingroup\fontsize{8}{10}\selectfont \endgroup & \begingroup\fontsize{8}{10}\selectfont \endgroup & \begingroup\fontsize{8}{10}\selectfont 0.20\endgroup & \begingroup\fontsize{8}{10}\selectfont 0.03\endgroup & \begingroup\fontsize{8}{10}\selectfont 0.30\endgroup\\
\begingroup\fontsize{8}{10}\selectfont L\endgroup & \begingroup\fontsize{8}{10}\selectfont VNM\endgroup & \begingroup\fontsize{8}{10}\selectfont 0.31\endgroup & \begingroup\fontsize{8}{10}\selectfont 1.19\endgroup & \begingroup\fontsize{8}{10}\selectfont 1.20\endgroup & \begingroup\fontsize{8}{10}\selectfont 0.51\endgroup & \begingroup\fontsize{8}{10}\selectfont 0.35\endgroup & \begingroup\fontsize{8}{10}\selectfont 0.08\endgroup & \begingroup\fontsize{8}{10}\selectfont 0.20\endgroup & \begingroup\fontsize{8}{10}\selectfont 0.02\endgroup & \begingroup\fontsize{8}{10}\selectfont \endgroup & \begingroup\fontsize{8}{10}\selectfont \endgroup & \begingroup\fontsize{8}{10}\selectfont 0.00\endgroup & \begingroup\fontsize{8}{10}\selectfont 0.00\endgroup & \begingroup\fontsize{8}{10}\selectfont 0.04\endgroup & \begingroup\fontsize{8}{10}\selectfont 0.30\endgroup\\
\midrule
\begingroup\fontsize{8}{10}\selectfont M\endgroup & \begingroup\fontsize{8}{10}\selectfont SEN\endgroup & \begingroup\fontsize{8}{10}\selectfont 0.53\endgroup & \begingroup\fontsize{8}{10}\selectfont 0.51\endgroup & \begingroup\fontsize{8}{10}\selectfont 0.21\endgroup & \begingroup\fontsize{8}{10}\selectfont 0.24\endgroup & \begingroup\fontsize{8}{10}\selectfont 0.06\endgroup & \begingroup\fontsize{8}{10}\selectfont 0.12\endgroup & \begingroup\fontsize{8}{10}\selectfont 0.12\endgroup & \begingroup\fontsize{8}{10}\selectfont 0.11\endgroup & \begingroup\fontsize{8}{10}\selectfont 0.14\endgroup & \begingroup\fontsize{8}{10}\selectfont \endgroup & \begingroup\fontsize{8}{10}\selectfont 0.02\endgroup & \begingroup\fontsize{8}{10}\selectfont 0.11\endgroup & \begingroup\fontsize{8}{10}\selectfont 0.19\endgroup & \begingroup\fontsize{8}{10}\selectfont 0.13\endgroup\\
\begingroup\fontsize{8}{10}\selectfont M\endgroup & \begingroup\fontsize{8}{10}\selectfont NGA\endgroup & \begingroup\fontsize{8}{10}\selectfont 0.46\endgroup & \begingroup\fontsize{8}{10}\selectfont 0.87\endgroup & \begingroup\fontsize{8}{10}\selectfont 0.33\endgroup & \begingroup\fontsize{8}{10}\selectfont 0.43\endgroup & \begingroup\fontsize{8}{10}\selectfont 0.04\endgroup & \begingroup\fontsize{8}{10}\selectfont 0.15\endgroup & \begingroup\fontsize{8}{10}\selectfont 0.20\endgroup & \begingroup\fontsize{8}{10}\selectfont 0.10\endgroup & \begingroup\fontsize{8}{10}\selectfont 0.16\endgroup & \begingroup\fontsize{8}{10}\selectfont \endgroup & \begingroup\fontsize{8}{10}\selectfont \endgroup & \begingroup\fontsize{8}{10}\selectfont 0.06\endgroup & \begingroup\fontsize{8}{10}\selectfont 0.20\endgroup & \begingroup\fontsize{8}{10}\selectfont 0.10\endgroup\\
\begingroup\fontsize{8}{10}\selectfont M\endgroup & \begingroup\fontsize{8}{10}\selectfont BGD\endgroup & \begingroup\fontsize{8}{10}\selectfont 0.36\endgroup & \begingroup\fontsize{8}{10}\selectfont 0.73\endgroup & \begingroup\fontsize{8}{10}\selectfont 1.04\endgroup & \begingroup\fontsize{8}{10}\selectfont 0.33\endgroup & \begingroup\fontsize{8}{10}\selectfont 0.22\endgroup & \begingroup\fontsize{8}{10}\selectfont 0.20\endgroup & \begingroup\fontsize{8}{10}\selectfont 0.15\endgroup & \begingroup\fontsize{8}{10}\selectfont 0.13\endgroup & \begingroup\fontsize{8}{10}\selectfont \endgroup & \begingroup\fontsize{8}{10}\selectfont \endgroup & \begingroup\fontsize{8}{10}\selectfont \endgroup & \begingroup\fontsize{8}{10}\selectfont 0.01\endgroup & \begingroup\fontsize{8}{10}\selectfont 0.12\endgroup & \begingroup\fontsize{8}{10}\selectfont 0.16\endgroup\\
\midrule*
\end{longtable}
\end{ThreePartTable}
\endgroup{}

\clearpage

\begin{table}[H]

\caption{Electricity generation in 87 countries (2021)}
\centering
\resizebox{\linewidth}{!}{
\begin{threeparttable}
\begin{tabular}[t]{l|r|rrrrrrrrrl|r|rrrrrrrrrl|r|rrrrrrrrrl|r|rrrrrrrrrl|r|rrrrrrrrrl|r|rrrrrrrrrl|r|rrrrrrrrrl|r|rrrrrrrrrl|r|rrrrrrrrrl|r|rrrrrrrrrl|r|rrrrrrrrr}
\toprule
\multicolumn{2}{c}{ } & \multicolumn{9}{c}{Share of electricity production by source in percent (2021)} \\
\cmidrule(l{3pt}r{3pt}){3-11}
Country & TWh & \rotatebox{90}{Hydro} & \rotatebox{90}{Wind} & \rotatebox{90}{Solar} & \rotatebox{90}{Bioenergy} & \rotatebox{90}{Renewables} & \rotatebox{90}{Nuclear} & \rotatebox{90}{Oil} & \rotatebox{90}{Gas} & \rotatebox{90}{Coal}\\
\midrule
Argentina & 147.0 & 14\% & 9\% & 1\% & 1\% & 0\% & 7\% & 5\% & 61\% & 2\%\\
Armenia & 7.3 & 30\% & 0\% & 1\% & 0\% & 0\% & 25\% & 0\% & 43\% & 0\%\\
Austria & 67.0 & 58\% & 10\% & 4\% & 7\% & 0\% & 0\% & 5\% & 16\% & 0\%\\
Bangladesh & 81.0 & 1\% & 0\% & 1\% & 0\% & 0\% & 0\% & 17\% & 68\% & 13\%\\
Barbados & 1.1 & 0\% & 0\% & 7\% & 0\% & 0\% & 0\% & 93\% & 0\% & 0\%\\
Belgium & 99.0 & 0\% & 12\% & 6\% & 5\% & 0\% & 51\% & 3\% & 23\% & 0\%\\
Benin & 0.2 & 0\% & 0\% & 4\% & 0\% & 0\% & 0\% & 96\% & 0\% & 0\%\\
Bolivia & 11.0 & 31\% & 1\% & 4\% & 5\% & 0\% & 0\% & 0\% & 58\% & 0\%\\
Brazil & 663.0 & 55\% & 11\% & 3\% & 9\% & 0\% & 2\% & 3\% & 14\% & 4\%\\
Bulgaria & 47.0 & 10\% & 3\% & 3\% & 5\% & 0\% & 35\% & 1\% & 6\% & 36\%\\
Burkina Faso & 1.8 & 6\% & 0\% & 7\% & 0\% & 0\% & 0\% & 87\% & 0\% & 0\%\\
Cambodia & 8.7 & 46\% & 0\% & 4\% & 3\% & 0\% & 0\% & 5\% & 0\% & 42\%\\
Canada & 626.0 & 60\% & 6\% & 1\% & 1\% & 0\% & 14\% & 0\% & 12\% & 6\%\\
Chile & 82.0 & 20\% & 9\% & 13\% & 0\% & 0\% & 0\% & 6\% & 18\% & 34\%\\
Colombia & 81.0 & 72\% & 0\% & 0\% & 1\% & 0\% & 0\% & 5\% & 15\% & 6\%\\
Costa Rica & 13.0 & 73\% & 12\% & 0\% & 0\% & 13\% & 0\% & 1\% & 0\% & 0\%\\
Cote d'Ivoire & 11.0 & 30\% & 0\% & 0\% & 0\% & 0\% & 0\% & 20\% & 50\% & 0\%\\
Croatia & 15.0 & 47\% & 14\% & 1\% & 7\% & 1\% & 0\% & 0\% & 21\% & 10\%\\
Cyprus & 5.1 & 0\% & 5\% & 9\% & 1\% & 0\% & 0\% & 85\% & 0\% & 0\%\\
Czechia & 84.0 & 3\% & 1\% & 3\% & 6\% & 0\% & 37\% & 1\% & 9\% & 41\%\\
Denmark & 33.0 & 0\% & 49\% & 4\% & 26\% & 0\% & 0\% & 3\% & 5\% & 13\%\\
Dominican Rep. & 18.0 & 6\% & 7\% & 3\% & 1\% & 0\% & 0\% & 20\% & 36\% & 26\%\\
Ecuador & 32.0 & 79\% & 0\% & 0\% & 4\% & 0\% & 0\% & 13\% & 4\% & 0\%\\
Egypt & 202.0 & 7\% & 2\% & 2\% & 0\% & 0\% & 0\% & 13\% & 76\% & 0\%\\
El Salvador & 6.6 & 30\% & 0\% & 17\% & 8\% & 24\% & 0\% & 20\% & 0\% & 0\%\\
Estonia & 7.2 & 0\% & 10\% & 5\% & 25\% & 0\% & 0\% & 60\% & 1\% & 0\%\\
Ethiopia & 15.0 & 95\% & 4\% & 0\% & 0\% & 0\% & 0\% & 0\% & 0\% & 0\%\\
Finland & 72.0 & 22\% & 11\% & 0\% & 19\% & 0\% & 33\% & 5\% & 5\% & 4\%\\
France & 550.0 & 11\% & 7\% & 3\% & 2\% & 0\% & 69\% & 2\% & 6\% & 1\%\\
Georgia & 13.0 & 81\% & 1\% & 0\% & 0\% & 0\% & 0\% & 0\% & 19\% & 0\%\\
Germany & 582.0 & 3\% & 20\% & 8\% & 8\% & 0\% & 12\% & 4\% & 16\% & 28\%\\
Ghana & 21.0 & 34\% & 0\% & 0\% & 0\% & 0\% & 0\% & 17\% & 48\% & 0\%\\
Greece & 55.0 & 11\% & 19\% & 10\% & 1\% & 0\% & 0\% & 9\% & 41\% & 10\%\\
Guatemala & 14.0 & 41\% & 2\% & 2\% & 20\% & 2\% & 0\% & 14\% & 0\% & 19\%\\
Guinea-Bissau & 0.1 & 0\% & 0\% & 0\% & 0\% & 0\% & 0\% & 100\% & 0\% & 0\%\\
Hungary & 36.0 & 1\% & 2\% & 11\% & 6\% & 0\% & 44\% & 1\% & 27\% & 8\%\\
India & 1,714.0 & 9\% & 4\% & 4\% & 2\% & 0\% & 3\% & 0\% & 4\% & 74\%\\
Indonesia & 309.0 & 8\% & 0\% & 0\% & 5\% & 5\% & 0\% & 2\% & 18\% & 61\%\\
Iraq & 97.0 & 5\% & 0\% & 0\% & 0\% & 0\% & 0\% & 17\% & 78\% & 0\%\\
Ireland & 32.0 & 2\% & 31\% & 0\% & 3\% & 0\% & 0\% & 7\% & 48\% & 9\%\\
Israel & 73.0 & 0\% & 0\% & 6\% & 0\% & 0\% & 0\% & 0\% & 66\% & 27\%\\
Italy & 286.0 & 16\% & 7\% & 9\% & 7\% & 2\% & 0\% & 4\% & 50\% & 5\%\\
Jordan & 22.0 & 0\% & 7\% & 16\% & 0\% & 0\% & 0\% & 8\% & 69\% & 0\%\\
Kenya & 12.0 & 34\% & 13\% & 1\% & 1\% & 43\% & 0\% & 8\% & 0\% & 0\%\\
Latvia & 5.8 & 46\% & 2\% & 0\% & 15\% & 0\% & 0\% & 0\% & 36\% & 0\%\\
Liberia & 0.9 & 58\% & 0\% & 0\% & 0\% & 0\% & 0\% & 42\% & 0\% & 0\%\\
Lithuania & 4.2 & 9\% & 33\% & 5\% & 17\% & 0\% & 0\% & 8\% & 29\% & 0\%\\
Luxembourg & 1.2 & 9\% & 25\% & 15\% & 32\% & 0\% & 0\% & 6\% & 14\% & 0\%\\
Malawi & 1.4 & 70\% & 0\% & 12\% & 1\% & 0\% & 0\% & 16\% & 0\% & 0\%\\
Maldives & 0.7 & 0\% & 0\% & 8\% & 0\% & 0\% & 0\% & 92\% & 0\% & 0\%\\
Mali & 3.4 & 29\% & 0\% & 1\% & 6\% & 0\% & 0\% & 64\% & 0\% & 0\%\\
Mexico & 337.0 & 10\% & 6\% & 4\% & 2\% & 1\% & 3\% & 10\% & 59\% & 4\%\\
Mongolia & 7.1 & 1\% & 7\% & 2\% & 0\% & 0\% & 0\% & 0\% & 0\% & 90\%\\
Morocco & 41.0 & 3\% & 12\% & 4\% & 0\% & 0\% & 0\% & 11\% & 12\% & 58\%\\
Mozambique & 20.0 & 80\% & 0\% & 0\% & 1\% & 0\% & 0\% & 6\% & 12\% & 0\%\\
Myanmar & 22.0 & 40\% & 0\% & 0\% & 1\% & 0\% & 0\% & 18\% & 36\% & 4\%\\
Netherlands & 122.0 & 0\% & 15\% & 9\% & 9\% & 0\% & 3\% & 5\% & 47\% & 12\%\\
Nicaragua & 4.6 & 12\% & 14\% & 1\% & 11\% & 17\% & 0\% & 45\% & 0\% & 0\%\\
Niger & 0.4 & 0\% & 0\% & 11\% & 0\% & 0\% & 0\% & 89\% & 0\% & 0\%\\
Nigeria & 31.0 & 25\% & 0\% & 0\% & 0\% & 0\% & 0\% & 0\% & 72\% & 2\%\\
Norway & 151.0 & 92\% & 7\% & 0\% & 0\% & 0\% & 0\% & 0\% & 0\% & 0\%\\
Pakistan & 150.0 & 26\% & 2\% & 1\% & 1\% & 0\% & 10\% & 11\% & 37\% & 12\%\\
Paraguay & 40.0 & 100\% & 0\% & 0\% & 0\% & 0\% & 0\% & 0\% & 0\% & 0\%\\
Peru & 58.0 & 55\% & 3\% & 1\% & 1\% & 0\% & 0\% & 7\% & 31\% & 1\%\\
Philippines & 108.0 & 7\% & 1\% & 1\% & 2\% & 10\% & 0\% & 18\% & 14\% & 45\%\\
Poland & 179.0 & 1\% & 9\% & 2\% & 5\% & 0\% & 0\% & 3\% & 9\% & 71\%\\
Portugal & 49.0 & 24\% & 27\% & 5\% & 8\% & 0\% & 0\% & 3\% & 31\% & 2\%\\
Romania & 59.0 & 29\% & 11\% & 3\% & 1\% & 0\% & 19\% & 2\% & 17\% & 18\%\\
Russia & 1,110.0 & 19\% & 0\% & 0\% & 0\% & 0\% & 20\% & 1\% & 42\% & 17\%\\
Rwanda & 0.9 & 53\% & 0\% & 7\% & 0\% & 0\% & 0\% & 40\% & 0\% & 0\%\\
Senegal & 5.6 & 6\% & 4\% & 8\% & 2\% & 0\% & 0\% & 45\% & 32\% & 2\%\\
Serbia & 37.0 & 30\% & 3\% & 0\% & 1\% & 0\% & 0\% & 1\% & 1\% & 64\%\\
Slovakia & 30.0 & 15\% & 0\% & 2\% & 6\% & 0\% & 53\% & 4\% & 15\% & 6\%\\
South Africa & 223.0 & 1\% & 4\% & 3\% & 0\% & 0\% & 5\% & 1\% & 0\% & 86\%\\
Spain & 271.0 & 11\% & 23\% & 10\% & 3\% & 0\% & 21\% & 4\% & 26\% & 2\%\\
Suriname & 2.0 & 50\% & 0\% & 1\% & 0\% & 0\% & 0\% & 50\% & 0\% & 0\%\\
Sweden & 172.0 & 43\% & 16\% & 1\% & 8\% & 0\% & 31\% & 2\% & 0\% & 0\%\\
Switzerland & 61.0 & 61\% & 0\% & 5\% & 0\% & 0\% & 30\% & 4\% & 0\% & 0\%\\
Taiwan & 288.0 & 1\% & 1\% & 3\% & 1\% & 0\% & 10\% & 2\% & 38\% & 45\%\\
Thailand & 187.0 & 3\% & 2\% & 2\% & 7\% & 0\% & 0\% & 0\% & 65\% & 21\%\\
Togo & 0.6 & 24\% & 0\% & 3\% & 0\% & 0\% & 0\% & 73\% & 0\% & 0\%\\
Turkey & 331.0 & 17\% & 9\% & 4\% & 2\% & 3\% & 0\% & 1\% & 33\% & 31\%\\
Uganda & 4.4 & 91\% & 0\% & 3\% & 3\% & 0\% & 0\% & 3\% & 0\% & 0\%\\
United Kingdom & 307.0 & 2\% & 21\% & 4\% & 13\% & 0\% & 15\% & 3\% & 40\% & 2\%\\
United States & 4,152.0 & 6\% & 9\% & 4\% & 1\% & 0\% & 19\% & 1\% & 38\% & 22\%\\
Uruguay & 16.0 & 33\% & 32\% & 3\% & 10\% & 0\% & 0\% & 2\% & 20\% & 0\%\\
Vietnam & 245.0 & 31\% & 1\% & 11\% & 0\% & 0\% & 0\% & 0\% & 11\% & 47\%\\
\bottomrule
\end{tabular}
\begin{tablenotes}
\item \textit{Note: } 
\item This table provides summary statistics for electricity generation in 87 different countries of our sample. It reports the share of electricity generated by each source in each country in 2021 [\%] as well as the total annual electricity production [TWh]. Source: \textcite{IEA.2021} and Our World in Data \autocite{HannahRitchie.2020}.
\end{tablenotes}
\end{threeparttable}}
\end{table}

\clearpage

\begin{refcontext}[sorting=nyt]
\printbibliography[heading=subbibliography, title ={References (Appendix)}]
\end{refcontext}
\end{refsection}

\clearpage

\section{Supplementary information}

\subsection{Data availability} \label{data_availability} Data from household budget surveys are available from statistical agencies subject to permission and possible allowances. See also table \ref{tab:datasets}. Data from GTAP are available through \href{https://www.gtap.agecon.purdue.edu/}{GTAP}, subject to academic subscription.  

\subsection{Code availability} \label{code}
We distribute all code written for cleaning and harmonizing household data, modelling carbon intensity of consumption and analysis through \href{https://github.com/lmissbach/Carbon-Intensity-2023}{GitHub}. This repository also contains matching tables for all countries.

\subsection{Acknowledgements} \label{acknowledgements}

%We thank Florian Nachtigall for helpful comments and suggestions. 
We thank Marlene Merchert, Charlotte Buder and Paula Blechschmidt for research assistance. LM acknowledges funding from the German Federal Environmental Foundation (DBU).

\end{document}

